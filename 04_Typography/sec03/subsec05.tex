\begin{frame}{Modes}\relax

\TeX\ has 3(6) modes:
\begin{enumerate}
    \item {\bfseries {\csk Vertical} mode.} [Building the main vertical list, from which the pages of
output are derived.]
\item {\bfseries Internal {\csk vertical} mode.} [Building a vertical list for a vbox.]
\item {\bfseries {\csk Horizontal} mode.} [Building a horizontal list for a paragraph.]
\item {\bfseries Restricted {\csk horizontal} mode.} [Building a horizontal list for an hbox.]
\item {\bfseries {\csk Math} mode.} [Building a mathematical formula to be placed in a horizontal list.]
\item {\bfseries Display {\csk math} mode.} [Building a mathematical formula to be placed on
a line by itself, temporarily interrupting the current paragraph.]
     
\end{enumerate}
     
     \skfootnote{\knuthc{13}[95]}
\end{frame}

\begin{frame}{Difference between modes}

    The modes have lots of differences. For example:
    \begin{itemize}
        \item in horizontal mode only first space is taking into account
        \item in math mode generic font is italic, all spaces are ignored
        \item in Display math mode operators are drawing bigger, than in the regular one 
        \item in vertical mode all spaces and <return>s are ignored 
         
    \end{itemize}
     \skfootnote{use \ccol\leavevmode to leave vertical mode}
\end{frame}

\begin{frame}{More about math mode\magicPage}
     Math actually has 4 different styles. When you see that superscript $x^y$ is smaller then the text --- it is a different style. The styles are:
     
     \centering 
     \begin{tabular}{l|l|l|p{7em}}
     \hline
     Display style & \ccol\displaystyle & $\displaystyle A$ & main style for displayed formula\\
     Text style & \ccol\textstyle & $\textstyle A$ & main style for in-text formula\\
     Script style & \ccol\scriptstyle & $\scriptstyle A$ & main style for scripts\\
     Script-script style & \ccol\scriptscriptstyle & $\scriptscriptstyle A$ & main style for scripts in scripts\\
     \end{tabular}
     
 \skfootnote{\knuthc{17}[151]}
\end{frame}




