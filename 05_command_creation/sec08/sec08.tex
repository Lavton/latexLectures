\graphicspath{{sec08/images/}{sec08/code/}}
\lstset{inputpath=sec08/code/}


\begin{frame}{Filesystem writing\magicPage}
\twocolImg{
\lstinputlisting[linerange={11-17}]{writedoc.tex}
}{writedoc}

\footnotesize
The output in pdf is the result of listing the {\csk outfile.txt}
\begin{itemize}
    \item \ccol\newrite\ to get a free register
    \item \ccol\openout\ to open filename for output
    \item \ccol\write<register>\ to actually write 
    \item \ccol\noexpand\ to stop expanding (also see \ccol\expandafter)
    \item \ccol\closeout\ to close the file 
\end{itemize}

You also can use numbers. Like \ccol\write4

\end{frame}

\begin{frame}[fragile]{Filesystem reading\magicPage}
\twocolImg{
\lstinputlisting[linerange={18-31}, basicstyle=\tt\tiny]{readdoc.tex}
}{readdoc}

\footnotesize
\begin{itemize}
    \item \ccol\newread\ to get a free register
    \item \ccol\openin\ to open filename for input
    \item \ccol\read<register>\ \ccol{to}\string\<newvariable> to actually read
    \item \verb"\ifeof" to check if file is still have lines
    \item \ccol\closein\ to close the file 
\end{itemize}

You also can use numbers. Like \ccol\read4

\end{frame}


\begin{frame}[fragile]{How BiBLaTeX works? Proof-of-concept\magicPage}
\twocolImg{
\lstinputlisting[linerange={8-22, 28-31}, basicstyle=\tt\tiny]{PoCIO.tex}
}{PoCIO}

\footnotesize
\begin{enumerate}
    \item write info into a file 
    \item use an external command to do something with the file 
    \item read content from a file in a different place
     
\end{enumerate}

\end{frame}

\begin{frame}[fragile]{Use command line\magicPage}\relax
    \twocolImg{
\lstinputlisting[linerange={12-14}]{commandline.tex}
}{commandline}
\small

Use 
\begin{itemize}
    \item \ccol\write18\ to call the command line 
    \item \ccol\immidiate\ to run it as it reached (otherwise only when \TeX\ will ``print'' the page)
    \item use \verb|--enable-write18 -interaction=nonstopmode| keys for run offline
    \item the commands with internet connection will not work at Papeeria
     
\end{itemize}
\skfootnote{\stExC{https://tex.stackexchange.com/questions/20444/what-are-immediate-write18-and-how-does-one-use-them}. Also look at \ccol{filecontents} env.}
\end{frame}

\begin{frame}{Command names manipulation\magicPage}\relax
    \twocolImg{
\lstinputlisting[linerange={12-13}]{csnamemy.tex}
}{csnamemy}

\begin{itemize}
    \item \ccol\csname\ --- \ccol\endcsname\ to ``compile'' command from name 
    \item \ccol\string\ to show the name 
     
\end{itemize}

\end{frame}

\begin{frame}{Catcodes\magicPage}\relax

\twocolImg{
\lstinputlisting[linerange={12-18}]{catcodesmy.tex}
}{catcodesmy}

\ccol\catcode\ shows what symbol will be responsible for the group, what for comment etc.

    \skfootnote{\tugc{https://www.tug.org/utilities/plain/cseq.html#catcode-rp} \knuthc{app. B}[354]}
\end{frame}

\begin{frame}{Lua\TeX\ Proof-of-concept\magicPage}\relax
\twocolImg{
\lstinputlisting[linerange={8-8, 10-13}]{luamy.tex}
}{luamy}
     
\skfootnote{\stExC{https://tex.stackexchange.com/questions/70/what-is-a-simple-example-of-something-you-can-do-with-luatex}}
\end{frame}
