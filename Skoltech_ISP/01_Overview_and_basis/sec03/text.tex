% \knuthc  knuth the TeXBook
% \lvoc   Lvovsky
% \lamc  lamport latex 
% \slshape different font for footnote
\graphicspath{{sec03/images/s2/}{sec03/code/s2/}}
\lstset{inputpath=sec03/code/s2/}

\begin{frame}[fragile]{Writing \textit{Text}}\relax
\inclassFrag{Please, write a single line of text

use {\url{https://en.lipsum.com/feed/html}} in case of \cancel{laziness} optimization of resources
}[1]
\cprotect\twocolImg{
    \lstinputlisting[linerange={10-10},basicstyle=\tt\normalsize,showspaces=true]{general}
}{general}

\cprotect\skfootnote{\knuthc{8}[53]\\ \verb|\lstinputlisting[showspaces=true]|}
\end{frame}


\cprotect\inclassframe{
\begin{frame}{\exFrame{add some spaces}}
     Do the following and compare results {\csk each} from previous step.
    \begin{enumerate}
         \item add a bunch of spaces between two arbitrary words
         \item create two more lines of text 
         \item add spaces at the beginning of the last line
         \item add \% at the end of the first line  
    \end{enumerate}
    
\end{frame}
}

\begin{frame}[fragile]{Spaces}\relax


\cprotect\twocolImg{
    \lstinputlisting[linerange={10-14},basicstyle=\tt\small,showspaces=true]{spaces}
}{spaces}
\skfootnote{\lvoc{III.9}[141], \overC{https://www.overleaf.com/learn/latex/Line_breaks_and_blank_spaces}}
\end{frame}

\begin{frame}[fragile]{Paragraph}\relax
\skfootnote{\overC{https://www.overleaf.com/learn/latex/Paragraphs_and_new_lines}}

\cprotect\twocolImg{
    \lstinputlisting[linerange={10-17},basicstyle=\tt\small,showspaces=true]{paragr}
}{paragr}
\end{frame}

\begin{frame}[fragile]{Indents}\relax
\cprotect\twocolImg{
    \lstinputlisting[linerange={10-11},basicstyle=\tt\small,showspaces=true]{indents}
}{indents}
\skfootnote{\knuthc{13}[97], \knuthc{24}[294] \wikiC{https://en.wikibooks.org/wiki/LaTeX/Paragraph_Formatting} \stExC{https://tex.stackexchange.com/questions/45501/how-to-add-indentation}}
\end{frame}

\begin{frame}[fragile]{More spaces}\relax
\begin{columns}
\begin{column}{0.4\textwidth}
     \lstinline[basicstyle=\tt\normalsize]| Use ~ for non-breakable space and ~~~more spaces. Or \ \ \ like this|
\end{column}
\begin{column}{0.4\textwidth}
     Use ~ for non-breakable space and ~~~more spaces. Or \ \ \ like this
\end{column}
\end{columns}
\hrule
\begin{columns}
\begin{column}{0.4\textwidth}
     \lstinline[basicstyle=\tt\normalsize]| Use \\ for new line. And more then one ~\\~\\~\\ new line|
\end{column}
\begin{column}{0.4\textwidth}
~\\
     Use \\ for new line. And more then one ~\\~\\~\\ new line
\end{column}
\end{columns}

\end{frame}

\begin{frame}[fragile]{Spaces and commands}\relax

\newcommand{\appendTline}[2]{\vspace*{10pt}\begin{columns}
        \begin{column}{0.45\textwidth}
          \hfill #1 
        \end{column}
        \begin{column}{0.45\textwidth}
             \hfill #2\hfill \hfill
        \end{column}
    \end{columns}
    \vphantom.
    \hrule
    }

    \cprotect[mm]\appendTline{\csk source}{\csk result}
    \hrule height 0.05pt
    \cprotect[mm]\appendTline{\lstinline[basicstyle=\tt\normalsize,showspaces=true]|\TeX book|}{\TeX book}
    \cprotect[mm]\appendTline{\lstinline[basicstyle=\tt\normalsize,showspaces=true]|\TeX{} book|}{\TeX{} book}
    \cprotect[mm]\appendTline{\lstinline[basicstyle=\tt\normalsize,showspaces=true]|\TeX\ book|}{\TeX\ book}

\end{frame}

\begin{frame}[fragile]{Fonts}{shape (form)}\relax

\let\oldOp\{
\let\oldCl\} 
\let\oldBck\textbackslash
\def\{{{\normalfont\oldOp}}
\def\}{{\normalfont\oldCl}}
\def\textbackslash{{\normalfont\oldBck}}

% `\csname textit\endcsname` creates \textit
\newcommand{\putinside}[1]{\csname #1\endcsname{{\csk \textbackslash #1}\{text\}} }\relax
\newcommand{\putoutside}[1]{ { \csname #1\endcsname \{{\csk \textbackslash #1} text\} } }

\begin{tabular}{rcc}
    normal shape & (normal text) & (normal text)\\
    \textup{Upright shape} & \putinside{textup} & \putoutside{upshape}\\
    \textit{Italic shape} & \putinside{textit} & \putoutside{itshape}\\
    \textsl{Slanted shape} & \putinside{textsl} & \putoutside{slshape}\\
    \textsc{Small caps shape} & \putinside{textsc} & \putoutside{scshape}\\
    \hphantom{\textsc{Small caps shape}} & \hphantom{\putinside{textsc}} & \hphantom{\putoutside{scshape}}\\
\end{tabular}
\skfootnote{{} \lvoc{III.5.2}[111] \lmanc{4}[26] \wikiC{https://en.wikibooks.org/wiki/LaTeX/Fonts} \overC{https://www.overleaf.com/learn/latex/Bold,_italics_and_underlining} \\ code generate this slide is interesting. Look at it :) }
\end{frame}


\begin{frame}[fragile]{Fonts}{saturation (series)}\relax

\let\oldOp\{
\let\oldCl\} 
\let\oldBck\textbackslash
\def\{{{\normalfont\oldOp}}
\def\}{{\normalfont\oldCl}}
\def\textbackslash{{\normalfont\oldBck}}
% `\csname textit\endcsname` creates \textit
\newcommand{\putinside}[1]{\csname #1\endcsname{{\csk \textbackslash #1}\{text\}} }\relax
\newcommand{\putoutside}[1]{ { \csname #1\endcsname \{{\csk \textbackslash #1} text\} } }
\begin{tabular}{rcc}
    \textmd{Medium series} & \putinside{textmd} & \putoutside{mdseries}\\
    \textbf{Boldface series} & \putinside{textbf} & \putoutside{bfseries}\\
    \hphantom{\textsc{Small caps shape}} & \hphantom{\putinside{textsc}} & \hphantom{\putoutside{scshape}}\\
\end{tabular}
\end{frame}


\begin{frame}[fragile]{Fonts}{garniture (family)}\relax

\let\oldOp\{
\let\oldCl\} 
\let\oldBck\textbackslash
\def\{{{\normalfont\oldOp}}
\def\}{{\normalfont\oldCl}}
\def\textbackslash{{\normalfont\oldBck}}
% `\csname textit\endcsname` creates \textit
\newcommand{\putinside}[1]{\csname #1\endcsname{{\csk \textbackslash #1}\{text\}} }\relax
\newcommand{\putoutside}[1]{ { \csname #1\endcsname \{{\csk \textbackslash #1} text\} } }
\begin{tabular}{rcc}

    \textrm{Roman family} & \putinside{textrm} & \putoutside{rmfamily}\\
    \textsf{Sans serif family} & \putinside{textsf} & \putoutside{sffamily}\\
    \texttt{Typewriter family} & \putinside{texttt} & \putoutside{ttfamily}\\
    \hphantom{\textsc{Small caps shape}} & \hphantom{\putinside{textsc}} & \hphantom{\putoutside{scshape}}\\
\end{tabular}
\end{frame}


\begin{frame}[fragile]{Fonts}{size}\relax
\newcommand{\putoutside}[1]{ { \csname #1\endcsname \{{\csk \textbackslash #1} text\} } }

\begin{multicols}{2}
\hspace{-20em}
\begin{itemize}
\item \hbox{\putoutside{Huge}}
\item \putoutside{huge}
\item \putoutside{LARGE}
\item \putoutside{Large}
\item \putoutside{large}
\item \putoutside{normalsize}
\item \putoutside{small}
\item \putoutside{footnotesize}
\item \putoutside{scriptsize}
\item \putoutside{tiny}
\end{itemize}
\end{multicols}

\cprotect\skfootnote{{ }\lvoc{III.5.1}[108] {\normalfont \url{https://texblog.org/2012/08/29/changing-the-font-size-in-latex/}}\\ \verb|\fontsize{<size>}{<line space>}\selectfont| for arbitrary sizes}

\end{frame}

\begin{frame}[fragile]{To default}

    \lstinline[basicstyle=\tt\normalsize]|\Huge text \ttfamily text \itshape text \normalfont\normalsize text|

     \Huge text \ttfamily text \itshape text \normalfont\normalsize text
\end{frame}

\begin{frame}[fragile]{to default: ``GROUPS''}\relax
    \begin{itemize}
    \item Lots of \LaTeX{} commands are ``local''
    \item Local commands loose their effect outside the group 
    \item ``group'' is
    \begin{itemize}
        \item \lstinline[basicstyle=\tt\normalsize]|{group}|
        \item \lstinline[basicstyle=\tt\normalsize]|\begingroup group\endgroup|
        \item \lstinline[basicstyle=\tt\normalsize]|$group$|
        \item \lstinline[basicstyle=\tt\normalsize]|\begin{env}group\end{env}|
    \end{itemize}
    \item often something inside \{group\} means ``indivisible'', ``atomic'', ``single'' for \TeX\ commands.
    \end{itemize}

\skfootnote{\lvoc{I.2.5}[21] \knuthc{5}[29]}
\end{frame}


\begin{frame}[fragile]{Enumerate}\relax
\cprotect\twocolImg{
    \inputminted[firstline=10, lastline=16,fontsize=\tt\footnotesize]{latex}{sec03/code/s2/enumimy.tex}
    % \lstinputlisting[linerange={10-16},basicstyle=\tt\footnotesize]{enumimy}
}{enumimy}

\cprotect\twocolImg{
    \inputminted[firstline=10, lastline=16,fontsize=\tt\footnotesize]{latex}{sec03/code/s2/itemimy.tex}
    % \lstinputlisting[linerange={10-16},basicstyle=\tt\footnotesize]{itemimy}
}{itemimy}
    
    \cprotect\skfootnote{{ }\lvoc{III.7.4}[131] \lmanc{8.7}[59] \lmanc{8.14}[65] \wikiC{https://en.wikibooks.org/wiki/LaTeX/List_Structures} \overC{https://www.overleaf.com/learn/latex/Lists}\\ 
    \slshape also look at env. \verb|description| \lmanc{8.4}[57] and at \verb|\item[x]| \lmanc{8.16.1}[71] }
     
\end{frame}
\newlength{\myboxlen}%

\begin{frame}[fragile]{Other languages}{accents}\relax
\setlength{\myboxlen}{9em}

\newcommand{\showacc}[2]{%
\makebox[\myboxlen]{\hfill\makebox[0.45\myboxlen]{\hfill{\bfseries\csk\string#1}\{#2\}}\makebox[0.25\myboxlen]{$\to$}\makebox[0.3\myboxlen]{#1{#2}\hfill}\hfill}%
}

\showacc{\`}{o}\hfill
\showacc{\'}{o}\hfill
\showacc{\^}{o}\hfill
\showacc{\"}{o}\hfill
\showacc{\H}{o}\hfill
\showacc{\c}{o}\hfill
\showacc{\k}{a}\hfill
\showacc{\=}{o}\hfill
\showacc{\b}{o}\hfill
\showacc{\.}{o}\hfill
\showacc{\d}{u}\hfill
\showacc{\r}{a}\hfill
\showacc{\u}{o}\hfill
\showacc{\v}{s}\hfill
\makebox[\myboxlen]{}\hfill
~\\[2ex]
\showacc{\l}{}\hfill
\showacc{\i}{}\hfill
\showacc{\j}{}\hfill

    \skfootnote{\url{https://tex.stackexchange.com/tags/accents/info} \wikiC{https://en.wikibooks.org/wiki/LaTeX/Special_Characters} \tugC{https://www.tug.org/utilities/plain/cseq.html\#accent-rp}}
\end{frame}

\begin{frame}[fragile, t]{Other languages}{complite solution: russian}\relax
     \begin{columns}[t]
          \begin{column}{0.45\textwidth}
          XeLaTeX
          \inputminted[firstline=2, lastline=5,fontsize=\tt\small]{latex}{sec03/code/s2/russianXe.tex}
            %   \lstinputlisting[linerange={2-5},basicstyle=\tt\small]{russianXe}
          \end{column}
          \begin{column}{0.45\textwidth}
          pdfLaTeX
          \inputminted[firstline=2, lastline=4,fontsize=\tt\small]{latex}{sec03/code/s2/russianpdf.tex}
            %   \lstinputlisting[linerange={2-4},basicstyle=\tt\small]{russianpdf}
          \end{column}
     \end{columns}
\end{frame}

\inclassframe{
\begin{frame}{\exFrame{Try to write the following}}{lvl 1}\relax
    Typography Rules
    \begin{enumerate}
         \item Use \textbf{one}, maximum \textit{two} different fonts to Attract attention
         \begin{itemize}
             \item Think twice before adding a new layer to you list
         \end{itemize}
    \end{enumerate}
\end{frame}

\begin{frame}{\exFrame{Try to write the following}}{lvl 2}\relax

{\scshape Typography \Large Rules}
\begin{enumerate}
     \item Use \textbf{one},\textit{maximum \slshape two} {\large different} fonts to \textbf{Attract \textit{attention}}
     \item {\footnotesize Remember, that the modifiers often interfere}.
     \begin{itemize}
         \item Think twice {\scshape before} adding a new layer to you list
         \item (better do not put anything in brackets)
     \end{itemize}
     \item \small two very {\Huge different} font sizes nearby is a movetone
     \item \Large \scshape  Always keep in mind: \ttfamily \bfseries You can break all typography rules if you are aware you are breaking them
     
\end{enumerate}
     
\end{frame}
}

