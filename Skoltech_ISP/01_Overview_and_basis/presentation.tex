\documentclass[aspectratio=169]{beamer}
\usepackage{ifxetex}
\usepackage{fontspec}
\usepackage{xunicode}
\usepackage{xltxtra}
\usepackage{xecyr}
\usepackage{polyglossia}
\usepackage{multicol}
\usepackage{dsfont}

\usepackage{minted}
\usepackage{beamerskoltech} 
% \renewcommand{\logoname}{sklogo.png} % <- default value for logo 
% \renewcommand{\logoname}{} % <- if you want no logo
\usepackage{cancel}


\usepackage{progressbar}
\colorlet{progressbarcolor}{skoltechgreen}
\usepackage[mode=in]{inoutclass}
\definecolor{inclasscolor}{RGB}{46, 228, 182}
\definecolor{outclasscolor}{RGB}{41, 0, 204}
\usepackage{xargs}
\usepackage{latexLectures}


\title{\LaTeX:\\ \Large from dummy to \TeX nician}
\subtitle{Overview and basis}
\author{Anton Lioznov, Pavel Kuzmin}
\institute{Skoltech, \\Project Center of Omics Technologies and Advanced Mass Spectrometry}
\date{ISP 2025,\\ \textit{lesson 1}}

\begin{document}


\frame{\titlepage}

\AtBeginSection[] 
{
  \begin{frame}{What we will know?}
  \tableofcontents[currentsection,hideallsubsections]
  \end{frame}
}
\AtBeginSubsection[]
{
  \begin{frame}{What we will know?}
  \tableofcontents[currentsubsection, hideothersubsections, sectionstyle=show/hide, subsectionstyle=show/shaded/hide]
  \end{frame}
}

{\supressprogressbartrue


\begin{frame}\frametitle{What we will know?}
\tableofcontents[hideallsubsections]
\end{frame}

\supressprogressbarfalse
}

% \begin{frame}
% % \begin{enumerate}
% %     \item Overview and basis
% %         \begin{tikzpicture}[remember picture, overlay]
% %             \draw[decoration={brace}, decorate] 
% %             ([yshift=0.5ex]pic cs:start) --
% %             ([yshift=-2ex]pic cs:end)
% %             node[midway, right=0.3cm] {User level};
% %         \end{tikzpicture}\tikz[remember picture]\coordinate(start);
% %     \item Document creation\tikz[remember picture]\coordinate(end);
% %     \item TikZ and Typography
% %     \item \TeX\ and Typography
% %     \item Command creation
% % \end{enumerate}
     
% %      \begin{itemize}
% %     \item[$\left.\begin{cases}
% %         \text{1.} & \text{First item in the list} \\
% %         \text{2.} & \text{Second item in the list}
% %     \end{cases}\right\}$] Text explaining items 1 and 2
    
% %     \item[$\left.\begin{cases}
% %         \text{3.} & \text{Third item in the list}
% %     \end{cases}\right\}$]
    
% %     \item[$\left.\begin{cases}
% %         \text{4.} & \text{Fourth item in the list} \\
% %         \text{5.} & \text{Fifth item in the list}
% %     \end{cases}\right\}$] Text explaining items 4 and 5
% % \end{itemize}

% % \begin{enumerate}
% %     \item \tikz\node(item1){Item 1};
% %     \item \tikz\node(item2){Item 2};
% %     \item \tikz\node(item3){Item 3};
% %     \item \tikz\node(item4){Item 4};
% %     \item \tikz\node(item5){Item 5};
% % \end{enumerate}

% % \begin{tikzpicture}[overlay, remember picture]
% %     % Bracket for items 1 and 2
% %     \draw[decorate, decoration={brace, amplitude=5pt}, thick]
% %         ([xshift=-5pt, yshift=3pt]item1.north west) -- ([xshift=-5pt, yshift=-3pt]item2.south west)
% %         node[midway, left=12pt] {Group A};

% %     % Bracket for item 3
% %     \draw[decorate, decoration={brace, amplitude=5pt}, thick]
% %         ([xshift=-5pt, yshift=3pt]item3.north west) -- ([xshift=-5pt, yshift=-3pt]item3.south west)
% %         node[midway, left=12pt] {Solo Item};

% %     % Bracket for items 4 and 5
% %     \draw[decorate, decoration={brace, amplitude=5pt}, thick]
% %         ([xshift=-5pt, yshift=3pt]item4.north west) -- ([xshift=-5pt, yshift=-3pt]item5.south west)
% %         node[midway, left=12pt] {Group B};
% % \end{tikzpicture}
% \end{frame}

\section{Technical agreements}

\begin{frame}{Agreements}{I}\relax
     {\Large inclass/outclass versions}
     \begin{itemize}
          \item two slightly different versions for class and home
          \item class version is more interactive and contains less information
          \inclass{\item \inclasshigh{} this line will be shown only in class} 
          \outclass{\item \outclasshigh{} this line will be shown only at home version}
     \end{itemize}
\end{frame}
\inclassframe{\begin{frame}{Frame for a class}
     
\end{frame}}
\outclassframe{\begin{frame}{Frame for home}
     
\end{frame}}

{\supressfootnotefalse
\begin{frame}[fragile]{Agreements}{II}\relax
\newcommand{\tikzmark}[1]{\tikz[overlay,remember picture] \node (#1) {};}

{ \Large Footnotes }
\begin{tikzpicture}[overlay,remember picture]
\draw[->,ultra thick] (0,0.1) to[out=0,in=45] (8, -1.5) to[out=225,in=90] (5,-5.5);
\end{tikzpicture}

\skfootnote{Like this}
\begin{itemize}
     \item For second reading
     \item Contains advanced usage of the command 
     \item Contains references to read more 
     \begin{itemize}
         \item to the exact chapter 
         \item (often) with the href to exact page  
     \end{itemize}
     \item Contains some comments
     \item Mostly for outclass version
\end{itemize}
\end{frame}

}

{\inclassmodetrue
\newcommand{\tikzmark}[1]{\tikz[overlay,remember picture] \node (#1) {};}

\begin{frame}{Agreements}{III}\relax
{ \Large Addition information -- ``magic'' \tikzmark{startM}} 

\begin{tikzpicture}[remember picture,overlay,shift={(current page.north east)}]
\node[anchor=north east,xshift=-0cm,yshift=-0cm](endM) {%
{\includegraphics[width=1cm]{images/magic}}%
};
\end{tikzpicture}%

\begin{tikzpicture}[overlay, remember picture]
    \draw[->,ultra thick] (startM) to[out=0, in=180] (endM);
\end{tikzpicture}

\begin{itemize}
     \item To have the full picture 
     \item Not to analyze or to puzzle out in class 
\end{itemize}
\end{frame}

}


\begin{frame}{\exFrame{Agreements}}{V}\relax


{ \Large Exercises } \begin{tikzpicture}[overlay]
% \draw[white] (0,0) -- (0, 0.6);
\draw[->,ultra thick] (0,0.1) to[out=0,in=-90](-2.1,3.15);
\end{tikzpicture}


\begin{itemize}
     \item To work in class 
\end{itemize}
\end{frame}
\begin{frame}{Special thanks to}\relax
    Our TAs:
     \begin{itemize}
         \item Peter Borisovets
         \item Pavel Kuzmin
         \item Anna Litvin
     \end{itemize}
\end{frame}
%%%%%%%%%%%%%%%%%%%%%%%%%% MAIN SECTION %%%%%%%%%%%%%%%%%%%%%%%%%%%%%%%%%

\section{Why \LaTeX? Beauty and fun}


\graphicspath{{sec01/images/}{sec01/code/}}
\lstset{inputpath=sec01/code/}

\subsection{Introduction: what is TikZ and when to use it}
\documentclass{article}
\usepackage{fontspec}
\pagestyle{empty}
\usepackage{geometry}
\geometry{paperwidth=50mm, paperheight=20mm, left=2mm, top=0mm, right=2mm, bottom=0mm} %, layoutwidth=60mm, layoutheight=35mm,
\parindent=0pt

\begin{document}
\vspace*{\fill} \vspace*{-5ex}

G\hskip0em lu\hskip0.5em e and k\kern0em e\kern0.5em rn provides...


\vspace*{\fill}
\end{document}

\subsection{General usage}
\documentclass{article}
\usepackage{fontspec}
\pagestyle{empty}
\usepackage{geometry}
\geometry{paperwidth=50mm, paperheight=65mm, left=5mm, top=5mm, right=5mm, bottom=5mm} %, layoutwidth=60mm, layoutheight=35mm,

\begin{document}
\vspace*{\fill} \vspace*{-5ex}

In most cases text is just a text. You write it and write and write. The system create line breaks by itself.

\vspace*{\fill}
\end{document}

\subsection{Graphs}
\outclassframe{
\begin{frame}{\ }\relax
    TikZ allows graphics to work according to ``rules''. This is a minus for an arbitrary drawing, but it is an advantage for those drawings that have a given structure and are also built according to ``rules''
\end{frame}
}

{\forcewidefootnote=1
\newcommand{\tikzmark}[1]{\tikz[overlay,remember picture] \node (#1) {};}

\begin{frame}{Graph example 1}\relax
    \twocolImg{
    \only<1>{\inputminted[firstline=9, lastline=10]{latex}{sec01/code/graphSample.tex}
    \inputminted[firstline= 16, lastline=27]{latex}{sec01/code/graphSample.tex}}
    % \only<2>{
    % \smash{\begin{tikzpicture}[overlay, remember picture]
    %         \draw[->, thick] (5.5, -0.5) to[out=0, in=80] +(3.8, 0.3);
    %     \end{tikzpicture}}
    % \inputminted[firstline=26, lastline=26, fontsize=\tt\nornalsize]{latex}{sec01/code/graphSample.tex}
        
    % }
    }{graphSample}
    
     
     \only<1>{You can write \ccol{below=of <label>} to have a relative coordinate}
     
\end{frame}
}

{\forcewidefootnote=1
\begin{frame}{Chain example\magicPage}{}\relax
    \includegraphics[width=\textwidth]{chainSample}
    
    \vspace{-2em}
    \inputminted[firstline= 9, lastline=10, fontsize=\tt\tiny]{latex}{sec01/code/chainSample.tex}
    \inputminted[firstline= 16, lastline=32, fontsize=\tt\tiny]{latex}{sec01/code/chainSample.tex}

     \skfootnote{\tikzc{I.5}[69]}
\end{frame}
}

\begin{frame}{Tree}

\twocolImg{
\inputminted[firstline=9, lastline=9]{latex}{sec01/code/treeSample.tex}
\inputminted[firstline= 16, lastline=25]{latex}{sec01/code/treeSample.tex}
}{treeSample}

We use \ccol\node\ and \ccol{child}.

\ccol{sibling distance} option provides a horizontal distance between nodes
     
\end{frame}

\inclassframe{
\begin{frame}{\exFrame{reproduce the following with tikZ}}{lvl 1. \textit{You can use absolute coodrinates}}\relax



     \includegraphics[width=0.4\textwidth]{exLvl1}
\end{frame}

\begin{frame}{\exFrame{reproduce the following with tikZ}}{lvl 2. \textit{You can't use absolute coodrinates}}\relax

     
     \includegraphics[width=0.4\textwidth]{exLvl2}
\end{frame}

}




\subsection{Arrangment}
\outclassframe{
\begin{frame}{\ }\relax
    Very few people make full-fledged charts in tikZ. But it is much more common to add tikZ as text decoration element.
\end{frame}
}


\begin{frame}[fragile]{Introduction}\relax

TikZ is often used not as ``independent picture'', but as a part of the presentation or document. 

\end{frame}

\begin{frame}[fragile]{Example: CV\magicPage}\relax
    \begin{columns}
        \begin{column}{0.45\textwidth}
              \only<1>{
              \expandafter{\inputminted[firstline= 16, lastline=20]{latex}{sec01/code/cv_examples.tex}
                }
                }
                \only<2>{\expandafter{\inputminted[firstline= 22, lastline=32]{latex}{sec01/code/cv_examples.tex}}}
        \end{column}
        \begin{column}{0.45\textwidth}
             \includegraphics[width=\textwidth, keepaspectratio]{cv_examples}
        \end{column}
    \end{columns}

    % \cprotect\twocolImg{
    % \only<1>{\inputminted[firstline= 16, lastline=20]{latex}{sec01/code/cv_examples.tex}}
    % \only<2>{\inputminted[firstline= 22, lastline=32]{latex}{sec01/code/cv_examples.tex}}
    % }{cv_examples}  
    
\end{frame}

\begin{frame}[fragile]{``Magic''}\relax

    How to produce ``magic'' \only<1>{\tikz[overlay, remember picture] \draw[->] (0, 0.1) to[out=0,in=180] (endM);}

\begin{tikzpicture}[remember picture,overlay,shift={(current page.north east)}]
    \node[anchor=north east,xshift=-0cm,yshift=-0cm](endM) {%
        {\includegraphics[width=1cm]{black_magic}}%
};
\end{tikzpicture}%

\inpause
\begin{minted}{latex}
\begin{tikzpicture}[remember picture,overlay,shift={(current page.north east)}]
\node[anchor=north east,xshift=-0cm,yshift=-0cm](endM) {%
    {\includegraphics[width=1cm]{images/magic}}%
};
\end{tikzpicture}
\end{minted}

\inpause
Notice \ccol{shift=\{(current page.north east)\}}
\end{frame}



\begin{frame}[fragile]{arrow to magic}\relax

How to produce this \tikz[overlay, remember picture] \node (arrstart) {}; arrow \tikz[overlay, remember picture] \node (startMMM) {};

\begin{tikzpicture}[remember picture,overlay,shift={(current page.north east)}]
    \node[anchor=north east,xshift=-0cm,yshift=-0cm](endMMM) {%
        {\includegraphics[width=1cm]{black_magic}}%
};
\end{tikzpicture}%

\tikz[overlay, remember picture] \draw[->, ultra thick] ($(startMMM)+(0,0.1)$) to[out=0,in=180] (endMMM);
\only<1>{\begin{tikzpicture}[overlay, remember picture]
     \path[->, ultra thick, name path=point c] ($(startMMM)+(0,0.1)$) to[out=0,in=180] (endMMM);
     
     \path[->, name path=point a] ($(startMMM)+(5,-1)$) -- +(-1, 3);
     \path[name intersections={of=point c and point a, by={intersection}}];
     \draw[thick, thick, ->] ($(arrstart)-(0.4ex,-0.6ex)$) to[out=45, in=110] (intersection);
\end{tikzpicture}}

\inpause

\begin{minted}{latex}
% remember position of (startM) node (and (endM) node from previous slide)
How to produce this arrow \tikz[overlay, remember picture] \node (startM) {};

% go from (startM) to (endM)
\tikz[overlay, remember picture] \draw[->, ultra thick] ($(startM)+(0,0.1)$) to[out=0,in=180] (endM);

\end{minted}

\skfootnote{What about arrow pointed on the thick arrow? It was produced with intersections library}
     
\end{frame}


\begin{frame}[fragile]{``Common belief''\magicPage}\relax
     \begin{center}
        \begin{tikzpicture}
             \node[align=center] (0,0) {
             \huge \LaTeX\ is only for use\\ \huge  in academic area
             };
             \uncover<2,3>{\node[rotate=30, bottom color=red!50, top color=red!50] (0,0) {\Huge WRONG};}
        \end{tikzpicture}
         
    \end{center}
    
\pause\pause was produce by 
\begin{minted}{latex}
\begin{tikzpicture}
    \node[align=center] (0,0) {
    \huge \LaTeX\ is only for use\\ \huge  in academic area
    };
    \uncover<2,3>{\node[rotate=30, bottom color=red!50, top color=red!50] (0,0) {\Huge WRONG}};
\end{tikzpicture}
\end{minted}
    
\end{frame}


\cprotect\inclassframe{
\begin{frame}[fragile]{\exFrame{Try to use other people's work}}{lvl 1}\relax

    \begin{enumerate}
        \item You have the code as below. Copy it to you document and make sure it is compiling. ``Play'' with the params and of \ccol\pictofraction\ and as the result give the enumerate list in pdf of what all params means. Ex: ``1. printed symbol, 2...''
         \item Go to \url{https://tikz.dev/library-mindmaps} and try to create your own mindmap
    \end{enumerate}

    \inputminted[firstline=7, lastline=17]{latex}{sec01/code/exExtLvl1.tex}
    \inputminted[firstline=21, lastline=22]{latex}{sec01/code/exExtLvl1.tex}
%7-17, 21-22
    %  \includegraphics[width=0.4\textwidth]{exLvl1}
\end{frame}

\begin{frame}[fragile, allowframebreaks]{\exFrame{Try to use other people's work}}{lvl 2}\relax
You copy a code, that produce a ``day cyrcle'' (from AltaCV). But some libraries and packages are missing. See logs and google errors, make this code works :)

\includegraphics[width=0.8\textwidth]{exExtLvl2full}

     \inputminted[fontsize=\tiny]{latex}{sec01/code/exExtLvl2notfull.tex}
    %  \includegraphics[width=0.4\textwidth]{exLvl2}
\end{frame}

}









\section{``Hello, world'':  first steps in \LaTeX}


\graphicspath{{sec01/images/}{sec01/code/}}
\lstset{inputpath=sec01/code/}

\subsection{Introduction: what is TikZ and when to use it}
\documentclass{article}
\usepackage{fontspec}
\pagestyle{empty}
\usepackage{geometry}
\geometry{paperwidth=50mm, paperheight=20mm, left=2mm, top=0mm, right=2mm, bottom=0mm} %, layoutwidth=60mm, layoutheight=35mm,
\parindent=0pt

\begin{document}
\vspace*{\fill} \vspace*{-5ex}

G\hskip0em lu\hskip0.5em e and k\kern0em e\kern0.5em rn provides...


\vspace*{\fill}
\end{document}

\subsection{General usage}
\documentclass{article}
\usepackage{fontspec}
\pagestyle{empty}
\usepackage{geometry}
\geometry{paperwidth=50mm, paperheight=65mm, left=5mm, top=5mm, right=5mm, bottom=5mm} %, layoutwidth=60mm, layoutheight=35mm,

\begin{document}
\vspace*{\fill} \vspace*{-5ex}

In most cases text is just a text. You write it and write and write. The system create line breaks by itself.

\vspace*{\fill}
\end{document}

\subsection{Graphs}
\outclassframe{
\begin{frame}{\ }\relax
    TikZ allows graphics to work according to ``rules''. This is a minus for an arbitrary drawing, but it is an advantage for those drawings that have a given structure and are also built according to ``rules''
\end{frame}
}

{\forcewidefootnote=1
\newcommand{\tikzmark}[1]{\tikz[overlay,remember picture] \node (#1) {};}

\begin{frame}{Graph example 1}\relax
    \twocolImg{
    \only<1>{\inputminted[firstline=9, lastline=10]{latex}{sec01/code/graphSample.tex}
    \inputminted[firstline= 16, lastline=27]{latex}{sec01/code/graphSample.tex}}
    % \only<2>{
    % \smash{\begin{tikzpicture}[overlay, remember picture]
    %         \draw[->, thick] (5.5, -0.5) to[out=0, in=80] +(3.8, 0.3);
    %     \end{tikzpicture}}
    % \inputminted[firstline=26, lastline=26, fontsize=\tt\nornalsize]{latex}{sec01/code/graphSample.tex}
        
    % }
    }{graphSample}
    
     
     \only<1>{You can write \ccol{below=of <label>} to have a relative coordinate}
     
\end{frame}
}

{\forcewidefootnote=1
\begin{frame}{Chain example\magicPage}{}\relax
    \includegraphics[width=\textwidth]{chainSample}
    
    \vspace{-2em}
    \inputminted[firstline= 9, lastline=10, fontsize=\tt\tiny]{latex}{sec01/code/chainSample.tex}
    \inputminted[firstline= 16, lastline=32, fontsize=\tt\tiny]{latex}{sec01/code/chainSample.tex}

     \skfootnote{\tikzc{I.5}[69]}
\end{frame}
}

\begin{frame}{Tree}

\twocolImg{
\inputminted[firstline=9, lastline=9]{latex}{sec01/code/treeSample.tex}
\inputminted[firstline= 16, lastline=25]{latex}{sec01/code/treeSample.tex}
}{treeSample}

We use \ccol\node\ and \ccol{child}.

\ccol{sibling distance} option provides a horizontal distance between nodes
     
\end{frame}

\inclassframe{
\begin{frame}{\exFrame{reproduce the following with tikZ}}{lvl 1. \textit{You can use absolute coodrinates}}\relax



     \includegraphics[width=0.4\textwidth]{exLvl1}
\end{frame}

\begin{frame}{\exFrame{reproduce the following with tikZ}}{lvl 2. \textit{You can't use absolute coodrinates}}\relax

     
     \includegraphics[width=0.4\textwidth]{exLvl2}
\end{frame}

}




\subsection{Arrangment}
\outclassframe{
\begin{frame}{\ }\relax
    Very few people make full-fledged charts in tikZ. But it is much more common to add tikZ as text decoration element.
\end{frame}
}


\begin{frame}[fragile]{Introduction}\relax

TikZ is often used not as ``independent picture'', but as a part of the presentation or document. 

\end{frame}

\begin{frame}[fragile]{Example: CV\magicPage}\relax
    \begin{columns}
        \begin{column}{0.45\textwidth}
              \only<1>{
              \expandafter{\inputminted[firstline= 16, lastline=20]{latex}{sec01/code/cv_examples.tex}
                }
                }
                \only<2>{\expandafter{\inputminted[firstline= 22, lastline=32]{latex}{sec01/code/cv_examples.tex}}}
        \end{column}
        \begin{column}{0.45\textwidth}
             \includegraphics[width=\textwidth, keepaspectratio]{cv_examples}
        \end{column}
    \end{columns}

    % \cprotect\twocolImg{
    % \only<1>{\inputminted[firstline= 16, lastline=20]{latex}{sec01/code/cv_examples.tex}}
    % \only<2>{\inputminted[firstline= 22, lastline=32]{latex}{sec01/code/cv_examples.tex}}
    % }{cv_examples}  
    
\end{frame}

\begin{frame}[fragile]{``Magic''}\relax

    How to produce ``magic'' \only<1>{\tikz[overlay, remember picture] \draw[->] (0, 0.1) to[out=0,in=180] (endM);}

\begin{tikzpicture}[remember picture,overlay,shift={(current page.north east)}]
    \node[anchor=north east,xshift=-0cm,yshift=-0cm](endM) {%
        {\includegraphics[width=1cm]{black_magic}}%
};
\end{tikzpicture}%

\inpause
\begin{minted}{latex}
\begin{tikzpicture}[remember picture,overlay,shift={(current page.north east)}]
\node[anchor=north east,xshift=-0cm,yshift=-0cm](endM) {%
    {\includegraphics[width=1cm]{images/magic}}%
};
\end{tikzpicture}
\end{minted}

\inpause
Notice \ccol{shift=\{(current page.north east)\}}
\end{frame}



\begin{frame}[fragile]{arrow to magic}\relax

How to produce this \tikz[overlay, remember picture] \node (arrstart) {}; arrow \tikz[overlay, remember picture] \node (startMMM) {};

\begin{tikzpicture}[remember picture,overlay,shift={(current page.north east)}]
    \node[anchor=north east,xshift=-0cm,yshift=-0cm](endMMM) {%
        {\includegraphics[width=1cm]{black_magic}}%
};
\end{tikzpicture}%

\tikz[overlay, remember picture] \draw[->, ultra thick] ($(startMMM)+(0,0.1)$) to[out=0,in=180] (endMMM);
\only<1>{\begin{tikzpicture}[overlay, remember picture]
     \path[->, ultra thick, name path=point c] ($(startMMM)+(0,0.1)$) to[out=0,in=180] (endMMM);
     
     \path[->, name path=point a] ($(startMMM)+(5,-1)$) -- +(-1, 3);
     \path[name intersections={of=point c and point a, by={intersection}}];
     \draw[thick, thick, ->] ($(arrstart)-(0.4ex,-0.6ex)$) to[out=45, in=110] (intersection);
\end{tikzpicture}}

\inpause

\begin{minted}{latex}
% remember position of (startM) node (and (endM) node from previous slide)
How to produce this arrow \tikz[overlay, remember picture] \node (startM) {};

% go from (startM) to (endM)
\tikz[overlay, remember picture] \draw[->, ultra thick] ($(startM)+(0,0.1)$) to[out=0,in=180] (endM);

\end{minted}

\skfootnote{What about arrow pointed on the thick arrow? It was produced with intersections library}
     
\end{frame}


\begin{frame}[fragile]{``Common belief''\magicPage}\relax
     \begin{center}
        \begin{tikzpicture}
             \node[align=center] (0,0) {
             \huge \LaTeX\ is only for use\\ \huge  in academic area
             };
             \uncover<2,3>{\node[rotate=30, bottom color=red!50, top color=red!50] (0,0) {\Huge WRONG};}
        \end{tikzpicture}
         
    \end{center}
    
\pause\pause was produce by 
\begin{minted}{latex}
\begin{tikzpicture}
    \node[align=center] (0,0) {
    \huge \LaTeX\ is only for use\\ \huge  in academic area
    };
    \uncover<2,3>{\node[rotate=30, bottom color=red!50, top color=red!50] (0,0) {\Huge WRONG}};
\end{tikzpicture}
\end{minted}
    
\end{frame}


\cprotect\inclassframe{
\begin{frame}[fragile]{\exFrame{Try to use other people's work}}{lvl 1}\relax

    \begin{enumerate}
        \item You have the code as below. Copy it to you document and make sure it is compiling. ``Play'' with the params and of \ccol\pictofraction\ and as the result give the enumerate list in pdf of what all params means. Ex: ``1. printed symbol, 2...''
         \item Go to \url{https://tikz.dev/library-mindmaps} and try to create your own mindmap
    \end{enumerate}

    \inputminted[firstline=7, lastline=17]{latex}{sec01/code/exExtLvl1.tex}
    \inputminted[firstline=21, lastline=22]{latex}{sec01/code/exExtLvl1.tex}
%7-17, 21-22
    %  \includegraphics[width=0.4\textwidth]{exLvl1}
\end{frame}

\begin{frame}[fragile, allowframebreaks]{\exFrame{Try to use other people's work}}{lvl 2}\relax
You copy a code, that produce a ``day cyrcle'' (from AltaCV). But some libraries and packages are missing. See logs and google errors, make this code works :)

\includegraphics[width=0.8\textwidth]{exExtLvl2full}

     \inputminted[fontsize=\tiny]{latex}{sec01/code/exExtLvl2notfull.tex}
    %  \includegraphics[width=0.4\textwidth]{exLvl2}
\end{frame}

}









\section{Mastering the base}


\graphicspath{{sec01/images/}{sec01/code/}}
\lstset{inputpath=sec01/code/}

\subsection{Introduction: what is TikZ and when to use it}
\documentclass{article}
\usepackage{fontspec}
\pagestyle{empty}
\usepackage{geometry}
\geometry{paperwidth=50mm, paperheight=20mm, left=2mm, top=0mm, right=2mm, bottom=0mm} %, layoutwidth=60mm, layoutheight=35mm,
\parindent=0pt

\begin{document}
\vspace*{\fill} \vspace*{-5ex}

G\hskip0em lu\hskip0.5em e and k\kern0em e\kern0.5em rn provides...


\vspace*{\fill}
\end{document}

\subsection{General usage}
\documentclass{article}
\usepackage{fontspec}
\pagestyle{empty}
\usepackage{geometry}
\geometry{paperwidth=50mm, paperheight=65mm, left=5mm, top=5mm, right=5mm, bottom=5mm} %, layoutwidth=60mm, layoutheight=35mm,

\begin{document}
\vspace*{\fill} \vspace*{-5ex}

In most cases text is just a text. You write it and write and write. The system create line breaks by itself.

\vspace*{\fill}
\end{document}

\subsection{Graphs}
\outclassframe{
\begin{frame}{\ }\relax
    TikZ allows graphics to work according to ``rules''. This is a minus for an arbitrary drawing, but it is an advantage for those drawings that have a given structure and are also built according to ``rules''
\end{frame}
}

{\forcewidefootnote=1
\newcommand{\tikzmark}[1]{\tikz[overlay,remember picture] \node (#1) {};}

\begin{frame}{Graph example 1}\relax
    \twocolImg{
    \only<1>{\inputminted[firstline=9, lastline=10]{latex}{sec01/code/graphSample.tex}
    \inputminted[firstline= 16, lastline=27]{latex}{sec01/code/graphSample.tex}}
    % \only<2>{
    % \smash{\begin{tikzpicture}[overlay, remember picture]
    %         \draw[->, thick] (5.5, -0.5) to[out=0, in=80] +(3.8, 0.3);
    %     \end{tikzpicture}}
    % \inputminted[firstline=26, lastline=26, fontsize=\tt\nornalsize]{latex}{sec01/code/graphSample.tex}
        
    % }
    }{graphSample}
    
     
     \only<1>{You can write \ccol{below=of <label>} to have a relative coordinate}
     
\end{frame}
}

{\forcewidefootnote=1
\begin{frame}{Chain example\magicPage}{}\relax
    \includegraphics[width=\textwidth]{chainSample}
    
    \vspace{-2em}
    \inputminted[firstline= 9, lastline=10, fontsize=\tt\tiny]{latex}{sec01/code/chainSample.tex}
    \inputminted[firstline= 16, lastline=32, fontsize=\tt\tiny]{latex}{sec01/code/chainSample.tex}

     \skfootnote{\tikzc{I.5}[69]}
\end{frame}
}

\begin{frame}{Tree}

\twocolImg{
\inputminted[firstline=9, lastline=9]{latex}{sec01/code/treeSample.tex}
\inputminted[firstline= 16, lastline=25]{latex}{sec01/code/treeSample.tex}
}{treeSample}

We use \ccol\node\ and \ccol{child}.

\ccol{sibling distance} option provides a horizontal distance between nodes
     
\end{frame}

\inclassframe{
\begin{frame}{\exFrame{reproduce the following with tikZ}}{lvl 1. \textit{You can use absolute coodrinates}}\relax



     \includegraphics[width=0.4\textwidth]{exLvl1}
\end{frame}

\begin{frame}{\exFrame{reproduce the following with tikZ}}{lvl 2. \textit{You can't use absolute coodrinates}}\relax

     
     \includegraphics[width=0.4\textwidth]{exLvl2}
\end{frame}

}




\subsection{Arrangment}
\outclassframe{
\begin{frame}{\ }\relax
    Very few people make full-fledged charts in tikZ. But it is much more common to add tikZ as text decoration element.
\end{frame}
}


\begin{frame}[fragile]{Introduction}\relax

TikZ is often used not as ``independent picture'', but as a part of the presentation or document. 

\end{frame}

\begin{frame}[fragile]{Example: CV\magicPage}\relax
    \begin{columns}
        \begin{column}{0.45\textwidth}
              \only<1>{
              \expandafter{\inputminted[firstline= 16, lastline=20]{latex}{sec01/code/cv_examples.tex}
                }
                }
                \only<2>{\expandafter{\inputminted[firstline= 22, lastline=32]{latex}{sec01/code/cv_examples.tex}}}
        \end{column}
        \begin{column}{0.45\textwidth}
             \includegraphics[width=\textwidth, keepaspectratio]{cv_examples}
        \end{column}
    \end{columns}

    % \cprotect\twocolImg{
    % \only<1>{\inputminted[firstline= 16, lastline=20]{latex}{sec01/code/cv_examples.tex}}
    % \only<2>{\inputminted[firstline= 22, lastline=32]{latex}{sec01/code/cv_examples.tex}}
    % }{cv_examples}  
    
\end{frame}

\begin{frame}[fragile]{``Magic''}\relax

    How to produce ``magic'' \only<1>{\tikz[overlay, remember picture] \draw[->] (0, 0.1) to[out=0,in=180] (endM);}

\begin{tikzpicture}[remember picture,overlay,shift={(current page.north east)}]
    \node[anchor=north east,xshift=-0cm,yshift=-0cm](endM) {%
        {\includegraphics[width=1cm]{black_magic}}%
};
\end{tikzpicture}%

\inpause
\begin{minted}{latex}
\begin{tikzpicture}[remember picture,overlay,shift={(current page.north east)}]
\node[anchor=north east,xshift=-0cm,yshift=-0cm](endM) {%
    {\includegraphics[width=1cm]{images/magic}}%
};
\end{tikzpicture}
\end{minted}

\inpause
Notice \ccol{shift=\{(current page.north east)\}}
\end{frame}



\begin{frame}[fragile]{arrow to magic}\relax

How to produce this \tikz[overlay, remember picture] \node (arrstart) {}; arrow \tikz[overlay, remember picture] \node (startMMM) {};

\begin{tikzpicture}[remember picture,overlay,shift={(current page.north east)}]
    \node[anchor=north east,xshift=-0cm,yshift=-0cm](endMMM) {%
        {\includegraphics[width=1cm]{black_magic}}%
};
\end{tikzpicture}%

\tikz[overlay, remember picture] \draw[->, ultra thick] ($(startMMM)+(0,0.1)$) to[out=0,in=180] (endMMM);
\only<1>{\begin{tikzpicture}[overlay, remember picture]
     \path[->, ultra thick, name path=point c] ($(startMMM)+(0,0.1)$) to[out=0,in=180] (endMMM);
     
     \path[->, name path=point a] ($(startMMM)+(5,-1)$) -- +(-1, 3);
     \path[name intersections={of=point c and point a, by={intersection}}];
     \draw[thick, thick, ->] ($(arrstart)-(0.4ex,-0.6ex)$) to[out=45, in=110] (intersection);
\end{tikzpicture}}

\inpause

\begin{minted}{latex}
% remember position of (startM) node (and (endM) node from previous slide)
How to produce this arrow \tikz[overlay, remember picture] \node (startM) {};

% go from (startM) to (endM)
\tikz[overlay, remember picture] \draw[->, ultra thick] ($(startM)+(0,0.1)$) to[out=0,in=180] (endM);

\end{minted}

\skfootnote{What about arrow pointed on the thick arrow? It was produced with intersections library}
     
\end{frame}


\begin{frame}[fragile]{``Common belief''\magicPage}\relax
     \begin{center}
        \begin{tikzpicture}
             \node[align=center] (0,0) {
             \huge \LaTeX\ is only for use\\ \huge  in academic area
             };
             \uncover<2,3>{\node[rotate=30, bottom color=red!50, top color=red!50] (0,0) {\Huge WRONG};}
        \end{tikzpicture}
         
    \end{center}
    
\pause\pause was produce by 
\begin{minted}{latex}
\begin{tikzpicture}
    \node[align=center] (0,0) {
    \huge \LaTeX\ is only for use\\ \huge  in academic area
    };
    \uncover<2,3>{\node[rotate=30, bottom color=red!50, top color=red!50] (0,0) {\Huge WRONG}};
\end{tikzpicture}
\end{minted}
    
\end{frame}


\cprotect\inclassframe{
\begin{frame}[fragile]{\exFrame{Try to use other people's work}}{lvl 1}\relax

    \begin{enumerate}
        \item You have the code as below. Copy it to you document and make sure it is compiling. ``Play'' with the params and of \ccol\pictofraction\ and as the result give the enumerate list in pdf of what all params means. Ex: ``1. printed symbol, 2...''
         \item Go to \url{https://tikz.dev/library-mindmaps} and try to create your own mindmap
    \end{enumerate}

    \inputminted[firstline=7, lastline=17]{latex}{sec01/code/exExtLvl1.tex}
    \inputminted[firstline=21, lastline=22]{latex}{sec01/code/exExtLvl1.tex}
%7-17, 21-22
    %  \includegraphics[width=0.4\textwidth]{exLvl1}
\end{frame}

\begin{frame}[fragile, allowframebreaks]{\exFrame{Try to use other people's work}}{lvl 2}\relax
You copy a code, that produce a ``day cyrcle'' (from AltaCV). But some libraries and packages are missing. See logs and google errors, make this code works :)

\includegraphics[width=0.8\textwidth]{exExtLvl2full}

     \inputminted[fontsize=\tiny]{latex}{sec01/code/exExtLvl2notfull.tex}
    %  \includegraphics[width=0.4\textwidth]{exLvl2}
\end{frame}

}









\progressend

\begin{frame}\frametitle{What we have learned today?}\relax


\tableofcontents
\end{frame}

\begin{frame}[allowframebreaks]{references}
color from the footnotes corresponds to references' color.
    \begin{itemize}
        \item \knuthc{Knuth ``The \TeX Book''}
        \item \lvoc{L'vovsky ``Nabor i verstka v sisteme \LaTeX''}
        \item \lamc{Lamport. ``\LaTeX. A Document Preparation System, User’s Guide and Reference Manual''}
        \item \lmanc{``\LaTeX 2e: An unofficial reference manual''} also at website \url{https://latexref.xyz/}
        \item \stExC{https://tex.stackexchange.com/questions} : \url{https://tex.stackexchange.com/questions}
        \item \wikiC{https://en.wikibooks.org/wiki/LaTeX} : \url{https://en.wikibooks.org/wiki/LaTeX}
        \item \overC{https://www.overleaf.com/learn/latex} : \url{https://www.overleaf.com/learn/latex}
        \item \tugC{https://www.tug.org/utilities/plain/cseq.html} : \url{https://www.tug.org/utilities/plain/cseq.html}
        \item \url{http://hostmath.com/} -- WYSiWYG math editor
        \item \url{http://detexify.kirelabs.org/classify.html} -- find symbols
    \end{itemize}
\end{frame}

\begin{frame}{Distribution}\relax
\begin{itemize}
     \item the pdf-version of the presentation and all printed materials can be distributed under license Creative Commons Attribution-ShareAlike 4.0 \url{https://creativecommons.org/licenses/by-sa/4.0/}
     \item The source code of the presentation is available on {\csk\url{https://github.com/Lavton/latexLectures}} and can be distributed under the MIT license \url{https://en.wikipedia.org/wiki/MIT_License\#License_terms}
\end{itemize}
     
\end{frame}
\end{document}