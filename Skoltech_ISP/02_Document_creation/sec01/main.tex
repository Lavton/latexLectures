%% It is just an empty TeX file.
%% Write your code here.
\graphicspath{{sec01/images/}{sec01/code/}}
\lstset{inputpath=sec01/code/}


\begin{frame}{How to structure \& refer the document}\relax
    \begin{center}
        \includegraphics{s2/refBegin2}
    \end{center}
    
\end{frame}

\begin{frame}{How to structure \& refer the document}\relax
    \begin{enumerate}
         $\left.
         \begin{tabular}{p{.5\textwidth}} \vspace{-2ex}
         \item add structure element. \TeX\ will automatically calculate it's serial number incrementing the previous one. 
         \item refer to the element added before or after label. Refer to specific page, to specific equation, to specific biblio record or item
         \end{tabular}
         \right\}\text{\TeX\ use ``counters'' implicitly}$
     \end{enumerate}     
\end{frame}

% общая идея: счётчики, подчинение счётчиков, двойной прогон
\begin{frame}{how all this works}\relax
% отдельно счётчики
% отдельно labels
% и слайд что это отдельно 
\begin{enumerate}
    \item \TeX\ has a counter for... lots of stuff 
    \item When you add an element (section, equation, etc), the command updates its counter and prints it near the element\inpause
    \item To reference an element, you need to explicitly tell \TeX\ about it.
    \item When \TeX\ ``sees'' this guideline, it saves the related counter to {\csk external} file
    \item When \TeX\ runs {\csk second time} and finds the place where you need to insert the reference, it looks at the file and gets counter value from the file \inpause
    \item Sometimes (e.g.: bibliography) you run external command (e.g.: bibtex) to create the mentioned file \inpause
    \item By the way: \TeX\ always ``reads'' document from top to bottom. \outclasshigh{The only way some command to affect anything \textit{before} the command appears is throw external file. The set of commands that do this is quite small}
\end{enumerate}

\inpause 
(We will look more precisely at the last lecture.)
     
\end{frame}

\subsection{Structural elements}
% \knuthc  knuth the TeXBook
% \lvoc   Lvovsky
% \lamc  lamport latex 
% \slshape different font for footnote
\graphicspath{{sec01/images/s1z/}{sec01/code/s1/}}
\lstset{inputpath=sec01/code/s1/}

\begin{frame}{What document consists of?}\relax
\begin{itemize}
    \item Title
    \item Authors
    \item Table of contents
    \item Table of figures
    \item Table of tables
    \item Sections, subsections,..
    % \item Bibliography
\end{itemize}
     
\end{frame}

%%%%%%%%%%%%%%%%%%%% Title %%%%%%%%%%%%%%%%%%%%%%
\begin{frame}[fragile]{Title}{title}\relax
\cprotect\twocolImg{
% \lstinputlisting[linerange={6-9}]{title01.tex}
\inputminted[firstline=6, lastline=9]{latex}{sec01/code/s1/title01.tex}
}{title01}
\inpause
\begin{itemize}
     \item \ccol{\title} before begin of the document
     \item \ccol{\maketitle} after begin of the document 
\end{itemize}
% \inclassFrag{Try it in your own paper!}[-1]
\skfootnote{\wikiC{https://en.wikibooks.org/wiki/LaTeX/Title_Creation} \lmanc{18.1}[163] \lmanc{8.26}[96]}
\end{frame}

\begin{frame}[fragile]{Title}{date}
 \cprotect\twocolImg{
% \lstinputlisting[linerange={6-10}]{title02.tex}
\inputminted[firstline=6, lastline=10]{latex}{sec01/code/s1/title02.tex}
}{title02}
\inpause
\begin{itemize}
     \item by defaut \LaTeX\ think you use {\csk \verb|\date{\today}|}
     \inclassFrag{what do you think command \ccol{\today} if for?}[3]
     \begin{itemize}
         \item \ccol{\today} is the date of last document compilation
     \end{itemize} 
     \inpause
     \item you can put anything inside \ccol{\date}\{\} command
     \item use \ccol{\date}\{\} without arguments to remove the string
\end{itemize}
\inpause
\inclasshigh{notice \ccol\the\ before year, month and day. We will return to it at the last lecture}
\end{frame}

\begin{frame}[fragile]{Title}{authors}
 \cprotect\twocolImg{
% \lstinputlisting[linerange={6-11}]{title03.tex}
\inputminted[firstline=6, lastline=11]{latex}{sec01/code/s1/title03.tex}
}{title03}

\inpause
\begin{itemize}
    \item \ccol\author\ for put the author
    \item \ccol\and\ (can be) used to concatinate several authors
    \begin{itemize}
        \item You always can use just plain text
    \end{itemize}
    \item \ccol\thanks\ for a footnote 
\end{itemize}
\end{frame}

\begin{frame}[fragile]{Abstract}\relax
\cprotect\twocolImg{
    \inputminted[firstline=8, lastline=10]{latex}{sec01/code/s1/abstractmy.tex}
    % \lstinputlisting[linerange={8-10}]{abstractmy.tex}
}{abstractmy}

% \inpause\inclasshigh{...Nothing happends, yeah?))}

% \outclasshigh{The style of abstract may change in specific journal, but by default there is no addition style}

\skfootnote{\lmanc{8.1}[54] \lvoc{IV.5.4}[170]}
\end{frame}

\begin{frame}[fragile]{Structure}\relax
\cprotect\twocolImg{
    \inputminted[firstline=7, lastline=15]{latex}{sec01/code/s1/struc.tex}
    % \lstinputlisting[linerange={7-15}]{struc.tex}
}{struc}

    \cprotect\skfootnote{\wikiC{https://en.wikibooks.org/wiki/LaTeX/Document_Structure\#Section_numbering}, \overC{https://www.overleaf.com/learn/latex/Sections_and_chapters}, \lmanc{6}[40], \lvoc{IV.5}[165]\\ 
    You can use \verb|\section[short name]{long name}| to put \verb|short name| to table of contents and \verb|\renewcommand{\chaptername}{new name}| (\lvoc{IV.5.3}[169]) to change the standard name
    }
\end{frame}


\begin{frame}[fragile]{Structure}{Tips}\relax

% \inclassFrag{Try \ccol\section*}[1]

\begin{itemize}
    \item Use {\csk \textbackslash<command>*} (with {\Large *}) to ommit the numbering
    \item The structure (and titles) is not pre-build into \LaTeX: they are defined inside class files $\Rightarrow$ not all classes contain all commands
     
\end{itemize}
\end{frame}

\begin{frame}[fragile]{Table of content}\relax
\cprotect\twocolImg{
    \inputminted[firstline=8, lastline=16]{latex}{sec01/code/s1/toc.tex}
    % \lstinputlisting[linerange={8-16}]{toc.tex}
}{toc}
     \inpause \ccol{\tableofcontents} for create it, \ccol{\newpage} for new page.
     
     Notice that not all structure elements are mentioned it ToC!

     \cprotect\skfootnote{\lmanc{25.1}[212]\\
     \verb|\contentsname| --- the name of the ToC; \verb|secnumdepth| counter (\lvoc{IX.3.1}[298], \lmanc{6}[40]) to change what will be included in ToC
     }
\end{frame}



\begin{frame}[fragile]{Table of...}\relax
\Large
     \begin{itemize}
          \item \ccol{\listoffigures} for figures 
          \item \ccol{\listoftables} for tables 
     \end{itemize}
     
     \cprotect\skfootnote{\lvoc{IV.8.1}[188], \lmanc{25.1}[202]\\ 
     You can use \verb|\caption[short name]{long name}| to put \verb|short name| to the lists, \verb|\listfigurename| and \verb|\listtablename| --- the names of the Lists.
     }
\end{frame}

%%%%%%%%%%%%%% IN CLASS EXERSISES
% different classes
\cprotect\inclassframe{
\begin{frame}[fragile]{\exFrame{Try to use the commands for different class files}}\relax
    \inputminted{latex}{sec01/code/s1/strucTask.tex}
    %  \lstinputlisting[linerange={1-1,7-15},basicstyle=\tt\small]{strucTask.tex}
    %  
     for {\csk book, report, article}
     
\end{frame}
}

\inclassframe{
\begin{frame}{\exFrame{Try to reproduce the following}}{lvl 1}\relax
    \begin{columns}
        \begin{column}{0.5\textwidth}
            \fbox{\includegraphics[height=0.7\textheight, page=1]{exStrutLvl1}}
        \end{column}
        \begin{column}{0.5\textwidth}
            \fbox{\includegraphics[height=0.7\textheight, page=2]{exStrutLvl1}}
        \end{column}
    \end{columns}
    
\end{frame}

\begin{frame}{\exFrame{Try to reproduce the following}}{lvl 2. The 2025 year must be ``current year'', not just ``2025''! \textit{You may need to google some stuff}}\relax
    
    \begin{columns}
        \begin{column}{0.5\textwidth}
            \fbox{\includegraphics[height=0.7\textheight, page=1]{exStrutLvl2}}
        \end{column}
        \begin{column}{0.5\textwidth}
            \fbox{\includegraphics[height=0.7\textheight, page=2]{exStrutLvl2}}
        \end{column}
    \end{columns}
\end{frame}
}


\subsection{References}
% \knuthc  knuth the TeXBook
% \lvoc   Lvovsky
% \lamc  lamport latex 
% \slshape different font for footnote
\graphicspath{{sec01/images/s2/}{sec01/code/s2/}}
\lstset{inputpath=sec01/code/s2/}

\begin{frame}[fragile]{How good it will be if...\only<2,3>{we could write like this}}\relax
\begin{center}
\only<1>{\includegraphics{refBegin}}
\only<2>{\includegraphics{refBegin2}}
\only<3>{\Huge We can!}
\end{center}

\cprotect\skfootnote{I use \verb|\only<2>{text}| for such effect. Also \verb|\setcounter| was used in source.}
\end{frame}

\begin{frame}[fragile]{Step 1: \ccol\label}\relax
    %  \lstinputlisting[linerange={12-15}]{refBegin2.tex}
     \inputminted[firstline=12, lastline=15, fontsize=\tt]{latex}{sec01/code/s2/refBegin2.tex}
     
     \skfootnote{for whole reference mechanism: \lmanc{7}[51] \lvoc{I.2.11}[27] \overC{https://www.overleaf.com/learn/latex/Cross_referencing_sections_and_equations}, \wikiC{https://en.wikibooks.org/wiki/LaTeX/Labels_and_Cross-referencing}}
\end{frame}

\begin{frame}[fragile]{Step 2: \ccol{\ref} and \ccol{\pageref}}
%, basicstyle=\tt
    %  \lstinputlisting[linerange={17-17}]{refBegin2.tex}
     \inputminted[firstline=17, lastline=17, fontsize=\tt]{latex}{sec01/code/s2/refBegin2.tex}
\end{frame}

\begin{frame}[fragile]{Combined}\relax
     \inputminted[firstline=12, lastline=17, fontsize=\tt]{latex}{sec01/code/s2/refBegin2.tex}
     
\inpause Notice \texttt{prefix\ccol{:}id} notation (\texttt{sec:nice}). It is rather common

\end{frame}

\begin{frame}[fragile]{Problem 1: lots of labels!}
    What if you have too many marks throughout the document?
    \inclasshigh{\inpause Will you have to open the .tex on the left, the .pdf on the right and compare them line by line? }
    \inpause
    
    Use package \ccol{showlabels}
\cprotect\twocolImg{
% , basicstyle=\tt\smal
    % \lstinputlisting[linerange={10-10, 13-16}]{refBeginShow.tex}
    \inputminted[firstline=10, lastline=10]{latex}{sec01/code/s2/refBeginShow.tex}
    \inputminted[firstline= 13, lastline=16]{latex}{sec01/code/s2/refBeginShow.tex}
}{refBeginShow}    
    
    
    \skfootnote{\url{http://ctan.altspu.ru/macros/latex/contrib/showlabels/showlabels.pdf}}
\end{frame}

\begin{frame}[fragile]{Problem 2: Typos}\relax
\cprotect\twocolImg{
% , basicstyle=\tt\small
    % \lstinputlisting[linerange={12-17}]{refBad.tex}
    \inputminted[firstline=12, lastline=17]{latex}{sec01/code/s2/refBad.tex}
}{refBad}

\inpause
Look at \textbf{?} in the document or inside the logs
\end{frame}

\begin{frame}[fragile]{Counter domination}\relax

look at the equation numbering style
\cprotect\twocolImg{
    \inputminted[firstline=11, lastline=19]{latex}{sec01/code/s2/refDomination.tex}
}{refDomination}

\end{frame}

\begin{frame}[fragile]{Bibliography}{How to cite}\relax
Use \ccol{\cite\{label\}}

\cprotect\twocolImg{
% , basicstyle=\tt\small
    % \lstinputlisting[linerange={9-9}]{citeFirst.tex}
    \inputminted[firstline=9, lastline=9, fontsize=\tt]{latex}{sec01/code/s2/citeFirst.tex}
}{citeFirst}  

\inpause
\cprotect\twocolImg{
% , basicstyle=\tt\small
    % \lstinputlisting[linerange={11-12}]{citeSecond.tex}
    \inputminted[firstline=11, lastline=12, fontsize=\tt]{latex}{sec01/code/s2/citeSecond.tex}
}{citeSecond}  

\skfootnote{\lmanc{8.24.2}[94] \wikiC{https://en.wikibooks.org/wiki/LaTeX/Bibliography_Management} \overC{https://www.overleaf.com/learn/latex/Bibliography_management_in_LaTeX}}     
\end{frame}

\begin{frame}[fragile]{Bibliography}{What to cite}\relax

{\csk .bib} files

\begin{verbatim}
@Book{landau,
    author = {Landau, L. D. and Lifshitz, E. M.},
    title = {The Classical Theory of Fields},
    journal = N,
    volume = {1},
    pages = {140},
    year = 1980
}
\end{verbatim}

You can have multiple records in one .bib file.
     
\end{frame}

\begin{frame}[fragile]{Offline - compile twice!}\relax
Running \LaTeX\ offline, you can get \textbf{(??)} in \ccol{\ref} and \textbf{[?]} in \ccol{\cite}. 

For 
\begin{itemize}
    \item References 
    \item Bibliography
    \item Table of content
    \item Indexing
    \item ...
\end{itemize}

\LaTeX\ collect addition data in extra files. \LaTeX\ need more then one run to get this data. 

Use \verb|latex; bibtex; latex; latex|

\inclasshigh{we will talk about the mechanism in the last lecture}

\end{frame}

\begin{frame}[fragile]{Bibliography. Where can you get .bib files?}\relax
\begin{itemize}
    \item Just google it! ``article\_name bibtex''
    \item at \url{scholar.google.ru} ask Cite -> BibTeX
    \item Go to your favorite journal and look at Citations -> ``.bib'' or ``bibtex''
    \item Ask Mendeley, Zotero or other programs to give you the .bib file
    \item Create it by yourself
\end{itemize}
\end{frame}

\begin{frame}[fragile]{Bibliography. Creating .bib file \magicPage}\relax
\begin{center}
\begin{tabular}{rl}
     \ccol{@article} & Journal or magazine article\\
\ccol{@book} & Book\\
\ccol{@conference} & Article in conference proceedings\\
\ccol{@misc} & If nothing else fits.
\end{tabular}
\end{center}

Than fill in 

{\obeylines 
\ author
\ title
\ journal
\ year
\ pages
\ volume}

following the example of other entries

\skfootnote{\wikiC{https://en.wikibooks.org/wiki/LaTeX/Bibliography_Management\#Standard_templates} for cite url checkout \stExC{https://tex.stackexchange.com/questions/3587/how-can-i-use-bibtex-to-cite-a-web-page}\\ check for full list \wikiC{https://en.wikibooks.org/wiki/LaTeX/Bibliography_Management\#BibTeX}
}     
\end{frame}

% \newcommand{\twocolImgS}[2]{  % add two columns: (1) is some text (2) is an image.
%     \begin{columns}
%         \begin{column}{0.45\textwidth}
%             #1
%         \end{column}
%         \begin{column}{0.45\textwidth}
%         \fbox{\includegraphics[width=\textwidth, keepaspectratio,page=1]{#2}}
        
%         \fbox{\includegraphics[width=\textwidth, keepaspectratio,page=2]{#2}}
%         % ниже мы делаем позиционирование, поднимая изображение на половину его высоты. Таким образом, центрированные слайды останутся центрированными и с т.з. изображения
%             % https://tex.stackexchange.com/questions/3166/how-can-the-dimensions-of-a-box-be-retrieved-with-latex
%             % \savebox{\Abox}{\includegraphics[width=\textwidth, keepaspectratio,page=1]{#2}\\\hrule\includegraphics[width=\textwidth, keepaspectratio,page=2]{#2}}
%             % \newlength{\myhh}
%             % \settoheight{\myhh}{\usebox{\Abox}}
%             % \raisebox{-0.5\myhh+1ex}[0pt][0pt]{\includegraphics[width=\textwidth, keepaspectratio,page=1]{#2}\\\hrule\includegraphics[width=\textwidth, keepaspectratio,page=2]{#2}}
%         \end{column}
%     \end{columns}
% }

% \begin{frame}[fragile]{Bibliography. Manually\magicPage}

% You can add Bibliography manually.

% \cprotect\twocolImg{
%     \inputminted[firstline=10, lastline=22]{latex}{sec01/code/s2/citeMan.tex}
%     % \lstinputlisting[linerange={10-22}]{citeMan.tex}
% }{citeMan}  

% \skfootnote{\lmanc{8.24}[92]}
% \end{frame}

% \begin{frame}[fragile]{Bibliography. Styles\magicPage}

% You can change styles.

% Mannually -- check \url{https://en.wikibooks.org/wiki/LaTeX/Bibliography_Management}

% Or with packages -- check \url{https://tex.stackexchange.com/questions/25701/bibtex-vs-biber-and-biblatex-vs-natbib}
% \end{frame}

% \begin{frame}[fragile]{Nomenclature\magicPage}\relax

%     \cprotect\twocolImg{
%         % \inputminted[firstline=9, lastline=16]{latex}{sec01/code/s2/indexmy.tex}
%         \inputminted[firstline=8, lastline=14]{latex}{sec01/code/s2/nomenclature.tex}
%     }{nomenclature}  

%     \skfootnote{\overC{https://www.overleaf.com/learn/latex/Nomenclatures}}     
% \end{frame}

% \begin{frame}[fragile]{Subject index\magicPage}\relax % предметный указатель

% \cprotect\twocolImgS{
%     \inputminted[firstline=9, lastline=16]{latex}{sec01/code/s2/indexmy.tex}
% }{indexmy}


% \skfootnote{\lvoc{IV.7}[175] \lmanc{25.2}[215]}     
% \end{frame}



\subsection{Useful commands}
% \knuthc  knuth the TeXBook
% \lvoc   Lvovsky
% \lamc  lamport latex 
% \slshape different font for footnote
\graphicspath{{sec01/images/s3/}{sec01/code/s3/}}
\lstset{inputpath=sec01/code/s3/}

\begin{frame}[fragile]{\ccol{\footnote}\magicPage}\relax
\cprotect\twocolImg{
    % \lstinputlisting[linerange={9-9}]{footnoteMy.tex}
    \inputminted[firstline=9, lastline=9]{latex}{sec01/code/s3/footnoteMy.tex}
}{footnoteMy}  


\skfootnote{\lmanc{8}[108] \wikiC{https://en.wikibooks.org/wiki/LaTeX/Footnotes_and_Margin_Notes} \overC{https://www.overleaf.com/learn/latex/Footnotes} \knuthc{15}[120]}
\end{frame}

\begin{frame}[fragile]{Horizontal aligment\magicPage}\relax
\cprotect\twocolImg{
    % \lstinputlisting[linerange={9-25}]{flushing.tex}
    \inputminted[firstline=9, lastline=25]{latex}{sec01/code/s3/flushing.tex}
}{flushing}  


\skfootnote{\lmanc{8.12}[63] \lmanc{8.13}[64] \knuthc{14}[112] and note \stExC{https://tex.stackexchange.com/questions/64644/centering-doesnt-seem-to-center-my-text}}
\end{frame}

\begin{frame}[fragile]{Page break\magicPage}\relax

{\centering \Large \ccol{\newpage} \ccol{\pagebreak}\par}

\skfootnote{\lmanc{10}[105] \stExC{https://tex.stackexchange.com/questions/736/pagebreak-vs-newpage}}
     
\end{frame}

\begin{frame}[fragile]{Quotes\magicPage}\relax

\cprotect\twocolImg{
    % \lstinputlisting[linerange={10-14}]{quotemy.tex}
    \inputminted[firstline=10, lastline=14]{latex}{sec01/code/s3/quotemy.tex}
}{quotemy}  


\cprotect\skfootnote{\lvoc{III.7.1}[129] \lmanc{8.20}[83]. Also see \verb|quotation| env }
     
\end{frame}

\begin{frame}[fragile]{Verses\magicPage}\relax

\cprotect\twocolImg{
    % \lstinputlisting[linerange={11-18}]{versemy.tex}
    \inputminted[firstline=11, lastline=18]{latex}{sec01/code/s3/versemy.tex}
}{versemy}  

\skfootnote{\lvoc{III.7.3}[130] \lmanc{8.28}[98]}
\end{frame}

\begin{frame}[fragile]{Marginal notes\magicPage}\relax

\cprotect\twocolImg{
    % \lstinputlisting[linerange={9-10}]{marginmy.tex}
    \inputminted[firstline=9, lastline=10]{latex}{sec01/code/s3/marginmy.tex}
}{marginmy}  

\skfootnote{\lvoc{IV.10}[194] \lmanc{15.4}[137]}
\end{frame}

 
 %%%%%%%%%% in class tasks %%%%%%%%%%%%%%%%%
 
 \cprotect\inclassframe{
 \begin{frame}{\exFrame{Create a document}}{lvl 1}\relax
 ...with following params
 
 \begin{itemize}
     \item it must be an article
     \item it must contains at least two pages 
     \item it must have an equation 
     \item it must refer to the equation number
      
 \end{itemize}
      
 \end{frame}
 
 \begin{frame}{\exFrame{Create a document}}{lvl 2}\relax
 ...with following params
 
 \begin{itemize}
     \item it must be an article
     \item it must contains at least two pages 
     \item it must have an equation 
     \item it must refer to the equation number
     \item it must contains a theorem (google for it)
     \item it must have at least two sections and subsections
     \item the equation numbering must be (section.subsection.equation) 
      
 \end{itemize}
      
 \end{frame}
 
 }
% \cprotect\inclassFrame{
% \begin{frame}[fragile]{Try to add a package}\relax
% \url{http://hanno-rein.de/downloads/coffee.sty}

% and use \verb|\newpage \coffee{2} hello world!|

% \inpause
% NOW you document is ready!

% \end{frame}
% }
