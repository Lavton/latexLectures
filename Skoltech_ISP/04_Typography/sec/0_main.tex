\graphicspath{{sec/images/}{sec/code/}}
\lstset{inputpath=sec/code/}

\section{\TeX\ primitives}

\begin{frame}{Entities}

    \begin{enumerate}
        \item Primitive commands
        \item Counters (=integer numbers)
        \item Lengths
        \item Boxes
        \item Glues
        \item Spaces 
        \item Toks (Strings)
        \item Inserts 
    \end{enumerate}
    \inpause
    commands and macros will wait in details the next lecture. No we discuss it just in a few words
\end{frame}

\subsection{Commands (Macros)}


\begin{frame}[fragile]{Simple command creation \lW}\relax
    \twocolImg{
    \inputminted[firstline=8, lastline=8]{latex}{sec/code/commandSimple.tex}
    \inputminted[firstline=13, lastline=20]{latex}{sec/code/commandSimple.tex}
    }{commandSimple}
    
    \inpause
    \inclasshigh{You are a developer now!}
\end{frame}

\subsection{Counters}
\graphicspath{{sec/images/}{sec/code/counters/}}
\lstset{inputpath=sec/code/counters}

\begin{frame}{What is ``counter''}\relax

    ``Counter'' is just an integer number.
    
    It's using in multiple places to count everything in \LaTeX: sections, equations, references, citation, enumerate lists,...
     
\end{frame}

\begin{frame}{Define and simple manipulation with counters \lW}\relax
    \twocolImg{
    % \lstinputlisting[linerange={11-18}]{countersimp.tex}
    \inputminted[firstline=11, lastline=18]{latex}{sec/code/counters/countersimp.tex}
    }{countersimp}
    
    \begin{itemize}
        \item \ccol\newcounter\ to define new counter 
        \item \ccol\setcounter\ to set counter to new value
        \item \ccol\addtocounter\ to add a number to the counter
    \end{itemize}
    
    \skfootnote{\lvoc{VII.3.1}[244] \lmanc{12.5}[117] \lmanc{13.4-5}[128] \wikiC{https://en.wikibooks.org/wiki/LaTeX/Counters\#Counter_manipulation}}
     
\end{frame}

\newcounter{tmptt}
\begin{frame}[fragile]{Print counter\lW}\relax
     
     \newcommand{\countab}[1]{
     \ccol#1\{countname\}&
     \setcounter{tmptt}{1} #1{tmptt} &
     \setcounter{tmptt}{2} #1{tmptt} &
     \setcounter{tmptt}{3} #1{tmptt} &
     \setcounter{tmptt}{4} #1{tmptt} &
     \setcounter{tmptt}{5} #1{tmptt} &
     \setcounter{tmptt}{6} #1{tmptt} &
     \setcounter{tmptt}{7} #1{tmptt} &
     \setcounter{tmptt}{8} #1{tmptt} &
     \setcounter{tmptt}{9} #1{tmptt}
     \\}
     
     \begin{tabular}{l|ccccccccc}
          \countab{\arabic}
          \countab{\alph}
          \countab{\Alph}
          \countab{\roman}
          \countab{\Roman}
          \countab{\fnsymbol}
     \end{tabular}
     \vspace{1ex}
     
     P.S. \ccol\value\ to get ``raw'' value of the counter
     \skfootnote{ \wikiC{https://en.wikibooks.org/wiki/LaTeX/Counters\#Counter_style} \lmanc{13.1}[126]\\ 
     see \lvoc{IX.2.3}[295] for russian analog of \ccol\alph}
\end{frame}

\begin{frame}[fragile]{pre-defined counters in standart classes}\relax
    \newlength{\myboxlen}%
    \setlength{\myboxlen}{5em}
    
    \newcommand{\showC}[1]{%
    \makebox[\myboxlen]{#1\hfill}%
    }
    
    \showC{part}\hfill
    \showC{chapter}\hfill
    \showC{section}\hfill
    \showC{subsection}\hfill
    \showC{subsubsection}\hfill
    \showC{paragraph}\hfill
    \showC{subparagraph}\hfill
    \showC{}\hfill
    \showC{}\hfill
    \showC{}\hfill
    \showC{}\hfill
    \showC{}\hfill
    
    \showC{page}\hfill
    \showC{figure}\hfill
    \showC{table}\hfill
    \showC{footnote}\hfill
    \showC{equation}\hfill
    
    \showC{enumi}\hfill
    \showC{enumii}\hfill
    \showC{enumiii}\hfill
    \showC{enumiv}\hfill
    \showC{}\hfill
    \showC{}\hfill
    \showC{}\hfill
    \showC{}\hfill
    \showC{}\hfill
    
    \TeX's counters (will talk later)
    \showC{\ccol\year}\hfill
    \showC{\ccol\month}\hfill
    \showC{\ccol\day}\hfill
    \showC{\ccol\time}\hfill
    \showC{}\hfill
    \showC{}\hfill
    \showC{}\hfill
    \showC{}\hfill
    \showC{}\hfill
    
    
    
    \skfootnote{\wikiC{https://en.wikibooks.org/wiki/LaTeX/Counters\#LaTeX_default_counters} \lmanc{13}[126]}
\end{frame}

\begin{frame}{Counter Domination\inclasshigh{\textit{(, conquering, humiliation)}}}{problem}\relax
    \begin{columns}
    \begin{column}{0.45\textwidth}
         You may want to write something like
         
         \vspace{-5ex}
         \includegraphics[width=\linewidth]{counterdom}
    \end{column}
    \inpause
    \begin{column}{0.45\textwidth}
    But the straightforward solution will give you 
    
    \vspace{-5ex}
    \includegraphics[width=\linewidth]{counterdomno}
         
    \end{column}
         
    \end{columns}
\end{frame}

\begin{frame}{Counter Domination\inclasshigh{\textit{(, conquering, humiliation)}}}{straightforward solution}\relax

    \twocolImg{
    % \lstinputlisting[linerange={11-20}]{counterdomno.tex}
    \inputminted[firstline=8, lastline=19]{latex}{sec/code/counters/counterdomno.tex}
    }{counterdomno}
    
    \inclassFrag{Try to compile it by your own}[0]
\end{frame}

\begin{frame}[fragile]{Counter Domination\inclasshigh{\textit{(, conquering, humiliation)}}}{The Way}\relax

    \newcommand{\modif}[3]{\fbox{\parbox{\textwidth}{\makebox[\textwidth]{\small\makebox[0.47\textwidth]{#1}\hfill$\to$\hfill\makebox[0.47\textwidth]{#2}}\\ #3}}}
    
    \centering
    \modif{\string\newcounter\{task\}}{\ocol\newcounter\{task\}[section]}{\ocol\newcounter\{<slave>\}[<master>] will resets the value of <slave> if the value of <master> is change}
    
    \inclassFrag{Please, modify your code after this and every step in this frame}
    % \inpause
    
    \modif{\ccol\addtocounter\{task\}\{1\}}{\ccol\refstepcounter\{task\}}{
    \ccol\refstepcounter\{<counter>\} use it to update \ccol\label--\ccol\ref\ mechanism
    }
    
    \inpause
    \modif{\string\textbf\{Task \string\#\string\arabic\{task\}.}{\tiny\string\textbf\{Task \string\#\string\arabic\{section\}.\string\arabic\{task\}.}{Inside \string\newcommand\{\string\tsk\} to redefine the labels}
    
    \inpause
    \modif{}{\tiny\string\renewcommand\{\ccol\thetask\}\{\string\arabic\{section\}.\string\arabic\{task\}\}}{\string\renewcommand\{\ccol{\the<counter>}\} to redefine the reference}
    
    \skfootnote{\lvoc{VII.3.3}[250] \lmanc{13.6}[128]}
\end{frame}

\begin{frame}{Counter Domination\inclasshigh{\textit{(, conquering, humiliation)}}}{solution}\relax
    \begin{columns}[t]
        \begin{column}{0.5\textwidth}
             \inputminted[firstline=8, lastline=19]{latex}{sec/code/counters/counterdomno.tex}
        \end{column}
        \begin{column}{0.6\textwidth}
             \inputminted[firstline=8, lastline=19]{latex}{sec/code/counters/counterdom.tex}
        \end{column}
         
    \end{columns}
     
\end{frame}

\begin{frame}{Counter Domination\inclasshigh{\textit{(, conquering, humiliation)}}}{solution}\relax
    \twocolImg{
    % \lstinputlisting[linerange={11-20}]{counterdom.tex}
    \inputminted[firstline=8, lastline=19]{latex}{sec/code/counters/counterdom.tex}
    }{counterdom}
\end{frame}



\begin{frame}[fragile]{Redefine existing counter domination\magicPage}{``equation'' example}\relax
    Package based solution:
    
    \lstinline|\usepackage{chngcntr}| 
    
    and \lstinline|\counterwith{equation}{chapter}|  to make the ``equation'' a slave or \lstinline|\counterwithout{equation}{chapter}| to ``free'' the counter.
    
    Core-based solution:

\begin{minted}{latex}
\makeatletter
\@removefromreset{equation}{section}
\@addtoreset{equation}{chapter}
\renewcommand{\theequation}{\thechapter.\@arabic\c@equation}
\makeatother
\end{minted}


    \skfootnote{\stExC{https://tex.stackexchange.com/questions/61756/how-to-change-equation-numbering-style} \stExC{https://tex.stackexchange.com/questions/54241/change-the-type-of-equation-numbering-in-document-class-article} \stExC{https://tex.stackexchange.com/questions/28333/continuous-v-per-chapter-section-numbering-of-figures-tables-and-other-docume} \normalfont\url{https://texfaq.org/FAQ-running-nos} \lvoc{IX.2.1}[293].\\ Also see \ccol{\p@} prefix \lvoc{IX.2.2}[295] and \stExC{https://tex.stackexchange.com/questions/61426/how-to-make-ref-display-only-subsection}}     
\end{frame}


\begin{frame}[fragile]{Define and simple manipulation\tW}\relax


     \textbf{Define new} \ccol\newcount\ccol{\<countname>} as \verb|\newcount\mycounter|
     
     \textbf{Set number} \ccol{\<countname>=<number>}  Or use \ccol\countdef. Like \verb|\countdef\mynumber=43|
     
     \textbf{Add number}  \ccol{\advance\string\<countname>\ by <number>}. Also there are \ccol\multiply\ and \ccol\divide. As well as \ccol\numexp.
     
     \textbf{Show number} \ccol{\the\string\<countname>} or \ccol\number\ or \ccol\romannumeral
     
     \skfootnote{\vspace{-3ex}\knuthc{15}[129] \stExC{https://tex.stackexchange.com/questions/245635/formal-syntax-rules-of-dimexpr-numexpr-glueexpr}\\Actually you can use \ccol\count<number> like \ccol\count212. What \ccol\newcount\ do is just find a free number and fix it to your defined name.}
\end{frame}

\begin{frame}{Define and simple manipulation\tW}{Example}\relax

    \twocolImg{
    % \lstinputlisting[linerange={11-16}]{counterTeX.tex}
        \inputminted[firstline=11, lastline=16]{latex}{sec/code/counters/counterTeX.tex}
    }{counterTeX}

\end{frame}

\graphicspath{{sec/images/}{sec/code/}}
\lstset{inputpath=sec/code/}

\subsection{Lengths}
\graphicspath{{sec02/images/}{sec02/code/}}
\lstset{inputpath=sec02/code/}

\begin{frame}[label=simple,fragile]{Simplest Beamer document}\relax
    \cprotect\twocolImg{\lstinputlisting{simplest01.tex}}{simplest01}
    \vspace{15mm}
    \inclassFrag{Try it! \hyperlink{style}{\beamerbutton{STYLE}} }[-1]
\end{frame}

\begin{frame}[fragile]{Beamer document structure}\relax
    \lstinputlisting[numbers=none]{structure.tex}
\end{frame}

\begin{frame}[fragile]{Title page (Preamble)}\relax
    \cprotect\twocolImg{\lstinputlisting[linerange={5-17}]{title.tex}}{title}
\end{frame}

\begin{frame}[fragile]{TOC}\relax
    \cprotect\twocolImg{\lstinputlisting[linerange={11-13}]{toc.tex}}{toc1}
\end{frame}

\begin{frame}[fragile]{TOC ({\tt AtBeginSection[]})}\relax
    \cprotect\twocolImg{\lstinputlisting[linerange={3-7}]{toc.tex}}{toc2}
\end{frame}


\begin{frame}[fragile]{Appendix}\relax
    \cprotect\twocolImg{\visible<2>{\lstinputlisting[linerange={4-4}]{appendix.tex}} \\
    \lstinputlisting[linerange={9-12}]{appendix.tex}}{appendix}
\end{frame}

\begin{frame}[fragile]{Bibliography (bibtex)}\relax
    \cprotect\twocolImg{\lstinputlisting[linerange={10-16}]{bibtex.tex}}{bibtex.pdf}
\end{frame}

\begin{frame}[fragile]{Bibliography (simple)}\relax
    \cprotect\twocolImg{\lstinputlisting[linerange={9-21}]{bibliography.tex}}{bibliography.pdf}
\end{frame}

\begin{frame}[fragile]{Frame: Columns}\relax
    \cprotect\twocolImg{\lstinputlisting[linerange={8-17}]{columns.tex}}{columns.pdf}
\end{frame}


\subsection{Boxes}
% \knuthc  knuth the TeXBook
% \lvoc   Lvovsky
% \lamc  lamport latex 
% \slshape different font for footnote
\graphicspath{{sec01/images/s2/}{sec01/code/s2/}}
\lstset{inputpath=sec01/code/s2/}

\begin{frame}[fragile]{How good it will be if...\only<2,3>{we could write like this}}\relax
\begin{center}
\only<1>{\includegraphics{refBegin}}
\only<2>{\includegraphics{refBegin2}}
\only<3>{\Huge We can!}
\end{center}

\cprotect\skfootnote{I use \verb|\only<2>{text}| for such effect. Also \verb|\setcounter| was used in source.}
\end{frame}

\begin{frame}[fragile]{Step 1: \ccol\label}\relax
     \lstinputlisting[linerange={12-15}, basicstyle=\tt]{refBegin2.tex}
     
     \skfootnote{for whole reference mechanism: \lmanc{7}[51] \lvoc{I.2.11}[27] \overC{https://www.overleaf.com/learn/latex/Cross_referencing_sections_and_equations}, \wikiC{https://en.wikibooks.org/wiki/LaTeX/Labels_and_Cross-referencing}}
\end{frame}

\begin{frame}[fragile]{Step 2: \ccol{\ref} and \ccol{\pageref}}
     \lstinputlisting[linerange={17-17}, basicstyle=\tt]{refBegin2.tex}
\end{frame}

\begin{frame}[fragile]{Problem 1: lots of labels!}
    What if you have a lot of labels?
    \inclass{\incPause Look throw whole document and compare \TeX\ and pdf?\incPause }
    
\cprotect\twocolImg{
    \lstinputlisting[linerange={10-10, 13-16}, basicstyle=\tt\small]{refBeginShow.tex}
}{refBeginShow}    
    
    
    \skfootnote{\url{http://ctan.altspu.ru/macros/latex/contrib/showlabels/showlabels.pdf}}
\end{frame}

\begin{frame}[fragile]{Problem 2: Typos}\relax
\cprotect\twocolImg{
    \lstinputlisting[linerange={12-17}, basicstyle=\tt\small]{refBad.tex}
}{refBad}           
\end{frame}

\begin{frame}[fragile]{Bibliography}{How to cite}\relax
Use \ccol{\cite\{label\}}

\cprotect\twocolImg{
    \lstinputlisting[linerange={9-9}, basicstyle=\tt\small]{citeFirst.tex}
}{citeFirst}  

\incPause
\cprotect\twocolImg{
    \lstinputlisting[linerange={11-12}, basicstyle=\tt\small]{citeSecond.tex}
}{citeSecond}  

\skfootnote{\lmanc{8.24.2}[94] \wikiC{https://en.wikibooks.org/wiki/LaTeX/Bibliography_Management} \overC{https://www.overleaf.com/learn/latex/Bibliography_management_in_LaTeX}}     
\end{frame}

\begin{frame}[fragile]{Bibliography}{What to cite}\relax

{\csk .bib} files

\begin{verbatim}
@Book{landau,
    author = {Landau, L. D. and Lifshitz, E. M.},
    title = {The Classical Theory of Fields},
    journal = N,
    volume = {1},
    pages = {140},
    year = 1980
}
\end{verbatim}

You can have multiple records in one .bib file.
     
\end{frame}

\begin{frame}[fragile]{Bibliography. Where can you get .bib files?\preMagicPage}\relax
\begin{itemize}
    \item Just google it! ``article\_name bibtex''
    \item Go to you favorite journal and look at Citations -> ``.bib'' or ``bibtex''
    \item Ask Mendeley or other programs to give you the .bib file
    \item Create it by yourself
\end{itemize}
\end{frame}

\begin{frame}[fragile]{Bibliography. Creating .bib file \magicPage}\relax
\begin{center}
\begin{tabular}{rl}
     \ccol{@article} & Journal or magazine article\\
\ccol{@book} & Book\\
\ccol{@conference} & Article in conference proceedings\\
\ccol{@misc} & If nothing else fits.
\end{tabular}
\end{center}
\skfootnote{\wikiC{https://en.wikibooks.org/wiki/LaTeX/Bibliography_Management\#Standard_templates} for cite url checkout \stExC{https://tex.stackexchange.com/questions/3587/how-can-i-use-bibtex-to-cite-a-web-page}
}     
\end{frame}

\begin{frame}[fragile]{Bibliography. Creating .bib file \magicPage}\relax
\begin{center}
{\obeylines
author
title
journal
year
pages
volume
...
}
\end{center}
\skfootnote{check for full list \wikiC{https://en.wikibooks.org/wiki/LaTeX/Bibliography_Management\#BibTeX}
}     
\end{frame}

\begin{frame}[fragile]{Bibliography. Offline\preMagicPage}\relax
Running \LaTeX\ offline, you can get \textbf{(??)} in \ccol{\ref} and \textbf{[?]} in \ccol{\cite}. 

For 
\begin{itemize}
    \item References 
    \item Bibliography
    \item Table of content
    \item Indexing
    \item ...
\end{itemize}

\LaTeX\ collect addition data in extra files. \LaTeX\ need more then one run to get this data. 

Use \verb|latex; bibtex; latex; latex|

\end{frame}

\begin{frame}[fragile]{Bibliography. Manually\magicPage}

You can add Bibliography manually.

\cprotect\twocolImg{
    \lstinputlisting[linerange={10-22}]{citeMan.tex}
}{citeMan}  

\skfootnote{\lmanc{8.24}[92]}
\end{frame}

\begin{frame}[fragile]{Bibliography. Styles\magicPage}

You can change styles.

Mannually -- check \url{https://en.wikibooks.org/wiki/LaTeX/Bibliography_Management}

Our with packages -- check \url{https://tex.stackexchange.com/questions/25701/bibtex-vs-biber-and-biblatex-vs-natbib}
\end{frame}

\begin{frame}[fragile]{Subject index\magicPage}\relax
\cprotect\twocolImg{
    \lstinputlisting[linerange={9-16}]{indexmy.tex}
}{indexmy}  

\skfootnote{\lvoc{IV.7}[175] \lmanc{25.2}[215]}     
\end{frame}
\subsection{Glue}
% \knuthc  knuth the TeXBook
% \lvoc   Lvovsky
% \lamc  lamport latex 
% \slshape different font for footnote
\graphicspath{{sec01/images/s3/}{sec01/code/s3/}}
\lstset{inputpath=sec01/code/s3/}

\begin{frame}[fragile]{\ccol{\footnote}\magicPage}\relax
\cprotect\twocolImg{
    \lstinputlisting[linerange={9-9}]{footnoteMy.tex}
}{footnoteMy}  


\skfootnote{\lmanc{8}[108] \wikiC{https://en.wikibooks.org/wiki/LaTeX/Footnotes_and_Margin_Notes} \overC{https://www.overleaf.com/learn/latex/Footnotes} \knuthc{15}[120]}
\end{frame}

\begin{frame}[fragile]{Horizontal aligment\magicPage}\relax
\cprotect\twocolImg{
    \lstinputlisting[linerange={9-25}]{flushing.tex}
}{flushing}  


\skfootnote{\lmanc{8.12}[63] \lmanc{8.13}[64] \knuthc{14}[112] and note \stExC{https://tex.stackexchange.com/questions/64644/centering-doesnt-seem-to-center-my-text}}
\end{frame}

\begin{frame}[fragile]{Page break\magicPage}\relax

{\centering \Large \ccol{\newpage} \ccol{\pagebreak}\par}

\skfootnote{\lmanc{10}[105] \stExC{https://tex.stackexchange.com/questions/736/pagebreak-vs-newpage}}
     
\end{frame}

\begin{frame}[fragile]{Quotes\magicPage}\relax

\cprotect\twocolImg{
    \lstinputlisting[linerange={10-14}]{quotemy.tex}
}{quotemy}  


\cprotect\skfootnote{\lvoc{III.7.1}[129] \lmanc{8.20}[83]. Also see \verb|quotation| env }
     
\end{frame}

\begin{frame}[fragile]{Verses\magicPage}\relax

\cprotect\twocolImg{
    \lstinputlisting[linerange={11-18}]{versemy.tex}
}{versemy}  

\skfootnote{\lvoc{III.7.3}[130] \lmanc{8.28}[98]}
\end{frame}

\begin{frame}[fragile]{Marginal notes\magicPage}\relax

\cprotect\twocolImg{
    \lstinputlisting[linerange={9-10}]{marginmy.tex}
}{marginmy}  

\skfootnote{\lvoc{IV.10}[194] \lmanc{15.4}[137]}
\end{frame}

 
\cprotect\inclassFrame{
\begin{frame}[fragile]{Try to add a package}\relax
\url{http://hanno-rein.de/downloads/coffee.sty}

and use \verb|\newpage \coffee{2} hello world!|

\incPause
NOW you document is ready!

\end{frame}
}
\subsection{Spaces}
\graphicspath{{sec/images/}{sec/code/}}
\lstset{inputpath=sec/code/}

% phantom, smash
\begin{frame}[fragile]{What spaces we have?}
     \def\showLength#1{\newbox\boxtodimen%
\newdimen\wb%
\setbox\boxtodimen=\hbox{#1}%
\wb=\the\wd\boxtodimen%
     \leavevmode\raise4pt\hbox{\vrule height 6pt depth2pt{\csk\rule{\wb}{4pt}}\vrule height 6pt depth2pt}%
     }
     
    \begin{columns}
    \begin{column}{0.45\textwidth}
    
\centering\Large
     \begin{tabular}{rl}
          \ccol\qquad& \showLength{\qquad}\\ 
          \ccol\quad& \showLength{\quad}\\ 
          \ccol\enspace & \showLength\enspace\\
          \ccol\  & \showLength{\ }\\
     \end{tabular}
     
     \ccol\quad\ is used below:
     $$x=y\quad\hbox{if y=0}$$
    \end{column}
    \begin{column}{0.45\textwidth}
    
\centering\Large
     \begin{tabular}{rl}
          \multicolumn{2}{c}{useful for math:}\\
          \ccol{\;} & \showLength{\;}\\
          \ccol{\>} & \showLength{$\>$}\\
          \ccol{\,} & \showLength{\,}\\
          \ccol{\!} & \showLength{\!}\\
     \end{tabular}
     
     the last one is negative space
    \end{column}
    \end{columns}
     
     
     \skfootnote{\knuthc{18}[178] \lmanc{19.1}[167]}
\end{frame}

\begin{frame}{Phantoms}\relax
    
    Phantoms have the same size as it's an argument without drawing.\inpause 
    
    \twocolImg{
    % \lstinputlisting[linerange={39-45}]{phantomsmy.tex}
    \inputminted[firstline=39, lastline=45]{latex}{sec/code/phantomsmy.tex}
    }{phantomsmy}
    \ccol\phantom\ leaves both dimentions. \ccol\hphantom\ and \ccol\vphantom\ leaves only one dimention.
    
    \ccol\strut\ is short for \ccol\vphantom\{(\}
    
    \ccol\smash\ is using to leave only the horizontal coordinate of a formula
    
    \skfootnote{\lmanc{16.6}[156] \knuthc{18}[178]}
     
\end{frame}
\subsection{Toks}
\begin{frame}{Toks manipulation\tW\magicPage}\relax
    \TeX\ has addition registers for storing strings. They have \ccol\toks\ prefix/suffix. The difference between toks and storage inside macros are in extension (We mention the extension mechanism at the last lecture). In toks \TeX store tokens (unexpanded).
     
     \skfootnote{\stExC{https://tex.stackexchange.com/questions/455208/token-register-vs-macro-register}}
\end{frame}
\subsection{Inserts}
\begin{frame}{Inserts\tW\magicPage}\relax

    \TeX\ has addition registers for storing floats. They have \ccol\insert\ prefix/suffix. 
     
     \skfootnote{\knuthc{15}[132] \stExC{https://tex.stackexchange.com/questions/527613/simple-example-of-using-newinsert-and-friends}\\ But I failed to make a working example...}
\end{frame}


\inclassframe{

\begin{frame}{\exFrame{Make footnote center}}{lvl 1}\relax
    Update the code of footnote generation from previous exercise and make the following change:
    \begin{itemize}
        \item Make a spaces between frame number, footnote an year (see next frame)
        \item Please, use only glue commands
         
    \end{itemize}
\end{frame}

\begin{frame}[t]{\exFrame{Implement footnotes}}{lvl 1}\relax
    \centering
    \vspace*{-1cm}
    \fbox{\includegraphics[scale=0.8]{exLvl1.pdf}}
\end{frame}

\begin{frame}{\exFrame{Make footnote center}}{lvl 2}\relax
     Update the previous template. 
     \begin{itemize}
         \item Remove year information from the right
         \item Make the footnote appears \textit{exactly} at the center independing of how width are slide number infrormation
         \item Check one and multiline footnote 
         \item You can use all build-in possibilities
     \end{itemize}
\end{frame}

}