\documentclass[aspectratio=169]{beamer}
\usepackage{ifxetex}
\usepackage{fontspec}
\usepackage{xunicode}
\usepackage{xltxtra}
\usepackage{xecyr}
\usepackage{polyglossia}
\usepackage{multicol}
\usepackage{pdfpages} 


\usepackage{beamerskoltech} 
% \renewcommand{\logoname}{sklogo.png} % <- default value for logo 
% \renewcommand{\logoname}{} % <- if you want no logo

\usepackage{progressbar}
\colorlet{progressbarcolor}{skoltechgreen}
\usepackage[mode=out]{inoutclass}
\definecolor{inclasscolor}{RGB}{46, 228, 182}
\definecolor{outclasscolor}{RGB}{41, 0, 204}
\usepackage{xargs}
\usepackage{latexLectures}
\usepackage{minted}
\usetikzlibrary{intersections}
\usetikzlibrary{calc}


\title{\LaTeX:\\ \Large from dummy to \TeX nician}
\subtitle{core Typography. How \TeX\ works-1}
\author{Anton Lioznov}
\institute{Skoltech, \\Project Center of Omics Technologies and Advanced Mass Spectrometry}
\date{ISP 2025,\\ \textit{lesson 4}}
\usetikzlibrary{decorations.pathreplacing}
\begin{document}


\frame{\titlepage}

\AtBeginSection[] 
{
  \begin{frame}{What we will know?}
  \tableofcontents[currentsection,hideallsubsections]
  \end{frame}
}
\AtBeginSubsection[]
{
  \begin{frame}{What we will know?}
  \tableofcontents[currentsubsection, hideothersubsections, sectionstyle=show/hide, subsectionstyle=show/shaded/hide]
  \end{frame}
}

{\supressprogressbartrue


\begin{frame}\frametitle{What we will know?}
\tableofcontents[hideallsubsections]
\end{frame}

\supressprogressbarfalse
}

% \begin{frame}
% % \begin{enumerate}
% %     \item Overview and basis
% %         \begin{tikzpicture}[remember picture, overlay]
% %             \draw[decoration={brace}, decorate] 
% %             ([yshift=0.5ex]pic cs:start) --
% %             ([yshift=-2ex]pic cs:end)
% %             node[midway, right=0.3cm] {User level};
% %         \end{tikzpicture}\tikz[remember picture]\coordinate(start);
% %     \item Document creation\tikz[remember picture]\coordinate(end);
% %     \item TikZ and Typography
% %     \item \TeX\ and Typography
% %     \item Command creation
% % \end{enumerate}
     
% %      \begin{itemize}
% %     \item[$\left.\begin{cases}
% %         \text{1.} & \text{First item in the list} \\
% %         \text{2.} & \text{Second item in the list}
% %     \end{cases}\right\}$] Text explaining items 1 and 2
    
% %     \item[$\left.\begin{cases}
% %         \text{3.} & \text{Third item in the list}
% %     \end{cases}\right\}$]
    
% %     \item[$\left.\begin{cases}
% %         \text{4.} & \text{Fourth item in the list} \\
% %         \text{5.} & \text{Fifth item in the list}
% %     \end{cases}\right\}$] Text explaining items 4 and 5
% % \end{itemize}

% % \begin{enumerate}
% %     \item \tikz\node(item1){Item 1};
% %     \item \tikz\node(item2){Item 2};
% %     \item \tikz\node(item3){Item 3};
% %     \item \tikz\node(item4){Item 4};
% %     \item \tikz\node(item5){Item 5};
% % \end{enumerate}

% % \begin{tikzpicture}[overlay, remember picture]
% %     % Bracket for items 1 and 2
% %     \draw[decorate, decoration={brace, amplitude=5pt}, thick]
% %         ([xshift=-5pt, yshift=3pt]item1.north west) -- ([xshift=-5pt, yshift=-3pt]item2.south west)
% %         node[midway, left=12pt] {Group A};

% %     % Bracket for item 3
% %     \draw[decorate, decoration={brace, amplitude=5pt}, thick]
% %         ([xshift=-5pt, yshift=3pt]item3.north west) -- ([xshift=-5pt, yshift=-3pt]item3.south west)
% %         node[midway, left=12pt] {Solo Item};

% %     % Bracket for items 4 and 5
% %     \draw[decorate, decoration={brace, amplitude=5pt}, thick]
% %         ([xshift=-5pt, yshift=3pt]item4.north west) -- ([xshift=-5pt, yshift=-3pt]item5.south west)
% %         node[midway, left=12pt] {Group B};
% % \end{tikzpicture}
% \end{frame}

\section{Technical agreements}

\begin{frame}{Agreements}{I}\relax
     {\Large inclass/outclass versions}
     \begin{itemize}
          \item two slightly different versions for class and home
          \item class version is more interactive and contains less information
          \inclass{\item \inclasshigh{} this line will be shown only in class} 
          \outclass{\item \outclasshigh{} this line will be shown only at home version}
     \end{itemize}
\end{frame}
\inclassframe{\begin{frame}{Frame for a class}
     
\end{frame}}
\outclassframe{\begin{frame}{Frame for home}
     
\end{frame}}

{\supressfootnotefalse
\begin{frame}[fragile]{Agreements}{II}\relax
\newcommand{\tikzmark}[1]{\tikz[overlay,remember picture] \node (#1) {};}

{ \Large Footnotes }
\begin{tikzpicture}[overlay,remember picture]
\draw[->,ultra thick] (0,0.1) to[out=0,in=45] (8, -1.5) to[out=225,in=90] (5,-5.5);
\end{tikzpicture}

\skfootnote{Like this}
\begin{itemize}
     \item For second reading
     \item Contains advanced usage of the command 
     \item Contains references to read more 
     \begin{itemize}
         \item to the exact chapter 
         \item (often) with the href to exact page  
     \end{itemize}
     \item Contains some comments
     \item Mostly for outclass version
\end{itemize}
\end{frame}

}

{\inclassmodetrue
\newcommand{\tikzmark}[1]{\tikz[overlay,remember picture] \node (#1) {};}

\begin{frame}{Agreements}{III}\relax
{ \Large Addition information -- ``magic'' \tikzmark{startM}} 

\begin{tikzpicture}[remember picture,overlay,shift={(current page.north east)}]
\node[anchor=north east,xshift=-0cm,yshift=-0cm](endM) {%
{\includegraphics[width=1cm]{images/magic}}%
};
\end{tikzpicture}%

\begin{tikzpicture}[overlay, remember picture]
    \draw[->,ultra thick] (startM) to[out=0, in=180] (endM);
\end{tikzpicture}

\begin{itemize}
     \item To have the full picture 
     \item Not to analyze or to puzzle out in class 
\end{itemize}
\end{frame}

}


\begin{frame}{\exFrame{Agreements}}{V}\relax


{ \Large Exercises } \begin{tikzpicture}[overlay]
% \draw[white] (0,0) -- (0, 0.6);
\draw[->,ultra thick] (0,0.1) to[out=0,in=-90](-2.1,3.15);
\end{tikzpicture}


\begin{itemize}
     \item To work in class 
\end{itemize}
\end{frame}
\begin{frame}{Special thanks to}\relax

     Our TAs:
     \begin{itemize}
         \item Peter Borisovets
         \item Pavel Kuzmin
         \item Anna Litvin
     \end{itemize}

\end{frame}

\supressfootnotefalse
\ifinclasstrue
\def\skfootnote#1{\myfootnote{{\color{white!70!black}#1}}}
\fi
%%%%%%%%%%%%%%%%%%%%%%%%%% MAIN SECTION %%%%%%%%%%%%%%%%%%%%%%%%%%%%%%%%%


\graphicspath{{sec01/images/}{sec01/code/}}
\lstset{inputpath=sec01/code/}

\subsection{Introduction: what is TikZ and when to use it}
\documentclass{article}
\usepackage{fontspec}
\pagestyle{empty}
\usepackage{geometry}
\geometry{paperwidth=50mm, paperheight=20mm, left=2mm, top=0mm, right=2mm, bottom=0mm} %, layoutwidth=60mm, layoutheight=35mm,
\parindent=0pt

\begin{document}
\vspace*{\fill} \vspace*{-5ex}

G\hskip0em lu\hskip0.5em e and k\kern0em e\kern0.5em rn provides...


\vspace*{\fill}
\end{document}

\subsection{General usage}
\documentclass{article}
\usepackage{fontspec}
\pagestyle{empty}
\usepackage{geometry}
\geometry{paperwidth=50mm, paperheight=65mm, left=5mm, top=5mm, right=5mm, bottom=5mm} %, layoutwidth=60mm, layoutheight=35mm,

\begin{document}
\vspace*{\fill} \vspace*{-5ex}

In most cases text is just a text. You write it and write and write. The system create line breaks by itself.

\vspace*{\fill}
\end{document}

\subsection{Graphs}
\outclassframe{
\begin{frame}{\ }\relax
    TikZ allows graphics to work according to ``rules''. This is a minus for an arbitrary drawing, but it is an advantage for those drawings that have a given structure and are also built according to ``rules''
\end{frame}
}

{\forcewidefootnote=1
\newcommand{\tikzmark}[1]{\tikz[overlay,remember picture] \node (#1) {};}

\begin{frame}{Graph example 1}\relax
    \twocolImg{
    \only<1>{\inputminted[firstline=9, lastline=10]{latex}{sec01/code/graphSample.tex}
    \inputminted[firstline= 16, lastline=27]{latex}{sec01/code/graphSample.tex}}
    % \only<2>{
    % \smash{\begin{tikzpicture}[overlay, remember picture]
    %         \draw[->, thick] (5.5, -0.5) to[out=0, in=80] +(3.8, 0.3);
    %     \end{tikzpicture}}
    % \inputminted[firstline=26, lastline=26, fontsize=\tt\nornalsize]{latex}{sec01/code/graphSample.tex}
        
    % }
    }{graphSample}
    
     
     \only<1>{You can write \ccol{below=of <label>} to have a relative coordinate}
     
\end{frame}
}

{\forcewidefootnote=1
\begin{frame}{Chain example\magicPage}{}\relax
    \includegraphics[width=\textwidth]{chainSample}
    
    \vspace{-2em}
    \inputminted[firstline= 9, lastline=10, fontsize=\tt\tiny]{latex}{sec01/code/chainSample.tex}
    \inputminted[firstline= 16, lastline=32, fontsize=\tt\tiny]{latex}{sec01/code/chainSample.tex}

     \skfootnote{\tikzc{I.5}[69]}
\end{frame}
}

\begin{frame}{Tree}

\twocolImg{
\inputminted[firstline=9, lastline=9]{latex}{sec01/code/treeSample.tex}
\inputminted[firstline= 16, lastline=25]{latex}{sec01/code/treeSample.tex}
}{treeSample}

We use \ccol\node\ and \ccol{child}.

\ccol{sibling distance} option provides a horizontal distance between nodes
     
\end{frame}

\inclassframe{
\begin{frame}{\exFrame{reproduce the following with tikZ}}{lvl 1. \textit{You can use absolute coodrinates}}\relax



     \includegraphics[width=0.4\textwidth]{exLvl1}
\end{frame}

\begin{frame}{\exFrame{reproduce the following with tikZ}}{lvl 2. \textit{You can't use absolute coodrinates}}\relax

     
     \includegraphics[width=0.4\textwidth]{exLvl2}
\end{frame}

}




\subsection{Arrangment}
\outclassframe{
\begin{frame}{\ }\relax
    Very few people make full-fledged charts in tikZ. But it is much more common to add tikZ as text decoration element.
\end{frame}
}


\begin{frame}[fragile]{Introduction}\relax

TikZ is often used not as ``independent picture'', but as a part of the presentation or document. 

\end{frame}

\begin{frame}[fragile]{Example: CV\magicPage}\relax
    \begin{columns}
        \begin{column}{0.45\textwidth}
              \only<1>{
              \expandafter{\inputminted[firstline= 16, lastline=20]{latex}{sec01/code/cv_examples.tex}
                }
                }
                \only<2>{\expandafter{\inputminted[firstline= 22, lastline=32]{latex}{sec01/code/cv_examples.tex}}}
        \end{column}
        \begin{column}{0.45\textwidth}
             \includegraphics[width=\textwidth, keepaspectratio]{cv_examples}
        \end{column}
    \end{columns}

    % \cprotect\twocolImg{
    % \only<1>{\inputminted[firstline= 16, lastline=20]{latex}{sec01/code/cv_examples.tex}}
    % \only<2>{\inputminted[firstline= 22, lastline=32]{latex}{sec01/code/cv_examples.tex}}
    % }{cv_examples}  
    
\end{frame}

\begin{frame}[fragile]{``Magic''}\relax

    How to produce ``magic'' \only<1>{\tikz[overlay, remember picture] \draw[->] (0, 0.1) to[out=0,in=180] (endM);}

\begin{tikzpicture}[remember picture,overlay,shift={(current page.north east)}]
    \node[anchor=north east,xshift=-0cm,yshift=-0cm](endM) {%
        {\includegraphics[width=1cm]{black_magic}}%
};
\end{tikzpicture}%

\inpause
\begin{minted}{latex}
\begin{tikzpicture}[remember picture,overlay,shift={(current page.north east)}]
\node[anchor=north east,xshift=-0cm,yshift=-0cm](endM) {%
    {\includegraphics[width=1cm]{images/magic}}%
};
\end{tikzpicture}
\end{minted}

\inpause
Notice \ccol{shift=\{(current page.north east)\}}
\end{frame}



\begin{frame}[fragile]{arrow to magic}\relax

How to produce this \tikz[overlay, remember picture] \node (arrstart) {}; arrow \tikz[overlay, remember picture] \node (startMMM) {};

\begin{tikzpicture}[remember picture,overlay,shift={(current page.north east)}]
    \node[anchor=north east,xshift=-0cm,yshift=-0cm](endMMM) {%
        {\includegraphics[width=1cm]{black_magic}}%
};
\end{tikzpicture}%

\tikz[overlay, remember picture] \draw[->, ultra thick] ($(startMMM)+(0,0.1)$) to[out=0,in=180] (endMMM);
\only<1>{\begin{tikzpicture}[overlay, remember picture]
     \path[->, ultra thick, name path=point c] ($(startMMM)+(0,0.1)$) to[out=0,in=180] (endMMM);
     
     \path[->, name path=point a] ($(startMMM)+(5,-1)$) -- +(-1, 3);
     \path[name intersections={of=point c and point a, by={intersection}}];
     \draw[thick, thick, ->] ($(arrstart)-(0.4ex,-0.6ex)$) to[out=45, in=110] (intersection);
\end{tikzpicture}}

\inpause

\begin{minted}{latex}
% remember position of (startM) node (and (endM) node from previous slide)
How to produce this arrow \tikz[overlay, remember picture] \node (startM) {};

% go from (startM) to (endM)
\tikz[overlay, remember picture] \draw[->, ultra thick] ($(startM)+(0,0.1)$) to[out=0,in=180] (endM);

\end{minted}

\skfootnote{What about arrow pointed on the thick arrow? It was produced with intersections library}
     
\end{frame}


\begin{frame}[fragile]{``Common belief''\magicPage}\relax
     \begin{center}
        \begin{tikzpicture}
             \node[align=center] (0,0) {
             \huge \LaTeX\ is only for use\\ \huge  in academic area
             };
             \uncover<2,3>{\node[rotate=30, bottom color=red!50, top color=red!50] (0,0) {\Huge WRONG};}
        \end{tikzpicture}
         
    \end{center}
    
\pause\pause was produce by 
\begin{minted}{latex}
\begin{tikzpicture}
    \node[align=center] (0,0) {
    \huge \LaTeX\ is only for use\\ \huge  in academic area
    };
    \uncover<2,3>{\node[rotate=30, bottom color=red!50, top color=red!50] (0,0) {\Huge WRONG}};
\end{tikzpicture}
\end{minted}
    
\end{frame}


\cprotect\inclassframe{
\begin{frame}[fragile]{\exFrame{Try to use other people's work}}{lvl 1}\relax

    \begin{enumerate}
        \item You have the code as below. Copy it to you document and make sure it is compiling. ``Play'' with the params and of \ccol\pictofraction\ and as the result give the enumerate list in pdf of what all params means. Ex: ``1. printed symbol, 2...''
         \item Go to \url{https://tikz.dev/library-mindmaps} and try to create your own mindmap
    \end{enumerate}

    \inputminted[firstline=7, lastline=17]{latex}{sec01/code/exExtLvl1.tex}
    \inputminted[firstline=21, lastline=22]{latex}{sec01/code/exExtLvl1.tex}
%7-17, 21-22
    %  \includegraphics[width=0.4\textwidth]{exLvl1}
\end{frame}

\begin{frame}[fragile, allowframebreaks]{\exFrame{Try to use other people's work}}{lvl 2}\relax
You copy a code, that produce a ``day cyrcle'' (from AltaCV). But some libraries and packages are missing. See logs and google errors, make this code works :)

\includegraphics[width=0.8\textwidth]{exExtLvl2full}

     \inputminted[fontsize=\tiny]{latex}{sec01/code/exExtLvl2notfull.tex}
    %  \includegraphics[width=0.4\textwidth]{exLvl2}
\end{frame}

}










\graphicspath{{sec01/images/}{sec01/code/}}
\lstset{inputpath=sec01/code/}

\subsection{Introduction: what is TikZ and when to use it}
\documentclass{article}
\usepackage{fontspec}
\pagestyle{empty}
\usepackage{geometry}
\geometry{paperwidth=50mm, paperheight=20mm, left=2mm, top=0mm, right=2mm, bottom=0mm} %, layoutwidth=60mm, layoutheight=35mm,
\parindent=0pt

\begin{document}
\vspace*{\fill} \vspace*{-5ex}

G\hskip0em lu\hskip0.5em e and k\kern0em e\kern0.5em rn provides...


\vspace*{\fill}
\end{document}

\subsection{General usage}
\documentclass{article}
\usepackage{fontspec}
\pagestyle{empty}
\usepackage{geometry}
\geometry{paperwidth=50mm, paperheight=65mm, left=5mm, top=5mm, right=5mm, bottom=5mm} %, layoutwidth=60mm, layoutheight=35mm,

\begin{document}
\vspace*{\fill} \vspace*{-5ex}

In most cases text is just a text. You write it and write and write. The system create line breaks by itself.

\vspace*{\fill}
\end{document}

\subsection{Graphs}
\outclassframe{
\begin{frame}{\ }\relax
    TikZ allows graphics to work according to ``rules''. This is a minus for an arbitrary drawing, but it is an advantage for those drawings that have a given structure and are also built according to ``rules''
\end{frame}
}

{\forcewidefootnote=1
\newcommand{\tikzmark}[1]{\tikz[overlay,remember picture] \node (#1) {};}

\begin{frame}{Graph example 1}\relax
    \twocolImg{
    \only<1>{\inputminted[firstline=9, lastline=10]{latex}{sec01/code/graphSample.tex}
    \inputminted[firstline= 16, lastline=27]{latex}{sec01/code/graphSample.tex}}
    % \only<2>{
    % \smash{\begin{tikzpicture}[overlay, remember picture]
    %         \draw[->, thick] (5.5, -0.5) to[out=0, in=80] +(3.8, 0.3);
    %     \end{tikzpicture}}
    % \inputminted[firstline=26, lastline=26, fontsize=\tt\nornalsize]{latex}{sec01/code/graphSample.tex}
        
    % }
    }{graphSample}
    
     
     \only<1>{You can write \ccol{below=of <label>} to have a relative coordinate}
     
\end{frame}
}

{\forcewidefootnote=1
\begin{frame}{Chain example\magicPage}{}\relax
    \includegraphics[width=\textwidth]{chainSample}
    
    \vspace{-2em}
    \inputminted[firstline= 9, lastline=10, fontsize=\tt\tiny]{latex}{sec01/code/chainSample.tex}
    \inputminted[firstline= 16, lastline=32, fontsize=\tt\tiny]{latex}{sec01/code/chainSample.tex}

     \skfootnote{\tikzc{I.5}[69]}
\end{frame}
}

\begin{frame}{Tree}

\twocolImg{
\inputminted[firstline=9, lastline=9]{latex}{sec01/code/treeSample.tex}
\inputminted[firstline= 16, lastline=25]{latex}{sec01/code/treeSample.tex}
}{treeSample}

We use \ccol\node\ and \ccol{child}.

\ccol{sibling distance} option provides a horizontal distance between nodes
     
\end{frame}

\inclassframe{
\begin{frame}{\exFrame{reproduce the following with tikZ}}{lvl 1. \textit{You can use absolute coodrinates}}\relax



     \includegraphics[width=0.4\textwidth]{exLvl1}
\end{frame}

\begin{frame}{\exFrame{reproduce the following with tikZ}}{lvl 2. \textit{You can't use absolute coodrinates}}\relax

     
     \includegraphics[width=0.4\textwidth]{exLvl2}
\end{frame}

}




\subsection{Arrangment}
\outclassframe{
\begin{frame}{\ }\relax
    Very few people make full-fledged charts in tikZ. But it is much more common to add tikZ as text decoration element.
\end{frame}
}


\begin{frame}[fragile]{Introduction}\relax

TikZ is often used not as ``independent picture'', but as a part of the presentation or document. 

\end{frame}

\begin{frame}[fragile]{Example: CV\magicPage}\relax
    \begin{columns}
        \begin{column}{0.45\textwidth}
              \only<1>{
              \expandafter{\inputminted[firstline= 16, lastline=20]{latex}{sec01/code/cv_examples.tex}
                }
                }
                \only<2>{\expandafter{\inputminted[firstline= 22, lastline=32]{latex}{sec01/code/cv_examples.tex}}}
        \end{column}
        \begin{column}{0.45\textwidth}
             \includegraphics[width=\textwidth, keepaspectratio]{cv_examples}
        \end{column}
    \end{columns}

    % \cprotect\twocolImg{
    % \only<1>{\inputminted[firstline= 16, lastline=20]{latex}{sec01/code/cv_examples.tex}}
    % \only<2>{\inputminted[firstline= 22, lastline=32]{latex}{sec01/code/cv_examples.tex}}
    % }{cv_examples}  
    
\end{frame}

\begin{frame}[fragile]{``Magic''}\relax

    How to produce ``magic'' \only<1>{\tikz[overlay, remember picture] \draw[->] (0, 0.1) to[out=0,in=180] (endM);}

\begin{tikzpicture}[remember picture,overlay,shift={(current page.north east)}]
    \node[anchor=north east,xshift=-0cm,yshift=-0cm](endM) {%
        {\includegraphics[width=1cm]{black_magic}}%
};
\end{tikzpicture}%

\inpause
\begin{minted}{latex}
\begin{tikzpicture}[remember picture,overlay,shift={(current page.north east)}]
\node[anchor=north east,xshift=-0cm,yshift=-0cm](endM) {%
    {\includegraphics[width=1cm]{images/magic}}%
};
\end{tikzpicture}
\end{minted}

\inpause
Notice \ccol{shift=\{(current page.north east)\}}
\end{frame}



\begin{frame}[fragile]{arrow to magic}\relax

How to produce this \tikz[overlay, remember picture] \node (arrstart) {}; arrow \tikz[overlay, remember picture] \node (startMMM) {};

\begin{tikzpicture}[remember picture,overlay,shift={(current page.north east)}]
    \node[anchor=north east,xshift=-0cm,yshift=-0cm](endMMM) {%
        {\includegraphics[width=1cm]{black_magic}}%
};
\end{tikzpicture}%

\tikz[overlay, remember picture] \draw[->, ultra thick] ($(startMMM)+(0,0.1)$) to[out=0,in=180] (endMMM);
\only<1>{\begin{tikzpicture}[overlay, remember picture]
     \path[->, ultra thick, name path=point c] ($(startMMM)+(0,0.1)$) to[out=0,in=180] (endMMM);
     
     \path[->, name path=point a] ($(startMMM)+(5,-1)$) -- +(-1, 3);
     \path[name intersections={of=point c and point a, by={intersection}}];
     \draw[thick, thick, ->] ($(arrstart)-(0.4ex,-0.6ex)$) to[out=45, in=110] (intersection);
\end{tikzpicture}}

\inpause

\begin{minted}{latex}
% remember position of (startM) node (and (endM) node from previous slide)
How to produce this arrow \tikz[overlay, remember picture] \node (startM) {};

% go from (startM) to (endM)
\tikz[overlay, remember picture] \draw[->, ultra thick] ($(startM)+(0,0.1)$) to[out=0,in=180] (endM);

\end{minted}

\skfootnote{What about arrow pointed on the thick arrow? It was produced with intersections library}
     
\end{frame}


\begin{frame}[fragile]{``Common belief''\magicPage}\relax
     \begin{center}
        \begin{tikzpicture}
             \node[align=center] (0,0) {
             \huge \LaTeX\ is only for use\\ \huge  in academic area
             };
             \uncover<2,3>{\node[rotate=30, bottom color=red!50, top color=red!50] (0,0) {\Huge WRONG};}
        \end{tikzpicture}
         
    \end{center}
    
\pause\pause was produce by 
\begin{minted}{latex}
\begin{tikzpicture}
    \node[align=center] (0,0) {
    \huge \LaTeX\ is only for use\\ \huge  in academic area
    };
    \uncover<2,3>{\node[rotate=30, bottom color=red!50, top color=red!50] (0,0) {\Huge WRONG}};
\end{tikzpicture}
\end{minted}
    
\end{frame}


\cprotect\inclassframe{
\begin{frame}[fragile]{\exFrame{Try to use other people's work}}{lvl 1}\relax

    \begin{enumerate}
        \item You have the code as below. Copy it to you document and make sure it is compiling. ``Play'' with the params and of \ccol\pictofraction\ and as the result give the enumerate list in pdf of what all params means. Ex: ``1. printed symbol, 2...''
         \item Go to \url{https://tikz.dev/library-mindmaps} and try to create your own mindmap
    \end{enumerate}

    \inputminted[firstline=7, lastline=17]{latex}{sec01/code/exExtLvl1.tex}
    \inputminted[firstline=21, lastline=22]{latex}{sec01/code/exExtLvl1.tex}
%7-17, 21-22
    %  \includegraphics[width=0.4\textwidth]{exLvl1}
\end{frame}

\begin{frame}[fragile, allowframebreaks]{\exFrame{Try to use other people's work}}{lvl 2}\relax
You copy a code, that produce a ``day cyrcle'' (from AltaCV). But some libraries and packages are missing. See logs and google errors, make this code works :)

\includegraphics[width=0.8\textwidth]{exExtLvl2full}

     \inputminted[fontsize=\tiny]{latex}{sec01/code/exExtLvl2notfull.tex}
    %  \includegraphics[width=0.4\textwidth]{exLvl2}
\end{frame}

}








\graphicspath{{sec/images/}{sec/code/}}
\lstset{inputpath=sec/code/}

\section{\TeX\ primitives}

\begin{frame}{Entities}

    \begin{enumerate}
        \item Primitive commands
        \item Counters (=integer numbers)
        \item Lengths
        \item Boxes
        \item Glues
        \item Spaces 
        \item Toks (Strings)
        \item Inserts 
    \end{enumerate}
    \inpause
    commands and macros will wait in details the next lecture. No we discuss it just in a few words
\end{frame}

\subsection{Commands (Macros)}


\begin{frame}[fragile]{Simple command creation \lW}\relax
    \twocolImg{
    \inputminted[firstline=8, lastline=8]{latex}{sec/code/commandSimple.tex}
    \inputminted[firstline=13, lastline=20]{latex}{sec/code/commandSimple.tex}
    }{commandSimple}
    
    \inpause
    \inclasshigh{You are a developer now!}
\end{frame}

\subsection{Counters}
\graphicspath{{sec/images/}{sec/code/counters/}}
\lstset{inputpath=sec/code/counters}

\begin{frame}{What is ``counter''}\relax

    ``Counter'' is just an integer number.
    
    It's using in multiple places to count everything in \LaTeX: sections, equations, references, citation, enumerate lists,...
     
\end{frame}

\begin{frame}{Define and simple manipulation with counters \lW}\relax
    \twocolImg{
    % \lstinputlisting[linerange={11-18}]{countersimp.tex}
    \inputminted[firstline=11, lastline=18]{latex}{sec/code/counters/countersimp.tex}
    }{countersimp}
    
    \begin{itemize}
        \item \ccol\newcounter\ to define new counter 
        \item \ccol\setcounter\ to set counter to new value
        \item \ccol\addtocounter\ to add a number to the counter
    \end{itemize}
    
    \skfootnote{\lvoc{VII.3.1}[244] \lmanc{12.5}[117] \lmanc{13.4-5}[128] \wikiC{https://en.wikibooks.org/wiki/LaTeX/Counters\#Counter_manipulation}}
     
\end{frame}

\newcounter{tmptt}
\begin{frame}[fragile]{Print counter\lW}\relax
     
     \newcommand{\countab}[1]{
     \ccol#1\{countname\}&
     \setcounter{tmptt}{1} #1{tmptt} &
     \setcounter{tmptt}{2} #1{tmptt} &
     \setcounter{tmptt}{3} #1{tmptt} &
     \setcounter{tmptt}{4} #1{tmptt} &
     \setcounter{tmptt}{5} #1{tmptt} &
     \setcounter{tmptt}{6} #1{tmptt} &
     \setcounter{tmptt}{7} #1{tmptt} &
     \setcounter{tmptt}{8} #1{tmptt} &
     \setcounter{tmptt}{9} #1{tmptt}
     \\}
     
     \begin{tabular}{l|ccccccccc}
          \countab{\arabic}
          \countab{\alph}
          \countab{\Alph}
          \countab{\roman}
          \countab{\Roman}
          \countab{\fnsymbol}
     \end{tabular}
     \vspace{1ex}
     
     P.S. \ccol\value\ to get ``raw'' value of the counter
     \skfootnote{ \wikiC{https://en.wikibooks.org/wiki/LaTeX/Counters\#Counter_style} \lmanc{13.1}[126]\\ 
     see \lvoc{IX.2.3}[295] for russian analog of \ccol\alph}
\end{frame}

\begin{frame}[fragile]{pre-defined counters in standart classes}\relax
    \newlength{\myboxlen}%
    \setlength{\myboxlen}{5em}
    
    \newcommand{\showC}[1]{%
    \makebox[\myboxlen]{#1\hfill}%
    }
    
    \showC{part}\hfill
    \showC{chapter}\hfill
    \showC{section}\hfill
    \showC{subsection}\hfill
    \showC{subsubsection}\hfill
    \showC{paragraph}\hfill
    \showC{subparagraph}\hfill
    \showC{}\hfill
    \showC{}\hfill
    \showC{}\hfill
    \showC{}\hfill
    \showC{}\hfill
    
    \showC{page}\hfill
    \showC{figure}\hfill
    \showC{table}\hfill
    \showC{footnote}\hfill
    \showC{equation}\hfill
    
    \showC{enumi}\hfill
    \showC{enumii}\hfill
    \showC{enumiii}\hfill
    \showC{enumiv}\hfill
    \showC{}\hfill
    \showC{}\hfill
    \showC{}\hfill
    \showC{}\hfill
    \showC{}\hfill
    
    \TeX's counters (will talk later)
    \showC{\ccol\year}\hfill
    \showC{\ccol\month}\hfill
    \showC{\ccol\day}\hfill
    \showC{\ccol\time}\hfill
    \showC{}\hfill
    \showC{}\hfill
    \showC{}\hfill
    \showC{}\hfill
    \showC{}\hfill
    
    
    
    \skfootnote{\wikiC{https://en.wikibooks.org/wiki/LaTeX/Counters\#LaTeX_default_counters} \lmanc{13}[126]}
\end{frame}

\begin{frame}{Counter Domination\inclasshigh{\textit{(, conquering, humiliation)}}}{problem}\relax
    \begin{columns}
    \begin{column}{0.45\textwidth}
         You may want to write something like
         
         \vspace{-5ex}
         \includegraphics[width=\linewidth]{counterdom}
    \end{column}
    \inpause
    \begin{column}{0.45\textwidth}
    But the straightforward solution will give you 
    
    \vspace{-5ex}
    \includegraphics[width=\linewidth]{counterdomno}
         
    \end{column}
         
    \end{columns}
\end{frame}

\begin{frame}{Counter Domination\inclasshigh{\textit{(, conquering, humiliation)}}}{straightforward solution}\relax

    \twocolImg{
    % \lstinputlisting[linerange={11-20}]{counterdomno.tex}
    \inputminted[firstline=8, lastline=19]{latex}{sec/code/counters/counterdomno.tex}
    }{counterdomno}
    
    \inclassFrag{Try to compile it by your own}[0]
\end{frame}

\begin{frame}[fragile]{Counter Domination\inclasshigh{\textit{(, conquering, humiliation)}}}{The Way}\relax

    \newcommand{\modif}[3]{\fbox{\parbox{\textwidth}{\makebox[\textwidth]{\small\makebox[0.47\textwidth]{#1}\hfill$\to$\hfill\makebox[0.47\textwidth]{#2}}\\ #3}}}
    
    \centering
    \modif{\string\newcounter\{task\}}{\ocol\newcounter\{task\}[section]}{\ocol\newcounter\{<slave>\}[<master>] will resets the value of <slave> if the value of <master> is change}
    
    \inclassFrag{Please, modify your code after this and every step in this frame}
    % \inpause
    
    \modif{\ccol\addtocounter\{task\}\{1\}}{\ccol\refstepcounter\{task\}}{
    \ccol\refstepcounter\{<counter>\} use it to update \ccol\label--\ccol\ref\ mechanism
    }
    
    \inpause
    \modif{\string\textbf\{Task \string\#\string\arabic\{task\}.}{\tiny\string\textbf\{Task \string\#\string\arabic\{section\}.\string\arabic\{task\}.}{Inside \string\newcommand\{\string\tsk\} to redefine the labels}
    
    \inpause
    \modif{}{\tiny\string\renewcommand\{\ccol\thetask\}\{\string\arabic\{section\}.\string\arabic\{task\}\}}{\string\renewcommand\{\ccol{\the<counter>}\} to redefine the reference}
    
    \skfootnote{\lvoc{VII.3.3}[250] \lmanc{13.6}[128]}
\end{frame}

\begin{frame}{Counter Domination\inclasshigh{\textit{(, conquering, humiliation)}}}{solution}\relax
    \begin{columns}[t]
        \begin{column}{0.5\textwidth}
             \inputminted[firstline=8, lastline=19]{latex}{sec/code/counters/counterdomno.tex}
        \end{column}
        \begin{column}{0.6\textwidth}
             \inputminted[firstline=8, lastline=19]{latex}{sec/code/counters/counterdom.tex}
        \end{column}
         
    \end{columns}
     
\end{frame}

\begin{frame}{Counter Domination\inclasshigh{\textit{(, conquering, humiliation)}}}{solution}\relax
    \twocolImg{
    % \lstinputlisting[linerange={11-20}]{counterdom.tex}
    \inputminted[firstline=8, lastline=19]{latex}{sec/code/counters/counterdom.tex}
    }{counterdom}
\end{frame}



\begin{frame}[fragile]{Redefine existing counter domination\magicPage}{``equation'' example}\relax
    Package based solution:
    
    \lstinline|\usepackage{chngcntr}| 
    
    and \lstinline|\counterwith{equation}{chapter}|  to make the ``equation'' a slave or \lstinline|\counterwithout{equation}{chapter}| to ``free'' the counter.
    
    Core-based solution:

\begin{minted}{latex}
\makeatletter
\@removefromreset{equation}{section}
\@addtoreset{equation}{chapter}
\renewcommand{\theequation}{\thechapter.\@arabic\c@equation}
\makeatother
\end{minted}


    \skfootnote{\stExC{https://tex.stackexchange.com/questions/61756/how-to-change-equation-numbering-style} \stExC{https://tex.stackexchange.com/questions/54241/change-the-type-of-equation-numbering-in-document-class-article} \stExC{https://tex.stackexchange.com/questions/28333/continuous-v-per-chapter-section-numbering-of-figures-tables-and-other-docume} \normalfont\url{https://texfaq.org/FAQ-running-nos} \lvoc{IX.2.1}[293].\\ Also see \ccol{\p@} prefix \lvoc{IX.2.2}[295] and \stExC{https://tex.stackexchange.com/questions/61426/how-to-make-ref-display-only-subsection}}     
\end{frame}


\begin{frame}[fragile]{Define and simple manipulation\tW}\relax


     \textbf{Define new} \ccol\newcount\ccol{\<countname>} as \verb|\newcount\mycounter|
     
     \textbf{Set number} \ccol{\<countname>=<number>}  Or use \ccol\countdef. Like \verb|\countdef\mynumber=43|
     
     \textbf{Add number}  \ccol{\advance\string\<countname>\ by <number>}. Also there are \ccol\multiply\ and \ccol\divide. As well as \ccol\numexp.
     
     \textbf{Show number} \ccol{\the\string\<countname>} or \ccol\number\ or \ccol\romannumeral
     
     \skfootnote{\vspace{-3ex}\knuthc{15}[129] \stExC{https://tex.stackexchange.com/questions/245635/formal-syntax-rules-of-dimexpr-numexpr-glueexpr}\\Actually you can use \ccol\count<number> like \ccol\count212. What \ccol\newcount\ do is just find a free number and fix it to your defined name.}
\end{frame}

\begin{frame}{Define and simple manipulation\tW}{Example}\relax

    \twocolImg{
    % \lstinputlisting[linerange={11-16}]{counterTeX.tex}
        \inputminted[firstline=11, lastline=16]{latex}{sec/code/counters/counterTeX.tex}
    }{counterTeX}

\end{frame}

\graphicspath{{sec/images/}{sec/code/}}
\lstset{inputpath=sec/code/}

\subsection{Lengths}
\graphicspath{{sec02/images/}{sec02/code/}}
\lstset{inputpath=sec02/code/}

\begin{frame}[label=simple,fragile]{Simplest Beamer document}\relax
    \cprotect\twocolImg{\lstinputlisting{simplest01.tex}}{simplest01}
    \vspace{15mm}
    \inclassFrag{Try it! \hyperlink{style}{\beamerbutton{STYLE}} }[-1]
\end{frame}

\begin{frame}[fragile]{Beamer document structure}\relax
    \lstinputlisting[numbers=none]{structure.tex}
\end{frame}

\begin{frame}[fragile]{Title page (Preamble)}\relax
    \cprotect\twocolImg{\lstinputlisting[linerange={5-17}]{title.tex}}{title}
\end{frame}

\begin{frame}[fragile]{TOC}\relax
    \cprotect\twocolImg{\lstinputlisting[linerange={11-13}]{toc.tex}}{toc1}
\end{frame}

\begin{frame}[fragile]{TOC ({\tt AtBeginSection[]})}\relax
    \cprotect\twocolImg{\lstinputlisting[linerange={3-7}]{toc.tex}}{toc2}
\end{frame}


\begin{frame}[fragile]{Appendix}\relax
    \cprotect\twocolImg{\visible<2>{\lstinputlisting[linerange={4-4}]{appendix.tex}} \\
    \lstinputlisting[linerange={9-12}]{appendix.tex}}{appendix}
\end{frame}

\begin{frame}[fragile]{Bibliography (bibtex)}\relax
    \cprotect\twocolImg{\lstinputlisting[linerange={10-16}]{bibtex.tex}}{bibtex.pdf}
\end{frame}

\begin{frame}[fragile]{Bibliography (simple)}\relax
    \cprotect\twocolImg{\lstinputlisting[linerange={9-21}]{bibliography.tex}}{bibliography.pdf}
\end{frame}

\begin{frame}[fragile]{Frame: Columns}\relax
    \cprotect\twocolImg{\lstinputlisting[linerange={8-17}]{columns.tex}}{columns.pdf}
\end{frame}


\subsection{Boxes}
% \knuthc  knuth the TeXBook
% \lvoc   Lvovsky
% \lamc  lamport latex 
% \slshape different font for footnote
\graphicspath{{sec01/images/s2/}{sec01/code/s2/}}
\lstset{inputpath=sec01/code/s2/}

\begin{frame}[fragile]{How good it will be if...\only<2,3>{we could write like this}}\relax
\begin{center}
\only<1>{\includegraphics{refBegin}}
\only<2>{\includegraphics{refBegin2}}
\only<3>{\Huge We can!}
\end{center}

\cprotect\skfootnote{I use \verb|\only<2>{text}| for such effect. Also \verb|\setcounter| was used in source.}
\end{frame}

\begin{frame}[fragile]{Step 1: \ccol\label}\relax
     \lstinputlisting[linerange={12-15}, basicstyle=\tt]{refBegin2.tex}
     
     \skfootnote{for whole reference mechanism: \lmanc{7}[51] \lvoc{I.2.11}[27] \overC{https://www.overleaf.com/learn/latex/Cross_referencing_sections_and_equations}, \wikiC{https://en.wikibooks.org/wiki/LaTeX/Labels_and_Cross-referencing}}
\end{frame}

\begin{frame}[fragile]{Step 2: \ccol{\ref} and \ccol{\pageref}}
     \lstinputlisting[linerange={17-17}, basicstyle=\tt]{refBegin2.tex}
\end{frame}

\begin{frame}[fragile]{Problem 1: lots of labels!}
    What if you have a lot of labels?
    \inclass{\incPause Look throw whole document and compare \TeX\ and pdf?\incPause }
    
\cprotect\twocolImg{
    \lstinputlisting[linerange={10-10, 13-16}, basicstyle=\tt\small]{refBeginShow.tex}
}{refBeginShow}    
    
    
    \skfootnote{\url{http://ctan.altspu.ru/macros/latex/contrib/showlabels/showlabels.pdf}}
\end{frame}

\begin{frame}[fragile]{Problem 2: Typos}\relax
\cprotect\twocolImg{
    \lstinputlisting[linerange={12-17}, basicstyle=\tt\small]{refBad.tex}
}{refBad}           
\end{frame}

\begin{frame}[fragile]{Bibliography}{How to cite}\relax
Use \ccol{\cite\{label\}}

\cprotect\twocolImg{
    \lstinputlisting[linerange={9-9}, basicstyle=\tt\small]{citeFirst.tex}
}{citeFirst}  

\incPause
\cprotect\twocolImg{
    \lstinputlisting[linerange={11-12}, basicstyle=\tt\small]{citeSecond.tex}
}{citeSecond}  

\skfootnote{\lmanc{8.24.2}[94] \wikiC{https://en.wikibooks.org/wiki/LaTeX/Bibliography_Management} \overC{https://www.overleaf.com/learn/latex/Bibliography_management_in_LaTeX}}     
\end{frame}

\begin{frame}[fragile]{Bibliography}{What to cite}\relax

{\csk .bib} files

\begin{verbatim}
@Book{landau,
    author = {Landau, L. D. and Lifshitz, E. M.},
    title = {The Classical Theory of Fields},
    journal = N,
    volume = {1},
    pages = {140},
    year = 1980
}
\end{verbatim}

You can have multiple records in one .bib file.
     
\end{frame}

\begin{frame}[fragile]{Bibliography. Where can you get .bib files?\preMagicPage}\relax
\begin{itemize}
    \item Just google it! ``article\_name bibtex''
    \item Go to you favorite journal and look at Citations -> ``.bib'' or ``bibtex''
    \item Ask Mendeley or other programs to give you the .bib file
    \item Create it by yourself
\end{itemize}
\end{frame}

\begin{frame}[fragile]{Bibliography. Creating .bib file \magicPage}\relax
\begin{center}
\begin{tabular}{rl}
     \ccol{@article} & Journal or magazine article\\
\ccol{@book} & Book\\
\ccol{@conference} & Article in conference proceedings\\
\ccol{@misc} & If nothing else fits.
\end{tabular}
\end{center}
\skfootnote{\wikiC{https://en.wikibooks.org/wiki/LaTeX/Bibliography_Management\#Standard_templates} for cite url checkout \stExC{https://tex.stackexchange.com/questions/3587/how-can-i-use-bibtex-to-cite-a-web-page}
}     
\end{frame}

\begin{frame}[fragile]{Bibliography. Creating .bib file \magicPage}\relax
\begin{center}
{\obeylines
author
title
journal
year
pages
volume
...
}
\end{center}
\skfootnote{check for full list \wikiC{https://en.wikibooks.org/wiki/LaTeX/Bibliography_Management\#BibTeX}
}     
\end{frame}

\begin{frame}[fragile]{Bibliography. Offline\preMagicPage}\relax
Running \LaTeX\ offline, you can get \textbf{(??)} in \ccol{\ref} and \textbf{[?]} in \ccol{\cite}. 

For 
\begin{itemize}
    \item References 
    \item Bibliography
    \item Table of content
    \item Indexing
    \item ...
\end{itemize}

\LaTeX\ collect addition data in extra files. \LaTeX\ need more then one run to get this data. 

Use \verb|latex; bibtex; latex; latex|

\end{frame}

\begin{frame}[fragile]{Bibliography. Manually\magicPage}

You can add Bibliography manually.

\cprotect\twocolImg{
    \lstinputlisting[linerange={10-22}]{citeMan.tex}
}{citeMan}  

\skfootnote{\lmanc{8.24}[92]}
\end{frame}

\begin{frame}[fragile]{Bibliography. Styles\magicPage}

You can change styles.

Mannually -- check \url{https://en.wikibooks.org/wiki/LaTeX/Bibliography_Management}

Our with packages -- check \url{https://tex.stackexchange.com/questions/25701/bibtex-vs-biber-and-biblatex-vs-natbib}
\end{frame}

\begin{frame}[fragile]{Subject index\magicPage}\relax
\cprotect\twocolImg{
    \lstinputlisting[linerange={9-16}]{indexmy.tex}
}{indexmy}  

\skfootnote{\lvoc{IV.7}[175] \lmanc{25.2}[215]}     
\end{frame}
\subsection{Glue}
% \knuthc  knuth the TeXBook
% \lvoc   Lvovsky
% \lamc  lamport latex 
% \slshape different font for footnote
\graphicspath{{sec01/images/s3/}{sec01/code/s3/}}
\lstset{inputpath=sec01/code/s3/}

\begin{frame}[fragile]{\ccol{\footnote}\magicPage}\relax
\cprotect\twocolImg{
    \lstinputlisting[linerange={9-9}]{footnoteMy.tex}
}{footnoteMy}  


\skfootnote{\lmanc{8}[108] \wikiC{https://en.wikibooks.org/wiki/LaTeX/Footnotes_and_Margin_Notes} \overC{https://www.overleaf.com/learn/latex/Footnotes} \knuthc{15}[120]}
\end{frame}

\begin{frame}[fragile]{Horizontal aligment\magicPage}\relax
\cprotect\twocolImg{
    \lstinputlisting[linerange={9-25}]{flushing.tex}
}{flushing}  


\skfootnote{\lmanc{8.12}[63] \lmanc{8.13}[64] \knuthc{14}[112] and note \stExC{https://tex.stackexchange.com/questions/64644/centering-doesnt-seem-to-center-my-text}}
\end{frame}

\begin{frame}[fragile]{Page break\magicPage}\relax

{\centering \Large \ccol{\newpage} \ccol{\pagebreak}\par}

\skfootnote{\lmanc{10}[105] \stExC{https://tex.stackexchange.com/questions/736/pagebreak-vs-newpage}}
     
\end{frame}

\begin{frame}[fragile]{Quotes\magicPage}\relax

\cprotect\twocolImg{
    \lstinputlisting[linerange={10-14}]{quotemy.tex}
}{quotemy}  


\cprotect\skfootnote{\lvoc{III.7.1}[129] \lmanc{8.20}[83]. Also see \verb|quotation| env }
     
\end{frame}

\begin{frame}[fragile]{Verses\magicPage}\relax

\cprotect\twocolImg{
    \lstinputlisting[linerange={11-18}]{versemy.tex}
}{versemy}  

\skfootnote{\lvoc{III.7.3}[130] \lmanc{8.28}[98]}
\end{frame}

\begin{frame}[fragile]{Marginal notes\magicPage}\relax

\cprotect\twocolImg{
    \lstinputlisting[linerange={9-10}]{marginmy.tex}
}{marginmy}  

\skfootnote{\lvoc{IV.10}[194] \lmanc{15.4}[137]}
\end{frame}

 
\cprotect\inclassFrame{
\begin{frame}[fragile]{Try to add a package}\relax
\url{http://hanno-rein.de/downloads/coffee.sty}

and use \verb|\newpage \coffee{2} hello world!|

\incPause
NOW you document is ready!

\end{frame}
}
\subsection{Spaces}
\graphicspath{{sec/images/}{sec/code/}}
\lstset{inputpath=sec/code/}

% phantom, smash
\begin{frame}[fragile]{What spaces we have?}
     \def\showLength#1{\newbox\boxtodimen%
\newdimen\wb%
\setbox\boxtodimen=\hbox{#1}%
\wb=\the\wd\boxtodimen%
     \leavevmode\raise4pt\hbox{\vrule height 6pt depth2pt{\csk\rule{\wb}{4pt}}\vrule height 6pt depth2pt}%
     }
     
    \begin{columns}
    \begin{column}{0.45\textwidth}
    
\centering\Large
     \begin{tabular}{rl}
          \ccol\qquad& \showLength{\qquad}\\ 
          \ccol\quad& \showLength{\quad}\\ 
          \ccol\enspace & \showLength\enspace\\
          \ccol\  & \showLength{\ }\\
     \end{tabular}
     
     \ccol\quad\ is used below:
     $$x=y\quad\hbox{if y=0}$$
    \end{column}
    \begin{column}{0.45\textwidth}
    
\centering\Large
     \begin{tabular}{rl}
          \multicolumn{2}{c}{useful for math:}\\
          \ccol{\;} & \showLength{\;}\\
          \ccol{\>} & \showLength{$\>$}\\
          \ccol{\,} & \showLength{\,}\\
          \ccol{\!} & \showLength{\!}\\
     \end{tabular}
     
     the last one is negative space
    \end{column}
    \end{columns}
     
     
     \skfootnote{\knuthc{18}[178] \lmanc{19.1}[167]}
\end{frame}

\begin{frame}{Phantoms}\relax
    
    Phantoms have the same size as it's an argument without drawing.\inpause 
    
    \twocolImg{
    % \lstinputlisting[linerange={39-45}]{phantomsmy.tex}
    \inputminted[firstline=39, lastline=45]{latex}{sec/code/phantomsmy.tex}
    }{phantomsmy}
    \ccol\phantom\ leaves both dimentions. \ccol\hphantom\ and \ccol\vphantom\ leaves only one dimention.
    
    \ccol\strut\ is short for \ccol\vphantom\{(\}
    
    \ccol\smash\ is using to leave only the horizontal coordinate of a formula
    
    \skfootnote{\lmanc{16.6}[156] \knuthc{18}[178]}
     
\end{frame}
\subsection{Toks}
\begin{frame}{Toks manipulation\tW\magicPage}\relax
    \TeX\ has addition registers for storing strings. They have \ccol\toks\ prefix/suffix. The difference between toks and storage inside macros are in extension (We mention the extension mechanism at the last lecture). In toks \TeX store tokens (unexpanded).
     
     \skfootnote{\stExC{https://tex.stackexchange.com/questions/455208/token-register-vs-macro-register}}
\end{frame}
\subsection{Inserts}
\begin{frame}{Inserts\tW\magicPage}\relax

    \TeX\ has addition registers for storing floats. They have \ccol\insert\ prefix/suffix. 
     
     \skfootnote{\knuthc{15}[132] \stExC{https://tex.stackexchange.com/questions/527613/simple-example-of-using-newinsert-and-friends}\\ But I failed to make a working example...}
\end{frame}


\inclassframe{

\begin{frame}{\exFrame{Make footnote center}}{lvl 1}\relax
    Update the code of footnote generation from previous exercise and make the following change:
    \begin{itemize}
        \item Make a spaces between frame number, footnote an year (see next frame)
        \item Please, use only glue commands
         
    \end{itemize}
\end{frame}

\begin{frame}[t]{\exFrame{Implement footnotes}}{lvl 1}\relax
    \centering
    \vspace*{-1cm}
    \fbox{\includegraphics[scale=0.8]{exLvl1.pdf}}
\end{frame}

\begin{frame}{\exFrame{Make footnote center}}{lvl 2}\relax
     Update the previous template. 
     \begin{itemize}
         \item Remove year information from the right
         \item Make the footnote appears \textit{exactly} at the center independing of how width are slide number infrormation
         \item Check one and multiline footnote 
         \item You can use all build-in possibilities
     \end{itemize}
\end{frame}

}



\progressend


\begin{frame}\frametitle{What we have learned today?}\relax
    \tableofcontents
\end{frame}

\begin{frame}[allowframebreaks]{references}
color from the footnotes corresponds to references' color.
    \begin{itemize}
        \item \knuthc{Knuth ``The \TeX Book''}
        \item \lvoc{L'vovsky ``Nabor i verstka v sisteme \LaTeX''}
        \item \lamc{Lamport. ``\LaTeX. A Document Preparation System, User’s Guide and Reference Manual''}
        \item \lmanc{``\LaTeX 2e: An unofficial reference manual''} also at website \url{https://latexref.xyz/}
        \item \stExC{https://tex.stackexchange.com/questions} : \url{https://tex.stackexchange.com/questions}
        \item \wikiC{https://en.wikibooks.org/wiki/LaTeX} : \url{https://en.wikibooks.org/wiki/LaTeX}
        \item \overC{https://www.overleaf.com/learn/latex} : \url{https://www.overleaf.com/learn/latex}
        \item \tugC{https://www.tug.org/utilities/plain/cseq.html} : \url{https://www.tug.org/utilities/plain/cseq.html}
    \end{itemize}
\end{frame}

\begin{frame}{Distribution}\relax
\begin{itemize}
     \item the pdf-version of the presentation and all printed materials can be distributed under license Creative Commons Attribution-ShareAlike 4.0 \url{https://creativecommons.org/licenses/by-sa/4.0/}
     \item The source code of the presentation is available on {\csk\url{https://github.com/Lavton/latexLectures}} and can be distributed under the MIT license \url{https://en.wikipedia.org/wiki/MIT_License\#License_terms}
\end{itemize}
     
\end{frame}
\end{document}