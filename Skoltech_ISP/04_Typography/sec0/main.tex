\graphicspath{{sec0/images/}{sec0/code/}}
\lstset{inputpath=sec0/code/}

\section{Introduction}

\begin{frame}{Separation ``how'' and ``what''}\relax

    \outclasshigh{This is \textit{the} point of non-WYSiWYG}
    \begin{itemize}
        \item Focus on what you want to do, not how to do it. \outclass{Leave the creation of ``how'' to someone else}\inpause
        \item But \textit{who} create these ``how'' rules?\inpause
        \item Who needed to :). In these last lectures it will be you
         
    \end{itemize}
     
\end{frame}

\begin{frame}[fragile]{Ugly ``backend'', beautiful ``frontend''}\relax

    \outclasshigh{
    \begin{itemize}
        \item \TeX\ is a very old language
        \item it allows you so many things, redefine almost anything...
        \item but its syntax may be unclear
    \end{itemize}
    }
    \begin{itemize}
        \item Why figures and footnotes don't rendered where they appears, but sent where they must be?
        \item Why can you write \verb|\pictofraction{xx}{blue}{3}{black!30}{3}{\tiny}| and the \string\tiny\ size will appears in the command result, and not {\tiny just change the following text?}
        \item Why can I use something like \ccol\magicPage\ command and don't change the template itself? (And why now you see the command, not its result?) 
    \end{itemize}
\end{frame}

{
\supressprogressbarfalse

\begin{frame}[fragile]{Examples\preMagicPage}\relax

I need to write as simple as
\begin{itemize}
    \item \ccol\magicPage\ to create a ``magic hat''
    \item \ccol\twocolImg\verb|{<anything>}{image}| to create the pattern ``code to the left, picture to the right''
    \item \ccol\overC\verb|{https://www.overleaf.com/learn/latex/Page_size_and_margins}| to shorten this (with hyperref!) to \overC{https://www.overleaf.com/learn/latex/Page_size_and_margins}
    \item no addition code for the progressbar at the top, except \verb|\usepackage{progressbar}| and define \texttt{progressbarcolor}
     
\end{itemize}
\inpause but the tricks I use... 
     
\end{frame}

\begin{frame}[fragile, allowframebreaks]{``Backend'' of the commands\magicPage}\relax
    ``Magic hat'':
    \inputminted[firstline=194, lastline=200, fontsize=\tiny]{latex}{sec0/code/latexLectures.sty}
    
    Code and image pattern:
    \inputminted[firstline=55, lastline=70, fontsize=\tiny]{latex}{sec0/code/latexLectures.sty}
    
    Overleaf short view:
    \inputminted[firstline=311, lastline=313, fontsize=\tiny]{latex}{sec0/code/latexLectures.sty}
    
    Progressbar:
    \inputminted[fontsize=\tiny]{latex}{sec0/code/progressbar.sty}
\end{frame}

}

\begin{frame}\relax

\centering\large We will learn how to do things like the above 
     
\end{frame}

\begin{frame}{\LaTeX\ vs \TeX}\relax
    \begin{columns} 
    \begin{column}{0.5\textwidth}
    \begin{tikzpicture}[sibling distance=10em,
      every node/.style = {shape=rectangle, rounded corners,
        draw, align=center,
        top color=white, bottom color=skoltechgreen!20}]]
      \node {\TeX}
        child { node {\LaTeX}
          child { node {XeLaTeX} }
          child { node {LuaTeX} }
          } ;
    \end{tikzpicture}
    \end{column}
    
    \begin{column}{0.5\textwidth}
    \small
    \begin{itemize}
        \item {\csk \LaTeX} --- is the most popular set of macro-extensions (or macro package) of the computer typesetting system \TeX, which facilitates a typesetting of complex documents.
        \item {\csk \TeX} --- is a typesetting system designed and mostly written by Donald Knuth --- the ``father of modern Computer Science''. TeX was designed with two main goals in mind: to allow anybody to produce high-quality books using minimal effort and to provide a system that would give exactly the same results on all computers, at any point in time
         
    \end{itemize}
    \end{column}
    \end{columns}
    \outclasshigh{
    Now it is become important to separate the \LaTeX\ ideas from \TeX's ones
    }
\end{frame}