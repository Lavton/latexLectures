\begin{frame}{Fonts classification}\relax
    % \begin{columns}
    %     \begin{column}{0.5\textwidth}
    %          Serif
    %     \end{column}
    %     \begin{column}{0.5\textwidth}
    %          San Serif
    %     \end{column}
         
    % \end{columns}
    \begin{itemize}
        \item {\csk\scshape Serif} --- for long texts, books,..  \strut\\  \includegraphics[width=0.9\textwidth]{fserif}
        \item {\csk\scshape Sans Serif} --- for short texts, titles, presentations,..  \strut\\  \includegraphics[width=0.9\textwidth]{fsunserif}
        \item {\csk\scshape Typewriter} --- emulate typewriter, write code and commands   \strut\\  \includegraphics[width=0.9\textwidth]{ftt}
        \item {\csk\scshape Other} --- Decoration etc \strut  \\  \includegraphics[width=0.9\textwidth]{fcali}
         
    \end{itemize}
    
    \skfootnote{ \url{http://www.tug.dk/FontCatalogue/accanthis/} \url{http://www.tug.dk/FontCatalogue/arev/} \url{http://www.tug.dk/FontCatalogue/ascii/} \url{http://www.tug.dk/FontCatalogue/janaskrivana/}}
\end{frame}

\begin{frame}{\TeX nical classification}\relax

you have: 

\centering
standard pdf\LaTeX\ engine with ``METAFONT'' fonts:
\vspace{0.2cm}

{\footnotesize
\begin{tabular}{c|c}
\parbox{5cm}{The package has \textbf{global} usage out-of-the-box\\ you want to use it \textbf{globally}} & \parbox{5cm}{The package has \textbf{only global} usage out-of-the-box\\ you want to use it \textbf{locally}} \\\hline
\parbox{5cm}{The package has \textbf{only local} usage out-of-the-box\\ you want to use it \textbf{globally}} & \parbox{5cm}{The package has \textbf{local} usage out-of-the-box\\ you want to use it \textbf{locally}} \\
\end{tabular}}

\vspace{0.5cm}
\XeLaTeX\ with support of system-installed fonts:
\vspace{0.2cm}

{\footnotesize
\begin{tabular}{c|c}
\parbox{5cm}{The font is \textbf{global}\\ you want to use it \textbf{globally}} & \parbox{5cm}{The font is \textbf{global}\\ you want to use in \textbf{locally}} \\\hline
\parbox{5cm}{The font is \textbf{local}\\ you want to use in \textbf{globally}} & \parbox{5cm}{tThe font is \textbf{local}\\ you want to use in \textbf{locally}} \\
\end{tabular}}
\end{frame}

\begin{frame}
     \centering\Huge pdf\LaTeX
     
\end{frame}


\begin{frame}[fragile]{Global Font usage throw package with pdf\LaTeX}{where to find a font}\relax

\begin{itemize}
    \item { \url{https://tug.org/FontCatalogue/allfonts.html}}
    \item { \url{https://www.ctan.org/tex-archive/fonts}}
    
\end{itemize}
\inclassFrag{Go to the first website and check some fonts!}

then just follow the instructions for the package

\end{frame}

\begin{frame}[fragile]{Fonts usage\preMagicPage}{notation}\relax

    You may have something like this in logs:\\ 
    {\csk \verb|LaTeX Font Warning: Font shape `T1/calligra/bx/n' undefined|}
    
    Or something like {\csk\verb|OT1/cmr/m/n/10|}
    
    How to read it?
    \strut
    \inpause
    
    \centering 
    \begin{tabular}{c|c|c|c|c}
    T1&calligra&bx&n&\\ 
    OT1&cmr&m&it&10\\\hline
    \large encoding & font family & series & shape & font size
    \end{tabular}
    \skfootnote{\normalfont \url{http://mirrors.ctan.org/macros/latex/doc/fntguide.pdf}: 2.1, \normalfont\url{http://mirrors.ctan.org/macros/latex/doc/encguide.pdf}}
\end{frame}

\begin{frame}[fragile]{Fonts usage\magicPage}{notation}\relax

    \begin{columns}[t]
    \begin{column}{0.5\textwidth}
    
    Most common encodings\strut\hrule
    
    \begin{tabular}{rl}
         OT1 & TEX text\\
         T1 & TEX extended text\\
         OML & TEX math italic\\
         OMS & TEX math symbols\\ 
         OMX & TEX math large symbols\\
         U & Unknown\\ 
         L<xx>& local encoding
    \end{tabular}
    \end{column}
    
    \begin{column}{0.5\textwidth}
    
    Some common families\strut\hrule
    
    \begin{tabular}{r>{\footnotesize}l}
    
         cmr & Computer Modern Roman\\ 
         cmss & Computer Modern Sans\\ 
         cmtt & Computer Modern Typewriter\\ 
         cmm & Computer Modern Math Italic\\ 
         cmsy & Computer Modern Math Symbols\\ 
         cmex &  Computer Modern Math Extensions\\ 
         ptm & Adobe Times\\ 
         phv & Adobe Helvetica\\ 
         pcr &Adobe Courier
    \end{tabular}
    \end{column}
         
    \end{columns}
    \cprotect\skfootnote{more families: \overC{https://www.overleaf.com/learn/latex/Font_typefaces} \normalfont \url{http://mirrors.ctan.org/macros/latex/doc/fntguide.pdf}: 2.1\\
    Look at the source code to understand how the right table was created (\verb|>{\footnotesize}| and \verb|\usepackage{array}|)
    }
\end{frame}


\begin{frame}[fragile]{Fonts usage\magicPage}{notation}\relax
    \centering
    \begin{columns}[t]
    \begin{column}{0.5\textwidth}
    \centering
    
    Most common values for series\strut\hrule
    
    \begin{tabular}{rl}
         t & thin\\
         m &Medium\\ 
         b &Bold \\
         bx & Bold extended\\ 
         sb &Semi-bold\\ 
         c & Condensed
    \end{tabular}
    \end{column}
    
    \begin{column}{0.5\textwidth}
    
    Most common values for shape\strut\hrule
    
    \begin{tabular}{rl}
    n & Normal \scriptsize(that is 'upright' or 'roman') \\ 
    it & Italic\\  
    sl & Slanted \scriptsize(or 'oblique') \\ 
    sc &Caps and small caps
    \end{tabular}
    \end{column}
         
    \end{columns}
    \cprotect\skfootnote{ \normalfont \url{http://mirrors.ctan.org/macros/latex/doc/fntguide.pdf}: 2.1
    }
\end{frame}


\begin{frame}[t, fragile]{Fonts usage\magicPage}{default}\relax

% \myminted{sec02/code/fnts/fontpdfdef.tex}{2}{2}
    \samePosPictureI{sec02/code}{fnts/fontpdfdef}{1-2,9-10,25-25,30-30,34-34,38-38,42-42}

% \samePosPicture{fnts/fontpdfdef}{1-2,9-10,25-25,30-30,34-34,38-38,42-42}

\ncol\usepackage{fontenc}

\end{frame}

\begin{frame}

\magicPage
     \centering\huge Global Font usage throw package with pdf\LaTeX, when the package is constructed to change defaults
     
\end{frame}

%http://ctan.altspu.ru/macros/latex/doc/encguide.pdf
%https://tex.stackexchange.com/questions/25249/how-do-i-use-a-particular-font-for-a-small-section-of-text-in-my-document
\begin{frame}[t, fragile]{Fonts usage\toFrameTop{pdfLaTeX}\magicPage}{Global font by loading package}\relax

\samePosPictureI{sec02/code}{fnts/fontpdfgl01}{8-10,25-25,30-30,34-34,38-38,42-42}\relax

\ncol\usepackage{<fontPackage>}\\
\Oncol\usepackage[T1]{fontenc}

     
\end{frame}

\begin{frame}[t, fragile]{Fonts usage\magicPage}{Global font by loading package}\relax

\samePosPictureI{sec02/code}{fnts/fontpdfgl02}{8-10,25-25,30-30,34-34,38-38,42-42}\relax

\ncol\usepackage{<fontPackage>}\\
\Oncol\usepackage[LY1]{fontenc}
\end{frame}

\begin{frame}[t, fragile]{Fonts usage\magicPage}{Global font by loading package}\relax

\samePosPictureI{sec02/code}{fnts/fontpdfgl03}{7-10,25-25,30-30,34-34,38-38,42-42}\relax

\end{frame}

\begin{frame}

\magicPage
     \centering\huge Local font usage throw package with pdf\LaTeX, when the package is constructed to use locally
\end{frame}

\begin{frame}[t, fragile]{Fonts usage\magicPage}{Local font by loading package}\relax

\samePosPictureI{sec02/code}{fnts/fontpdflcl01}{8-10,24-25}\relax

All that is done here and bellow is just follow \url{http://www.tug.dk/FontCatalogue/allfonts.html}

\end{frame}

\begin{frame}[t, fragile]{Fonts usage\magicPage}{Local font by loading package}\relax

\samePosPictureI{sec02/code}{fnts/fontpdflcl03}{7-10,24-25,29-29}\relax

There are some beautiful fonts!

\end{frame}

% \inclassframe{
% \begin{frame}{Try it out!}\relax

% Go to check the font \url{http://www.tug.dk/FontCatalogue/calligra/}

% apply the font to the text (by example --- the following document)

% \inputminted{latex}{sec02/code/fnts/fontpdfTASK.tex}

% after you succeeded --- add \ccol\bfseries\ to make the font bold
     
% \end{frame}
% }

\begin{frame}[t, fragile]{Fonts usage \magicPage}{Local font by loading package}\relax

\samePosPictureI{sec02/code}{fnts/fontpdflcl02}{8-10,24-25, 31-31, 36-36}\relax

\end{frame}


\begin{frame}
\magicPage
     \centering\huge Global font usage throw package with pdf\LaTeX, when the package is constructed to use locally
     
\end{frame}

\begin{frame}[fragile]{Algorithm\magicPage}\relax

    You need to figuraute the {\csk Font Family}
    \begin{enumerate}
         \item Check the package documentation
         \item (Remember, that not all fonts provide all series and shapes!)
         \item If manual is unreachable, get the Family directly: \ccol{\showthe}\ccol{\font} and see {\csk logs}:
    \end{enumerate}
    
    \logshow{fnts/fontpdfNONDEFgl01}{8-10, 19-21}{1-4}
    
    \begin{enumerate}
        \setcounter{enumi}{3}
        \item get the family ({\csk hmin}) and use it! (next slide)
         
    \end{enumerate}
    \skfootnote{\stExC{https://tex.stackexchange.com/questions/25249/how-do-i-use-a-particular-font-for-a-small-section-of-text-in-my-document/25251\#25251}}     
\end{frame}

\begin{frame}[t, fragile]{Fonts usage\magicPage}{global}\relax
    
    \samePosPictureI{sec02/code}{fnts/fontpdfNONDEFgl01}{8-10,15-15,23-23,27-27,31-31}\relax
    
    \verb|\renewcommand{|\ccol{\rmdefault}\verb|}<family_name>|
    
    \skfootnote{\stExC{https://tex.stackexchange.com/questions/25249/how-do-i-use-a-particular-font-for-a-small-section-of-text-in-my-document/25251\#25251} \normalfont \url{http://mirrors.ctan.org/macros/latex/doc/fntguide.pdf}: 2.4}
\end{frame}

\begin{frame}

\magicPage
     \centering\huge Local font usage throw package with pdf\LaTeX, when the package is constructed to change defaults
     
\end{frame}

\begin{frame}[fragile]{Algorithm\magicPage}\relax

    You need to figuraute the {\csk Font Family}
    \begin{enumerate}
         \item Check the package documentation
         \item If manual is unreachable, get the Family directly:
         \ccol{\rmdefault} or \ccol{\familydefault}
    \end{enumerate}
    
    \samePosPictureI{sec02/code}{fnts/fontpdfNONDEFlcl01}{8-10, 17-18}
    
    \begin{enumerate}
        \setcounter{enumi}{2}
        \item remember the family ({\csk ybd2j}) to use it (next slide)
         
    \end{enumerate}

\end{frame}

\begin{frame}[fragile]{Algorithm\magicPage}\relax

    \samePosPictureI{sec02/code}{fnts/fontpdfNONDEFlcl02}{8-10, 15-16, 22-23}
    
    \begin{enumerate}
        \setcounter{enumi}{3}
        \item Change the encoding and font family to defaults (\verb|\renewcommand{|\ccol\encodingdefault\verb|}{|\ccol{OT1}\verb|}|, \verb|\renewcommand{|\ccol\rmdefault\verb|}{|\ccol{cmr}\verb|}|)
         
    \end{enumerate}
    \skfootnote{\stExC{https://tex.stackexchange.com/questions/25249/how-do-i-use-a-particular-font-for-a-small-section-of-text-in-my-document/25251\#25251} \normalfont \url{http://mirrors.ctan.org/macros/latex/doc/fntguide.pdf}: 2.2}
\end{frame}

\begin{frame}[t, fragile]{Fonts usage\magicPage}{locally}\relax
    How to change font:
    \begin{itemize}
        \item \ccol\fontencoding\ will change the encoding
        \item \ccol\fontfamily\ will change family
        \item \ccol\fontseries\ wil change series 
        \item \ccol\fontshape\ will change shape 
        \item \ccol\fontsize\ will change font size 
    \end{itemize}
    ... and {\large \ccol\selectfont}\ after font change!
    
    \inclassFrag{why do we need the ending-command like \ccol\selectfont ?}
    We need \ccol{\selectfont} because while changing the font we can be in an inconsistent state: for example, we change the encoding, but now there is no such family as an old one!
    
    \skfootnote{\stExC{https://tex.stackexchange.com/questions/25249/how-do-i-use-a-particular-font-for-a-small-section-of-text-in-my-document/25251\#25251} \normalfont \url{http://mirrors.ctan.org/macros/latex/doc/fntguide.pdf}: 2.2}
\end{frame}

\begin{frame}

    \magicPage
     \centering\Huge \XeLaTeX\ and Lua\TeX
     
\end{frame}

\begin{frame}{Fonts usage\magicPage}{XeLaTeX and LuaTeX}\relax
     \begin{enumerate}
         \item You can use practically all fonts from pdf\LaTeX
         \item You can use OpenType (OTF), TrueType (TTF) fonts. They usually install in your system.
          
     \end{enumerate}
\end{frame}

\begin{frame}

    \magicPage
     \centering\huge Global font usage throw global-available font with \XeLaTeX
     
\end{frame}

\begin{frame}[fragile]{Font usage\magicPage}{default}\relax

    \samePosPictureI{sec02/code}{fnts/fontXedef}{9-9,24-24,29-29,33-33,37-37,41-41}
    
    \verb|\usepackage{|\ccol{fontspec}\verb|}|
\end{frame}

\begin{frame}[fragile]{Font usage\magicPage}{global}\relax

    \samePosPictureI{sec02/code}{fnts/fontXegl01}{7-8,24-24,29-29,33-33,37-37,41-41}
    
    \ccol\setmainfont<font-name>
    \skfootnote{\wikiC{https://en.wikibooks.org/wiki/LaTeX/Fonts\#Using_TTF_and_OTF_fonts} \overC{https://www.overleaf.com/learn/latex/XeLaTeX} \stExC{https://tex.stackexchange.com/questions/25249/how-do-i-use-a-particular-font-for-a-small-section-of-text-in-my-document/37251\#37251}}
\end{frame}

\begin{frame}[fragile]{Font usage\magicPage}{global}\relax

    \samePosPictureI{sec02/code}{fnts/fontXegl02}{7-8,24-24,29-29,33-33,37-37,41-41}
    \ccol\setmainfont<font-name>
    
    \skfootnote{\wikiC{https://en.wikibooks.org/wiki/LaTeX/Fonts\#Using_TTF_and_OTF_fonts} \overC{https://www.overleaf.com/learn/latex/XeLaTeX} \stExC{https://tex.stackexchange.com/questions/25249/how-do-i-use-a-particular-font-for-a-small-section-of-text-in-my-document/37251\#37251}}
\end{frame}


\begin{frame}[fragile]{Font usage\magicPage}{global}\relax
    \begin{itemize}
        \item \ccol\setmainfont\ sets the roman font
        \item \ccol\setsansfont\ sets the sans font
        \item \ccol\setmonofont\ sets the monospace font 
         
    \end{itemize}
    
    \skfootnote{\wikiC{https://en.wikibooks.org/wiki/LaTeX/Fonts\#Using_TTF_and_OTF_fonts} \overC{https://www.overleaf.com/learn/latex/XeLaTeX} \stExC{https://tex.stackexchange.com/questions/25249/how-do-i-use-a-particular-font-for-a-small-section-of-text-in-my-document/37251\#37251}}
\end{frame}

\begin{frame}

\magicPage
     \centering\huge Local font usage throw global-avaliable font with \XeLaTeX
\end{frame}

\begin{frame}[fragile]{Font usage\magicPage}{local}\relax

    \samePosPictureI{sec02/code}{fnts/fontXelcl01}{8-8,23-24,33-34}
    \ccol\fontspec<font-name>
    
    \skfootnote{\stExC{https://tex.stackexchange.com/questions/25249/how-do-i-use-a-particular-font-for-a-small-section-of-text-in-my-document/37251\#37251}}
\end{frame}

\begin{frame}[fragile]{Font usage\magicPage}{local}\relax

    \samePosPictureI{sec02/code}{fnts/fontXelcl02}{8-8,18-18,23-24,33-34}
    \ccol\newfontfamily\ --- more effective way
    
    \skfootnote{\stExC{https://tex.stackexchange.com/questions/25249/how-do-i-use-a-particular-font-for-a-small-section-of-text-in-my-document/37251\#37251}}
\end{frame}

\begin{frame}

\magicPage
     \centering\huge Global font usage throw local-avaliable font with \XeLaTeX
\end{frame}
\begin{frame}[fragile]{Font usage\magicPage}{global}\relax

    \samePosPictureI{sec02/code}{fnts/fontXeNNgl01}{7-8,24-24,29-29,33-33,37-37,41-41}
    \ccol\setmainfont<font-filename>
    
    \skfootnote{\overC{https://www.overleaf.com/learn/latex/XeLaTeX} \\
from \url{https://www.fontsquirrel.com/fonts/lato}}
\end{frame}

\begin{frame}[fragile]{fontspec's commands optional params\magicPage}\relax
    \begin{itemize}
        \item BoldFont = ⟨font name⟩
    \item ItalicFont = ⟨font name⟩
    \item BoldItalicFont = ⟨font name⟩
    \item SlantedFont = ⟨font name⟩
    \item BoldSlantedFont = ⟨font name⟩
    \item SmallCapsFont = ⟨font name⟩
    \item UprightFont = ⟨font name⟩
    \end{itemize}
     
     \skfootnote{\url{http://mirrors.ctan.org/macros/latex/contrib/fontspec/fontspec.pdf}: 3.1}
\end{frame}

\begin{frame}

    \magicPage
     \centering\huge Local font usage throw local-avaliable font with \XeLaTeX
\end{frame}

\begin{frame}[fragile]{Font usage\magicPage}{local}\relax

    \samePosPictureI{sec02/code}{fnts/fontXeNNlcl01}{8-8,23-24,33-34}
    \ccol\fontspec<font-name>

\end{frame}

\begin{frame}[fragile]{Font usage\magicPage\preMagicPage}{local}\relax

    \samePosPictureI{sec02/code}{fnts/fontXeNNlcl02}{8-8,18-18,23-24,33-34}
    \ccol\newfontfamily\ --- more effective way


\end{frame}

\begin{frame}{How to find a font}\relax
\centering
     \includegraphics[height=0.8\textheight]{choosing-fonts}
     
     \skfootnote{\url{https://www.labnol.org/home/choose-fonts-with-flowchart/13488/}}
\end{frame}

\begin{frame}{Useful links}\relax
\begin{itemize}
    \item {\csk \url{https://tug.org/FontCatalogue/allfonts.html}} avaliable at \LaTeX\ fonts 
    \item {\csk \url{https://www.fontsquirrel.com/}} font catalogue 
    \item {\csk \url{https://www.fontsquirrel.com/matcherator}} identify font by picture
    \item {\csk \url{https://www.fonts-online.ru/fonts/russian}} fonts with cyrillic 
    \item {\csk \url{http://allfont.ru/free/}} fonts with cyrillic 
    \item {\csk \url{https://fonts.google.com/?subset=cyrillic}} fonts with cyrillic
    \item {\csk \url{https://wordmark.it/}} quick look of how your text will look like
     
\end{itemize}
\end{frame}

\begin{frame}{Useful tips: Font pairs}\relax
    Don't use too many fonts in your document! The best choice is two-three different fonts.
    
    How to choose font pairs? 
    
    \begin{itemize}
        \item \url{https://www.fastprint.co.uk/blog/the-art-of-mixing-typefaces.html} cheat list
        \item \url{https://www.canva.com/font-combinations/} combinator
        \item \url{https://fontpair.co/} list 
        \item \url{http://font-combinator.com/} combinator
        \item \url{http://www.joustmultimedia.com/blog/post/the-art-of-combining-fonts} some tips
         
    \end{itemize}
     \skfootnote{from \url{https://freelance.today/poleznoe/15-servisov-dlya-podbora-shriftovyh-par.html}}
\end{frame}

\inclassframe{

\begin{frame}{\exFrame{Change the font!}}{lvl 1}\relax
     
     Write some text with italic and bold. Change the font globally (in whole document) using global-avaliable font ``Times New Roman''.
\end{frame}

\begin{frame}{\exFrame{Change the font!}}{lvl 2}\relax
     
     Change the font locally (in parts of document) using local-avaliable font .ttf (don't forget to upload it)
\end{frame}

}