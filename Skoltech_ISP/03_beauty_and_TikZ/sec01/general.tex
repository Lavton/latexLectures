\begin{frame}{Path}
     
     Path is the main TikZ essence.
    \inputminted[firstline=8, lastline=8]{latex}{sec01/code/pathmy.tex}
    \inputminted[firstline= 11, lastline=13]{latex}{sec01/code/pathmy.tex}


\inclassFrag{what do you think ``(0,0)'', ``(0,1)'' for?}

``(0,0)'', ``(0,1)'' is the simplest coordinate assignment. The (x, y) coordinate in units (typically 1cm)
The \ccol\path\ is not draw anything by itself!
\end{frame}

{\forcewidefootnote=1
\begin{frame}[fragile]{Draw}\relax

    \twocolImg{
    % \lstinputlisting[linerange={8-8, 14-16}]{drawmy1.tex}
    \only<2>{\begin{tikzpicture}[overlay]
            \draw[->, thick] (6.3, -1.2) to[out=0, in=180] (10, 0.3);
            \draw[->, thick] (6.3, -1.9) to[out=0, in=180] (10, -0.7);
        \end{tikzpicture}}
    \inputminted[firstline=9, lastline=9]{latex}{sec01/code/draw_combined.tex}
    \inputminted[firstline= 15, lastline=22]{latex}{sec01/code/draw_combined.tex}
    }{draw_combined}
    
     \ccol{\draw} $\equiv$ \ocol\path[draw],\\ \ccol{\fill} $\equiv$ \ocol\path[fill]\\ etc

\end{frame}
}


\begin{frame}{Nodes}\relax

    \twocolImg{
    \inputminted[firstline=9, lastline=9]{latex}{sec01/code/nodemy.tex}
    \inputminted[firstline= 15, lastline=17]{latex}{sec01/code/nodemy.tex}
    }{nodemy}
    
    \ccol\node\ or \ocol\path[node]. Without optional arguments a node has no border.\\[1cm]
    
    \inpause Nodes are important for graph creation
     
\end{frame}

% \inclassframe{
% \begin{frame}[fragile]{Draw a snowman!}\relax

% \begin{columns}
% \begin{column}{0.45\textwidth}
%      \centering
%     \begin{tikzpicture}
%         \draw (0, 0) circle[radius=1.5cm];
%         \draw (0, 2.4) circle[radius=1cm];
%         \draw (0, 4) circle[radius=0.7cm];
%         \filldraw[fill=gray] (-0.5, 4.5) rectangle (0.5, 5.5);
%         \filldraw[fill=orange, draw=orange!50!red] (0.3, 4.1) -- (1.5, 4) -- (0.3, 3.9) -- cycle;
%         \draw[very thick, ->] (-4, 4.2) -- (-1, 3); 
%         \node at (-4, 4.5) {snowman!};
%         \path (1, 3) -- (4, 5);
%     \end{tikzpicture}
% \end{column}

% \begin{column}{0.45\textwidth}
% \expandafter{\begin{minted}{latex}
% \begin{tikzpicture}
%     \draw (0, 0) circle[radius=1.5cm];
%     \draw (0, 2.4) circle[radius=1cm];
%     \draw (0, 4) circle[radius=0.7cm];
%     \filldraw[fill=gray] (-0.5, 4.5) rectangle (0.5, 5.5);
%     \filldraw[fill=orange, draw=orange!50!red] (0.3, 4.1) -- (1.5, 4) -- (0.3, 3.9) -- cycle;
%     \draw[very thick, ->] (-4, 4.2) -- (-1, 3); 
%     \node at (-4, 4.5) {snowman!};
%     \path (1, 3) -- (4, 5);
% \end{tikzpicture}
% \end{minted}}
% \end{column}
% \end{columns}

% \end{frame}
% }

% сократить часть про координаты, оставив node
\begin{frame}[fragile]{``Node''-based coordinates \preMagicPage}\relax
    \twocolImg{
    \inputminted[firstline=15, lastline=22]{latex}{sec01/code/coordinatesNode.tex}%
    \only<1>{\begin{tikzpicture}[overlay]
        \path (0, 0) -- (1, 0);
    \end{tikzpicture}}
    \only<2>{\smash{\begin{tikzpicture}[overlay]
        \fill[color=white, opacity=0.75] (0, 5.5) rectangle (6.5, 2.1);
        \fill[color=white, opacity=0.75] (0, 0) rectangle (6.5, 1.9);
        \draw[->, thick] (5.5, 2) to[out=0, in=180] (9.2, 2.3);
    \end{tikzpicture}}}%
    }{coordinatesNode}
     
    \begin{enumerate}
        \item label node \verb|\node| \ccol{(<label>)}
        \item refer to the node as \ccol{(node cs:name=<label>)} 
        \item you can also use things like <label>.west or <label>.right
         
    \end{enumerate}
\end{frame}

\begin{frame}{Vertical and horizontal\magicPage}\relax
    \twocolImg{
    % \lstinputlisting[linerange={9-9, 15-17}]{curveverthor.tex}
    \inputminted[firstline=9, lastline=9]{latex}{sec01/code/curveverthor.tex}
    \inputminted[firstline= 15, lastline=17]{latex}{sec01/code/curveverthor.tex}
    }{curveverthor}
    
    Use {\csk -|} for ``first horizontal, than vertical''. Use {\csk |-} for ``first vertical, than horizontal''
    
    
    \skfootnote{\tikzc{14.2.2}[152]}
     
\end{frame}


\begin{frame}{Curves\magicPage}\relax
    \vspace{-0.5cm}
    \twocolImg{
    % \lstinputlisting[linerange={9-9, 15-20}]{curveinout.tex}
    \inputminted[firstline=9, lastline=9]{latex}{sec01/code/curveinout.tex}
    \inputminted[firstline= 15, lastline=20]{latex}{sec01/code/curveinout.tex}
    }{curveinout}
    \footnotesize
    \ccol{to [out=.., in=..]} the angle on which curve flows out and the angle on which curve flows in.
    
    \twocolImg{
    % \lstinputlisting[linerange={9-9, 15-19}]{curveibez.tex}
    \inputminted[firstline=9, lastline=9]{latex}{sec01/code/curveibez.tex}
    \inputminted[firstline= 15, lastline=19]{latex}{sec01/code/curveibez.tex}
    }{curveibez}
    
    \ccol{ .. controls <coord> and <coord> .. }
    
    
    \skfootnote{\url{https://en.wikipedia.org/wiki/Bezier_curve}}
     
\end{frame}


\begin{frame}{Coordinates\magicPage}\relax
    \twocolImg{
        \inputminted[firstline=15, lastline=33]{latex}{sec01/code/coordinatesCombined.tex}
    }{coordinatesCombined}
    \skfootnote{\tikzc{13.2}[133]}
\end{frame}


\begin{frame}[fragile]{``++'' and ``+'' coordinates\magicPage}\relax
    \twocolImg{
    % \lstinputlisting[linerange={14-21}]{coordinatesPlus.tex}
    \inputminted[firstline=14, lastline=21]{latex}{sec01/code/coordinatesPlus.tex}
    }{coordinatesPlus}
     \begin{itemize}
         \item ``++'' use relative coordinate and set this new coordinate as ``current''
         \item ``+'' use relative coordinate and DOESN't set this new coordinate as ``current''
          
     \end{itemize}
\end{frame}


\begin{frame}{Coordinate calculation\magicPage}\relax
    \twocolImg{
    % \lstinputlisting[linerange={9-10, 15-20}]{coordinatesCalc.tex}
    \inputminted[firstline=9, lastline=10]{latex}{sec01/code/coordinatesCalc.tex}
    \inputminted[firstline= 15, lastline=20]{latex}{sec01/code/coordinatesCalc.tex}
    }{coordinatesCalc}
     
    \begin{enumerate}
        \item \ccol\usetikzlibrary\{calc\} 
        \item syntax: {\csk \$<coord1>!fraction!<coord2>\$}
        \item in this slide you can also see \ccol\foreach!
         
    \end{enumerate}
     
\end{frame}

\begin{frame}{Coordinate intersection\magicPage}\relax
    \twocolImg{
    % \lstinputlisting[linerange={9-10, 15-22}]{coordinatesInter.tex}
    \inputminted[firstline=9, lastline=10]{latex}{sec01/code/coordinatesInter.tex}
    \inputminted[firstline= 15, lastline=22]{latex}{sec01/code/coordinatesInter.tex}
    }{coordinatesInter}
 
\end{frame}

