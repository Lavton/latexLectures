\begin{frame}[fragile]{What is TikZ?}\relax
``TikZ ist kein Zeichenprogramm'' 

which translates to ``TikZ is not a drawing program''

TikZ defines a number of \TeX\ commands that produce graphics: \tikz \fill[orange] (1ex,1ex) circle (1ex); produced by \verb|\tikz \fill[orange] (1ex,1ex) circle (1ex);|
\end{frame}

\begin{frame}{Pros and Cons}

\Huge\centering Pros and Cons
     
\end{frame}

\begin{frame}[fragile]{Cons}\relax
     \begin{itemize}
        \item[$-$] it is most likely that you don't need TikZ
        \item[$-$] write visual-based thinks like graphics is really annoying in a not-WYSiWYG way
    \end{itemize}
    
\end{frame}

\begin{frame}{Pros}\relax
     \begin{itemize}
        \item[$+$] it is most likely that you need some TikZ elements
        \item[$+$] some graphics (graphs for example) are so good structured, that it is OK to program them
        \item[$+$] TikZ has perfect integration with \LaTeX\ (and beamer):
        \begin{itemize}
            \item You can use all \LaTeX\ commands inside TikZ, creating beautiful pictures with math 
            \item You can pose elements using TikZ
            \item You can show just part of the picture in beamer Overlays   
        \end{itemize}
        \item[$+$] You don't need to have an external file
        \item[$+$] TikZ is using in CV and lots of other templates. It is good to be able to read the code
    \end{itemize}
\end{frame}


\begin{frame}[fragile]{How to setup TikZ picture?}\relax

\verb|\usepackage{tikz}|

and then 

\verb|\begin{tikzpicture} <code> \end{tikzpicture}| or, for short inline graphics, \ccol\tikz. 

\inputminted[firstline=8, lastline=17]{latex}{sec01/code/start.tex}

\skfootnote{\tikzc{12.1}[126]}
     
\end{frame}

\begin{frame}{Examples}\relax

\begin{itemize}
    \item \url{https://tikz.net/}
    \item \url{https://texample.net/bar-chart/}
\end{itemize}


\inclassFrag{  Please, look at these sites for a few minutes  }[-1]
     
\end{frame}