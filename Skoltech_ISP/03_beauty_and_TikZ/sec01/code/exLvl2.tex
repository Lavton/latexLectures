\documentclass{article}
\usepackage{fontspec}
\pagestyle{empty}
\usepackage{geometry}
\geometry{paperwidth=50mm, paperheight=50mm, left=2mm, top=0mm, right=2mm, bottom=0mm}
\parindent=0pt
\usepackage{color}
\usepackage{xcolor}
\usepackage{tikz}

\begin{document}
\tikzset{int/.style  = {draw, circle, fill=blue!20, minimum size=2em}}
\begin{tikzpicture}[auto,>=latex, remember picture]
  \node [int] (a) {$b=3$};
  \node (b) [below of=a, left of=a, node distance=1cm, coordinate] {a};
  \node (c) [above of=a, left of=a, node distance=1cm, coordinate] {a};
  \node [int] (bnode) [left of=b, distance=0cm] {$x_1$};
  \node [int] (cnode) [left of=c, distance=0cm] {$x_2$};
  \path[->] (bnode) edge node[below] {$-2$} (a);
  \path[->] (cnode) edge node[above] {$-2$} (a);
\end{tikzpicture}
\\ \\ 

It is \tikz[overlay,remember picture] \draw (0,0) to[out=90, in=-90] (a.south); some scheme
% https://tikz.net/nand-perceptron/
\end{document}