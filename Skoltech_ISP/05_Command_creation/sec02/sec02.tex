\graphicspath{{sec02/images/}{sec02/code/}}
\lstset{inputpath=sec02/code/}

\begin{frame}{Create new command\lW}{without arguments}\relax
    \inputminted[firstline=8, lastline=10]{latex}{sec02/code/commandwithout.tex}
    \inputminted[firstline= 13, lastline=15]{latex}{sec02/code/commandwithout.tex}
    \inputminted[firstline= 18, lastline=18]{latex}{sec02/code/commandwithout.tex}
    \skfootnote{\lmanc{12.1}[113] \lvoc{VII.1.1}[234] \wikiC{https://en.wikibooks.org/wiki/LaTeX/Macros} \overC{https://www.overleaf.com/learn/latex/Commands\#Defining_a_new_command}}
    
     \inclassFrag{Reproduce this code by yourselves}[-1]
\end{frame}

\begin{frame}{Create new command\lW}{without arguments}\relax

\ccol\newcommand\{<\ccol\commandname>\}\{<code>\} to create new macros

    \twocolImg{
    % \lstinputlisting[linerange={8-12, 14-17, 19-19}]{commandwithout.tex}
    \inputminted[firstline=8, lastline=10]{latex}{sec02/code/commandwithout.tex}
    \inputminted[firstline= 13, lastline=15]{latex}{sec02/code/commandwithout.tex}
    \inputminted[firstline= 18, lastline=18]{latex}{sec02/code/commandwithout.tex}
    }{commandwithout}
    
    
    \skfootnote{\lmanc{12.1}[113] \lvoc{VII.1.1}[234] \wikiC{https://en.wikibooks.org/wiki/LaTeX/Macros} \overC{https://www.overleaf.com/learn/latex/Commands\#Defining_a_new_command}}
\end{frame}

\begin{frame}{Recreate command\lW\hasError}\relax
    \inclassFrag{Try to change command to previous example:
    
    % \lstinputlisting[linerange={8-11}]{recommandwithoutTASK.tex}
        \inputminted[firstline=8, lastline=9]{latex}{sec02/code/recommandwithoutTASK.tex}
        
        {\footnotesize you'll have an error}
    }[1]
    \ccol\renewcommand\ to recreate already created command
    
    \twocolImg{
    % \lstinputlisting[linerange={8-12, 15-17, 20-20}]{recommandwithout.tex}
        \inputminted[firstline=8, lastline=13]{latex}{sec02/code/recommandwithout.tex}
        \inputminted[firstline= 15, lastline=17]{latex}{sec02/code/recommandwithout.tex}
        \inputminted[firstline= 20, lastline=20]{latex}{sec02/code/recommandwithout.tex}
    }{recommandwithout}
    
    

\end{frame}

\begin{frame}{Create new command\lW}{with arguments}\relax
    \inputminted[firstline=8, lastline=10]{latex}{sec02/code/commandwith.tex}
    \inputminted[firstline= 13, lastline=15]{latex}{sec02/code/commandwith.tex}
    \inputminted[firstline= 18, lastline=18]{latex}{sec02/code/commandwith.tex}
    
    \inclassFrag{Reproduce this code by yourselves}[-1]
    \skfootnote{\lmanc{12.1}[113] \lvoc{VII.1.1}[234] \wikiC{https://en.wikibooks.org/wiki/LaTeX/Macros} \overC{https://www.overleaf.com/learn/latex/Commands\#Defining_a_new_command}\\ to define command with star use \ccol{\@ifstar} \lmanc{12.4}[116]}
\end{frame}

\begin{frame}{Create new command\lW}{with arguments}\relax
    \ocol\newcommand\{<commandname>\}[<number of args]\{<code>\}. Refer to arg as \#1, \#2, ...
     \twocolImg{
        \inputminted[firstline=8, lastline=10]{latex}{sec02/code/commandwith.tex}
        \inputminted[firstline= 13, lastline=15]{latex}{sec02/code/commandwith.tex}
        \inputminted[firstline= 18, lastline=18]{latex}{sec02/code/commandwith.tex}
    }{commandwith}
    % \inclassFrag{Mod}[-1]
\end{frame}

\begin{frame}{Define macros \tW}
     In \TeX\ you can define new macros via \ccol\def.

        \inputminted[firstline=8, lastline=10]{latex}{sec02/code/commanddef.tex}
        \inputminted[firstline=13, lastline=15]{latex}{sec02/code/commanddef.tex}
        \inputminted[firstline=18, lastline=18]{latex}{sec02/code/commanddef.tex}
    \skfootnote{\tugC{https://www.tug.org/utilities/plain/cseq.html\#def-rp} \knuthc{20}[209]}
\end{frame}

\begin{frame}{Define macros \tW}
     In \TeX\ you can define new macros via \ccol\def.

    \twocolImg{
        \inputminted[firstline=8, lastline=10]{latex}{sec02/code/commanddef.tex}
        \inputminted[firstline=13, lastline=15]{latex}{sec02/code/commanddef.tex}
        \inputminted[firstline=18, lastline=18]{latex}{sec02/code/commanddef.tex}
    }{commanddef}
     
    Use \ccol\global\ prefix to define macros not just inside ``group''. 
    
    Use \ccol\long\ prefix to define macros that can have multiple paragraphs as an argument.
    
\end{frame}

\begin{frame}[fragile]{Compare \LaTeX\ and \TeX}\relax
      \inputminted[firstline=8, lastline=10]{latex}{sec02/code/commandwith.tex}
        \inputminted[firstline= 13, lastline=15]{latex}{sec02/code/commandwith.tex}
        \inputminted[firstline= 18, lastline=18]{latex}{sec02/code/commandwith.tex}
        
        \hrule
        \inputminted[firstline=8, lastline=10]{latex}{sec02/code/commanddef.tex}
        \inputminted[firstline=13, lastline=15]{latex}{sec02/code/commanddef.tex}
        \inputminted[firstline=18, lastline=18]{latex}{sec02/code/commanddef.tex}
        
\end{frame}

\begin{frame}[fragile]{\{\}}\relax

\cprotect\inclassFrag{
     Try the following:
        \inputminted[firstline=8, lastline=8]{latex}{sec02/code/commandtrick.tex}
        \inputminted[firstline=13, lastline=15]{latex}{sec02/code/commandtrick.tex}
     }[1]
   \twocolImg{
        \inputminted[firstline=8, lastline=8]{latex}{sec02/code/commandtrick.tex}
        \inputminted[firstline=13, lastline=15]{latex}{sec02/code/commandtrick.tex}
    }{commandtrick}}
    
    \outclasshigh{The braces just indicates atomicy!}
\end{frame}

\begin{frame}[fragile]{\{\}}\relax

    \twocolImg{
        \inputminted[firstline=8, lastline=8]{latex}{sec02/code/commandtrKnown.tex}
        \inputminted[firstline=13, lastline=23]{latex}{sec02/code/commandtrKnown.tex}
    }{commandtrKnown}
    \outclasshigh{The braces just indicates atomicy!}
\end{frame}

\begin{frame}{what is ``atomic'' in \TeX}\relax
\label{sl:atom}
    \begin{itemize}
        \item Commands \ccol\somecommand
        \item Symbols
        \item Everything in braces {\csk \{\}}
         
    \end{itemize}
     
\end{frame}

\begin{frame}[fragile]{Command creation inside command creation}\relax
    As simple as \verb|\newcommand{\name}{\newcommand{\othername}{smth}}|
    
    \begin{enumerate}
        \item In the inner command, you can refer to the argument of outer command as \#1
        \item In the inner command, you can refer to the argument of inner command as \#\#1
    \end{enumerate}
    \inpause
    Sometimes you can see something like
    \lstinline[basicstyle=\tt\small]|\newcommand{\photo}[1]{\renewcommand{\photo}[#1]}|
    
    \inclassFrag{What it is for?}[3] It provides the following usage: You can store something at first usage as \verb|\photo{myface.png}| and then refer to it as just \verb|\photo|
\end{frame}

\begin{frame}[fragile]{The scope\preMagicPage}\relax
     \cprotect\inclassFrag{
     Try the following:
     
     \lstinline[basicstyle=\tt]|\newcommand{\htext}[1]{\Huge text}|
     
     \lstinline[basicstyle=\tt]|text \htext{text} text|
     }[1]
     The braces at command definition and at command usage ommited. If you want your code to have local effect -- provide an extra braces:
     
     not \\  \lstinline[basicstyle=\tt]|\newcommand{\htext}[1]{\Huge text}|
     
     but \\  \lstinline[basicstyle=\tt]|\newcommand{\htext}[1]{{\Huge text}}|
\end{frame}


\begin{frame}{New environment\magicPage}\relax

    use \ccol\newenvironment\{<name>\}\{<code at begin>\}\{<code at end>\}
    
    or \ccol\renewenvironment     
    
    \skfootnote{\lmanc{12.8}[118]}
     
\end{frame}


\begin{frame}[fragile]{New command with optional arguments\magicPage}\relax
    \twocolImg{
    % \lstinputlisting[linerange={10-16}]{commandoptdef.tex}
    \inputminted[firstline=10, lastline=16]{latex}{sec02/code/adv/commandoptdef.tex}
    }{adv/commandoptdef}
    
    Use syntax \verb|\newcommand{\<cmdname>}[total_arg_num][defaults]{<code>}|
\end{frame}

\begin{frame}[fragile]{New command with optional arguments\magicPage}{using package}\relax
    \twocolImg{
    % \lstinputlisting[linerange={7-7, 11-18}]{commandoptpack.tex}
    \inputminted[firstline=7, lastline=7]{latex}{sec02/code/adv/commandoptpack.tex}
    \inputminted[firstline= 11, lastline=18]{latex}{sec02/code/adv/commandoptpack.tex}
    }{adv/commandoptpack}
    
    Use \ncol\usepackage{xargs}\ and \ccol\newcommandx\ (notice \textbf{x} at the end)

\end{frame}

\begin{frame}[fragile]{New command with key=value syntax\magicPage}{keyval package}\relax
    \twocolImg{
    % \lstinputlisting[linerange={9-16, 22-22}]{commandkeyval.tex}
    \inputminted[firstline=9, lastline=16]{latex}{sec02/code/adv/commandkeyval.tex}
    \inputminted[firstline= 22, lastline=22]{latex}{sec02/code/adv/commandkeyval.tex}
    }{adv/commandkeyval}
    
    Use \ncol\usepackage{keyval}\ and \ccol{\define@key}, \ccol{\setkeys}
    
    \skfootnote{\stExC{https://tex.stackexchange.com/questions/58069/newcommand-key-value}}
\end{frame}


\begin{frame}[fragile]{New command with key=value syntax\magicPage}{keyval package}\relax
    \twocolImg{
    % \lstinputlisting[linerange={9-9, 15-16}]{commandkeyvalpgf.tex}
        \inputminted[firstline=9, lastline=9]{latex}{sec02/code/adv/commandkeyvalpgf.tex}
        \inputminted[firstline= 15, lastline=16]{latex}{sec02/code/adv/commandkeyvalpgf.tex}
    }{adv/commandkeyvalpgf}
    
    Use \ncol\usepackage{pgfkeys}
    
    \skfootnote{\stExC{https://tex.stackexchange.com/questions/34312/how-to-create-a-command-with-key-values}}
\end{frame}


\cprotect\inclassframe{
\begin{frame}[fragile]{\exFrame{Try to create commands}}{lvl 1}

    Try to type the following, using the macros creation (also see next slide):
    \newcommand{\mypart}[2]{\frac{\partial^2 #1}{\partial #2^2}}
        $$\Delta f = \mypart{f}{r} = \mypart{f}{x}+\mypart{f}{y}+\mypart{f}{z}$$
\end{frame}

\begin{frame}[fragile]{\exFrame{Try to create commands}}{lvl 1}
    Try to type the following, using the macros creation:
    \newcommand{\mypart}[2]{\frac{\partial^2 #1}{\partial #2^2}}
        $$\Delta g = \mypart{g}{r} = \mypart{g}{x}+\mypart{g}{y}+\mypart{g}{z}$$
\end{frame}

\begin{frame}[fragile]{\exFrame{Try to create commands}}{lvl 2}
    Try to type the following, using the macros creation (also see next slide):
    \newcommand{\mypart}[2]{\frac{\partial^2 #1}{\partial #2^2}}
    \newcommand{\lName}{Lusy}
        $$\Delta g = \mypart{g}{r} = \mypart{g}{x}+\mypart{g}{y}+\mypart{g}{z}$$
        {
        -- Hello. I'm \lName.\\
        -- What do you wish, \lName?\\
        -- \lName\ wishes a big Cola and French fries\\
        -- Here are they, \lName! \\
    }     

\end{frame}


\begin{frame}[fragile]{\exFrame{Try to create commands}}{lvl 2}

    Try to type the following, using the macros creation (also see next slide):
    \newcommand{\mypart}[2]{\frac{\partial^2 #1}{\partial #2^2}}
    \newcommand{\lName}{Mantius de Virr, the Future Lord of the Universe and conqueror of the Galaxy}
        $$\Delta f = \mypart{f}{r} = \mypart{f}{x}+\mypart{f}{y}+\mypart{f}{z}$$
        {
        -- Hello. I'm \lName.\\
        -- What do you wish, \lName?\\
        -- \lName\ wishes a big Cola and French fries\\
        -- Here are they, \lName! \\
    }
\end{frame}
}

\cprotect\inclassframe{
\begin{frame}[fragile]{\exFrame{Modify the previous code}}{lvl 2}

    use \verb|\usepackage{color}| and \verb|\color{green}|
    
    Try to type the following, using the macros creation:
    \newcommand{\mypart}[2]{\frac{\partial^2 #1}{\partial #2^2}}
    \newcommand{\lName}{{\color{green}Lusy}}
        $$\Delta g = \mypart{g}{r} = \mypart{g}{x}+\mypart{g}{y}+\mypart{g}{z}$$
        {
        -- Hello. I'm \lName.\\
        -- What do you wish, \lName?\\
        -- \lName\ wishes a big Cola and French fries\\
        -- Here are they, \lName! \\
    }

\end{frame}

\begin{frame}[fragile]{\exFrame{Implement the following}}{lvl 2}
     Imagine that you are creating a package and want to provide a simple ui for the user. Create somehow commands \string\generate\ and \string\lName\ to reproduce the following behaviour:\\[1cm]
     
     \twocolImg{
    \inputminted[firstline=18, lastline=21]{latex}{sec02/code/exLvl2.tex}
    }{exLvl2}
    
\end{frame}

}




\begin{frame}{Where to put your own commands}\relax
    
    \begin{enumerate}
        \item You can put it into document preamble
        \item You can put it inside document whenever you want. Then:
        \begin{itemize}
            \item The command can be used only after it's definition
            \item The command definition is LOCAL: the scope of the visibility is the GROUP
        \end{itemize}
        \inpause
        \item You can put it into style or class files
         
    \end{enumerate}
         
\end{frame}