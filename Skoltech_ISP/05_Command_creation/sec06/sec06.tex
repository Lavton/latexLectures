%% It is just an empty TeX file.
%% Write your code here.
\graphicspath{{sec06/images/}{sec06/code/}}
\lstset{inputpath=sec06/code/}

\subsection{Define macros}


\begin{frame}\relax
    
    {\centering\Huge \TeX\ is Turing-complete language
    
    }
    \vspace{1em}
    \inclassFrag{Who understand, what does it means?}
    In basic words, it means, that you can write in \TeX\ and \LaTeX\ any algoritms, that you can write in C++, Java, Python... Moreover, some \TeX syntax is really familiar to functional languages
    \skfootnote{\vspace{-3ex}\url{https://stackoverflow.com/questions/2968411/ive-heard-that-latex-is-turing-complete-are-there-any-programs-written-in-late} \overC{https://www.overleaf.com/learn/latex/Articles/LaTeX_is_More_Powerful_than_you_Think_-_Computing_the_Fibonacci_Numbers_and_Turing_Completeness} \url{http://sdh33b.blogspot.com/2008/07/icfp-contest-2008.html}}
     
\end{frame}


\begin{frame}{Reminder: Define macros \tW\magicPage}
     In \TeX\ you can define new macros via \ccol\def.

    \twocolImg{
    % \lstinputlisting[linerange={8-9, 14-16}]{commandwith.tex}
        \inputminted[firstline=8, lastline=9]{latex}{sec06/code/commandwith.tex}
        \inputminted[firstline= 14, lastline=16]{latex}{sec06/code/commandwith.tex}
    }{commandwith}
     
    Use \ccol\global\ prefix to define macros not just inside ``group''. 
    
    Use \ccol\long\ prefix to define macros that can have multiple paragraphs as an argument.
    
    \skfootnote{\tugC{https://www.tug.org/utilities/plain/cseq.html\#def-rp} \knuthc{20}[209]}
\end{frame}

\begin{frame}{Define with pattern matching\tW\magicPage}\relax

    The syntax with writing each argument seems to be an over-use. But it is needed because of \textit{pattern matching}
      
    \twocolImg{
    % \lstinputlisting[linerange={11-13}]{commandpattern.tex}
        \inputminted[firstline=11, lastline=13]{latex}{sec06/code/commandpattern.tex}
    }{commandpattern}   

\end{frame}

\subsection{Conditions}

\begin{frame}[fragile]{Compare strings (macros)\tW}\relax

    \twocolImg{
    % \lstinputlisting[linerange={12-21}]{ifmy.tex}
    \inputminted[firstline=12, lastline=22]{latex}{sec06/code/ifmy.tex}
    }{ifmy}   

    {\csk \verb|\ifx\<first>\<second>  <code1>  [\else  <code2>]  \fi|}
    
\end{frame}

\begin{frame}[fragile]{Compare numbers\tW}\relax

    \twocolImg{
    % \lstinputlisting[linerange={12-20}]{ifnummy.tex}
    \inputminted[firstline=12, lastline=26]{latex}{sec06/code/ifnummy.tex}
    }{ifnummy}   

    {\csk \verb|\ifnum<first><operator><second>  <code1>  [\else  <code2>]  \fi|}. Only ``='', ``>'' or ``<'' are allowed.
    
    Use {\csk \verb|\ifcase|} to check different stuff. Also use {\csk \verb|\ifodd|} to check if num is odd or even
    
\end{frame}

\begin{frame}{Compare in \LaTeX\lW}\relax
    \twocolImg{
    % \lstinputlisting[linerange={7-7, 12-16}]{ifstring.tex}
        \inputminted[firstline=7, lastline=7]{latex}{sec06/code/ifstring.tex}
        \inputminted[firstline= 12, lastline=16]{latex}{sec06/code/ifstring.tex}
    }{ifstring}  
      \ncol\usepackage{xstring}
      
      also see \ncol\usepackage{ifthen}
      
      \hrule 
      
      you can check if you are in \XeLaTeX by \ncol\usepackage{ifxetex}
      
      \skfootnote{read about the xstring package! It has lots of things: strings, substrings,.. what to do with them,.. replace strings etc\\ \stExC{https://tex.stackexchange.com/questions/47576/combining-ifxetex-and-ifluatex-with-the-logical-or-operation}}
\end{frame}

\begin{frame}[fragile]{Check modes\tW\magicPage}\relax

    \twocolImg{
    % \lstinputlisting[linerange={12-22}]{ifmodemy.tex}
    \inputminted[firstline=12, lastline=22]{latex}{sec06/code/ifmodemy.tex}
    }{ifmodemy}   
    \begin{itemize}
        \item {\csk\verb|\ifmmode|} to check if in mathematical mode
        \item {\csk\verb|\ifvmode|} to check if in vertical mode
        \item {\csk\verb|\ifhmode|} to check if in horizontal mode
        \item {\csk\verb|\ifinner|} to check if \TeX\ is in internal vertical mode, or restricted horizontal mode, or (nondisplay) mathmode
     
    \end{itemize}
    
    \skfootnote{other ``if'''s can be found in \knuthc{20}[221]}
\end{frame}



%%%%%%%%%%%%%%%%%%%%%%%%%%% LOOPS %%%%%%%%%%%%%%%%%%%%%%%%%%%%%%%%
\subsection{Loops and recursion}

\begin{frame}[fragile]{Loop \tW}\relax
    \twocolImg{
        % \lstinputlisting[linerange={12-17}]{loopmy.tex}
        \inputminted[firstline=12, lastline=17]{latex}{sec06/code/loopmy.tex}
    }{loopmy}

    \ccol\loop\ for start loop, <code> inside, then {\csk\verb|\if<..>|}-family, another bunch of <code>, ended with \ccol\repeat.

     \skfootnote{\knuthc{20}[228]}
\end{frame}



\begin{frame}[fragile]{For loop\lW}\relax
\label{sl:for}
    \twocolImg{
    % \lstinputlisting[linerange={8-8, 13-16}]{forloopmy.tex}
        \inputminted[firstline=8, lastline=8]{latex}{sec06/code/forloopmy.tex}
        \inputminted[firstline= 13, lastline=16]{latex}{sec06/code/forloopmy.tex}
    }{forloopmy}
    \ncol \usepackage{forloop}
    
    \twocolImg{
    % \lstinputlisting[linerange={8-8, 13-15}]{foreachmy.tex}
        \inputminted[firstline=8, lastline=8]{latex}{sec06/code/foreachmy.tex}
        \inputminted[firstline= 13, lastline=18]{latex}{sec06/code/foreachmy.tex}
    }{foreachmy}
    
    \ncol \usepackage{pgffor}, part of pgf, part of TikZ
    
    \skfootnote{\url{https://stackoverflow.com/questions/2561791/iteration-in-latex}}
\end{frame}

\begin{frame}[fragile]{Reqursion \tW\magicPage}\relax
    \twocolImg{
    % \lstinputlisting[linerange={12-13}]{reqmy.tex}
    \inputminted[firstline=12, lastline=13]{latex}{sec06/code/reqmy.tex}
    }{reqmy}

\end{frame}

%%%%%%%%%%%%%%%%%%% related %%%%%%%%%%%%%%%%%%%%%%%%%%%%%%%%%%
\subsection{Related things to macros creation}

\begin{frame}[fragile]{``let'' command\magicPage}\relax

    \twocolImg{
    % \lstinputlisting[linerange={11-21}]{letmy.tex}
        \inputminted[firstline=11, lastline=21]{latex}{sec06/code/letmy.tex}
    }{letmy}
    
    
    The statement ``\ccol\let\verb|\a=\b|'' gives \string\a\ the current meaning of \string\b. If \string\b\ changes after the assignment is made, \string\a\ does not change.
    
    \skfootnote{\knuthc{20}[217] \tugC{https://www.tug.org/utilities/plain/cseq.html\#let-rp}}
     
\end{frame}

\begin{frame}[fragile]{Usecase with \ccol\let: ``decorator''\magicPage}\relax

    Imagine: you have some \string\command\ used inside the document multiple times. You want to \textit{add some addition behaviour} to the command -- decorate (or ``wrap'', or ``redefine with the use of itself''). You can do it with \ccol\let:
    
    \begin{minted}{latex}
    \let\oldCommand=\command
    \def\command#1{<some code>\oldCommand}
    \end{minted}
    
    And the same for enviruments using \ncol\usepackage{etoolbox} or \ccol{\g@addto@macro}
    
    
    
    \skfootnote{\stExC{https://tex.stackexchange.com/questions/47351/can-i-redefine-a-command-to-contain-itself} \stExC{https://tex.stackexchange.com/questions/467435/environment-decorators}}
     
\end{frame}


\subsection{Programming examples}

\begin{frame}[fragile]{99 Bottles of Beer\magicPage}\relax

    \twocolImg{
        % \lstinputlisting[linerange={9-26}]{bottles.tex}
        \inputminted[firstline=9, lastline=26]{latex}{sec06/code/bottles.tex}
    }{bottles}
    
    \skfootnote{\url{https://rosettacode.org/wiki/99_Bottles_of_Beer\#LaTeX}}
     
\end{frame}

\begin{frame}[fragile]{not-AND logical gate\magicPage}\relax

    \twocolImg{
    % \lstinputlisting[linerange={11-29}]{nand.tex}
        \inputminted[firstline=11, lastline=29]{latex}{sec06/code/nand.tex}
    }{nand}
    
    
\end{frame}

\begin{frame}[fragile]{Split words\magicPage}\relax
    \twocolImg{
    % , basicstyle=\tiny
        % \lstinputlisting[linerange={11-31}]{splitmy.tex}
        \inputminted[firstline=11, lastline=31, fontsize=\tiny]{latex}{sec06/code/splitmy.tex}
    }{splitmy}

     \skfootnote{\stExC{https://tex.stackexchange.com/questions/12810/how-do-i-split-a-string/12811}}
\end{frame}
