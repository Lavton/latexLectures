\documentclass[aspectratio=169]{beamer}
\usepackage{ifxetex}
\usepackage{fontspec}
\usepackage{xunicode}
\usepackage{xltxtra}
\usepackage{xecyr}
\usepackage{polyglossia}
\usepackage{multicol}
\usepackage{pdfpages} 


\usepackage{beamerskoltech} 
% \renewcommand{\logoname}{sklogo.png} % <- default value for logo 
% \renewcommand{\logoname}{} % <- if you want no logo

\usepackage{progressbar}
\colorlet{progressbarcolor}{skoltechgreen}
\usepackage[mode=out]{inoutclass}
\definecolor{inclasscolor}{RGB}{46, 228, 182}
\definecolor{outclasscolor}{RGB}{41, 0, 204}
\usepackage{xargs}
\usepackage{latexLectures}
\usepackage{minted}
\usetikzlibrary{intersections}
\usetikzlibrary{calc}


\title{\LaTeX:\\ \Large from dummy to \TeX nician}
\subtitle{Command creation. How \TeX\ works-2}
\author{Anton Lioznov}
\institute{Skoltech, \\Project Center of Omics Technologies and Advanced Mass Spectrometry}
\date{ISP 2025,\\ \textit{lesson 5}}
\usetikzlibrary{decorations.pathreplacing}
\def\hasError{\inclass{
\begin{tikzpicture}[remember picture,overlay,shift={(current page.north east)}]
\draw[fill, red, anchor=north east,xshift=-1cm,yshift=-0.7cm] circle [radius=0.06];
\end{tikzpicture}}%
}
\begin{document}


\frame{\titlepage}

\AtBeginSection[] 
{
  \begin{frame}{What we will know?}
  \tableofcontents[currentsection,hideallsubsections]
  \end{frame}
}
\AtBeginSubsection[]
{
  \begin{frame}{What we will know?}
  \tableofcontents[currentsubsection, hideothersubsections, sectionstyle=show/hide, subsectionstyle=show/shaded/hide]
  \end{frame}
}

{\supressprogressbartrue


\begin{frame}\frametitle{What we will know?}
\tableofcontents[hideallsubsections]
\end{frame}

\supressprogressbarfalse
}

% \begin{frame}
% % \begin{enumerate}
% %     \item Overview and basis
% %         \begin{tikzpicture}[remember picture, overlay]
% %             \draw[decoration={brace}, decorate] 
% %             ([yshift=0.5ex]pic cs:start) --
% %             ([yshift=-2ex]pic cs:end)
% %             node[midway, right=0.3cm] {User level};
% %         \end{tikzpicture}\tikz[remember picture]\coordinate(start);
% %     \item Document creation\tikz[remember picture]\coordinate(end);
% %     \item TikZ and Typography
% %     \item \TeX\ and Typography
% %     \item Command creation
% % \end{enumerate}
     
% %      \begin{itemize}
% %     \item[$\left.\begin{cases}
% %         \text{1.} & \text{First item in the list} \\
% %         \text{2.} & \text{Second item in the list}
% %     \end{cases}\right\}$] Text explaining items 1 and 2
    
% %     \item[$\left.\begin{cases}
% %         \text{3.} & \text{Third item in the list}
% %     \end{cases}\right\}$]
    
% %     \item[$\left.\begin{cases}
% %         \text{4.} & \text{Fourth item in the list} \\
% %         \text{5.} & \text{Fifth item in the list}
% %     \end{cases}\right\}$] Text explaining items 4 and 5
% % \end{itemize}

% % \begin{enumerate}
% %     \item \tikz\node(item1){Item 1};
% %     \item \tikz\node(item2){Item 2};
% %     \item \tikz\node(item3){Item 3};
% %     \item \tikz\node(item4){Item 4};
% %     \item \tikz\node(item5){Item 5};
% % \end{enumerate}

% % \begin{tikzpicture}[overlay, remember picture]
% %     % Bracket for items 1 and 2
% %     \draw[decorate, decoration={brace, amplitude=5pt}, thick]
% %         ([xshift=-5pt, yshift=3pt]item1.north west) -- ([xshift=-5pt, yshift=-3pt]item2.south west)
% %         node[midway, left=12pt] {Group A};

% %     % Bracket for item 3
% %     \draw[decorate, decoration={brace, amplitude=5pt}, thick]
% %         ([xshift=-5pt, yshift=3pt]item3.north west) -- ([xshift=-5pt, yshift=-3pt]item3.south west)
% %         node[midway, left=12pt] {Solo Item};

% %     % Bracket for items 4 and 5
% %     \draw[decorate, decoration={brace, amplitude=5pt}, thick]
% %         ([xshift=-5pt, yshift=3pt]item4.north west) -- ([xshift=-5pt, yshift=-3pt]item5.south west)
% %         node[midway, left=12pt] {Group B};
% % \end{tikzpicture}
% \end{frame}

\section{Technical agreements}

\begin{frame}{Agreements}{I}\relax
     {\Large inclass/outclass versions}
     \begin{itemize}
          \item two slightly different versions for class and home
          \item class version is more interactive and contains less information
          \inclass{\item \inclasshigh{} this line will be shown only in class} 
          \outclass{\item \outclasshigh{} this line will be shown only at home version}
     \end{itemize}
\end{frame}
\inclassframe{\begin{frame}{Frame for a class}
     
\end{frame}}
\outclassframe{\begin{frame}{Frame for home}
     
\end{frame}}

{\supressfootnotefalse
\begin{frame}[fragile]{Agreements}{II}\relax
\newcommand{\tikzmark}[1]{\tikz[overlay,remember picture] \node (#1) {};}

{ \Large Footnotes }
\begin{tikzpicture}[overlay,remember picture]
\draw[->,ultra thick] (0,0.1) to[out=0,in=45] (8, -1.5) to[out=225,in=90] (5,-5.5);
\end{tikzpicture}

\skfootnote{Like this}
\begin{itemize}
     \item For second reading
     \item Contains advanced usage of the command 
     \item Contains references to read more 
     \begin{itemize}
         \item to the exact chapter 
         \item (often) with the href to exact page  
     \end{itemize}
     \item Contains some comments
     \item Mostly for outclass version
\end{itemize}
\end{frame}

}

{\inclassmodetrue
\newcommand{\tikzmark}[1]{\tikz[overlay,remember picture] \node (#1) {};}

\begin{frame}{Agreements}{III}\relax
{ \Large Addition information -- ``magic'' \tikzmark{startM}} 

\begin{tikzpicture}[remember picture,overlay,shift={(current page.north east)}]
\node[anchor=north east,xshift=-0cm,yshift=-0cm](endM) {%
{\includegraphics[width=1cm]{images/magic}}%
};
\end{tikzpicture}%

\begin{tikzpicture}[overlay, remember picture]
    \draw[->,ultra thick] (startM) to[out=0, in=180] (endM);
\end{tikzpicture}

\begin{itemize}
     \item To have the full picture 
     \item Not to analyze or to puzzle out in class 
\end{itemize}
\end{frame}

}


\begin{frame}{\exFrame{Agreements}}{V}\relax


{ \Large Exercises } \begin{tikzpicture}[overlay]
% \draw[white] (0,0) -- (0, 0.6);
\draw[->,ultra thick] (0,0.1) to[out=0,in=-90](-2.1,3.15);
\end{tikzpicture}


\begin{itemize}
     \item To work in class 
\end{itemize}
\end{frame}
\begin{frame}{Special thanks to}\relax

     Our TAs:
     \begin{itemize}
         \item Peter Borisovets
         \item Pavel Kuzmin
         \item Anna Litvin
     \end{itemize}

\end{frame}
\supressfootnotefalse
\ifinclasstrue
\def\skfootnote#1{\myfootnote{{\color{white!70!black}#1}}}
\fi

%%%%%%%%%%%%%%%%%%%%%%%%%% MAIN SECTION %%%%%%%%%%%%%%%%%%%%%%%%%%%%%%%%%


\section{Simple command creation and style files creation} 
%% It is just an empty TeX file.
%% Write your code here.
\graphicspath{{sec01/images/}{sec01/code/}}
\lstset{inputpath=sec01/code/}

\begin{frame}[fragile]{What is TikZ?}\relax
``TikZ ist kein Zeichenprogramm'' 

which translates to ``TikZ is not a drawing program''

TikZ defines a number of \TeX\ commands that produce graphics: \tikz \fill[orange] (1ex,1ex) circle (1ex); produced by \verb|\tikz \fill[orange] (1ex,1ex) circle (1ex);|
\end{frame}

\begin{frame}{Pros and Cons}

\Huge\centering Pros and Cons
     
\end{frame}

\begin{frame}[fragile]{Cons}\relax
     \begin{itemize}
        \item[$-$] it is most likely that you don't need TikZ
        \item[$-$] write visual-based thinks like graphics is really annoying in a not-WYSiWYG way
    \end{itemize}
    
\end{frame}

\begin{frame}{Pros}\relax
     \begin{itemize}
        \item[$+$] it is most likely that you need some TikZ elements
        \item[$+$] some graphics (graphs for example) are so good structured, that it is OK to program them
        \item[$+$] TikZ has perfect integration with \LaTeX\ (and beamer):
        \begin{itemize}
            \item You can use all \LaTeX commands inside TikZ, creating beautiful pictures with math 
            \item You can pose elements using TikZ
            \item You can show just part of the picture in beamer Overlays   
        \end{itemize}
        \item[$+$] You don't need to have an external file
        \item[$+$] TikZ is using in CV and lots of other templates. It is good to be able to read the code
    \end{itemize}
\end{frame}


\begin{frame}[fragile]{How to setup TikZ picture?}\relax

\verb|\usepackage{tikz}|

and then 

\verb|\begin{tikzpicture} <code> \end{tikzpicture}| or, for short inline graphics, \ccol\tikz. 



\skfootnote{\tikzc{12.1}[126]}
     
\end{frame}
\subsection{Command creation}
\graphicspath{{sec02/images/}}
\lstset{inputpath=sec02/code}

\begin{frame}{WYSiWYG vs not-WYSiWYG editors}
\cwpa{
{\csk WYSiWYG} -- \textit{What You See is What You Get} editor\\~\\

\inclassFrag{Examples?}[0]
}{
Microsoft Word
\insImg{wysiwyg.png}
}
\end{frame}


\begin{frame}[fragile]{not-WYSiWYG}


\cprotect\skfootnote{``\verb|\begin{frame}[fragile]|'' to put code inside}


\inclassFrag{Examples?}[1]
\cprotect[mm]\cwpa{
{\small HTML }

\begin{lstlisting}[language=html]
<html>
  <head>
    <meta charset="utf-8">
  </head>
  <body>
    <h1>Header</h1>
    <i>Hello</i>,<br/> world!  <!-- comment -->
  </body>
</html>
\end{lstlisting}

}{
{\small \LaTeX}
\lstinputlisting{helloworld.tex}

}[t]

\end{frame}

\begin{frame}[fragile]{Commands}
    \lstinline[basicstyle=\tt\large]|\command[o1, o2]{n1, n2=value}[o3]{n3}|\par
    (o = optional argument, n = necessary argument)~\\[2ex]

\pause
    \skfootnote{\knuthc{ch 3}, \lvoc{ch 2.3, 2.7}}

    {\csk Command symbols}
    \verb|\$ \# \{ \} \^{} \& \_ \~{} \\|
    \vfil
    {\csk Command words}
    \verb|\sin \LaTeX \Rightarrow \qquad|
    
    {\csk Enviruments}
    \verb|\begin{frame}\end{frame}  \begin{equation}\end{equation}|
    
\end{frame}

\begin{frame}[fragile]{Document structure}{overview}
    \lstinputlisting{structureOfDocument.tex}
\end{frame}

\begin{frame}[fragile]{Document structure}{class files}

Class of the document responsable for the large-scale settings
    \cprotect\skfootnote{\lvoc{IV.1}, \lamc{2.2.2} \url{https://texblog.org/2013/02/13/latex-documentclass-options-illustrated/}\\ envirament \lstinline|tabbing|} 
{
\scriptsize
\begin{tabbing}
\lstinline|\documentclass|\hspace{-1ex} \= \lstinline|[10pt,|  \= \lstinline|onecolumn,|  \=  \lstinline|a4paper]|\hspace{-1ex} \= \lstinline|{article}| \kill
\> \> \> \> \lstinline|{beamer} %presentation, poster| \\
\> \> \> \> \lstinline|{report}| \\
\> \> \> \> \lstinline|{book}| \\
\> \> \> \> \lstinline|{standalone} %for one picture/equation| \\
\> \> \> \> \lstinline|{extarticle} %if you want 14pt font size| \\
\lstinline|\documentclass| \> \lstinline|[10pt,|  \> \lstinline|onecolumn,|  \>  \lstinline|a4paper]| \> \lstinline|{article}| \\ 
\> \lstinline|[12pt] %fontsize| \> \\ 
\> \> \lstinline|[twocolumns] %number of columns in document| \> \\ 
\> \> \> \lstinline|[a5paper] %paper size| \> \\ 
\end{tabbing}}
\cprotect\skfootnote{\slshape in this presentation \verb|[aspectratio=169]|}
\end{frame}
\note{
but there is no ``fixed set'' of classes. Can build our own 
}

%%TSK: предложить создать документ и поиграться с параметрами
\begin{frame}[fragile]{Document structure}{style files}
     Style files are responsable for settings and providing new commands 
     
     \vfill
     \hspace{-1ex}
     \lstinline[basicstyle=\tt\large]|\usepackage[optional]{necessary}{packagename}|
     \vfill
\end{frame}

\note{practicully everything is from package }


\inclassFrame{
\begin{frame}{This slide means it is ``in-class task time!''}\relax
\centering
     \LARGE This slide means it is ``in-class task time!''
\end{frame}
}
\cprotect\inclassFrame{
\begin{frame}[fragile]{Try to compile your first doc}\relax
     \lstinputlisting[basicstyle=\tt\small]{first.tex}
\end{frame}
}
\subsection{Style and class files}
%% It is just an empty TeX file.
%% Write your code here.
\graphicspath{{sec03/images/}{sec03/code/}}
\lstset{inputpath=sec03/code/}


\section{Programming}
%% It is just an empty TeX file.
%% Write your code here.
\graphicspath{{sec06/images/}{sec06/code/}}
\lstset{inputpath=sec06/code/}

\subsection{Define macros}


\begin{frame}\relax
    
    {\centering\Huge \TeX\ is Turing-complete language
    
    }
    \vspace{1em}
    \inclassFrag{Who understand, what does it means?}
    In basic words, it means, that you can write in \TeX\ and \LaTeX\ any algoritms, that you can write in C++, Java, Python... Moreover, some \TeX syntax is really familiar to functional languages
    \skfootnote{\vspace{-3ex}\url{https://stackoverflow.com/questions/2968411/ive-heard-that-latex-is-turing-complete-are-there-any-programs-written-in-late} \overC{https://www.overleaf.com/learn/latex/Articles/LaTeX_is_More_Powerful_than_you_Think_-_Computing_the_Fibonacci_Numbers_and_Turing_Completeness} \url{http://sdh33b.blogspot.com/2008/07/icfp-contest-2008.html}}
     
\end{frame}


\begin{frame}{Reminder: Define macros \tW\magicPage}
     In \TeX\ you can define new macros via \ccol\def.

    \twocolImg{
    % \lstinputlisting[linerange={8-9, 14-16}]{commandwith.tex}
        \inputminted[firstline=8, lastline=9]{latex}{sec06/code/commandwith.tex}
        \inputminted[firstline= 14, lastline=16]{latex}{sec06/code/commandwith.tex}
    }{commandwith}
     
    Use \ccol\global\ prefix to define macros not just inside ``group''. 
    
    Use \ccol\long\ prefix to define macros that can have multiple paragraphs as an argument.
    
    \skfootnote{\tugC{https://www.tug.org/utilities/plain/cseq.html\#def-rp} \knuthc{20}[209]}
\end{frame}

\begin{frame}{Define with pattern matching\tW\magicPage}\relax

    The syntax with writing each argument seems to be an over-use. But it is needed because of \textit{pattern matching}
      
    \twocolImg{
    % \lstinputlisting[linerange={11-13}]{commandpattern.tex}
        \inputminted[firstline=11, lastline=13]{latex}{sec06/code/commandpattern.tex}
    }{commandpattern}   

\end{frame}

\subsection{Conditions}

\begin{frame}[fragile]{Compare strings (macros)\tW}\relax

    \twocolImg{
    % \lstinputlisting[linerange={12-21}]{ifmy.tex}
    \inputminted[firstline=12, lastline=22]{latex}{sec06/code/ifmy.tex}
    }{ifmy}   

    {\csk \verb|\ifx\<first>\<second>  <code1>  [\else  <code2>]  \fi|}
    
\end{frame}

\begin{frame}[fragile]{Compare numbers\tW}\relax

    \twocolImg{
    % \lstinputlisting[linerange={12-20}]{ifnummy.tex}
    \inputminted[firstline=12, lastline=26]{latex}{sec06/code/ifnummy.tex}
    }{ifnummy}   

    {\csk \verb|\ifnum<first><operator><second>  <code1>  [\else  <code2>]  \fi|}. Only ``='', ``>'' or ``<'' are allowed.
    
    Use {\csk \verb|\ifcase|} to check different stuff. Also use {\csk \verb|\ifodd|} to check if num is odd or even
    
\end{frame}

\begin{frame}{Compare in \LaTeX\lW}\relax
    \twocolImg{
    % \lstinputlisting[linerange={7-7, 12-16}]{ifstring.tex}
        \inputminted[firstline=7, lastline=7]{latex}{sec06/code/ifstring.tex}
        \inputminted[firstline= 12, lastline=16]{latex}{sec06/code/ifstring.tex}
    }{ifstring}  
      \ncol\usepackage{xstring}
      
      also see \ncol\usepackage{ifthen}
      
      \hrule 
      
      you can check if you are in \XeLaTeX by \ncol\usepackage{ifxetex}
      
      \skfootnote{read about the xstring package! It has lots of things: strings, substrings,.. what to do with them,.. replace strings etc\\ \stExC{https://tex.stackexchange.com/questions/47576/combining-ifxetex-and-ifluatex-with-the-logical-or-operation}}
\end{frame}

\begin{frame}[fragile]{Check modes\tW\magicPage}\relax

    \twocolImg{
    % \lstinputlisting[linerange={12-22}]{ifmodemy.tex}
    \inputminted[firstline=12, lastline=22]{latex}{sec06/code/ifmodemy.tex}
    }{ifmodemy}   
    \begin{itemize}
        \item {\csk\verb|\ifmmode|} to check if in mathematical mode
        \item {\csk\verb|\ifvmode|} to check if in vertical mode
        \item {\csk\verb|\ifhmode|} to check if in horizontal mode
        \item {\csk\verb|\ifinner|} to check if \TeX\ is in internal vertical mode, or restricted horizontal mode, or (nondisplay) mathmode
     
    \end{itemize}
    
    \skfootnote{other ``if'''s can be found in \knuthc{20}[221]}
\end{frame}



%%%%%%%%%%%%%%%%%%%%%%%%%%% LOOPS %%%%%%%%%%%%%%%%%%%%%%%%%%%%%%%%
\subsection{Loops and recursion}

\begin{frame}[fragile]{Loop \tW}\relax
    \twocolImg{
        % \lstinputlisting[linerange={12-17}]{loopmy.tex}
        \inputminted[firstline=12, lastline=17]{latex}{sec06/code/loopmy.tex}
    }{loopmy}

    \ccol\loop\ for start loop, <code> inside, then {\csk\verb|\if<..>|}-family, another bunch of <code>, ended with \ccol\repeat.

     \skfootnote{\knuthc{20}[228]}
\end{frame}



\begin{frame}[fragile]{For loop\lW}\relax
\label{sl:for}
    \twocolImg{
    % \lstinputlisting[linerange={8-8, 13-16}]{forloopmy.tex}
        \inputminted[firstline=8, lastline=8]{latex}{sec06/code/forloopmy.tex}
        \inputminted[firstline= 13, lastline=16]{latex}{sec06/code/forloopmy.tex}
    }{forloopmy}
    \ncol \usepackage{forloop}
    
    \twocolImg{
    % \lstinputlisting[linerange={8-8, 13-15}]{foreachmy.tex}
        \inputminted[firstline=8, lastline=8]{latex}{sec06/code/foreachmy.tex}
        \inputminted[firstline= 13, lastline=18]{latex}{sec06/code/foreachmy.tex}
    }{foreachmy}
    
    \ncol \usepackage{pgffor}, part of pgf, part of TikZ
    
    \skfootnote{\url{https://stackoverflow.com/questions/2561791/iteration-in-latex}}
\end{frame}

\begin{frame}[fragile]{Reqursion \tW\magicPage}\relax
    \twocolImg{
    % \lstinputlisting[linerange={12-13}]{reqmy.tex}
    \inputminted[firstline=12, lastline=13]{latex}{sec06/code/reqmy.tex}
    }{reqmy}

\end{frame}

%%%%%%%%%%%%%%%%%%% related %%%%%%%%%%%%%%%%%%%%%%%%%%%%%%%%%%
\subsection{Related things to macros creation}

\begin{frame}[fragile]{``let'' command\magicPage}\relax

    \twocolImg{
    % \lstinputlisting[linerange={11-21}]{letmy.tex}
        \inputminted[firstline=11, lastline=21]{latex}{sec06/code/letmy.tex}
    }{letmy}
    
    
    The statement ``\ccol\let\verb|\a=\b|'' gives \string\a\ the current meaning of \string\b. If \string\b\ changes after the assignment is made, \string\a\ does not change.
    
    \skfootnote{\knuthc{20}[217] \tugC{https://www.tug.org/utilities/plain/cseq.html\#let-rp}}
     
\end{frame}

\begin{frame}[fragile]{Usecase with \ccol\let: ``decorator''\magicPage}\relax

    Imagine: you have some \string\command\ used inside the document multiple times. You want to \textit{add some addition behaviour} to the command -- decorate (or ``wrap'', or ``redefine with the use of itself''). You can do it with \ccol\let:
    
    \begin{minted}{latex}
    \let\oldCommand=\command
    \def\command#1{<some code>\oldCommand}
    \end{minted}
    
    And the same for enviruments using \ncol\usepackage{etoolbox} or \ccol{\g@addto@macro}
    
    
    
    \skfootnote{\stExC{https://tex.stackexchange.com/questions/47351/can-i-redefine-a-command-to-contain-itself} \stExC{https://tex.stackexchange.com/questions/467435/environment-decorators}}
     
\end{frame}


\subsection{Programming examples}

\begin{frame}[fragile]{99 Bottles of Beer\magicPage}\relax

    \twocolImg{
        % \lstinputlisting[linerange={9-26}]{bottles.tex}
        \inputminted[firstline=9, lastline=26]{latex}{sec06/code/bottles.tex}
    }{bottles}
    
    \skfootnote{\url{https://rosettacode.org/wiki/99_Bottles_of_Beer\#LaTeX}}
     
\end{frame}

\begin{frame}[fragile]{not-AND logical gate\magicPage}\relax

    \twocolImg{
    % \lstinputlisting[linerange={11-29}]{nand.tex}
        \inputminted[firstline=11, lastline=29]{latex}{sec06/code/nand.tex}
    }{nand}
    
    
\end{frame}

\begin{frame}[fragile]{Split words\magicPage}\relax
    \twocolImg{
    % , basicstyle=\tiny
        % \lstinputlisting[linerange={11-31}]{splitmy.tex}
        \inputminted[firstline=11, lastline=31, fontsize=\tiny]{latex}{sec06/code/splitmy.tex}
    }{splitmy}

     \skfootnote{\stExC{https://tex.stackexchange.com/questions/12810/how-do-i-split-a-string/12811}}
\end{frame}

\subsection{Relatate stuff}
\graphicspath{{sec08/images/}{sec08/code/}}
\lstset{inputpath=sec08/code/}


\begin{frame}{Filesystem writing}
    \twocolImg{
    % \lstinputlisting[linerange={11-17}]{writedoc.tex}
    \inputminted[firstline=11, lastline=17]{latex}{sec08/code/writedoc.tex}
    }{writedoc}
    
    \footnotesize
    The output in pdf is the result of listing the {\csk outfile.txt}
    \begin{itemize}
        \item \ccol\newrite\ to get a free register
        \item \ccol\openout\ to open filename for output
        \item \ccol\write<register>\ to actually write 
        \item \ccol\noexpand\ to stop expanding (also see \ccol\expandafter)
        \item \ccol\closeout\ to close the file 
    \end{itemize}
    
    You also can use numbers. Like \ccol\write4

\end{frame}

\begin{frame}[fragile]{Filesystem reading}
    \twocolImg{
    \inputminted[firstline=18, lastline=31, fontsize=\tiny]{latex}{sec08/code/readdoc.tex}
    }{readdoc}
    
    \footnotesize
    \begin{itemize}
        \item \ccol\newread\ to get a free register
        \item \ccol\openin\ to open filename for input
        \item \ccol\read<register>\ \ccol{to}\string\<newvariable> to actually read
        \item \verb"\ifeof" to check if file is still have lines
        \item \ccol\closein\ to close the file 
    \end{itemize}
    
    You also can use numbers. Like \ccol\read4

\end{frame}


\begin{frame}[fragile]{How BiBLaTeX works? Proof-of-concept\magicPage}\relax
    \twocolImg{
    % basicstyle=\tt\tiny
        \inputminted[firstline=8, lastline=22, fontsize=\tiny]{latex}{sec08/code/PoCIO.tex}
        \inputminted[firstline=28, lastline=31, fontsize=\tiny]{latex}{sec08/code/PoCIO.tex}
    }{PoCIO}
    
    \footnotesize
    \begin{enumerate}
        \item write info into a file 
        \item use an external command to do something with the file 
        \item read content from a file in a different place
         
    \end{enumerate}

\end{frame}

\begin{frame}[fragile]{Use command line\magicPage}\relax
        \twocolImg{
    % \lstinputlisting[linerange={12-14}]{commandline.tex}
    \inputminted[firstline=12, lastline=16]{latex}{sec08/code/commandline.tex}
    }{commandline}
    \small
    
    Use 
    \begin{itemize}
        \item \ccol\write18\ to call the command line 
        \item \ccol\immidiate\ to run it as it reached (otherwise only when \TeX\ will ``print'' the page)
        \item use \verb|--enable-write18 -interaction=nonstopmode| keys for run offline
        \item the commands with internet connection will not work at Papeeria
         
    \end{itemize}
    \skfootnote{\stExC{https://tex.stackexchange.com/questions/20444/what-are-immediate-write18-and-how-does-one-use-them}. Also look at \ccol{filecontents} env.}
\end{frame}

\begin{frame}[fragile]{Command names manipulation\magicPage}\relax
    \twocolImg{
    % \lstinputlisting[linerange={12-13}]{csnamemy.tex}
    \inputminted[firstline=12, lastline=13]{latex}{sec08/code/csnamemy.tex}
    }{csnamemy}
    
    \begin{itemize}
        \item \ccol\csname\ --- \ccol\endcsname\ to ``compile'' command from name 
        \item \ccol\string\ to show the name 
         
    \end{itemize}

\end{frame}

\begin{frame}[fragile]{Catcodes\magicPage}\relax

    \twocolImg{
    % \lstinputlisting[linerange={12-18}]{catcodesmy.tex}
    \inputminted[firstline=12, lastline=18]{latex}{sec08/code/catcodesmy.tex}
    }{catcodesmy}

    \ccol\catcode\ shows what symbol will be responsible for the group, what for comment etc.

    \skfootnote{\tugC{https://www.tug.org/utilities/plain/cseq.html\#catcode-rp} \knuthc{app. B}[354]}
\end{frame}

\begin{frame}{Expanding mechanism\magicPage}\relax

    \begin{itemize}
        \item \TeX\ support overriding practically everything
        \item You need a detail mechanism sometimes: so you want to run a command, print it or just put it deeper in macros 
        \item \ccol\expandafter\ expands the token after the next one before the next token itself.
        \item \ccol\noexpand\ Prevents the next token from being expanded.
    \end{itemize}

    \skfootnote{\overC{https://www.overleaf.com/learn/latex/Articles/How_does_\%5Cexpandafter_work\%3A_The_meaning_of_expansion} \overC{https://www.overleaf.com/learn/latex/Articles/How_does_\%5Cexpandafter_work\%3A_A_detailed_macro_case_study}}     
\end{frame}

\begin{frame}[fragile]{Lua\TeX\ Proof-of-concept\magicPage}\relax
    \twocolImg{
    % \lstinputlisting[linerange={8-8, 10-13}]{luamy.tex}
    \inputminted[firstline=8, lastline=8]{latex}{sec08/code/luamy.tex}
    \inputminted[firstline= 10, lastline=13]{latex}{sec08/code/luamy.tex}
    }{luamy}
         
    \skfootnote{\stExC{https://tex.stackexchange.com/questions/70/what-is-a-simple-example-of-something-you-can-do-with-luatex}}
\end{frame}

% %write/read, expand,...

\section{Debugging}
%% It is just an empty TeX file.
%% Write your code here.
\graphicspath{{sec05/images/}{sec05/code/}}
\lstset{inputpath=sec05/code/}

\subsection{Show-family}

\begin{frame}[fragile]{Expand macros\magicPage}\relax

\logshow{showmy}{7-10,38-41}{1-18}

\ccol{\show<macros>} 
     \skfootnote{\tugC{https://www.tug.org/utilities/plain/cseq.html\#show-rp} \knuthc{3}[21] \stExC{https://tex.stackexchange.com/questions/364942/show-family-debugging-commands}}
\end{frame}

\begin{frame}[fragile]{Expand macros\magicPage}{\ccol{show} with not macros}\relax

\logshow{showmy}{12-15,43-44}{21-26}
\end{frame}

\begin{frame}[fragile]{Show length and counts\tW\magicPage}\relax

\logshow{showthemy}{12-20,39-45}{1-14}
\ccol{\showthe<var>}
% TODO: \showthe\font
\skfootnote{\url{https://texfaq.org/FAQ-printvar} \tugC{https://www.tug.org/utilities/plain/cseq.html\#showthe-rp} \knuthc{15}[132]}
\end{frame}


\begin{frame}[fragile]{Show boxes\tW\magicPage}\relax

\logshow{showboxmy}{31-39}{1-14}
\ccol{\showbox<box>}

\skfootnote{\url{https://texfaq.org/FAQ-printvar} \tugC{https://www.tug.org/utilities/plain/cseq.html\#showbox-rp} \tugC{https://www.tug.org/utilities/plain/cseq.html\#showboxbreadth-rp} \tugC{https://www.tug.org/utilities/plain/cseq.html\#showboxdepth-rp} \knuthc{11}[77]\\ \ccol{\showboxbreadth} and \ccol{\showboxdepth} provides how much info to show}
\end{frame}

% \begin{frame}[fragile]{Show stack of modes\tW\magicPage}\relax

% \logshow{showlistsmy}{12-17}{1-18}
% \ccol{\showlists}
% \skfootnote{\vspace{-3ex}\tugC{https://www.tug.org/utilities/plain/cseq.html\#showlists-rp}  \knuthc{13}[99]\\``it causes TEX to display the lists that are being worked
% on, in the current mode and in all enclosing modes where the work has been suspended:''}
% \end{frame}

\begin{frame}{Show-family list}\relax
\small
\begin{description}
    \item[\ccol{\show}] log macros insides
    \item[\ccol\showthe] log length or counter value
    \item[\ccol\showbox] log box insides 
    \begin{description}
        \item[\ccol\showboxdepth] the value of the deepest level of box nesting
        \item[\ccol\showboxbreadth] the maximum number of items shown per level 
    \end{description}
    \item[\ccol\showlists] writes the content of partial box lists in all of the 4 non-math TeX modes
    \item[\ccol\showhyphens\{W\}] displays the hyphenation of W on the terminal/log according to the hyphenation rules.
\end{description}
     \skfootnote{\stExC{https://tex.stackexchange.com/questions/364942/show-family-debugging-commands}}
\end{frame}

\subsection{Tracing-family}

\begin{frame}[fragile]{Trace modes and commands\tW\magicPage}\relax
\logshow{tracingcommandsmy}{11-11,16-18}{1-10}
\ccol{\tracingcommands=1}

\skfootnote{\tugC{https://www.tug.org/utilities/plain/cseq.html\#tracingcommands-rp} \knuthc{13}[99]}
\end{frame}

\begin{frame}[fragile]{Trace macros (recursively)\tW\magicPage}\relax
\logshow{tracingmacrosmy}{7-10,15-20}{1-15}
\ccol{\tracingmacros=1}

\skfootnote{\tugC{https://www.tug.org/utilities/plain/cseq.html\#tracingmacros-rp} \knuthc{20}[216]}
\end{frame}


\begin{frame}{Tracing-family list}\relax
\footnotesize

\begin{description}
    \item[\ccol\tracingcommands] if positive, writes commands to the .log file
    \item[\ccol\tracinglostchars] if positive, writes characters not in the current font to the .log file
    \item[\ccol\tracingmacros] if positive, writes to the .log file when expanding macros and arguments
    \item[\ccol\tracingonline] if positive, writes diagnostic output to the terminal as well as to the .log file
    \item[\ccol\tracingoutput] if positive, writes contents of shipped out boxes to the .log file
    \item[\ccol\tracingpages] if positive, writes the page-cost calculations to the .log file
    \item[\ccol\tracingparagraphs] if positive, writes a summary of the line-breaking calculations to the .log file
    \item[\ccol\tracingrestores] if positive, writes save-stack details to the .log file
    \item[\ccol\tracingstats] if positive, writes memory usage statistics to the .log file
    \item[\ccol\tracingall] turns on every possible mode of interaction
\end{description}
     \skfootnote{\stExC{https://tex.stackexchange.com/questions/60491/latex-tracing-commands-list}}
\end{frame}

\subsection{Other debugging ways}

\begin{frame}[fragile]{Message to log file\magicPage}\relax
\logshow{messagemy}{7-7,14-15,17-18,31-32,15-20}{1-2}
\ccol{\message\{<msg>\}} -- \TeX-command, \ccol{\typeout\{<msg>\}} -- \LaTeX-command

\skfootnote{\tugC{https://www.tug.org/utilities/plain/cseq.html\#message-rp} \knuthc{20}[228] \lmanc{27.2}[228]}
\end{frame}

% 

\graphicspath{{code_ex/}}
\lstset{inputpath=code_ex/}
\cprotect\inclassframe{
\begin{frame}{\exFrame{Implement for loop}}{lvl 1}

     Implement a function that, given the syntax in the code on the left, would output the result as in the image on the right.
    \twocolImg{
    \inputminted[firstline=24, lastline=27]{latex}{code_ex/exForLvl1.tex}
    }{exForLvl1.pdf}
    
    Tips: 
    \begin{itemize}
        \item You need a counter
        \item Use \ccol\foreach
        \item if you increment counter, but it remains the same, try increment in with \ccol\global
    \end{itemize}
\end{frame}
\begin{frame}[fragile]{\exFrame{Implement for loop}}{lvl 2}
     Implement a function that, given the syntax in the code on the left, would output the result as in the image on the right. You can use \ccol\foreach, but not any of addition packages
    \twocolImg{
    \inputminted[firstline=24, lastline=28]{latex}{code_ex/exForLvl2.tex}
    }{exForLvl2.pdf}
    
    Tips: 
    \begin{itemize}
        \item The problem will be how to to split \{2-5\}\{Caption 1\}
        \item You may need a \ccol\expandafter
        \item Create a helper function that would do 1 line and play around with how you can pass parameters to it
    \end{itemize}
\end{frame}
\begin{frame}[fragile]{\exFrame{Implement for loop}}{lvl 2. Hardcore mode!}
     Implement a function that, given the syntax in the code on the left, would output the result as in the image on the right. You can't use \ccol\foreach\ or any of addition packages! Try use \TeX\ power
    \twocolImg{
    \inputminted[firstline=31, lastline=35]{latex}{code_ex/exForLvl2.5.tex}
    }{exForLvl2.pdf}
    
    Tips: 
    \begin{itemize}
        \item You need to use pattern matching tehnique
        \item You need to use reqursion
        \item One of your helper macros may have the following declaration: \verb|\def\<funName>#1#2, #3\relax|
    \end{itemize}
\end{frame}
}
\cprotect\inclassframe{
\begin{frame}[fragile]{\exFrame{Implement for loop}}{lvl 1: more tips}\relax
\pause
\begin{itemize}
    \item Firstly try to make it without counter \pause
    \item The \ccol\foreach\ command was in the slide \ref{sl:for} \pause
    \item There the format like ``\verb|apples,burgers,cake|'', you have ``\verb|{Caption 1}, {Caption, 2}|''. But braces just indicates atomicy (slide \ref{sl:atom})! \pause
    \item So, ``\verb|\foreach \n in {apples,burgers,cake}|'' is the same as ``\verb|\foreach \n in {{Caption 1}, {Caption, 2}}|''\pause
    \item The only difference if you just pass ``\verb|\foreach \n in #1|'' it will be expanded as ``\verb|\foreach \n in {Caption 1}, {Caption, 2}|'', not ``\verb|\foreach \n in {{Caption 1}, {Caption, 2}}|''. Guess what to do!\pause
    \item After stuff without counter works, try to add a counter. How to make counter see at previous lecture. Don't forget to use \ccol\global\ if you try to add 1 to it at it is not working
\end{itemize}
     
\end{frame}
}

\progressend


\begin{frame}\frametitle{What we have learned today?}\relax
    \tableofcontents
\end{frame}

\begin{frame}[allowframebreaks]{references}
color from the footnotes corresponds to references' color.
    \begin{itemize}
        \item \knuthc{Knuth ``The \TeX Book''}
        \item \lvoc{L'vovsky ``Nabor i verstka v sisteme \LaTeX''}
        \item \lamc{Lamport. ``\LaTeX. A Document Preparation System, User’s Guide and Reference Manual''}
        \item \lmanc{``\LaTeX 2e: An unofficial reference manual''} also at website \url{https://latexref.xyz/}
        \item \stExC{https://tex.stackexchange.com/questions} : \url{https://tex.stackexchange.com/questions}
        \item \wikiC{https://en.wikibooks.org/wiki/LaTeX} : \url{https://en.wikibooks.org/wiki/LaTeX}
        \item \overC{https://www.overleaf.com/learn/latex} : \url{https://www.overleaf.com/learn/latex}
        \item \tugC{https://www.tug.org/utilities/plain/cseq.html} : \url{https://www.tug.org/utilities/plain/cseq.html}
    \end{itemize}
\end{frame}

\begin{frame}{Distribution}\relax
\begin{itemize}
     \item the pdf-version of the presentation and all printed materials can be distributed under license Creative Commons Attribution-ShareAlike 4.0 \url{https://creativecommons.org/licenses/by-sa/4.0/}
     \item The source code of the presentation is available on {\csk\url{https://github.com/Lavton/latexLectures}} and can be distributed under the MIT license \url{https://en.wikipedia.org/wiki/MIT_License\#License_terms}
\end{itemize}
     
\end{frame}
\end{document}