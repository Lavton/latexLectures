%https://tex.stackexchange.com/questions/114219/add-notes-to-latex-beamer
\documentclass[12pt]{beamer} %, aspectratio=169
\usepackage{fontspec}
\usepackage{xunicode}
\usepackage{xltxtra}
\usepackage{dsfont}
\usepackage{multicol}
\usepackage{array}
\usepackage[darktherm, useprogressbar, classmode=outclass]{cscbeamer}

\usepackage{lecturelatex}

\usepackage{xecyr}
\usepackage{hyperref}

\usepackage{polyglossia}
\usepackage{graphicx}
\usepackage{listings}
% \renewcommand{\outClass}[1]{}
\newcommand{\lookCode}[1]{\inclassframe{\begin{frame}{Посмотрим на код}\relax


#1
\end{frame}}}
\newdimen\standartDeptht%
\newcommand{\rulecommand}[2][12pt]{\smash{{%
    \standartDeptht=#1%
    \multiply\standartDeptht by #2%
    \advance\standartDeptht by 1pt%
    \color{cscgreen!50!cscdarkback}{\vrule depth\standartDeptht height 3pt width 0.7pt}}}%
}

\begin{document}


\title{Продвинутый \LaTeX}
\subtitle{Типография и работа с командами}
\author{Антон Лиознов}
\institute{CSC}
\date{\the\year}
\frame{\titlepage}

\AtBeginSection[] 
{
  \begin{frame}{Итак, сегодня...}
  \tableofcontents[currentsection,hideallsubsections]
  \end{frame}
}

\AtBeginSubsection[]
{
  \begin{frame}{Итак, сегодня...}
  \tableofcontents[currentsubsection, hideothersubsections, sectionstyle=show/hide, subsectionstyle=show/shaded/hide]
  
  \end{frame}
}
% основа:
% https://github.com/Lavton/latexLectures/blob/master/LaTeX04_typography.pdf
% https://github.com/Lavton/latexLectures/blob/master/LaTeX05_command_creation.pdf
\graphicspath{{sec00/images/}{sec00/code/}}
\lstset{inputpath=sec00/code/}

%%%%%% предыстория лекция Смаля
\begin{frame}{Предыстория}\relax
    \centering
    \includegraphics[width=0.9\linewidth]{prevLaTeX}
    % В 2015 году была лекция ``LaTeX: краткое введение в качественную типографику'' от Александра Владимировича Смаля
    
    % \cscfootnote{aaaa}
    \cscfootnote{\url{https://compscicenter.ru/videos/latex/}}
     
\end{frame}

\outclassframe{
\begin{frame}{Предыстория}\relax
    \begin{itemize}
        \item В 2015 году в CSC была лекция ``LaTeX: краткое введение в качественную типографику'' от Александра Владимировича Смаля
        \item Она знакомила с основами \LaTeX\ и показывала прелесть этой системы
        \item Если вы видите эти строки в записи и лишь начинаете изучать \LaTeX, советую ознакомиться с лекцией
        \item И прочитать книгу Львовского ``набор и верстка в системе \LaTeX''
    \end{itemize}
    
    
    \cscfootnote{\url{https://compscicenter.ru/videos/latex/}}
     
\end{frame}
}

\begin{frame}{Что сегодня узнаем?}\relax
    \begin{enumerate}
        \item Примитивы в \LaTeX
        \begin{enumerate}
            \item Как \TeX\ видит наш документ: боксы и клей
            \item Какие примитивы существуют: длины, счётчики и другое
            \item Как манипулировать примитивами
        \end{enumerate}
        
        \item Программирование в \TeX\ и \LaTeX
        \begin{enumerate}
            \item Создание макросов
            \item Условные операторы
            \item Циклы и рекурсия
            \item Операции ввода-вывода
            \item Отладка
        \end{enumerate}
         
    \end{enumerate}
    \inpause Эта лекция для тех, кто уже знает \LaTeX\ хотя бы на уровне книги Львовского
\end{frame}


\begin{frame}{Для чего вам эти знания}\relax
    \moveleft1.4cm\hbox{\vbox{
     \begin{description}
         \item[\textbf{{\huge-}}]~
         \begin{itemize}
              \item Скорее всего не понадобится
              \item Документы можно составлять и из готовых шаблонов
         \end{itemize}\pause
         \item[\textbf{{\huge+}}]~
         \begin{itemize}
              \item Для общей эрудиции
              \item ``подправлять'' используемые шаблоны 
              \item писать свои шаблоны
              \item автоматизировать работу
         \end{itemize}
     \end{description}
     }}
\end{frame}


\begin{frame}{Обо мне}\relax
     \begin{itemize}
         \item закончил CSC в 2015 году
         \item стажировался в Papeeria, онлайн \LaTeX -редакторе
         \item младший научный сотрудник в Сколковском институте науки и технологий
          
     \end{itemize}
\end{frame}

{\supressfootnotefalse
\begin{frame}{Некоторые соглашения}{Сноски}\relax
    
    \cscfootnote{Как эта}
    \begin{itemize}
         \item Для повторного прочтения
         \item Некоторые детали по работе с командами
         \item Ссылки на источники
         \item Комментарии
         \item будут видны вне класса
    \end{itemize}
\end{frame}
}

\begin{frame}{Некоторые соглашения}{``магические'' слайды}\relax
Слайды с дополнительной информацией
\magicPage

\begin{itemize}
     \item Для полноты картины
     \item Но не для анализа в классе
\end{itemize}
\end{frame}

{\renewcommand{\inclass}[1]{#1}\renewcommand{\outclass}[1]{#1}
\inclassframe{
\begin{frame}{Некоторые соглашения}{Слайды только для класса}\relax
     Такие слайды исчезнут в выкладываемой лекции.~\\~\\

    \inclasshigh{Так обозначаем сноску, которая будет только в классе видна}
\end{frame}
}
\outclassframe{
\begin{frame}{Некоторые соглашения}{Слайды только для чтения}\relax
    Такие слайды появятся в выкладываемой лекции.~\\~\\

    \outclasshigh{Так обозначаем сноску, которая будет только вне классе видна}
\end{frame}
}
}

\begin{frame}{Задача -- сделать шаблон выступлений}\relax
    Наша практическая задача сегодня-- реализовать шаблон, с которым я рассказываю =)
    \begin{itemize}
        \item плашка слева
        \item номера страниц и логотип
        \item возможность писать сноски 
        \item работа с заголовком
        \item и всё это автоматизировать!
    \end{itemize}
    
    \outclasshigh{\textit{примечание:} моя задача -- проиллюстрировать основные идеи. Не везде предлагаемое решение будет State~of~the~Art, где-то будет немного костылей. Но будет работать}
    
\end{frame}
%% 7-8 минут по сюда

%% start point
\outclassframe{
\begin{frame}{Начнём с пустого шаблона}\relax
     \centering
    \fbox{\includegraphics[width=0.9\linewidth]{01_init}}
    \outclass{\cscfootnote{см папку ``01\_init''}}
\end{frame}}

\lookCode{Сделаем простой файл презентации}

%%%%%%%%%%%%%%%%%%%%%%%%%%%%%%%%% создание команд 
\section{Слегка продвинутый \LaTeX: Типографика и создание команд}
\subsection{Простое создание команд}


\begin{frame}{Где пишем код}\relax
     Код можно писать 
     \begin{itemize}
         \item Прям в начале документа
         \item Внутри \{\} (будет локален)
         \item В классах
         \item В стилевых файлах
     \end{itemize}
     \outclass{На уровне языка отличий класса от стиля нет. Концептуальное различие: класс это нечто глобальное, стили могут подходить к разным классам.}
\end{frame}


\begin{frame}[fragile]{Создание команд}\relax
    
    % \lstset
    % {
    %     basicstyle=\tt\normalsize,
    % }
    \footnotesize
     \begin{tabbing}
     
        \lstinline|\newcommand{|\= \lstinline|\mycommand| \= \lstinline|[1]|  \=  \lstinline|{something #1}|\kill

        \rulecommand{1}\lstinline|\newcommand{|\> \rulecommand{2}\lstinline|\mycommand}| \> \rulecommand{3}\lstinline|[1]|  \>  \rulecommand{4}\lstinline|{something #1}| \\ \footnotesize
        создаём команду\>  \>   \>  \\ 
        \>\footnotesize имя команды \>   \>  \\ 
        \>  \>\footnotesize число аргументов (может  отсутствовать)  \>  \\
        \>  \>  \>\footnotesize тело \\
    \end{tabbing}
    
    \cprotect\outclasshigh{\lstinline|\renewcommand| чтобы пересоздать}
\end{frame}

\lookCode{Создадим команду для быстрого набора ``Computer Science Center''}

\cprotect\outclassframe{
\begin{frame}[fragile]{Начало .cls и .sty файлов}\relax
     Класс:
     \begin{lstlisting}
     \NeedsTeXFormat{LaTeX2e}
     \ProvidesClass{<class-name>}[<date in YYYY/MM/DD> <other info>]
     \end{lstlisting}
     
     Стиль:
     \begin{lstlisting}
     \NeedsTeXFormat{LaTeX2e}
     \ProvidesPackage{<package-name>}[<date in YYYY/MM/DD> <other info>]
     \end{lstlisting}
     
     
     \cscfootnote{\lmanc{3.3.1}[19] \normalfont\url{https://www.latex-project.org/help/documentation/clsguide.pdf}}
    
\end{frame}
}

\begin{frame}{Специальный синтаксис}\relax
    Внутри пакетов синтаксис меняется:

    {\centering
    \begin{tabular}{rcl}
         \ccol\newcommand& $\to$ & \ccol\providecommand\\
         \ccol\usepackage & $\to$ & \ccol\RequirePackage\\
         \ccol\documentclass & $\to$ & \ccol\LoadClass
    \end{tabular}}
    ~\\~\\ 
        

    \inclasshigh{\pause Обратите внимание на нотацию двух последних команд}

    \inpause
    \outclasshigh{происходят дополнительные проверки на отсутствие дублирования}

\end{frame}

\lookCode{Переименуем создание команды \ccol\CSC,\\ Создадим новую команду, выделяющую по цвету\pause 

Вопрос: как сделать линейку сверху слайда?
}

% %%%%%%%%%%%%%%%%%%%%%%%%%%%%%%%%%%%%%%%%%%%%%%%% Длины %%%%%%%%%%%%%%%%%%%%%%

\subsection{Длины: единицы измерения}
\def\showLength#1{
    \raise4pt\hbox{%
        \vrule height 6pt depth 2pt%
        \highgreen{\rule{#1}{4pt}}%
        \vrule height 6pt depth 2pt%
    } 
    #1
}

\begin{frame}{Длины}{абсолютные значения}\relax

    \centering
    чаще всего используется:
    
    \begin{tabular}{r|cc|l}
         pt& points & $\simeq$0.35mm & \showLength{12pt} \\
         mm& millimeters & $\simeq$2.84pt & \showLength{10mm} \\
         cm& centimeter & $\simeq$28.4pt, 10mm & \showLength{1cm} \\
         in& inch & $\simeq$72.27pt, 25.4mm  & \showLength{1in} \\
    \end{tabular}
    
    
    \cscfootnote{\wikiC{https://en.wikibooks.org/wiki/LaTeX/Lengths} \knuthc{10}[68] \lvoc{I.2.10}[26]}
\end{frame}


\begin{frame}[fragile]{Длины}{Относительные значения}\relax
    
    \centering
    
    
    \begin{tabular}{r|c|l}
        %  pt& points  & \showLength{12pt} \\
        %  mm& millimeters & \showLength{10mm} \\\hline
         em& примерно \highgreen{ширина} буквы \highgreen{'M'} & \showLength{1em} \\
         ex& примерно \highgreen{высота} буквы \highgreen{'x'} & \showLength{1ex} \\\hline
    \end{tabular}
    
    \inpause пример: если добавим команду \verb|\Huge|
    
    \begin{tabular}{r|l}
         mm & \showLength{5mm} \\
         em & \showLength{1em} \\\hline
         \Huge mm & \Huge \showLength{5mm} \\
         \Huge em & \Huge \showLength{1em} \\
    \end{tabular}
    
    % use {\ccsc em} for horizontal and {\ccsc ex} for vertical cases
    
    \cscfootnote{\wikiC{https://en.wikibooks.org/wiki/LaTeX/Lengths} \knuthc{10}[68] \lvoc{I.2.10}[26]}
\end{frame}

\cprotect\outclassframe{
\begin{frame}[fragile]{Предзаданные длины}{Наиболее используемые}\relax

\centering
\begin{tabular}{r|l}
    \multicolumn{2}{c}{\tiny\TeX's}\\\hline
     \ccol\parindent & Размер отступа в параграфе\\
     \ccol\parskip & \footnotesize Вертикальный отступ для нового параграфа \\
     \hline\multicolumn{2}{c}{\tiny\LaTeX's}\\\hline
     \ccol\textwidth & Ширина текста на странице\\
     \ccol\textheight & Высота текста на странице\\
     \ccol\linewidth & Ширина текста в ``боксе''\\
     \ccol\lineheight & Высота текста в ``боксе''\\
\end{tabular}

     \cscfootnote{\wikiC{https://en.wikibooks.org/wiki/LaTeX/Lengths\#LaTeX_default_lengths} \lmanc{5.5}[34] \stExC{https://tex.stackexchange.com/questions/16942/difference-between-textwidth-linewidth-and-hsize}\\ 
     \ccol\parskip\ на самом деле это клей}
\end{frame}
}

\cprotect\outclassframe{
\begin{frame}[fragile]{Арифметика с длинами}\relax
    \lstset
    {
        basicstyle=\tt\normalsize,
    }
    \begin{itemize}
        \item Можно домножать как \lstinline|0.5\textwidth|
        \item Нельзя просто так использовать +, -, *, /
        \begin{itemize}
            \item Это можно сделать с помощью команды \ccol\dimexpr: \lstinline|\dimexpr\textwidth - 30pt|
        \end{itemize}
    \end{itemize}

    \cscfootnote{\stExC{https://tex.stackexchange.com/questions/245635/formal-syntax-rules-of-dimexpr-numexpr-glueexpr}}
\end{frame}
}

\lookCode{Пустим по верху линейку}
\lookCode{...И по низу пустим лого, заметки и отображение страниц \pause 

Вопрос: Как сделать, чтобы длинные заметки не уводили номер страницы за экран?}


%%%%%%%%%%%%%%%%%%% boxes и glue (счётчик лишь слегка)

\subsection{Боксы и клей}
\begin{frame}{Основная идея \TeX а}\relax
    \centering \large
    \tabcolsep=0.15em
    \begin{tabular}{r>{\ccsc}l}
         Символ -- это & бокс\\
         он -- часть слова, которое & бокс\\ 
         слова соеденины & клеем\\ 
         в предложения и параграфы.&\\ 
         Параграф, кстати, это & бокс\\
         он соединён с другими & клеем\\
         в страницу. Которая -- & бокс\\
         &\\ 
         таблица, картинки, ... -- это & бокс 
    \end{tabular}
    
    \cscfootnote{\wikiC{https://en.wikibooks.org/wiki/LaTeX/Boxes} \knuthc{11}[73]}
\end{frame}


\begin{frame}{Параметры бокса: высота, глубина, ширина}
     
    \centering
    \newbox\boxtodimen%
    \newdimen\hb%
    \newdimen\db%
    \newdimen\wb%
    
    \setbox\boxtodimen=\hbox{\fontsize{120}{126}\selectfont y}%
    \hb=\ht\boxtodimen%
    \db=\dp\boxtodimen%
    \wb=\wd\boxtodimen%
    
    \newcommand{\boxingDimF}{%
    \leavevmode 
    
    \hbox to \wb{%
        \hbox to 0pt{\box\boxtodimen}%
        \hbox to 0pt{\vbox to 0pt{\hbox{%
            \leavevmode 
            \hbox to 0pt{\raisebox{0pt}[0pt][0pt]{\color{green!40!black}\rule{\wb}{1.7pt}}}%width
            \hbox to 0pt{\raisebox{0pt}[0pt][0pt]{\rule{\wb}{0.4pt}}}%width2
            \hbox to 0pt{\raisebox{0pt}[0pt][0pt]{\color{red}\rule{1.7pt}{\hb}}}%height
            \hbox to 0pt{\raisebox{-\db}[0pt][0pt]{\color{red!40!yellow}\rule{1.7pt}{\db}}}%depth
        }}}%
        \hbox to 0pt{\vbox to 0pt{\hbox{%
            \hbox to 0pt{\raisebox{-\db}[0pt][0pt]{\rule{0.4pt}{\dimexpr\hb+\db}}}%left
            \hbox to 0pt{\hbox to \wb{}\raisebox{-\db}[0pt][0pt]{\rule{0.4pt}{\dimexpr\hb+\db}}}%right
            \hbox to 0pt{\raisebox{\hb}[0pt][0pt]{\rule{\wb}{0.4pt}}}%top
            \hbox to 0pt{\raisebox{-\db}[0pt][0pt]{\rule{\wb}{0.4pt}}}%bottom
        }}}%
    }%
    }
    
    \centering
    \begin{tikzpicture}
        \inclass{\uncover<1>}{\node at (0.5\wb, 0) {\raisebox{\hb}{\fontsize{120}{126}\selectfont y}};}
         \inclass{\uncover<2,3>}{\node at (0.5\wb, 0) {\raisebox{\hb}{\boxingDimF}};}
         \inclass{\uncover<3>}{\fill(0,0) circle (0.1cm);
         \node(rp) at (0,0) {};
         \node(rpt) at (-6em, 0) {reference point};
         \draw[->] (rpt) -- (rp);
         \node(bl) at (\wb+8em, 0) {base line};
         \draw (\wb,0) -- (bl);
         
         \node(wdth) at (0.5\wb, -\db-1.4ex) {\color{green!40!black}width};
         \draw[<-] (0, -\db-1.4ex) -- (wdth);
         \draw[->] (wdth) -- (\wb, -\db-1.4ex);
         \draw (0, -\db) -- +(0, -2.8ex);
         \draw (\wb, -\db) -- +(0, -2.8ex);
         
         \node(dpth) at (\wb+1.6em, -0.5\db) {\color{red!40!yellow}depth};
         \draw[<-] (\wb+1.6em, 0) -- (dpth.north);
         \draw[->] (dpth.south) -- (\wb+1.6em, -\db);
         \draw (\wb, -\db) -- +(3.2em, 0);
    
         \node(hght) at (\wb+1.6em, +0.5\hb) {\color{red}height};
         \draw[<-] (\wb+1.6em, 0) -- (hght.south);
         \draw[->] (hght.north) -- (\wb+1.6em, \hb);
         \draw (\wb, \hb) -- +(3.2em, 0);}
    \end{tikzpicture}
     
     \cscfootnote{\stExC{https://tex.stackexchange.com/questions/40977/confused-with-tex-terminology-height-depth-width}}
\end{frame}

    \newcommand{\boxingDim}[1]{{%
    \ifcsname boxtodimen\endcsname%
    \else%
    \newbox\boxtodimen%
    \newdimen\hb%
    \newdimen\db%
    \newdimen\wb%
    \fi%
    \leavevmode 
    \setbox\boxtodimen=\hbox{#1}%
    \hb=\the\ht\boxtodimen%
    \db=\the\dp\boxtodimen%
    \wb=\the\wd\boxtodimen%
    \hbox to \wb{%
        \hbox to 0pt{\box\boxtodimen}%
        \hbox to 0pt{\vbox to 0pt{\hbox{%
            \leavevmode 
            \hbox to 0pt{\raisebox{0pt}[0pt][0pt]{\color{green!40!black}\rule{\wb}{1.7pt}}}%width
            \hbox to 0pt{\raisebox{0pt}[0pt][0pt]{\rule{\wb}{0.4pt}}}%width2
            \hbox to 0pt{\raisebox{0pt}[0pt][0pt]{\color{red}\rule{1.7pt}{\hb}}}%height
            \hbox to 0pt{\raisebox{-\db}[0pt][0pt]{\color{red!40!yellow}\rule{1.7pt}{\db}}}%depth
        }}}%
        \hbox to 0pt{\vbox to 0pt{\hbox{%
            \hbox to 0pt{\raisebox{-\db}[0pt][0pt]{\rule{0.4pt}{\dimexpr\hb+\db}}}%left
            \hbox to 0pt{\hbox to \wb{}\raisebox{-\db}[0pt][0pt]{\rule{0.4pt}{\dimexpr\hb+\db}}}%right
            \hbox to 0pt{\raisebox{\hb}[0pt][0pt]{\rule{\wb}{0.4pt}}}%top
            \hbox to 0pt{\raisebox{-\db}[0pt][0pt]{\rule{\wb}{0.4pt}}}%bottom
        }}}%
    }%
    }%
    }
\begin{frame}{Как \TeX\ объединяет боксы}
    
    \inclass{{\centering
    % \boxingDim{
    \only<1,2,3>{\leavevmode\hbox{\fontsize{120}{126}\selectfont\boxingDim{{g}}\inpause\boxingDim{f}\inpause\boxingDim{\textit{y}}\boxingDim{\textit{:}}\boxingDim{.}}}
    % }
    \only<4>{\boxingDim{{\fontsize{120}{126}\selectfont\boxingDim{{g}}\inpause\boxingDim{f}\inpause\boxingDim{\textit{y}}\boxingDim{\textit{:}}\boxingDim{.}}}}

    }}
    \outclass{{\centering
    % \boxingDim{
    % {\leavevmode\hbox{\fontsize{120}{126}\selectfont\boxingDim{{g}}\inpause\boxingDim{f}\inpause\boxingDim{\textit{y}}\boxingDim{\textit{:}}\boxingDim{.}}}
    % }
    {\boxingDim{{\fontsize{120}{126}\selectfont\boxingDim{{g}}\inpause\boxingDim{f}\inpause\boxingDim{\textit{y}}\boxingDim{\textit{:}}\boxingDim{.}}}}

    }}

    \outclasshigh{Обратите внимание, что бокс не совпадает полностью с буквой. ``f'' выходит за неё, точка занимает не всё пространство бокса}

\end{frame}

\begin{frame}{Два типа боксов}\relax
     
    Есть два типа боксов:
    \begin{itemize}
        \item {\ccsc``Горизонтальный''} бокс -- стыкуется к другим горизонтальным боксам. Его параметр -- это ширина.
        \item {\ccsc``Вертикальный''} бокс -- стыкуется с другими вертикальными боксами. Его параметры -- высота и глубина.
         
    \end{itemize}
\end{frame}

\begin{frame}[fragile]{\TeX : Горизонтальные и вертикальные боксы}\relax
     \lstset
    {
        basicstyle=\tt\normalsize,
    }
     \begin{tabbing}
     
        \lstinline|\hbox |\= \lstinline|to 20pt| \= \lstinline|{hello world}|\kill

        \rulecommand[14pt]{1}\lstinline|\hbox |\> \rulecommand[14pt]{2}\lstinline|to 20pt| \> \rulecommand[14pt]{3}\lstinline|{hello world}| \\ \footnotesize
        ``горизонтальный'' бокс\>  \>   \\ 
        \>\footnotesize какой длины его считает \TeX\ (может отсутствовать)  \>  \\ 
        \>  \>\footnotesize тело бокса \\
        
    \end{tabbing}
    
    \begin{tabbing}
     
        \lstinline|\vbox |\= \lstinline|to 20pt| \= \lstinline|{hello world}|\kill

        \rulecommand[14pt]{1}\lstinline|\vbox |\> \rulecommand[14pt]{2}\lstinline|to 20pt| \> \rulecommand[14pt]{3}\lstinline|{hello world}| \\ \footnotesize
        ``вертикальный'' бокс\>  \>   \\ 
        \>\footnotesize какой высоты его считает \TeX\ (может отсутствовать)  \>  \\ 
        \>  \>\footnotesize тело бокса \\
        
    \end{tabbing}
\end{frame}

\cprotect\outclassframe{
\begin{frame}[fragile, t]{Горизонтальный бокс с разными параметрами}\relax\vskip-3ex
    \samePosPicture{hboxmy}{38-41,44-47} 
    \cscfootnote{\tugC{https://www.tug.org/utilities/plain/cseq.html\#hbox-rp}}
\end{frame}
}


\begin{frame}[fragile]{``Горизонтальный'' бокс}{Использование}\relax\magicPage
        \begin{columns}
        \begin{column}{0.4\textwidth}
         \verb|\hbox to -1pt{/}=|    
        \end{column}
        \begin{column}{0.4\textwidth}
             \hbox{\hbox to -1pt{/}=}
        \end{column}
        \end{columns}
        
        \begin{columns}
        \begin{column}{0.4\textwidth}
        \small
        \begin{verbatim}
    \begin{tabbing}
    \hbox to 4em{} \= \hbox to 4em{}\kill
    a \> b\\
    hello\> world!
    \end{tabbing}
        \end{verbatim}     
        \end{column}
        \begin{column}{0.4\textwidth}
         \begin{tabbing}
              \hbox to 4em{} \= \hbox to 4em{}\kill
              a \> b\\
              hello\> world!
         \end{tabbing}
             
        \end{column}
        \end{columns}
     \cscfootnote{Ещё есть \ccol\llap\ и \ccol\rlap -- они создают бокс нулевой длины, который накладывается на следующий символ)}
\end{frame}

\cprotect\outclassframe{
\begin{frame}[fragile, t]{Вертикальный бокс с разными параметрами}\relax\vskip-3ex
    \samePosPicture{vboxmy}{46-49,52-55,58-61} 
    \cscfootnote{\tugC{https://www.tug.org/utilities/plain/cseq.html\#vbox-rp}, \tugC{https://www.tug.org/utilities/plain/cseq.html\#vtop-rp}}
\end{frame}
}

\begin{frame}{Двигаем буквы}\relax\magicPage\vskip-3ex
    \samePosPicture{moveboxmy}{39-39,41-46} \vskip-4ex
    
    Можно использовать команды
    \ccol\raise, \ccol\lower, \ccol\moveleft, \ccol\moveright
    
     
    \cscfootnote{\tugC{https://www.tug.org/utilities/plain/cseq.html\#raise-rp} \tugC{https://www.tug.org/utilities/plain/cseq.html\#lower-rp} \tugC{https://www.tug.org/utilities/plain/cseq.html\#moveleft-rp} \tugC{https://www.tug.org/utilities/plain/cseq.html\#moveright-rp}}
\end{frame}


\begin{frame}[fragile]{\LaTeX: Горизонтальные боксы}\relax
     \lstset
    {
        basicstyle=\tt\normalsize,
    }
     \begin{tabbing}
     
        \lstinline|\mbox| \= \lstinline|{hello world}|\kill

        \rulecommand[14pt]{1}\lstinline|\mbox| \> \rulecommand[14pt]{2}\lstinline|{hello world}| \\ \footnotesize
        ``горизонтальный'' бокс\>   \\ 
        \>\footnotesize тело бокса \\
        
    \end{tabbing}
    
    \begin{tabbing}
     
        \lstinline|\makebox |\= \lstinline|[20mm]| \= \lstinline|[c]| \= \lstinline|{hello world}|\kill

        \rulecommand[14pt]{1}\lstinline|\makebox|\> \rulecommand[14pt]{2}\lstinline|[20mm]| \> \rulecommand[14pt]{3}\lstinline|[c]| \> \rulecommand[14pt]{4}\lstinline|{hello world}| \\ \footnotesize
        ``горизонтальный'' бокс\>  \> \>  \\ 
        \>\footnotesize какой длины бокс  \> \> \\ 
        \>  \>\footnotesize выравнивание внутри бокса  \> \\
        \>  \>  \> \footnotesize тело бокса \\ 
        
    \end{tabbing}
    \cscfootnote{\lmanc{20.1}[179] \wikiC{https://en.wikibooks.org/wiki/LaTeX/Boxes\#makebox_and_mbox} \stExC{https://tex.stackexchange.com/questions/83930/what-are-the-different-kinds-of-boxes-in-latex}} 
\end{frame}

\cprotect\outclassframe{
\begin{frame}[fragile, t]{Горизонтальный бокс}\relax\vskip-3ex
    \samePosPicture{mboxmy}{39-41,43-46} 
\end{frame}
}

\begin{frame}[fragile]{\LaTeX: Вертикальные боксы}\relax

     \lstset
    {
        basicstyle=\tt\normalsize,
    }
    \begin{tabbing}
     
        \lstinline|\parbox |\= \lstinline|[c]| \= \lstinline|[20pt]| \= \lstinline|{100pt}| \= \lstinline|{hello world}|\kill

        \rulecommand[14pt]{1}\lstinline|\parbox|\> \rulecommand[14pt]{2}\lstinline|[c]| \> \rulecommand[14pt]{3}\lstinline|[20pt]| \> \rulecommand[14pt]{4}\lstinline|{100pt}| \> \rulecommand[14pt]{5}\lstinline|{hello world}| \\ \footnotesize
        ``вертикальный'' бокс\>  \> \> \>  \\ 
        \>\footnotesize выравнивание внутри бокса  \> \> \> \\ 
        \>  \>\footnotesize какой высоты бокс  \> \> \\
        \>  \> \> \footnotesize какая ширина бокса  \> \\
        \>  \>  \> \> \footnotesize тело бокса \\ 
        
    \end{tabbing}
    
    \vspace*{-4ex}
    \begin{tabbing}
     
        \lstinline|\raisebox |\= \lstinline|{20pt}| \= \lstinline|[10pt]| \= \lstinline|[50pt]| \= \lstinline|{hello world}|\kill

        \rulecommand[14pt]{1}\lstinline|\raisebox|\> \rulecommand[14pt]{2}\lstinline|{20pt}| \> \rulecommand[14pt]{3}\lstinline|[10pt]| \> \rulecommand[14pt]{4}\lstinline|[50pt]| \> \rulecommand[14pt]{5}\lstinline|{hello world}| \\ \footnotesize
        ``подъёмный''вертикальный бокс\>  \> \> \>  \\ 
        \>\footnotesize насколько поднимаем  \> \> \> \\ 
        \>  \>\footnotesize высота бокса  \> \> \\
        \>  \> \> \footnotesize глубина бокса  \> \\
        \>  \>  \> \> \footnotesize тело бокса \\ 
        
    \end{tabbing}

     
\end{frame}

\outclassframe{
\begin{frame}{Вертикальные боксы}\relax\vskip-3ex
    \samePosPicture{parboxmy}{44-44, 48-50} 
    
    \cscfootnote{\lmanc{20.3}[181] \lmanc{8.18}[73] \wikiC{https://en.wikibooks.org/wiki/LaTeX/Boxes\#parbox,_minipage,_and_pbox}\\ \lmanc{20.4}[182] \lmanc{22.3.2}[198] \lmanc{22.3.3}[199] \wikiC{https://en.wikibooks.org/wiki/LaTeX/Boxes\#box_modifiers}}
\end{frame}
}
\lookCode{Теперь можем ограничить размер текста! \pause

Вопрос: как сделать, чтобы между элементами было пустое пространство?
}


\begin{frame}{Пробелы}\relax
    {\LARGE \ccol{Клей} и \ccol{Керны} предоставляют пробелы между боксами.}
    \samePosPicture{intro}{11-11}

    \cscfootnote{\tugC{https://www.tug.org/utilities/plain/cseq.html\#kern-rp} \knuthc{12}[79]}
\end{frame}

\begin{frame}[fragile]{Что такое клей}\relax
    Но клей -- это больше, чем просто ``пробел'' между боксами.
    
    Клей это {\bfseries \ccsc растяжимый} пробел между боксами.
    
    \inpause
    %      \lstset
    % {
    %     basicstyle=\tt\normalsize,
    % }
    \footnotesize
        \begin{tabbing}
     
        \lstinline|\hskip |\= \lstinline|2em | \= \lstinline|plus 0.5em | \= \lstinline|minus 0.6em|\kill

        \rulecommand{1}\lstinline|\hskip |\> \rulecommand{2}\lstinline|2em | \> \rulecommand{3}\lstinline|plus 0.5em | \> \rulecommand{4}\lstinline|minus 0.6em| \\ \footnotesize
        добавить горизонтальный клей (\TeX). \ccol\vskip\ добавит вертикальный\>  \> \>  \\ 
        \>\footnotesize базовый отступ  \> \> \\ 
        \>  \>\footnotesize насколько может растянуться (опционно)  \> \\
        \>  \> \> \footnotesize насколько может сужаться (опционно) \\
        
    \end{tabbing}
    
    \begin{tabbing}
     
        \lstinline|\hspace{|\= \lstinline|2em | \= \lstinline|plus 0.5em | \= \lstinline|minus 0.6em}|\kill

        \rulecommand{1}\lstinline|\hspace{|\> \lstinline|2em | \> \lstinline|plus 0.5em| \> \lstinline|minus 0.6em}| \\ \footnotesize
        добавить горизонтальный клей (\LaTeX). \ccol\vspace\ добавит вертикальный.\>  \> \>  \\         
    \end{tabbing}
    \cscfootnote{\tugC{https://www.tug.org/utilities/plain/cseq.html\#hskip-rp} \tugC{https://www.tug.org/utilities/plain/cseq.html\#vskip-rp} \lmanc{19.2}[167] \lmanc{19.14}[176]\\ 
        ещё есть \ccol{\vspace*} -- добавит вертикальный клей в начале страницы
    }
     
\end{frame}


\begin{frame}{Где клей добавляется неявно}\relax
    \obeylines
    \hbox spread -10pt {Между словами и предложениями. Тут много клея.}
    \hbox spread -5pt  {Между словами и предложениями. Тут много клея.}
    \hbox              {Между словами и предложениями. Тут много клея.}
    \hbox spread 5pt   {Между словами и предложениями. Тут много клея.}
    \hbox spread 10pt  {Между словами и предложениями. Тут много клея.}
    \hbox spread 20pt  {Между словами и предложениями. Тут много клея.}
    \hbox spread 40pt  {Между словами и предложениями. Тут много клея.}

    \outclasshigh{\small P.S. Тут \string\hbox\ растянутый в диапазоне -10pt---40pt. Заметьте, что между предложениями растяжение больше, чем между словами. И что дефолтное растяжение -- третья строка -- не минимально.

    А ещё клей есть между параграфами.}
\end{frame}

\begin{frame}{Бесконечный клей}\relax\vspace*{-3ex}
     \lstset
    {
        basicstyle=\tt\tiny,
    }
    \samePosPicture{fillmy}{11-16} 
     
    \ccol{fil}, \ccol{fill}, \ccol{filll} добавляют бесконечность разной ``степени''
    
    \cscfootnote{\knuthc{12}[83] }
\end{frame}

\begin{frame}{Аббревиатуры}\relax
     Можно использовать:
     
     \ccol\hfil \hfill \ccol\hfill \hfill \ccol\hspace\{\ccol\fil\} \hfill  \ccol\hspace\{\ccol\fill\}
     
     \ccol\vfil  \hfill \ccol\vfill  \hfill \ccol\vspace\{\ccol\fil\} \hfill  \ccol\vspace\{\ccol\fill\}
     
     \ccol\hss, \ccol\vss\ -- бесконечный клей как в \textit{plus} так и в \textit{minus}.
\end{frame}

\lookCode{Добавим бесконечный клей между элементами}
\lookCode{И давайте изменим отображение заголовка\pause


Проблема: плохо, что чёрточка под заголовком не прижимается к нему, когда нет подзаголовка}
% %%%%%%%%%%%%%%%%%%%%%%%%%%%%%%%%%%%%%%%%%%%%%%% МОДЫ И ПАРАГРАФЫ %%%%%%%%%%%%%%%%%%%%%%%
\subsection{Моды и создание параграфов}
\begin{frame}{Моды\magicPage}\relax
\small
\TeX\ имеет 3(6) мод:
\footnotesize
\begin{enumerate}
    \item {\bfseries {\ccsc Vertical} mode.} [Создание главного вертикального списка, из которого получаются страницы.]
\item {\bfseries Internal {\ccsc vertical} mode.} [Вертикальный список для vbox.]
\item {\bfseries {\ccsc Horizontal} mode.} [Создание горизонтального списка для параграфов.]
\item {\bfseries Restricted {\ccsc horizontal} mode.} [Создание горизонтального списка для hbox.]
\item {\bfseries {\ccsc Math} mode.} [Создание математической формулы внутри горизонтального списка.]
\item {\bfseries Display {\ccsc math} mode.} [Создание математической формулы и положение её на отдельную строку, прерывание параграфа.]
     
\end{enumerate}
     
     \cscfootnote{\knuthc{13}[95]}
\end{frame}

\begin{frame}{Разница между модами\magicPage}

    Много мелких различий. Например:
    \begin{itemize}
        \item в горизонтальной моде только первый пробел имеет значение
        \item в математической моде шрифт по умолчанию - италик, пробелы игнорируются
        \item в выделенной математической моде операторы рисуется больше, чем в обычной
        \item в вертикальной моде все пробелы и <return>ы игнорируется
         
    \end{itemize}
     \cscfootnote{можно использовать \ccol\leavevmode\ чтобы прерывать вертикальную моду}
\end{frame}

\outclassframe{
\begin{frame}{Ещё чуть-чуть о математической моде}
    В реальности у нас есть 4 стиля:\small
     
     \centering 
     \begin{tabular}{l|l|l|p{7em}}
     \hline
     \footnotesize Display style &\footnotesize \ccol\displaystyle & $\displaystyle A$ &\scriptsize главный стиль для выделенной формулы\\
     \footnotesize Text style & \footnotesize\ccol\textstyle & $\textstyle A$ & \scriptsize главный стиль для внутритекстовой формулы\\
     \footnotesize Script style & \footnotesize\ccol\scriptstyle & $\scriptstyle A$ & \scriptsize главный стиль для индексов\\
     \footnotesize Script-script style & \footnotesize\ccol\scriptscriptstyle & $\scriptscriptstyle A$ & \scriptsize главный стиль для под-индексов\\
     \end{tabular}
     
    \cscfootnote{\knuthc{17}[151]}
\end{frame}
}

\begin{frame}{Создание параграфов}{Для перфекционистов}\relax
     \centering
    \includegraphics[width=0.6\textwidth]{twcomp}
    
    \includegraphics[width=0.6\textwidth]{twcomp2}
    
    \cscfootnote{\stExC{https://tex.stackexchange.com/questions/110133/visual-comparison-between-latex-and-word-output-hyphenation-typesetting-ligat} \url{http://www.rtznet.nl/zink/latex.php?lang=en}}
\end{frame}

\begin{frame}{Создание параграфов}\relax\magicPage
    \Huge\centering ``это, по факту, наверно самый интересный аспект всей системы TEX''\\\hfill \large D. Knuth, the \TeX Book

\end{frame}

\inclassframe{
\begin{frame}{Создание параграфов}
\Large ... но посмотрите его дома :)
\end{frame}
}

\cprotect\outclassframe{
\begin{frame}{Обзор}\relax
    \begin{itemize}
        \item Каждый параграф создаётся целиком: слова в конце параграфа могут повлиять на расположение строк в начале.
        \item \TeX\ никогда не положит слова ближе, чем позволяет клей.
        \item \TeX\ проблует все комбинации разрывов строк. Для каждого варианта и каждой строки \TeX\ вычисляет параметр \textit{badness}. Если он меньше \ccol\tolerance, \TeX\ попробует создать параграф с минимальным числом переносов слов.
        \item если у \TeX 'а не получается, он выдаст \textbf{Overfull} или \textbf{Underfull} ворнинги.
    \end{itemize}
    \cscfootnote{\knuthc{14}[101] \lvoc{3.6}[114] \lmanc{9}[100]}
\end{frame}

\begin{frame}[fragile]{Как управлять переносами слов}\relax

    Локально: использовать \ccol\- ~ \verb|as in this ve\-ry long se\-nta\-nce|
    
    Глобально: \ccol\hyphenation\verb|{some-thing poss-ible}|
    
    \cprotect\cscfootnote{\vspace{-3ex}\ccol\- это короткая версия для команды \ccol\discretionary\verb|{hpre-break texti}{hpost-break texti}{hno-break texti}|\\ 
    так же: \TeX\ никогда не перенесёт слово с ``/''. используйте вместо этого \ccol\slash, если перенос нужен. А \ccol\uchyph=0 запретит перенос слов с Большой буквы.}
\end{frame}

\begin{frame}{Управление разрывом строк вручную}\relax

    Никогда не прерываются: неразрывный пробел \~, \ccol\nobreak, \ccol\nolinebreak
    
    Всегда: \ccol\\, \ccol\break, \ccol\linebreak
    
    Ещё можно использовать \ccol\obeylines\ И тогда разрывы будут происходить там же, где в исходном коде.
     
     \cscfootnote{кстати: \ccol\\ \ имеет опционный параметр: вертикальный отступ после себя. А \ccol\smallskipamount, \ccol\medskipamount\ и \ccol\bigskipamount\ отвечают за разрыв после параграфа. \ccol\linebreak\ имеет опционный параметр: [0-4] насколько сильно ваше желание прервать строку после этого.}
\end{frame}

\begin{frame}{Алгоритм: часть 1}\relax

    \begin{enumerate}
        \item \TeX\ создаёт варианты без переноса слов. Проверяет параметр \textit{badness} с параметром \ccol\pretolerance. 
        \item \textit{badness} $\simeq$ 100$\cdot$<proportion-between-the-normal-glue-and-its-stretching/compression>$^3$
        \item если проверка с \ccol\pretolerance\ провалилась, \TeX\ попробует все комбинации разрывов строк, чтобы badness была меньше, чем \ccol\tolerance
         
    \end{enumerate}
     
     \cscfootnote{Умолчания: \ccol\pretolerance=100, \ccol\tolerance=200}
\end{frame}

\begin{frame}{Алгоритм: часть 2}\relax
    \footnotesize
    \begin{enumerate}
        \item разрывы строк допустимы лишь в следующих местах:
        \begin{enumerate}
            \item клей
            \item керн, после которого идёт клей 
            \item математика (\$) с последующим клеем
            \item ручное или автоматическое прохождение штрафа
            \item  discretionary разрыва
        \end{enumerate}
        \item Пенальти для клея -- $0$. Для разрыва это \ccol\hyphenpenalty=\ или \ccol\exhyphenpenalty=. Пенальти можно добавить вручную через \ccol\penalty
        \item Пенальти может быть как положительная, так и отрицательная. Если оно $>10^4$ тут никогда не будет разрыва, а если $<-10^4$ разрывы будут всегда
         
    \end{enumerate}
         
    \cscfootnote{умолчания: \ccol\hyphenpenalty=50, \ccol\exhyphenpenalty=50\\ 
    а ещё можно добавлять доп пространства между текстом и математикой с \ccol\mathsurround}
\end{frame}


\begin{frame}{Алгоритм: часть 3}\relax

    \begin{enumerate}
    \item В реальности, \TeX\ пытается минимизировать \textit{demerits}. Он пропорциональен {\ccsc badnesses}, \ccol\linepenalty\ (определяет, насколько сильно вы просите \TeX\ уменьшить число строк) и {\ccsc penalty}
    \item \TeX\ так же берёт в расчёт и добавляет пенальти, если две строки подряд имеют перенос слов (\ccol\doublehyphendemerits), и если строки \textit{визуально несовместимы} (например: растянутая линия с ужатой) (\ccol\adjdemerits) и если предпоследняя строка абзаца заканчивается discretionary (\ccol\finalhyphendemerits)
    
    \end{enumerate}
     
     \cscfootnote{Умолчания: \ccol\linepenalty=10, \ccol\adjdemerits=10000, \ccol\doublehyphendemerits=10000, \ccol\finalhyphendemerits=5000.\\ 
     \ccol\hfuzz=... добавит максимальную длину выравнивания строки}
\end{frame}

\begin{frame}{Что ещё }\relax 
    \def\strutdepth{\dp\strutbox}
    \def\marginalstar{\strut\vadjust{\kern-\strutdepth\specialstar}}
    \def\specialstar{\vtop to \strutdepth{
    \baselineskip\strutdepth
    \vss\llap{* }\null}}
    
    \begin{itemize}
        \item Используйте \ccol\narrow\ чтобы сузить строки
        \item \ccol\looseness=-1 чтобы попросить \TeX\ попробовать уменьшить число строк в параграфе
        \item \ccol\prevgraf\ показывает текущую строку параграфа
        \item \ccol\vadjust\ добавляет что-то в вертикальный список после каждого параграфа. Например, так мы добавили звёздочку слева \marginalstar
        \item \ccol\everypar\ добавляет что-то после каждого параграфа
        \item \ccol\parfillskip\ --- клей после каждой строки
        \item \ccol\parskip\ --- вертикальный клей между параграфами
         
    \end{itemize}
     
\end{frame}

\begin{frame}{Создание страниц}\relax
    \begin{itemize}
        \item Кнут создал \TeX\ тогда, когда памяти оптимизировать всю страницу ещё не хватало.
        \item \TeX\ ищет лучший разрыв для текущей страницы и удаляет страницу из памяти.
        \item В целом алгоритм примерно такой же.
        \item Можно использовать \ccol\penalty\ or \ccol\nobreak\ в вертикальном моде
        \item Можно использовать \raggedbottom\ чтобы убрать привязку вертикальную к низу страницы
        \item По аналогии с параграфами, можно использовать \ccol\newpage, \ccol\pagebreak, \ccol\nopagebreak
         
    \end{itemize}
     
     \cscfootnote{\knuthc{15}[119] \lvoc{III.9}[141] \lmanc{10}[105]}
\end{frame}
}
%%%%%%%%%%%%%%%%%%%%%%% Передача параметров, опционные аргументы %%%%%%%%%%%%%%%%
\subsection{Возможности создания команд и передачи параметров в \LaTeX}

\begin{frame}[fragile]{Опционные аргументы в создании команд\magicPage }\relax
    \footnotesize
     \begin{tabbing}
     
        \lstinline|\newcommand{\test}|\= \lstinline|[2]| \= \lstinline|[my default arg]| \= \lstinline|{имеет умолчание #1, задаётся #2}|\kill

        \lstinline|\newcommand{\test}|\> \rulecommand{1}\lstinline|[2]| \> \rulecommand{2}\lstinline|[my default arg]| \> \lstinline|{имеет умолчание #1, задаётся #2}| \\ \footnotesize
        \>\footnotesize общее число аргументов  \> \> \\ 
        \>  \>\footnotesize какие аргументы по умолчанию \> \\
        \> \>~~~~(по порядку с начала) \>
    \end{tabbing}
    
    \vspace{-3ex}
    \lstinline|\usepackage{xargs}|\\ 
    \lstinline|\newcommandx {\ testOpt }[3][3= def opt val]|
    
    Для именнованных команд можно использовать пакет \textit{keyval} или \textit{pgfkeys}
    
    \cscfootnote{\stExC{https://tex.stackexchange.com/questions/58069/newcommand-key-value} \stExC{https://tex.stackexchange.com/questions/34312/how-to-create-a-command-with-key-values} \stExC{https://tex.stackexchange.com/questions/26771/a-big-list-of-every-keyval-package}}
     
\end{frame}

\begin{frame}[fragile]{Передача параметров в пакет}\relax
    \footnotesize
    \lstinline|\RequirePackage{kvoptions}| -- используем пакет \\ 
    \lstinline|\SetupKeyvalOptions{family=KVCSC, prefix=KVCSC@}| -- задаём namespace
    \vspace{-1ex}
    {\scriptsize
    \lstset{basicstyle=\tt\tiny}
    \begin{tabbing}
     
     \lstinline|\DeclareStringOption|\= \lstinline|[noarg]| \= \lstinline|{argname}| \= \lstinline|[default arg value]|\kill

        \lstset{basicstyle=\tt\tiny}\lstinline|\DeclareStringOption|\> \rulecommand[10pt]{1}\lstinline|[noarg]| \> \rulecommand[10pt]{2}\lstinline|{argname}| \> \rulecommand[10pt]{4}\lstinline|[default arg value]| \\ 
        \> значение, если аргумент не передан совсем \lstinline|\usepackage{mypackage}|  \> \> \\
        \>  \> имя аргумента для передачи, может быть использована как  \> \\
        \>  \> ~~~~~~~~~~~~~~~~~~~~~~~~~~ \lstinline|\usepackage[myarg=smth]{mypackage}|  \> \\
        \>  \>  \> дефолтный аргумент в случае, если пакет вызывает-  \\
        \>  \>  \> ~~~~~~~~ся как \lstinline|\usepackage[smth]{mypackage}|  \\
    \end{tabbing}
    }

    \vspace{-2ex}
    \lstinline|\ProcessKeyvalOptions*| -- непосредственно подставляем полученные значения
    
    \lstinline|\KVCSC@argname| -- под таким именем аргумент будет доступен в командах
\end{frame}
\lookCode{Давайте передадим в пакет возможность менять логотип}

\inclassframe{
\begin{frame}{За эту часть мы узнали}\relax
    \begin{itemize}
        \item Как создавать стилевые файлы
        \item Как сделать футлайн и заголовок
        \item Как работать с боксами
        \item Как добавить клей
        \item Как создавать команды и передавать параметры в пакеты
    \end{itemize}
    Фактически мы умеем создать любую картинку\inpause, если она статична
\end{frame}
\begin{frame}{Пока мы не знаем}\relax
    \begin{itemize}
        \item Как убрать номер слайда с титульника
        \item Как пододвинуть чёрточку под заголовком ближе к нему, когда подзаголовка нет
        \item Как делать динамические элементы типа прогрессбара
        \item Как можно манипулировать элементами
    \end{itemize}\inpause
    Т.е. для полноты картины нужно научиться работать с:
    \begin{itemize}
        \item условными операторами
        \item переменными
        \item циклами
    \end{itemize}\inpause
    научиться программировать на \TeX!
\end{frame}
}
%% It is just an empty TeX file.
%% Write your code here.
\graphicspath{{sec01/images/}{sec01/code/}}
\lstset{inputpath=sec01/code/}

\begin{frame}[fragile]{What is TikZ?}\relax
``TikZ ist kein Zeichenprogramm'' 

which translates to ``TikZ is not a drawing program''

TikZ defines a number of \TeX\ commands that produce graphics: \tikz \fill[orange] (1ex,1ex) circle (1ex); produced by \verb|\tikz \fill[orange] (1ex,1ex) circle (1ex);|
\end{frame}

\begin{frame}{Pros and Cons}

\Huge\centering Pros and Cons
     
\end{frame}

\begin{frame}[fragile]{Cons}\relax
     \begin{itemize}
        \item[$-$] it is most likely that you don't need TikZ
        \item[$-$] write visual-based thinks like graphics is really annoying in a not-WYSiWYG way
    \end{itemize}
    
\end{frame}

\begin{frame}{Pros}\relax
     \begin{itemize}
        \item[$+$] it is most likely that you need some TikZ elements
        \item[$+$] some graphics (graphs for example) are so good structured, that it is OK to program them
        \item[$+$] TikZ has perfect integration with \LaTeX\ (and beamer):
        \begin{itemize}
            \item You can use all \LaTeX commands inside TikZ, creating beautiful pictures with math 
            \item You can pose elements using TikZ
            \item You can show just part of the picture in beamer Overlays   
        \end{itemize}
        \item[$+$] You don't need to have an external file
        \item[$+$] TikZ is using in CV and lots of other templates. It is good to be able to read the code
    \end{itemize}
\end{frame}


\begin{frame}[fragile]{How to setup TikZ picture?}\relax

\verb|\usepackage{tikz}|

and then 

\verb|\begin{tikzpicture} <code> \end{tikzpicture}| or, for short inline graphics, \ccol\tikz. 



\skfootnote{\tikzc{12.1}[126]}
     
\end{frame}


\outclassframe{
\begin{frame}{Ссылки на литературу}
color from the footnotes corresponds to references' color.
    \begin{itemize}
        \item \knuthc{Д. Кнут ``The \TeX Book''}
        \item \lvoc{Львовский ``Набор и вёрстка в системе \LaTeX''}
        \item \lmanc{``\LaTeX 2e: An unofficial reference manual''} так же на сайте \url{https://latexref.xyz/}
        \item \stExC{https://tex.stackexchange.com/questions} : \url{https://tex.stackexchange.com/questions}
        \item \wikiC{https://en.wikibooks.org/wiki/LaTeX} : \url{https://en.wikibooks.org/wiki/LaTeX}
        \item \overC{https://www.overleaf.com/learn/latex} : \url{https://www.overleaf.com/learn/latex}
        \item \tugC{https://www.tug.org/utilities/plain/cseq.html} : \url{https://www.tug.org/utilities/plain/cseq.html}
    \end{itemize}
\end{frame}

\begin{frame}{Распространение}\relax
\begin{itemize}
     \item the pdf-version of the presentation and all printed materials can be distributed under license Creative Commons Attribution-ShareAlike 4.0 \url{https://creativecommons.org/licenses/by-sa/4.0/}
     \item The source code of the presentation is available on {\ccsc\url{https://github.com/Lavton/latexLectures}} and can be distributed under the MIT license \url{https://en.wikipedia.org/wiki/MIT_License\#License_terms}
\end{itemize}
     
\end{frame}
}
\end{document}