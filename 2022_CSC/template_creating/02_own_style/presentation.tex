%https://tex.stackexchange.com/questions/114219/add-notes-to-latex-beamer
\documentclass[14pt, aspectratio=43]{beamer}
\usepackage{fontspec}
\usepackage{xunicode}
\usepackage{xltxtra}
\usepackage{dsfont}
\usepackage{multicol}
\usepackage{array}

\setsansfont[Mapping=tex-text]{Liberation Sans}
\setromanfont[Mapping=tex-text]{Liberation Sans}
\setmonofont[Mapping=tex-text]{Liberation Mono}


\usepackage{xecyr}
\usepackage{hyperref}

\usepackage{polyglossia}
\usepackage{graphicx}
\usepackage{listings}

% \newcommand{\CSC}{Computer Science Center}% <-- сначала тут ввели команду
\usepackage{cscbeamer}  % <-- создали стилевой файл



\begin{document}


    \title{Шаблон \LaTeX}
    \subtitle{для \CSC}  % <-- ввели для примера команду \CSC='Computer Science Center' и применили её
    \author{Антон Лиознов}
    \institute{\CSC}  % <-- ввели для примера команду \CSC='Computer Science Center' и применили её
    \date{\the\year}
    \frame{\titlepage}
    
    \begin{frame}{Будет клёвый шаблон}\relax
    
        С кучей всего))
         
    \end{frame}
\end{document}