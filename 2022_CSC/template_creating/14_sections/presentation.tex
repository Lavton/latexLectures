%https://tex.stackexchange.com/questions/114219/add-notes-to-latex-beamer
\documentclass[14pt, aspectratio=43]{beamer}
\usepackage{fontspec}
\usepackage{xunicode}
\usepackage{xltxtra}
\usepackage{dsfont}
\usepackage{multicol}
\usepackage{array}

\setsansfont[Mapping=tex-text]{Liberation Sans}
\setromanfont[Mapping=tex-text]{Liberation Sans}
\setmonofont[Mapping=tex-text]{Liberation Mono}


\usepackage{xecyr}
\usepackage{hyperref}

\usepackage{polyglossia}
\usepackage{graphicx}
\usepackage{listings}

\usepackage[logo]{cscbeamer}

\begin{document}

    \title{Шаблон \LaTeX}
    \subtitle{для \CSC}
    \author{Антон Лиознов}
    \institute{\CSC}
    \date{\the\year}
    \frame{\titlepage}
    \AtBeginSection[] 
    {
      \begin{frame}{Итак, сегодня...}
      \tableofcontents[currentsection,hideallsubsections]
      \end{frame}
    }
    

    \begin{frame}{Будет клёвый шаблон}\relax
    
        С кучей {\ccsc всего})). 
        
        Код будет доступен на \github{https://github.com/Lavton/latexLectures}
        
        Размер страницы, кстати, \the\paperwidth $\times$\the\paperheight  
        
        \cscfootnote{Заметка настолько длинная, что ради неё стоит убрать номер слайда и лого Lorem ipsum dolor sit amet, consectetur adipiscing elit, sed do eiusmod tempor incididunt ut labore et dolore magna aliqua. Ut enim ad minim veniam, quis nostrud exercitation ullamco laboris nisi ut aliquip ex ea commodo consequat. Duis aute irure. Lorem ipsum dolor sit amet, consectetur adipiscing elit, sed do eiusmod tempor incididunt ut labore et dolore magna aliqua. Ut enim ad minim veniam, quis nostrud exercitation}% <-- заметка стала ещё длиннее!
        \hardslide 
        
        Всего сложных слайдов \thehardslidecount !
    \end{frame}
    
    \section{Первая секция}%  <-- добавим секцию
    
    \begin{frame}[t]{Заголовок}{и подзаголовок}\relax
         Вот так вот {\ccsc всё}
         
        
         \cscfootnote{Заметка:  \CSC\ -- это круто}
    \end{frame}
    
    % <---- добавим ещё несколько секций и слайдов
    \begin{frame}{ещё слайд}
         
    \end{frame}
    \begin{frame}{ещё слайд}
         
    \end{frame}
    \begin{frame}{ещё слайд}
         
    \end{frame}
    \section{Вторая секция}
    \begin{frame}{ещё слайд}
         секция
    \end{frame}
    \begin{frame}{ещё слайд}
         
    \end{frame}
    \tracingmacros=1
    \section{Третья секция}  % <-- проверяем, из чего состоит секция
    \tracingmacros=0
    \begin{frame}[t]{завершаем}
         секция
    \end{frame}
    
\end{document}