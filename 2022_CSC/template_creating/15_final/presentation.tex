%https://tex.stackexchange.com/questions/114219/add-notes-to-latex-beamer
\documentclass[12pt, aspectratio=43]{beamer}
\usepackage{fontspec}
\usepackage{xunicode}
\usepackage{xltxtra}
\usepackage{dsfont}
\usepackage{multicol}
\usepackage{array}

% \setsansfont[Mapping=tex-text]{Liberation Sans}
% \setromanfont[Mapping=tex-text]{Liberation Sans}
% \setmonofont[Mapping=tex-text]{Liberation Mono}


\usepackage{xecyr}
\usepackage{hyperref}

\usepackage{polyglossia}
\usepackage{graphicx}
\usepackage{listings}

\usepackage[darktherm, useprogressbar, classmode=all]{cscbeamer}

\begin{document}
    \AtBeginSection[] 
    {
      \begin{frame}{Содержание}
      \tableofcontents[currentsection,hideallsubsections]
      \end{frame}
    }
    % \title{Презентации CS центра}
    \title{Шаблон \LaTeX}
    \subtitle{для \CSC}
    \author{Антон Лиознов}
    \institute{\CSC}
    \date{\the\year}
    \frame{\titlepage}
    \begin{frame}\relax
    Перед вами -- шаблон для презентаций \LaTeX\ в CSC. 
    
    Надеюсь, он подойдёт и для выступлений-презентаций-проектов, и для чтения лекций.
    
    \texttt{https://github.com/Lavton/latexLectures}
    
    Если есть вопросы -- пишите \texttt{vk.com/lavton}
    
    \end{frame}
    
    \section{Общий внешний вид}
    \begin{frame}{Шрифт}\relax
        Используем Fira Sans:
        \begin{itemize}
            \item обычный
            \item \textbf{жирный}
            \item \textit{италик}
            \item \textit{\textbf{жирный италик}}
            \item \texttt{моноширинный}
        \end{itemize}
    \end{frame}
    
    \begin{frame}{Низ слайда}\relax
        Внизу можно найти:
        \begin{itemize}
            \item Логотип CSC. Можно заменить, переопределив команду \texttt{\string\logoname}
            \item Заметки на полях. Задаются командой \texttt{\string\cscfootnote}
            \item Счётчик слайдов
             
        \end{itemize}
        \cscfootnote{Пример заметки: CSC -- крутой!}
    \end{frame}
    
    \begin{frame}{Возможные настройки внешнего вида}\relax
        При подключении пакета можно указать следующие опции:
        \begin{itemize}
            \item darktherm -- для тёмной темы. По умолчанию светлая
            \item nologo -- убрать логотип 
            \item useprogressbar -- использовать прогрессбар сверху
            \item classmod -- отображение слайдов в классе и вне его (подробнее в разделе)
             
        \end{itemize}
    \end{frame}

    \begin{frame}{Прогрессбар}\relax
        Прогрессбар идёт по верхней кромке слайда. 
        
        Он показывает количество слайдов, прошедших между секциями.
        Например, этот слайд -- последний в текущей секции и прогрессбар полон. На следующем слайде он обновится.
    \end{frame}
    \AtBeginSubsection[]
    {
      \begin{frame}{подсекции}
      \tableofcontents[currentsubsection, hideothersubsections, sectionstyle=show/hide, subsectionstyle=show/shaded/hide]
      
      \end{frame}
    }
    \section{Как выглядят подсекции}
    \subsection{прямо}
    \subsection{загляденье,}
    \subsection{залюбуешься}
    
    \section{Команды}

    \forcewidefootnote=1
    \begin{frame}{Заметки на полях}{ширина}\relax
        Заметки, как описано выше, задаются командой \texttt{\string\cscfootnote}. Они распалагаются между лого и номером слайда.
        
        Но если заметка длинная, она автоматически займёт всё пространство футера. Это поведение можно регулировать:
        \begin{itemize}
            \item \texttt{\string\forcewidefootnote=-1} -- всегда показывать лого и номер страницы
            \item \texttt{\string\forcewidefootnote=-1} -- иметь зависимость от длины заметки
            \item \texttt{\string\forcewidefootnote=1} -- всегда убирать лого и номер старницы
             
        \end{itemize}
     \cscfootnote{Пример короткой заметки, форсирование уборки лого и номера слайда}    
    \end{frame}
    \forcewidefootnote=0
    
    \begin{frame}{Выделение мыслей}\relax
         \begin{itemize}
            \item \highgreen{\texttt{\string\highgreen\{text\}}}
            \item \highorange{\texttt{\string\highorange\{text\}}}
            \item \highblue{\texttt{\string\highblue\{text\}}}
            \item \highbackgreen{\texttt{\string\highbackgreen\{text\}}}
         \end{itemize}
    \end{frame}
    
    \section{Слайды ``в классе'' / ``дома''}
    
    \begin{frame}{Идея}\relax
        \begin{itemize}
            \item \highgreen{В классе} презентация -- лишь поддержка лектора. Нам нужна анимация, нужен фокус слушателей на главном, нужно задавать вопросы по ходу лекции, но при этом не нужно много текста: все пояснения ведь даются голосом
            \item \highgreen{Дома} презентация -- документ в себе. Люди читают её без прояснений лектора, им не нужна анимация, но может понадобится бОльше пояснений. Например, ссылки на источники конкретного слайда важны для изучения, но лишь отвлекают на лекции.
             
        \end{itemize}
    \end{frame}
        
    \inclassframe{
    \begin{frame}{Пример ``в классе''}\relax
        \begin{itemize}
            \item Если слайд целиком показывается только в классе, он выделяется слева от заголовка. Оберните создание слайда командой \texttt{\string\inclassframe}
            \item \inclasshigh{На слайде можно произвольный кусок текста показать только в классе с помощью \texttt{\string\inclasshigh\{<text>\}}.}
            \inclass{\item Можно без выделения, используя просто \texttt{\string\inclass\{text\}}}
            \inpause\item И \texttt{\string\inpause} -- аналог \string\pause, которая исчезает вне класса
             
        \end{itemize}
         
    \end{frame}}
    \outclassframe{
    \begin{frame}{Пример ``вне класса''}\relax
        \begin{itemize}
            \item Если слайд целиком показывается только вне класса, он выделяется слева от заголовка. Оберните создание слайда командой \texttt{\string\outclassframe}
            \item \outclasshigh{На слайде можно произвольный кусок текста показать только вне класса с помощью \texttt{\string\outclasshigh\{<text>\}}.}
            \inclass{\item Можно без выделения, используя просто \texttt{\string\outclass\{text\}}}
        \end{itemize}
         
    \end{frame}}
    
    \begin{frame}{classmode как параметр пакета}\relax
        \begin{itemize}
            \item без параметра -- то же, что \texttt{classmode=all}
            \item classmode -- то же, что classmode=inclass 
            \item classmode=all -- отображение всей информации
            \item classmode=in -- скрываем слайды вне класса 
            \item classmode=out -- скрываем слайды в классе
            \item classmode=inclass -- скрываем слайды ``вне класса'' + скрываем заметки внизу 
            \item classmode=outclass -- скрываем слайды ``в классе'' + скрываем прогрессбар 
             
        \end{itemize}
         
    \end{frame}
    
    
    

\end{document}