\graphicspath{{sec02/images/}{sec02/code/}}
\lstset{inputpath=sec02/code/}

% включи в рассказ о бимере следующее (но не ограничивайся им):
% темы бимера, структурные элементы (титульник, содержание, нумерация слайдов), 
% оверлеи (можешь посмотреть в лекции за 15 год о них), 
% создание и использование цветов (в т.ч. синтаксис color!50!color!20), 
% колонок (в т.ч. multicols, columns), 
% анимацию в бимере (https://tex.stackexchange.com/questions/240243/getting-gif-and-or-moving-images-into-a-latex-presentation  ), 
% использование конкретно шаблона Сколтеха для презентации

\begin{frame}[t]{Beamer}
    \begin{columns}
    \begin{column}{0.65\textwidth}
        \begin{block}{What is Beamer?}
            Beamer is a \LaTeX\ document class for creating slides for presentations. \\
            It supports pdflatex, latex+dvips, lualatex and xelatex.
        \end{block}
        \visible<2->{\url{https://ctan.org/pkg/beamer}}
    \end{column}
    \begin{column}{0.3\textwidth}
        \begin{center}
            \includegraphics[width=.9\linewidth]{beamerlogo}
        \end{center}
    \end{column}
    \end{columns}
\end{frame}

\begin{frame}{Pros and Cons}

\Huge\centering Pros and Cons
     
\end{frame}

\begin{frame}{Cons}\relax
    Beamer is \alert{not} the best choice...
     \begin{itemize}
        \item[$-$] when you want to put something in {\csk arbitrary position} %\vspace{-1ex}
        \item[$-$] when you want to put {\csk a lot} on a slide % \vspace{-1ex}
        \item<2>[$-$] when you like {\csk transition} effects
    \end{itemize}
    \transdissolve<2>
\end{frame}

\begin{frame}{Pros}\relax
    Beamer is what you need...
     \begin{itemize}
        \item[$+$] When you have lots of {\csk formulas} %\vspace{-1ex}
        \item[$+$] When you {\csk already} have notes prepared in \LaTeX %\vspace{-1ex}
        \item[$+$] When you carry about {\csk device--independent} view and edit %\vspace{-1ex}
        \item<2->[$+$] When you like \only<2>{\csk stepwise viewing }\visible<3>{\alert{overlays}}
        \item<3>[$+$] When you like {\csk stepwise viewing}
    \end{itemize}
\end{frame}


\subsection{Document structure}
\graphicspath{{sec02/images/}{sec02/code/}}
\lstset{inputpath=sec02/code/}

\begin{frame}[label=simple,fragile]{Simplest Beamer document}\relax
    \cprotect\twocolImg{\lstinputlisting{simplest01.tex}}{simplest01}
    \vspace{15mm}
    \inclassFrag{Try it! \hyperlink{style}{\beamerbutton{STYLE}} }[-1]
\end{frame}

\begin{frame}[fragile]{Beamer document structure}\relax
    \lstinputlisting[numbers=none]{structure.tex}
\end{frame}

\begin{frame}[fragile]{Title page (Preamble)}\relax
    \cprotect\twocolImg{\lstinputlisting[linerange={5-17}]{title.tex}}{title}
\end{frame}

\begin{frame}[fragile]{TOC}\relax
    \cprotect\twocolImg{\lstinputlisting[linerange={11-13}]{toc.tex}}{toc1}
\end{frame}

\begin{frame}[fragile]{TOC ({\tt AtBeginSection[]})}\relax
    \cprotect\twocolImg{\lstinputlisting[linerange={3-7}]{toc.tex}}{toc2}
\end{frame}


\begin{frame}[fragile]{Appendix}\relax
    \cprotect\twocolImg{\visible<2>{\lstinputlisting[linerange={4-4}]{appendix.tex}} \\
    \lstinputlisting[linerange={9-12}]{appendix.tex}}{appendix}
\end{frame}

\begin{frame}[fragile]{Bibliography (bibtex)}\relax
    \cprotect\twocolImg{\lstinputlisting[linerange={10-16}]{bibtex.tex}}{bibtex.pdf}
\end{frame}

\begin{frame}[fragile]{Bibliography (simple)}\relax
    \cprotect\twocolImg{\lstinputlisting[linerange={9-21}]{bibliography.tex}}{bibliography.pdf}
\end{frame}

\begin{frame}[fragile]{Frame: Columns}\relax
    \cprotect\twocolImg{\lstinputlisting[linerange={8-17}]{columns.tex}}{columns.pdf}
\end{frame}



\subsection{Style: themes and colors}
% \knuthc  knuth the TeXBook
% \lvoc   Lvovsky
% \lamc  lamport latex 
% \slshape different font for footnote
\graphicspath{{sec01/images/s2/}{sec01/code/s2/}}
\lstset{inputpath=sec01/code/s2/}

\begin{frame}[fragile]{How good it will be if...\only<2,3>{we could write like this}}\relax
\begin{center}
\only<1>{\includegraphics{refBegin}}
\only<2>{\includegraphics{refBegin2}}
\only<3>{\Huge We can!}
\end{center}

\cprotect\skfootnote{I use \verb|\only<2>{text}| for such effect. Also \verb|\setcounter| was used in source.}
\end{frame}

\begin{frame}[fragile]{Step 1: \ccol\label}\relax
     \lstinputlisting[linerange={12-15}, basicstyle=\tt]{refBegin2.tex}
     
     \skfootnote{for whole reference mechanism: \lmanc{7}[51] \lvoc{I.2.11}[27] \overC{https://www.overleaf.com/learn/latex/Cross_referencing_sections_and_equations}, \wikiC{https://en.wikibooks.org/wiki/LaTeX/Labels_and_Cross-referencing}}
\end{frame}

\begin{frame}[fragile]{Step 2: \ccol{\ref} and \ccol{\pageref}}
     \lstinputlisting[linerange={17-17}, basicstyle=\tt]{refBegin2.tex}
\end{frame}

\begin{frame}[fragile]{Problem 1: lots of labels!}
    What if you have a lot of labels?
    \inclass{\incPause Look throw whole document and compare \TeX\ and pdf?\incPause }
    
\cprotect\twocolImg{
    \lstinputlisting[linerange={10-10, 13-16}, basicstyle=\tt\small]{refBeginShow.tex}
}{refBeginShow}    
    
    
    \skfootnote{\url{http://ctan.altspu.ru/macros/latex/contrib/showlabels/showlabels.pdf}}
\end{frame}

\begin{frame}[fragile]{Problem 2: Typos}\relax
\cprotect\twocolImg{
    \lstinputlisting[linerange={12-17}, basicstyle=\tt\small]{refBad.tex}
}{refBad}           
\end{frame}

\begin{frame}[fragile]{Bibliography}{How to cite}\relax
Use \ccol{\cite\{label\}}

\cprotect\twocolImg{
    \lstinputlisting[linerange={9-9}, basicstyle=\tt\small]{citeFirst.tex}
}{citeFirst}  

\incPause
\cprotect\twocolImg{
    \lstinputlisting[linerange={11-12}, basicstyle=\tt\small]{citeSecond.tex}
}{citeSecond}  

\skfootnote{\lmanc{8.24.2}[94] \wikiC{https://en.wikibooks.org/wiki/LaTeX/Bibliography_Management} \overC{https://www.overleaf.com/learn/latex/Bibliography_management_in_LaTeX}}     
\end{frame}

\begin{frame}[fragile]{Bibliography}{What to cite}\relax

{\csk .bib} files

\begin{verbatim}
@Book{landau,
    author = {Landau, L. D. and Lifshitz, E. M.},
    title = {The Classical Theory of Fields},
    journal = N,
    volume = {1},
    pages = {140},
    year = 1980
}
\end{verbatim}

You can have multiple records in one .bib file.
     
\end{frame}

\begin{frame}[fragile]{Bibliography. Where can you get .bib files?\preMagicPage}\relax
\begin{itemize}
    \item Just google it! ``article\_name bibtex''
    \item Go to you favorite journal and look at Citations -> ``.bib'' or ``bibtex''
    \item Ask Mendeley or other programs to give you the .bib file
    \item Create it by yourself
\end{itemize}
\end{frame}

\begin{frame}[fragile]{Bibliography. Creating .bib file \magicPage}\relax
\begin{center}
\begin{tabular}{rl}
     \ccol{@article} & Journal or magazine article\\
\ccol{@book} & Book\\
\ccol{@conference} & Article in conference proceedings\\
\ccol{@misc} & If nothing else fits.
\end{tabular}
\end{center}
\skfootnote{\wikiC{https://en.wikibooks.org/wiki/LaTeX/Bibliography_Management\#Standard_templates} for cite url checkout \stExC{https://tex.stackexchange.com/questions/3587/how-can-i-use-bibtex-to-cite-a-web-page}
}     
\end{frame}

\begin{frame}[fragile]{Bibliography. Creating .bib file \magicPage}\relax
\begin{center}
{\obeylines
author
title
journal
year
pages
volume
...
}
\end{center}
\skfootnote{check for full list \wikiC{https://en.wikibooks.org/wiki/LaTeX/Bibliography_Management\#BibTeX}
}     
\end{frame}

\begin{frame}[fragile]{Bibliography. Offline\preMagicPage}\relax
Running \LaTeX\ offline, you can get \textbf{(??)} in \ccol{\ref} and \textbf{[?]} in \ccol{\cite}. 

For 
\begin{itemize}
    \item References 
    \item Bibliography
    \item Table of content
    \item Indexing
    \item ...
\end{itemize}

\LaTeX\ collect addition data in extra files. \LaTeX\ need more then one run to get this data. 

Use \verb|latex; bibtex; latex; latex|

\end{frame}

\begin{frame}[fragile]{Bibliography. Manually\magicPage}

You can add Bibliography manually.

\cprotect\twocolImg{
    \lstinputlisting[linerange={10-22}]{citeMan.tex}
}{citeMan}  

\skfootnote{\lmanc{8.24}[92]}
\end{frame}

\begin{frame}[fragile]{Bibliography. Styles\magicPage}

You can change styles.

Mannually -- check \url{https://en.wikibooks.org/wiki/LaTeX/Bibliography_Management}

Our with packages -- check \url{https://tex.stackexchange.com/questions/25701/bibtex-vs-biber-and-biblatex-vs-natbib}
\end{frame}

\begin{frame}[fragile]{Subject index\magicPage}\relax
\cprotect\twocolImg{
    \lstinputlisting[linerange={9-16}]{indexmy.tex}
}{indexmy}  

\skfootnote{\lvoc{IV.7}[175] \lmanc{25.2}[215]}     
\end{frame}

\subsection{Tricks: overlays, animation, notes}
% \knuthc  knuth the TeXBook
% \lvoc   Lvovsky
% \lamc  lamport latex 
% \slshape different font for footnote
\graphicspath{{sec01/images/s3/}{sec01/code/s3/}}
\lstset{inputpath=sec01/code/s3/}

\begin{frame}[fragile]{\ccol{\footnote}\magicPage}\relax
\cprotect\twocolImg{
    \lstinputlisting[linerange={9-9}]{footnoteMy.tex}
}{footnoteMy}  


\skfootnote{\lmanc{8}[108] \wikiC{https://en.wikibooks.org/wiki/LaTeX/Footnotes_and_Margin_Notes} \overC{https://www.overleaf.com/learn/latex/Footnotes} \knuthc{15}[120]}
\end{frame}

\begin{frame}[fragile]{Horizontal aligment\magicPage}\relax
\cprotect\twocolImg{
    \lstinputlisting[linerange={9-25}]{flushing.tex}
}{flushing}  


\skfootnote{\lmanc{8.12}[63] \lmanc{8.13}[64] \knuthc{14}[112] and note \stExC{https://tex.stackexchange.com/questions/64644/centering-doesnt-seem-to-center-my-text}}
\end{frame}

\begin{frame}[fragile]{Page break\magicPage}\relax

{\centering \Large \ccol{\newpage} \ccol{\pagebreak}\par}

\skfootnote{\lmanc{10}[105] \stExC{https://tex.stackexchange.com/questions/736/pagebreak-vs-newpage}}
     
\end{frame}

\begin{frame}[fragile]{Quotes\magicPage}\relax

\cprotect\twocolImg{
    \lstinputlisting[linerange={10-14}]{quotemy.tex}
}{quotemy}  


\cprotect\skfootnote{\lvoc{III.7.1}[129] \lmanc{8.20}[83]. Also see \verb|quotation| env }
     
\end{frame}

\begin{frame}[fragile]{Verses\magicPage}\relax

\cprotect\twocolImg{
    \lstinputlisting[linerange={11-18}]{versemy.tex}
}{versemy}  

\skfootnote{\lvoc{III.7.3}[130] \lmanc{8.28}[98]}
\end{frame}

\begin{frame}[fragile]{Marginal notes\magicPage}\relax

\cprotect\twocolImg{
    \lstinputlisting[linerange={9-10}]{marginmy.tex}
}{marginmy}  

\skfootnote{\lvoc{IV.10}[194] \lmanc{15.4}[137]}
\end{frame}

 
\cprotect\inclassFrame{
\begin{frame}[fragile]{Try to add a package}\relax
\url{http://hanno-rein.de/downloads/coffee.sty}

and use \verb|\newpage \coffee{2} hello world!|

\incPause
NOW you document is ready!

\end{frame}
}
