\graphicspath{{sec02/images/}{sec02/code/}}
\lstset{inputpath=sec02/code/}

\begin{frame}[fragile]{Stepwise viewing}
\skfootnote{\href{https://ctan.org/pkg/beamer}{man: sec 9.1}}
    Command \lstinline[basicstyle=\tt]|\pause|\pause\
    is the simplest way to create an overlay.
    \pause
    \[
        \zeta(s) = \sum^{\infty}_{k=1}{\frac{1}{k^s}},  \quad  \operatorname{Re}s > 1.
    \]
    \onslide\lstinline[basicstyle=\tt]|\onslide| command tells to show material from the first slide.\\
    \onslide+<3->\lstinline[basicstyle=\tt]|\onslide<3->| tells to show material from the third slide on.\\ 
    \pause \lstinline[basicstyle=\tt]|\pause| command then leads to the next slide.
\end{frame}

\begin{frame}[fragile]{Stepwise viewing}
\skfootnote{\href{https://ctan.org/pkg/beamer}{man: sec 9}}
    Most of the commands are self--explanatory.\\
    \lstinline[basicstyle=\tt]|\pause<#>| --- following text shown only after slide \#  \\
    \lstinline[basicstyle=\tt]|\onslide<#>|  --- visible only slide \#\\
    \lstinline[basicstyle=\tt]|\FromSlide{#}| --- equivalent to \lstinline[basicstyle=\tt]|\onslide<#->|. \\
    \lstinline[basicstyle=\tt]|\only<#>| --- visible on particular slides, otherwise absent  \\
    \lstinline[basicstyle=\tt]|\uncover<#>| --- visible on particular slides, otherwise transparent \\
    \lstinline[basicstyle=\tt]|\visble<#>|  --- visible on particular slides, otherwise invisible  \\
    \lstinline[basicstyle=\tt]|\invisble<#>| --- opposite of \lstinline[basicstyle=\tt]|\visble<#>|. \\
    \inclassFrag{Check the difference between \texttt{only} and \texttt{onslide} }[-1]
\end{frame}

\begin{frame}[fragile]{Overlays}
    Overlay specifications can also be written behud some commands like 
    \lstinline[basicstyle=\tt]|\textbf|,
    \lstinline[basicstyle=\tt]|\item|,
    \lstinline[basicstyle=\tt]|\color|,
    \lstinline[basicstyle=\tt]|\alert|. \\
    \pause
    \lstinline[basicstyle=\tt]|\begin{enumerate}| \\
    \lstinline[basicstyle=\tt]|    \item<1-> Every \alert<3>{thing}|   \\
    \lstinline[basicstyle=\tt]|    \item<only@3,4> that has|   \\
    \lstinline[basicstyle=\tt]|    \item<2-> beginning| \\
    \lstinline[basicstyle=\tt]|    \item<1,4>  has end.|   \\
    \lstinline[basicstyle=\tt]|\end{enumerate}|
\end{frame}

\begin{frame}{Overlays}
    \begin{enumerate}
        \item<1-> Every \alert<3>{thing}
        \item<only@3,4> that has
        \item<2-> beginning
        \item<1,4> has end.
    \end{enumerate}
\end{frame}

\cprotect\inclassFrame{
\begin{frame}[fragile]{Can you guess what does this code produce?}\relax
    \lstinline[basicstyle=\tt]|\begin{itemize}[<+->]| \\
    \lstinline[basicstyle=\tt]|    \item Every thing|   \\
    \lstinline[basicstyle=\tt]|    \item that has|   \\
    \lstinline[basicstyle=\tt]|    \item beginning| \\
    \lstinline[basicstyle=\tt]|    \item has end.|   \\
    \lstinline[basicstyle=\tt]|\end{itemize}|
\end{frame}
}

\begin{frame}{Overlays}
    \begin{itemize}[<+->]
        \item Every thing
        \item that has
        \item beginning
        \item has end.
    \end{itemize}
\end{frame}

\begin{frame}{Animation}
\magicPage
    \transduration<1-4>{.5}
    \begin{itemize}
        \temporal<1>{\color{gray}}{\color{blue}}{\color{red!75}}{\item Every thing}
        \temporal<2>{\color{gray}}{\color{blue}}{\color{red!75}}{\item that has}
        \temporal<3>{\color{gray}}{\color{blue}}{\color{red!75}}{\item beginning}
        \temporal<4>{\color{gray}}{\color{blue}}{\color{red!75}}{\item has end.}
    \end{itemize}\vspace{5mm}
    \visible<5>{Cool, right?}
\end{frame}

\begin{frame}[fragile]{Animation}
    \lstinline[basicstyle=\tt]|\animate<1-4>| \\
    \lstinline[basicstyle=\tt]|\begin{itemize}[<+->]| \\
    \lstinline[basicstyle=\tt]|    \item Every thing|   \\
    \lstinline[basicstyle=\tt]|    \item that has|   \\
    \lstinline[basicstyle=\tt]|    \item beginning| \\
    \lstinline[basicstyle=\tt]|    \item has end.|   \\
    \lstinline[basicstyle=\tt]|\end{itemize}|
\end{frame}

\begin{frame}[fragile]{Animation}
    \lstinline[basicstyle=\tt]|\transduration<1-4>{.5}| \\
    \lstinline[basicstyle=\tt]|\begin{itemize}[<+->]| \\
    \lstinline[basicstyle=\tt]|    \item Every thing|   \\
    \lstinline[basicstyle=\tt]|    \item that has|   \\
    \lstinline[basicstyle=\tt]|    \item beginning| \\
    \lstinline[basicstyle=\tt]|    \item has end.|   \\
    \lstinline[basicstyle=\tt]|\end{itemize}|
\end{frame}