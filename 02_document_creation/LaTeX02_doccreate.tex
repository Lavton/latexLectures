%https://tex.stackexchange.com/questions/114219/add-notes-to-latex-beamer
\documentclass[14pt, aspectratio=169]{beamer}
\usepackage{fontspec}
\usepackage{xunicode}
\usepackage{xltxtra}
\usepackage{dsfont}
\usepackage{multicol}
\usepackage{multirow}
% \usepackage{pgfpages}
% \setbeameroption{show notes}
%\setbeameroption{hide notes} % Only slides
%\setbeameroption{show only notes} % Only notes
% \setbeameroption{show notes on second screen=right} % Both
%% USAGE: 
% \usepackage[logo="sklogo.png"]{beamerskoltech} 
%   if you have a stand-alone image file for Sk logo 
% or
% \usepackage[logo]{beamerskoltech} 
%   if you has no logo-file, but want LaTeX to generate it. 
%   In this case you probably will need to use `--enable-write18 -interaction=nonstopmode` arguments running the latex command.
%   in papeeria and overleaf all works fine
% or 
% \usepackage{beamerskoltech}
%   In case you don't want logo at all 
%
% provided commands:
% color `skoltechgreen` -- the dark-green color for structure elements 
% command `\logoname` -- the name of logo file if exist 
% command `{\csk <text>}` -- the shortcut from `\color{skoltechgreen}`
% command `skfootnote{text}` -- put some text for current slide
% `\renewcommand{\skbeforetitle}{\vspace{-3ex}}` is useful in case you use `aspectratio=169`
%%%%%%%%%%%%%%%%
\usepackage[logo]{beamerskoltech} 
\usepackage{lecturessk}
\renewcommand{\skbeforetitle}{\vspace{-3ex}}

\usepackage{xecyr}
\usepackage{hyperref}

\usepackage{polyglossia}
\usepackage{graphicx}
\usepackage{listings}



\begin{document}


\title{\LaTeX:\\ \Large from dummy to \TeX nician}
\subtitle{Create articles, presentations, posters}
\author{David Saykin \& Anton Lioznov}
\institute{Skoltech}
\date{ISP 2019,\\ \textit{lesson 2}}
\frame{\titlepage}
\begin{frame}\frametitle{What we will know?}
\tableofcontents[hideallsubsections]
\end{frame}
\begin{frame}{Acknowledgments}\relax
We acknowledge 
\begin{description}
    \item[Vlad Yurchenko] for beeing co-author of the first version of this presentation in 2015 
    \item[Dmitry Barashev] for useful comments, that was included in the presentation
    \item[Alexander Kulikov] for useful comments, that was included in the presentation
\end{description}
\end{frame}

\begin{frame}{Agreements}{I}\relax
{ \Large Footnotes }
\smash{
\raisebox{-5.4cm}{
\begin{tikzpicture}
\draw[white] (0,0) -- (0, 0.0);
\draw[->,ultra thick] (0,0) to[out=0,in=45] (8, -1.5) -- (4,-5.5);
\end{tikzpicture}}
}

\skfootnote{Like this}
\begin{itemize}
    \item Only in the ``out-class'' version
     \item For second reading
     \item Containe advanced usage of the command 
     \item Containe references to read more 
     \begin{itemize}
         \item to the exact chapter 
         \item (often) with the href to exact page  
     \end{itemize}
     \item Containe some comments 
\end{itemize}
\end{frame}

%\renewcommand{\skfootnote}[1]{}

\begin{frame}{Agreements}{II}\relax
{ \Large Addition information -- ``magic'' } 
\smash{\begin{tikzpicture}
\draw[white] (0,0) -- (0, 0.6);
\draw[->,ultra thick] (0,0.3) --(3,1.8);
\end{tikzpicture}}
\magicPage

\begin{itemize}
     \item To have the full picture 
     \item Not to analyze or to puzzle out in class 
\end{itemize}
\end{frame}

\inclassFrame{
\begin{frame}{Agreements}{III}\relax
\centering
     {\LARGE This slide means it is ``in-class task time''\par
     \inclassFrag{This box means it is ``in-class task time''}}
     These slides and boxes will be removed in the ``out-class'' version. However, they still be in the source code. 
\end{frame}
}

\AtBeginSection[] 
{
  \begin{frame}{What we will know?}
  \tableofcontents[currentsection,hideallsubsections]
  \end{frame}
}

\AtBeginSubsection[]
{
  \begin{frame}{What we will know?}
  \tableofcontents[currentsubsection, hideothersubsections, sectionstyle=show/hide, subsectionstyle=show/shaded/hide]
  \end{frame}
}

\section{Document Creation}

% pro and contr
\begin{frame}[t, fragile]{Pros and Cons}
\vspace{-3.5ex}
\cwpa{
    \small
    \begin{itemize}
        \item[$+$] When you have lots of equations\vspace{-1ex}
        \item[$+$] When you have a complex, but typical document \vspace{-1ex}
        \item[$+$] When you carry about device-independant view\vspace{-1ex}
        \item[$+$] When you don't want not care about the beauty, but want it\vspace{-1ex}
        \item[$+$] When you are care about the beauty wery much\vspace{-1ex}
        \item[$+$] When you hate all these humanitarian GUI programs
        % \item[$+$] When you love text files 
        % \item[``+''] When you want to impress your partner
        % yeah, I leave the commented item for you, source-code reader!
    \end{itemize}
}{
    \small
    \begin{itemize}
        \item[$-$] when you want to put something in attributary position\vspace{-1ex}
        \item[$-$] when you want to do something ``against the rules''\vspace{-1ex}
        \item[$-$] when you want to work with visual-based things (tables,...)\vspace{-1ex}
        \item[$-$] when you want to do something really simple \vspace{-1ex}
        \item[$-$] when you want to do something ``quick and dirty'' \vspace{-1ex}
    \end{itemize}
}[t]

\cprotect\skfootnote{
\verb|\item[$+$]|
}
\end{frame}

\begin{frame}{Science reseach about \LaTeX}\relax
     \begin{center}
     ``We show that LaTeX users were {\color{red}slower} than Word users <...> and produced {\color{red}more typesetting},<...>. LaTeX users, however, more often report {\color{green}enjoying using} their respective software.''          
     \end{center}

     
\skfootnote{\url{https://journals.plos.org/plosone/article?id=10.1371/journal.pone.0115069} (or in russian \url{https://habr.com/post/375109/})}
\end{frame}

\begin{frame}{Illustration}\relax
\begin{center}
     
     \begin{tikzpicture}[axes/.style=,important line/.style={very thick},]
        \begin{scope}[axes]
        \draw[->] (0,0) -- coordinate (x axis mid) (8,0);
    	\draw[->] (0,0) -- coordinate (y axis mid) (0,5);
    	\node[below] at (x axis mid) {document complexity and size};
    	\node[rotate=90, above] at (y axis mid) {effort and time consumption};
        \end{scope}
        \draw (0, 1)  .. controls (3,1) and (6,2) .. (8, 4) node[below=0.5cm]{\LaTeX};
        \draw[dashed] (0, 0.1) .. controls (3, 0.5) and (4,2) .. (5, 5) node[left]{MS Word};
     \end{tikzpicture}
\end{center}
     \skfootnote{based on \url{http://www.pinteric.com/miktex.html} picture\\ tikZ for plotting \url{http://www.texample.net/tikz/examples/line-plot-example/}}
\end{frame}


\begin{frame}{Common belief}\relax
    \begin{center}
        \begin{tikzpicture}
             \node[align=center] (0,0) {
             \huge \LaTeX\ is only for use\\ \huge  in academic area
             };
             \uncover<2>{\node[rotate=30, bottom color=red!50, top color=red!50] (0,0) {\Huge WRONG}};
        \end{tikzpicture}
         
    \end{center}
    \skfootnote{It was done with \tt TikZ}
\end{frame}

\begin{frame}[t]{The power of \LaTeX\ in it's templates and flexability!}\relax
\vspace{-1ex}
     Look at examples at:
     \begin{itemize}
         \item \url{https://www.latextemplates.com/}
         \item \url{https://tex.stackexchange.com/questions/158668/nice-scientific-pictures-show-off}
         \item \url{https://tex.stackexchange.com/questions/1319/showcase-of-beautiful-typography-done-in-tex-friends}
        \item $\cdots$
     \end{itemize}
     
     \vspace{-1ex}
\inclassFrag{  Please, look at these sites for two minutes  }[-1]
     
     \skfootnote{here are some more links \stExC{https://tex.stackexchange.com/questions/1362/latex-template-gallery}
     }
\end{frame}

\begin{frame}{Conclusion}\relax 

Now, in \the\year , using \LaTeX\ to write scientific articles with no math inside is more metter of joy, not productivity: MS Office took over lots of \LaTeX 's ideas.\\[2ex]

But \LaTeX\ becoming better too! because of packages, online tools and developing \LaTeX 3.

And for something as complex as this presentation you'll spend way too more time, trying to reproduce it with MS Office.
     
     \skfootnote{If you compile this presentation from source, you'll find that the year in the slide -- is the year of compilation :)}
\end{frame}


% view
\begin{frame}{What we have}
\skfootnote{\wikiC{https://en.wikibooks.org/wiki/LaTeX/Introduction}\\ \normalfont\url{http://www.texample.net/tikz/examples/tree/}}
\begin{center}
\begin{tikzpicture}[sibling distance=10em,
  every node/.style = {shape=rectangle, rounded corners,
    draw, align=center,
    top color=white, bottom color=skoltechgreen!20}]]
  \node {\TeX}
    child { node {\LaTeX}
      child { node {XeLaTeX} }
      child { node {LuaTeX} }
      } ;
\end{tikzpicture}

\end{center}
\end{frame}

\note[itemize]{
\item Start from \LaTeX
}

% LaTeX
\begin{frame}{Definitions}
\skfootnote{\normalfont\url{https://ru.wikipedia.org/wiki/LaTeX}}
\begin{columns}
\begin{column}{0.5\textwidth}
\begin{tikzpicture}[sibling distance=10em,
  every node/.style = {shape=rectangle, rounded corners,
    draw, align=center,
    top color=white, bottom color=skoltechgreen!20}]]
  \node {\TeX}
    child { node[text=red] {\LaTeX}
      child { node {XeLaTeX} }
      child { node {LuaTeX} }
      } ;
\end{tikzpicture}
\end{column}

\begin{column}{0.5\textwidth}
\small
{\csk \LaTeX} --- is the most popular set of macro-extensions (or macro package) of the computer typesetting system \TeX, which facilitates a typesetting of complex documents.

\end{column}
\end{columns}
\end{frame}

\note[itemize]{
\item What is macro 
\item macros = shortcuts 
}

% TeX
\begin{frame}{Definitions}
\skfootnote{\normalfont\url{https://en.wikipedia.org/wiki/TeX}}
\begin{columns}
\begin{column}{0.5\textwidth}
\begin{tikzpicture}[sibling distance=10em,
  every node/.style = {shape=rectangle, rounded corners,
    draw, align=center,
    top color=white, bottom color=skoltechgreen!20}]]
  \node[text=red] {\TeX}
    child { node {\LaTeX}
      child { node {XeLaTeX} }
      child { node {LuaTeX} }
      } ;
\end{tikzpicture}
\end{column}

\note{
how to pronaunce\\
Book The Art of Computer Programming \\ 
History 1978 (pdf=1993)
}

\begin{column}{0.5\textwidth}
\small
{\csk \TeX} --- is a typesetting system designed and mostly written by Donald Knuth. TeX was designed with two main goals in mind: to allow anybody to produce high-quality books using minimal effort, and to provide a system that would give exactly the same results on all computers, at any point in time

\end{column}
\end{columns}
\end{frame}

% LuaTeX and XeLaTeX
\begin{frame}{Definitions\preMagicPage}
\skfootnote{\normalfont\url{https://en.wikipedia.org/wiki/LuaTeX}, \url{https://en.wikipedia.org/wiki/XeTeX}}
\begin{columns}
\begin{column}{0.5\textwidth}
\begin{tikzpicture}[sibling distance=10em,
  every node/.style = {shape=rectangle, rounded corners,
    draw, align=center,
    top color=white, bottom color=skoltechgreen!20}]]
  \node {\TeX}
    child { node {\LaTeX}
      child { node[text=red] {XeLaTeX} }
      child { node[text=red] {LuaTeX} }
      } ;
\end{tikzpicture}
\end{column}

\begin{column}{0.5\textwidth}
\small
{\csk XeLaTeX} --- XeTeX is a \TeX typesetting engine using Unicode and supporting modern font technologies such as OpenType, Graphite and Apple Advanced Typography

{\csk LuaTeX} --- LuaTeX is a \TeX-based computer typesetting system which started as a version of pdfTeX with a Lua scripting engine embedded
\end{column}
\end{columns}
\end{frame}



\begin{frame}{Resourses\magicPage}%{Books etc}

\vspace{-3.0ex}
\begin{itemize}
    \item Knuth ``The \TeX Book'' (en, ru)
    \item L'vovsky ``Nabor i verstka v sisteme \LaTeX'' (ru)
    \item Lamport. ``\LaTeX. A Document Preparation System, User’s Guide and Reference Manual'' (en)
    \item Gratzer ``Math into \LaTeX'' (en) 
    \item Oetiker ``The Not So Short Introduction to \LaTeX'' (en, ru)
    \item \url{https://www.overleaf.com/learn}
    \item \url{https://www.latex-project.org/help/}
    \item \url{https://texfaq.org/}
\end{itemize}
\end{frame}

\note{
We will use papeeria\\
pointed about footnotes and colors!
}

\begin{frame}[fragile]{Resourses \magicPage}{Interesting links}
\cprotect\skfootnote{
\verb|\begin{description} \item[questions about \TeX]\end{description}|
}
\small
\begin{description}
    \item[questions about \TeX] \url{https://tex.stackexchange.com} \vspace{-1ex}
    \item[knowing a command of the symbol] \url{http://detexify.kirelabs.org/classify.html} \vspace{-1ex}
    \item[beauty of TikZ] \url{http://www.texample.net/tikz/examples/} \vspace{-1ex}
    \item[beauty of pictures] \url{https://tex.stackexchange.com/questions/158668/nice-scientific-pictures-show-off}\vspace{-1ex}
    \item[beauty of typesetting] \url{https://tex.stackexchange.com/questions/1319/showcase-of-beautiful-typography-done-in-tex-friends}\vspace{-1ex}
\end{description}
\end{frame}

\begin{frame}[fragile]{where to get}
\begin{enumerate}
    \item Online
    \begin{itemize}
        \item \url{http://papeeria.com}
        \item \url{https://overleaf.com}
    \end{itemize}
    \item Offline
    \begin{itemize}
        \item \LaTeX{} \url{https://www.latex-project.org/get/}
        \item package manager \verb'tlmgr'
    \end{itemize}
\end{enumerate}
     
\end{frame}








\section{Presentation Creation}

\graphicspath{{sec02/images/}}
\lstset{inputpath=sec02/code}

\begin{frame}{WYSiWYG vs not-WYSiWYG editors}
\cwpa{
{\csk WYSiWYG} -- \textit{What You See is What You Get} editor\\~\\

\inclassFrag{Examples?}[0]
}{
Microsoft Word
\insImg{wysiwyg.png}
}
\end{frame}


\begin{frame}[fragile]{not-WYSiWYG}


\cprotect\skfootnote{``\verb|\begin{frame}[fragile]|'' to put code inside}


\inclassFrag{Examples?}[1]
\cprotect[mm]\cwpa{
{\small HTML }

\begin{lstlisting}[language=html]
<html>
  <head>
    <meta charset="utf-8">
  </head>
  <body>
    <h1>Header</h1>
    <i>Hello</i>,<br/> world!  <!-- comment -->
  </body>
</html>
\end{lstlisting}

}{
{\small \LaTeX}
\lstinputlisting{helloworld.tex}

}[t]

\end{frame}

\begin{frame}[fragile]{Commands}
    \lstinline[basicstyle=\tt\large]|\command[o1, o2]{n1, n2=value}[o3]{n3}|\par
    (o = optional argument, n = necessary argument)~\\[2ex]

\pause
    \skfootnote{\knuthc{ch 3}, \lvoc{ch 2.3, 2.7}}

    {\csk Command symbols}
    \verb|\$ \# \{ \} \^{} \& \_ \~{} \\|
    \vfil
    {\csk Command words}
    \verb|\sin \LaTeX \Rightarrow \qquad|
    
    {\csk Enviruments}
    \verb|\begin{frame}\end{frame}  \begin{equation}\end{equation}|
    
\end{frame}

\begin{frame}[fragile]{Document structure}{overview}
    \lstinputlisting{structureOfDocument.tex}
\end{frame}

\begin{frame}[fragile]{Document structure}{class files}

Class of the document responsable for the large-scale settings
    \cprotect\skfootnote{\lvoc{IV.1}, \lamc{2.2.2} \url{https://texblog.org/2013/02/13/latex-documentclass-options-illustrated/}\\ envirament \lstinline|tabbing|} 
{
\scriptsize
\begin{tabbing}
\lstinline|\documentclass|\hspace{-1ex} \= \lstinline|[10pt,|  \= \lstinline|onecolumn,|  \=  \lstinline|a4paper]|\hspace{-1ex} \= \lstinline|{article}| \kill
\> \> \> \> \lstinline|{beamer} %presentation, poster| \\
\> \> \> \> \lstinline|{report}| \\
\> \> \> \> \lstinline|{book}| \\
\> \> \> \> \lstinline|{standalone} %for one picture/equation| \\
\> \> \> \> \lstinline|{extarticle} %if you want 14pt font size| \\
\lstinline|\documentclass| \> \lstinline|[10pt,|  \> \lstinline|onecolumn,|  \>  \lstinline|a4paper]| \> \lstinline|{article}| \\ 
\> \lstinline|[12pt] %fontsize| \> \\ 
\> \> \lstinline|[twocolumns] %number of columns in document| \> \\ 
\> \> \> \lstinline|[a5paper] %paper size| \> \\ 
\end{tabbing}}
\cprotect\skfootnote{\slshape in this presentation \verb|[aspectratio=169]|}
\end{frame}
\note{
but there is no ``fixed set'' of classes. Can build our own 
}

%%TSK: предложить создать документ и поиграться с параметрами
\begin{frame}[fragile]{Document structure}{style files}
     Style files are responsable for settings and providing new commands 
     
     \vfill
     \hspace{-1ex}
     \lstinline[basicstyle=\tt\large]|\usepackage[optional]{necessary}{packagename}|
     \vfill
\end{frame}

\note{practicully everything is from package }


\inclassFrame{
\begin{frame}{This slide means it is ``in-class task time!''}\relax
\centering
     \LARGE This slide means it is ``in-class task time!''
\end{frame}
}
\cprotect\inclassFrame{
\begin{frame}[fragile]{Try to compile your first doc}\relax
     \lstinputlisting[basicstyle=\tt\small]{first.tex}
\end{frame}
}

\begin{frame}\frametitle{What we have learned today?}
\tableofcontents
\end{frame}


\begin{frame}[allowframebreaks]{references}
color from the footnotes corresponds to references' color.
    \begin{itemize}
        \item \knuthc{Knuth ``The \TeX Book''}
        \item \lvoc{L'vovsky ``Nabor i verstka v sisteme \LaTeX''}
        \item \lamc{Lamport. ``\LaTeX. A Document Preparation System, User’s Guide and Reference Manual''}
        \item \lmanc{``\LaTeX 2e: An unofficial reference manual''} also at website \url{https://latexref.xyz/}
        \item \stExC{https://tex.stackexchange.com/questions} : \url{https://tex.stackexchange.com/questions}
        \item \wikiC{https://en.wikibooks.org/wiki/LaTeX} : \url{https://en.wikibooks.org/wiki/LaTeX}
        \item \overC{https://www.overleaf.com/learn/latex} : \url{https://www.overleaf.com/learn/latex}
        \item \tugC{https://www.tug.org/utilities/plain/cseq.html} : \url{https://www.tug.org/utilities/plain/cseq.html}
    \end{itemize}
\end{frame}

\begin{frame}{Distribution}\relax
\begin{itemize}
     \item the pdf-version of the presentation and all printed materials can be distributed under license Creative Commons Attribution-ShareAlike 4.0 \url{https://creativecommons.org/licenses/by-sa/4.0/}
     \item The source code of the presentation is available on {\csk\url{https://github.com/Lavton/latexLectures}} and can be distributed under the MIT license \url{https://en.wikipedia.org/wiki/MIT_License\#License_terms}
\end{itemize}
     
\end{frame}
\end{document}