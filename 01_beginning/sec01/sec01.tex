% view
\begin{frame}{What we have}
\skfootnote{\url{http://www.texample.net/tikz/examples/tree/}}
\begin{center}
\begin{tikzpicture}[sibling distance=10em,
  every node/.style = {shape=rectangle, rounded corners,
    draw, align=center,
    top color=white, bottom color=skoltechgreen!20}]]
  \node {\TeX}
    child { node {\LaTeX}
      child { node {XeLaTeX} }
      child { node {LuaTeX} }
      } ;
\end{tikzpicture}

\end{center}
\end{frame}

\note[itemize]{
\item Start from \LaTeX
}

% LaTeX
\begin{frame}{Definitions}
\skfootnote{\url{https://ru.wikipedia.org/wiki/LaTeX}}
\begin{columns}
\begin{column}{0.5\textwidth}
\begin{tikzpicture}[sibling distance=10em,
  every node/.style = {shape=rectangle, rounded corners,
    draw, align=center,
    top color=white, bottom color=skoltechgreen!20}]]
  \node {\TeX}
    child { node[text=red] {\LaTeX}
      child { node {XeLaTeX} }
      child { node {LuaTeX} }
      } ;
\end{tikzpicture}
\end{column}

\begin{column}{0.5\textwidth}
\small
{\csk \LaTeX} --- is the most popular set of macro-extensions (or macro package) of the computer typesetting system \TeX, which facilitates a typesetting of complex documents.

\end{column}
\end{columns}
\end{frame}

\note[itemize]{
\item What is macro 
\item macros = shortcuts 
}

% TeX
\begin{frame}{Definitions}
\skfootnote{\url{https://en.wikipedia.org/wiki/TeX}}
\begin{columns}
\begin{column}{0.5\textwidth}
\begin{tikzpicture}[sibling distance=10em,
  every node/.style = {shape=rectangle, rounded corners,
    draw, align=center,
    top color=white, bottom color=skoltechgreen!20}]]
  \node[text=red] {\TeX}
    child { node {\LaTeX}
      child { node {XeLaTeX} }
      child { node {LuaTeX} }
      } ;
\end{tikzpicture}
\end{column}

\note{
how to pronaunce\\
Book The Art of Computer Programming \\ 
History 1978 (pdf=1993)
}

\begin{column}{0.5\textwidth}
\small
{\csk \TeX} --- is a typesetting system designed and mostly written by Donald Knuth. TeX was designed with two main goals in mind: to allow anybody to produce high-quality books using minimal effort, and to provide a system that would give exactly the same results on all computers, at any point in time

\end{column}
\end{columns}
\end{frame}

% LuaTeX and XeLaTeX
\begin{frame}{Definitions}
\skfootnote{\url{https://en.wikipedia.org/wiki/LuaTeX}, \url{https://en.wikipedia.org/wiki/XeTeX}}
\begin{columns}
\begin{column}{0.5\textwidth}
\begin{tikzpicture}[sibling distance=10em,
  every node/.style = {shape=rectangle, rounded corners,
    draw, align=center,
    top color=white, bottom color=skoltechgreen!20}]]
  \node {\TeX}
    child { node {\LaTeX}
      child { node[text=red] {XeLaTeX} }
      child { node[text=red] {LuaTeX} }
      } ;
\end{tikzpicture}
\end{column}

\begin{column}{0.5\textwidth}
\small
{\csk XeLaTeX} --- XeTeX is a \TeX typesetting engine using Unicode and supporting modern font technologies such as OpenType, Graphite and Apple Advanced Typography

{\csk LuaTeX} --- LuaTeX is a \TeX-based computer typesetting system which started as a version of pdfTeX with a Lua scripting engine embedded
\end{column}
\end{columns}
\end{frame}


% pro and contr
\begin{frame}[t, fragile]{Pros and Cons}
\vspace{-3.5ex}
\cwpa{
    \small
    \begin{itemize}
        \item[$+$] When you have lots of equations\vspace{-1ex}
        \item[$+$] When you have a complex, but typical document \vspace{-1ex}
        \item[$+$] When you carry about device-independant view\vspace{-1ex}
        \item[$+$] When you don't want not care about the beauty, but want it\vspace{-1ex}
        \item[$+$] When you are care about the beauty wery much\vspace{-1ex}
        \item[$+$] When you hate all these humanitarian GUI programs
        % \item[$+$] When you love text files 
        % \item[``+''] When you want to impress your partner
        % yeah, I leave the commented item for you, source-code reader!
    \end{itemize}
}{
    \small
    \begin{itemize}
        \item[$-$] when you want to put something in attributary position
        \item[$-$] when you want to do something ``against the rules''
        \item[$-$] when you want to work with visual-based things (tables,...)
        \item[$-$] when you want to do something really simple
    \end{itemize}
}[t]

\cprotect\skfootnote{
\verb|\item[$+$]|
}

\end{frame}

\begin{frame}{Resourses}%{Books etc}

\vspace{-3.0ex}
\begin{itemize}
    \item Knuth ``The \TeX Book'' (en, ru)
    \item L'vovsky ``Nabor i verstka v sisteme \LaTeX'' (ru)
    \item Lamport. ``\LaTeX. A Document Preparation System, User’s Guide and Reference Manual'' (en)
    \item Gratzer ``Math into \LaTeX'' (en) 
    \item Oetiker ``The Not So Short Introduction to \LaTeX'' (en, ru)
    \item \url{https://www.overleaf.com/learn}
    \item \url{https://www.latex-project.org/help/}
    \item \url{https://texfaq.org/}
\end{itemize}
\end{frame}

\note{
We will use papeeria\\
pointed about footnotes and colors!
}

\begin{frame}[fragile]{Resourses}{Interesting links}
\cprotect\skfootnote{
\verb|\begin{description} \item[questions about \TeX]\end{description}|
}
\small
\begin{description}
    \item[questions about \TeX] \url{https://tex.stackexchange.com} \vspace{-1ex}
    \item[knowing a command of the symbol] \url{http://detexify.kirelabs.org/classify.html} \vspace{-1ex}
    \item[beauty of TikZ] \url{http://www.texample.net/tikz/examples/} \vspace{-1ex}
    \item[beauty of pictures] \url{https://tex.stackexchange.com/questions/158668/nice-scientific-pictures-show-off}\vspace{-1ex}
    \item[beauty of typesetting] \url{https://tex.stackexchange.com/questions/1319/showcase-of-beautiful-typography-done-in-tex-friends}\vspace{-1ex}
\end{description}
\end{frame}

\begin{frame}[fragile]{where to get}
\begin{enumerate}
    \item Online
    \begin{itemize}
        \item \url{http://papeeria.com}
        \item \url{https://overleaf.com}
    \end{itemize}
    \item Offline
    \begin{itemize}
        \item \LaTeX{} \url{https://www.latex-project.org/get/}
        \item package manager \verb'tlmgr'
    \end{itemize}
\end{enumerate}
     
\end{frame}






