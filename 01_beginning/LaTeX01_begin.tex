%https://tex.stackexchange.com/questions/114219/add-notes-to-latex-beamer
\documentclass[14pt, aspectratio=169]{beamer}
\usepackage{fontspec}
\usepackage{xunicode}
\usepackage{xltxtra}
\usepackage{dsfont}
\usepackage{multicol}
% \usepackage{pgfpages}
% \setbeameroption{show notes}
%\setbeameroption{hide notes} % Only slides
%\setbeameroption{show only notes} % Only notes
% \setbeameroption{show notes on second screen=right} % Both
%% USAGE: 
% \usepackage[logo="sklogo.png"]{beamerskoltech} 
%   if you have a stand-alone image file for Sk logo 
% or
% \usepackage[logo]{beamerskoltech} 
%   if you has no logo-file, but want LaTeX to generate it. 
%   In this case you probably will need to use `--enable-write18 -interaction=nonstopmode` arguments running the latex command.
%   in papeeria and overleaf all works fine
% or 
% \usepackage{beamerskoltech}
%   In case you don't want logo at all 
%
% provided commands:
% color `skoltechgreen` -- the dark-green color for structure elements 
% command `\logoname` -- the name of logo file if exist 
% command `{\csk <text>}` -- the shortcut from `\color{skoltechgreen}`
% command `skfootnote{text}` -- put some text for current slide
% `\renewcommand{\skbeforetitle}{\vspace{-3ex}}` is useful in case you use `aspectratio=169`
%%%%%%%%%%%%%%%%
\usepackage[logo]{beamerskoltech} 
\usepackage{lecturessk}
\renewcommand{\skbeforetitle}{\vspace{-3ex}}

\usepackage{xecyr}
\usepackage{hyperref}

\usepackage{polyglossia}
\usepackage{graphicx}
\usepackage{listings}



\begin{document}


\title{\LaTeX:\\ \Large from dummy to \TeX nician}
\subtitle{Overview and basis}
\author{Anton Lioznov}
\institute{Skoltech}
\date{ISP 2019,\\ \textit{lesson 1}}
\frame{\titlepage}
\begin{frame}\frametitle{What we will know?}
\tableofcontents[hideallsubsections]
\end{frame}

\AtBeginSection[] 
{
  \begin{frame}{What we will know?}
  \tableofcontents[currentsection,hideallsubsections]
  \end{frame}
}
% https://tex.stackexchange.com/questions/103444/show-toc-of-only-a-section-in-beamer-completely-hide-other-subsections
\AtBeginSubsection[]
{
  \begin{frame}{What we will know?}
  \tableofcontents[currentsubsection, hideothersubsections, sectionstyle=show/hide, subsectionstyle=show/shaded/hide]
  
  \end{frame}
}

\section{Introduction: on approaches to \LaTeX}

%% It is just an empty TeX file.
%% Write your code here.
\graphicspath{{sec01/images/}{sec01/code/}}
\lstset{inputpath=sec01/code/}

\begin{frame}[fragile]{What is TikZ?}\relax
``TikZ ist kein Zeichenprogramm'' 

which translates to ``TikZ is not a drawing program''

TikZ defines a number of \TeX\ commands that produce graphics: \tikz \fill[orange] (1ex,1ex) circle (1ex); produced by \verb|\tikz \fill[orange] (1ex,1ex) circle (1ex);|
\end{frame}

\begin{frame}{Pros and Cons}

\Huge\centering Pros and Cons
     
\end{frame}

\begin{frame}[fragile]{Cons}\relax
     \begin{itemize}
        \item[$-$] it is most likely that you don't need TikZ
        \item[$-$] write visual-based thinks like graphics is really annoying in a not-WYSiWYG way
    \end{itemize}
    
\end{frame}

\begin{frame}{Pros}\relax
     \begin{itemize}
        \item[$+$] it is most likely that you need some TikZ elements
        \item[$+$] some graphics (graphs for example) are so good structured, that it is OK to program them
        \item[$+$] TikZ has perfect integration with \LaTeX\ (and beamer):
        \begin{itemize}
            \item You can use all \LaTeX commands inside TikZ, creating beautiful pictures with math 
            \item You can pose elements using TikZ
            \item You can show just part of the picture in beamer Overlays   
        \end{itemize}
        \item[$+$] You don't need to have an external file
        \item[$+$] TikZ is using in CV and lots of other templates. It is good to be able to read the code
    \end{itemize}
\end{frame}


\begin{frame}[fragile]{How to setup TikZ picture?}\relax

\verb|\usepackage{tikz}|

and then 

\verb|\begin{tikzpicture} <code> \end{tikzpicture}| or, for short inline graphics, \ccol\tikz. 



\skfootnote{\tikzc{12.1}[126]}
     
\end{frame}

\section{``Hello, world'':  first steps in \LaTeX}

\graphicspath{{sec02/images/}}
\lstset{inputpath=sec02/code}

\begin{frame}{WYSiWYG vs not-WYSiWYG editors}
\cwpa{
{\csk WYSiWYG} -- \textit{What You See is What You Get} editor\\~\\

\inclassFrag{Examples?}[0]
}{
Microsoft Word
\insImg{wysiwyg.png}
}
\end{frame}


\begin{frame}[fragile]{not-WYSiWYG}


\cprotect\skfootnote{``\verb|\begin{frame}[fragile]|'' to put code inside}


\inclassFrag{Examples?}[1]
\cprotect[mm]\cwpa{
{\small HTML }

\begin{lstlisting}[language=html]
<html>
  <head>
    <meta charset="utf-8">
  </head>
  <body>
    <h1>Header</h1>
    <i>Hello</i>,<br/> world!  <!-- comment -->
  </body>
</html>
\end{lstlisting}

}{
{\small \LaTeX}
\lstinputlisting{helloworld.tex}

}[t]

\end{frame}

\begin{frame}[fragile]{Commands}
    \lstinline[basicstyle=\tt\large]|\command[o1, o2]{n1, n2=value}[o3]{n3}|\par
    (o = optional argument, n = necessary argument)~\\[2ex]

\pause
    \skfootnote{\knuthc{ch 3}, \lvoc{ch 2.3, 2.7}}

    {\csk Command symbols}
    \verb|\$ \# \{ \} \^{} \& \_ \~{} \\|
    \vfil
    {\csk Command words}
    \verb|\sin \LaTeX \Rightarrow \qquad|
    
    {\csk Enviruments}
    \verb|\begin{frame}\end{frame}  \begin{equation}\end{equation}|
    
\end{frame}

\begin{frame}[fragile]{Document structure}{overview}
    \lstinputlisting{structureOfDocument.tex}
\end{frame}

\begin{frame}[fragile]{Document structure}{class files}

Class of the document responsable for the large-scale settings
    \cprotect\skfootnote{\lvoc{IV.1}, \lamc{2.2.2} \url{https://texblog.org/2013/02/13/latex-documentclass-options-illustrated/}\\ envirament \lstinline|tabbing|} 
{
\scriptsize
\begin{tabbing}
\lstinline|\documentclass|\hspace{-1ex} \= \lstinline|[10pt,|  \= \lstinline|onecolumn,|  \=  \lstinline|a4paper]|\hspace{-1ex} \= \lstinline|{article}| \kill
\> \> \> \> \lstinline|{beamer} %presentation, poster| \\
\> \> \> \> \lstinline|{report}| \\
\> \> \> \> \lstinline|{book}| \\
\> \> \> \> \lstinline|{standalone} %for one picture/equation| \\
\> \> \> \> \lstinline|{extarticle} %if you want 14pt font size| \\
\lstinline|\documentclass| \> \lstinline|[10pt,|  \> \lstinline|onecolumn,|  \>  \lstinline|a4paper]| \> \lstinline|{article}| \\ 
\> \lstinline|[12pt] %fontsize| \> \\ 
\> \> \lstinline|[twocolumns] %number of columns in document| \> \\ 
\> \> \> \lstinline|[a5paper] %paper size| \> \\ 
\end{tabbing}}
\cprotect\skfootnote{\slshape in this presentation \verb|[aspectratio=169]|}
\end{frame}
\note{
but there is no ``fixed set'' of classes. Can build our own 
}

%%TSK: предложить создать документ и поиграться с параметрами
\begin{frame}[fragile]{Document structure}{style files}
     Style files are responsable for settings and providing new commands 
     
     \vfill
     \hspace{-1ex}
     \lstinline[basicstyle=\tt\large]|\usepackage[optional]{necessary}{packagename}|
     \vfill
\end{frame}

\note{practicully everything is from package }


\inclassFrame{
\begin{frame}{This slide means it is ``in-class task time!''}\relax
\centering
     \LARGE This slide means it is ``in-class task time!''
\end{frame}
}
\cprotect\inclassFrame{
\begin{frame}[fragile]{Try to compile your first doc}\relax
     \lstinputlisting[basicstyle=\tt\small]{first.tex}
\end{frame}
}

\section{Mastering the base}

%% It is just an empty TeX file.
%% Write your code here.
\graphicspath{{sec03/images/}{sec03/code/}}
\lstset{inputpath=sec03/code/}


\begin{frame}\frametitle{What we have learned today?}
\tableofcontents
\end{frame}

\begin{frame}[allowframebreaks]{references}
color from the footnotes corresponds to references' color.
    \begin{itemize}
        \item \knuthc{Knuth ``The \TeX Book''}
        \item \lvoc{L'vovsky ``Nabor i verstka v sisteme \LaTeX''}
        \item \lamc{Lamport. ``\LaTeX. A Document Preparation System, User’s Guide and Reference Manual''}
    \end{itemize}
\end{frame}

\begin{frame}{Distribution}\relax
\begin{itemize}
     \item the pdf-version of the presentation and all printed materials can be distributed under license Creative Commons Attribution-ShareAlike 4.0 \url{https://creativecommons.org/licenses/by-sa/4.0/}
     \item The source code of the presentation is available on \url{https://github.com/lavton/} and can be distributed under the MIT license \url{https://en.wikipedia.org/wiki/MIT_License#License_terms}
\end{itemize}
     
\end{frame}
\end{document}