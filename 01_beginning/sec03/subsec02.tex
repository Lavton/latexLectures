% \knuthc  knuth the TeXBook
% \lvoc   Lvovsky
% \lamc  lamport latex 
% \slshape different font for footnote
\graphicspath{{sec03/images/s2/}{sec03/code/s2/}}
\lstset{inputpath=sec03/code/s2/}

\begin{frame}[fragile]{Writing \textit{Text}}\relax
\inclassFrag{Please, write a line of a text}[1]
\cprotect\twocolImg{
    \lstinputlisting[linerange={10-10},basicstyle=\tt\normalsize,showspaces=true]{general}
}{general}

\skfootnote{showspaces=true}
\end{frame}

\begin{frame}[fragile]{Spaces}\relax
\inclassFrag{
Do the following and compare results from previous step.
\begin{itemize}
     \item add spaces between two arbitrary words
     \item add line break (<return>) between two arbitrary words
     \item add spaces at the begin of a line
     \item add \% at the end of the first line  
\end{itemize}
}[1]
\cprotect\twocolImg{
    \lstinputlisting[linerange={10-14},basicstyle=\tt\small,showspaces=true]{spaces}
}{spaces}
\end{frame}

\begin{frame}[fragile]{Paragraph}\relax
\inclassFrag{
Do the following and compare results from previous step.
\begin{itemize}
     \item add \textbf{two} line breaks (<return>) between arbitrary words
     \item add \textbf{three} line breaks (<return>) between arbitrary words
     \item add \textbf{four} line breaks (<return>) between arbitrary words
\end{itemize}
}[1]
\cprotect\twocolImg{
    \lstinputlisting[linerange={10-17},basicstyle=\tt\small,showspaces=true]{paragr}
}{paragr}
\end{frame}

\begin{frame}[fragile]{Indents}\relax
\cprotect\twocolImg{
    \lstinputlisting[linerange={10-11},basicstyle=\tt\small,showspaces=true]{indents}
}{indents}
\end{frame}

\begin{frame}[fragile]{More spaces}\relax
\begin{columns}
\begin{column}{0.4\textwidth}
     \lstinline[basicstyle=\tt\normalsize]| Use ~ for non brocken space and ~~~more spaces. Or \ \ \ like this|
\end{column}
\begin{column}{0.4\textwidth}
     Use ~ for non brocken space and ~~~more spaces. Or \ \ \ like this
\end{column}
\end{columns}
\hrule
\begin{columns}
\begin{column}{0.4\textwidth}
     \lstinline[basicstyle=\tt\normalsize]| Use \\ for new line. And more then one ~\\~\\~\\ new line|
\end{column}
\begin{column}{0.4\textwidth}
~\\
     Use \\ for new line. And more then one ~\\~\\~\\ new line
\end{column}
\end{columns}

\end{frame}

\begin{frame}[fragile]{Spaces and commands}\relax

\newcommand{\appendTline}[2]{\vspace*{10pt}\begin{columns}
        \begin{column}{0.45\textwidth}
          \hfill #1 
        \end{column}
        \begin{column}{0.45\textwidth}
             \hfill #2\hfill \hfill
        \end{column}
    \end{columns}
    \vphantom.
    \hrule
    }

    \cprotect[mm]\appendTline{\csk source}{\csk result}
    \hrule height 0.05pt
    \cprotect[mm]\appendTline{\lstinline[basicstyle=\tt\normalsize,showspaces=true]|\TeX book|}{\TeX book}
    \cprotect[mm]\appendTline{\lstinline[basicstyle=\tt\normalsize,showspaces=true]|\TeX{} book|}{\TeX{} book}
    \cprotect[mm]\appendTline{\lstinline[basicstyle=\tt\normalsize,showspaces=true]|\TeX\ book|}{\TeX\ book}

\end{frame}

\begin{frame}[fragile]{Fonts}{shape (form)}\relax


% `\csname textit\endcsname` creates \textit
\newcommand{\putinside}[1]{\csname #1\endcsname{{\csk \textbackslash #1}\{text\}} }\relax
\newcommand{\putoutside}[1]{ { \csname #1\endcsname \{{\csk \textbackslash #1} text\} } }

\begin{tabular}{rcc}
    \textup{Upright shape} & \putinside{textup} & \putoutside{upshape}\\
    \textit{Italic shape} & \putinside{textit} & \putoutside{itshape}\\
    \textsl{Slanted shape} & \putinside{textsl} & \putoutside{slshape}\\
    \textsc{Small caps shape} & \putinside{textsc} & \putoutside{scshape}\\
    \hphantom{\textsc{Small caps shape}} & \hphantom{\putinside{textsc}} & \hphantom{\putoutside{scshape}}\\
\end{tabular}
\skfootnote{{} \lvoc{III.5.2}\\ code generate this slide is interesting. Look at it :) }
\end{frame}


\begin{frame}[fragile]{Fonts}{saturation (series)}\relax


% `\csname textit\endcsname` creates \textit
\newcommand{\putinside}[1]{\csname #1\endcsname{{\csk \textbackslash #1}\{text\}} }\relax
\newcommand{\putoutside}[1]{ { \csname #1\endcsname \{{\csk \textbackslash #1} text\} } }
\begin{tabular}{rcc}
    \textmd{Medium series} & \putinside{textmd} & \putoutside{mdseries}\\
    \textbf{Boldface series} & \putinside{textbf} & \putoutside{bfseries}\\
    \hphantom{\textsc{Small caps shape}} & \hphantom{\putinside{textsc}} & \hphantom{\putoutside{scshape}}\\
\end{tabular}
\end{frame}


\begin{frame}[fragile]{Fonts}{garniture (family)}\relax


% `\csname textit\endcsname` creates \textit
\newcommand{\putinside}[1]{\csname #1\endcsname{{\csk \textbackslash #1}\{text\}} }\relax
\newcommand{\putoutside}[1]{ { \csname #1\endcsname \{{\csk \textbackslash #1} text\} } }
\begin{tabular}{rcc}

    \textrm{Roman family} & \putinside{textrm} & \putoutside{rmfamily}\\
    \textsf{Sans serif family} & \putinside{textsf} & \putoutside{sffamily}\\
    \texttt{Typewriter family} & \putinside{texttt} & \putoutside{ttfamily}\\
    \hphantom{\textsc{Small caps shape}} & \hphantom{\putinside{textsc}} & \hphantom{\putoutside{scshape}}\\
\end{tabular}
\end{frame}


\begin{frame}[fragile]{Fonts}{size}\relax
\newcommand{\putoutside}[1]{ { \csname #1\endcsname \{{\csk \textbackslash #1} text\} } }

\begin{multicols}{2}
\hspace{-20em}
\begin{itemize}
\item \hbox{\putoutside{Huge}}
\item \putoutside{huge}
\item \putoutside{LARGE}
\item \putoutside{Large}
\item \putoutside{large}
\item \putoutside{normalsize}
\item \putoutside{small}
\item \putoutside{footnotesize}
\item \putoutside{scriptsize}
\item \putoutside{tiny}
\end{itemize}
\end{multicols}

\skfootnote{{ }\lvoc{III.5.1}  \url{https://texblog.org/2012/08/29/changing-the-font-size-in-latex/}}

\end{frame}

\begin{frame}[fragile]{To Default}

    \lstinline[basicstyle=\tt\normalsize]|\Huge text \ttfamily text \itshape text \normalfont\normalsize text|

     \Huge text \ttfamily text \itshape text \normalfont\normalsize text
\end{frame}

\begin{frame}[fragile]{to Default: ``GROUPS''}\relax
    \begin{itemize}
    \item Lots of \LaTeX{} commands are ``local''
    \item Local commands loose their effect outside the group 
    \item ``group'' is
    \begin{itemize}
        \item \lstinline[basicstyle=\tt\normalsize]|{group}|
        \item \lstinline[basicstyle=\tt\normalsize]|\begingroup group\endgroup|
        \item \lstinline[basicstyle=\tt\normalsize]|$group$|
        \item \lstinline[basicstyle=\tt\normalsize]|\begin{env}group\end{env}|
    \end{itemize}
    \item often something inside \{group\} means ``indivisible'', ``atomic'', ``single'' for \TeX\ commands.
    \end{itemize}

\end{frame}


% \begin{frame}[fragile]{More symbols}{Accents}\relax
% % \footnotesize

% \skfootnote{\knuthc{17}}
% \newlength{\myboxlen}%
% \setlength{\myboxlen}{9em}

% \newcommand{\showmacc}[2]{%
% \makebox[\myboxlen]{\hfill\makebox[0.45\myboxlen]{\hfill{\csk\string#1}\{#2\}}\makebox[0.25\myboxlen]{$\to$}\makebox[0.3\myboxlen]{$#1{#2}$\hfill}\hfill}%
% }
% \showmacc{\hat}{a}\hfill
% \showmacc{\check}{a}\hfill
% \showmacc{\tilde}{a}\hfill
% \showmacc{\acute}{a}\hfill
% \showmacc{\grave}{a}\hfill
% \showmacc{\dot}{a}\hfill
% \showmacc{\ddot}{a}\hfill
% \showmacc{\breve}{a}\hfill
% \showmacc{\vec}{a}\hfill
% ~\\[2ex]
% \pause

% \setlength{\myboxlen}{12em}
% \Large \showmacc{\check}{a}\hfill \to \hfill
% \setlength{\myboxlen}{13em}
% {\csk \textbackslash{}skew5\textbackslash{}check}\{a\}{$\to$}$\skew5\check{a}$%

% \end{frame}


\begin{frame}[fragile]{Other languages}{accents}\relax
\newlength{\myboxlen}%
\setlength{\myboxlen}{9em}

\newcommand{\showacc}[2]{%
\makebox[\myboxlen]{\hfill\makebox[0.45\myboxlen]{\hfill{\bfseries\csk\string#1}\{#2\}}\makebox[0.25\myboxlen]{$\to$}\makebox[0.3\myboxlen]{#1{#2}\hfill}\hfill}%
}

\showacc{\`}{o}\hfill
\showacc{\'}{o}\hfill
\showacc{\^}{o}\hfill
\showacc{\"}{o}\hfill
\showacc{\H}{o}\hfill
\showacc{\c}{o}\hfill
\showacc{\k}{a}\hfill
\showacc{\=}{o}\hfill
\showacc{\b}{o}\hfill
\showacc{\.}{o}\hfill
\showacc{\d}{u}\hfill
\showacc{\r}{a}\hfill
\showacc{\u}{o}\hfill
\showacc{\v}{s}\hfill
\makebox[\myboxlen]{}\hfill
~\\[2ex]
\showacc{\l}{}\hfill
\showacc{\i}{}\hfill
\showacc{\j}{}\hfill

    \skfootnote{\url{https://tex.stackexchange.com/tags/accents/info} \url{https://en.wikibooks.org/wiki/LaTeX/Special_Characters}}
\end{frame}

\begin{frame}[fragile, t]{Other languages}{complite solution: russian}\relax
     \begin{columns}[t]
          \begin{column}{0.45\textwidth}
          XeLaTeX
          
              \lstinputlisting[linerange={2-5},basicstyle=\tt\small]{russianXe}
          \end{column}
          \begin{column}{0.45\textwidth}
          pdfLaTeX
          
              \lstinputlisting[linerange={2-4},basicstyle=\tt\small]{russianpdf}
          \end{column}
     \end{columns}
\end{frame}

\begin{frame}[fragile]{Enumirate}\relax
\cprotect\twocolImg{
    \lstinputlisting[linerange={10-16},basicstyle=\tt\footnotesize]{enumimy}
}{enumimy}

\cprotect\twocolImg{
    \lstinputlisting[linerange={10-16},basicstyle=\tt\footnotesize]{itemimy}
}{itemimy}
    
    \cprotect\skfootnote{{ }\lvoc{III.7.2}\\ 
    \slshape also look at env. \verb|description| and at \verb|\item[x]|}
     
\end{frame}
