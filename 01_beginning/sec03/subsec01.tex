% \knuthc  knuth the TeXBook
% \lvoc   Lvovsky
% \lamc  lamport latex 
% \slshape different font for footnote
\graphicspath{{sec03/images/s1/}{sec03/code/s1/}}
\lstset{inputpath=sec03/code/s1/}

\begin{frame}[fragile]{Going $\to$ Math}\relax
\cprotect\skfootnote{
\vspace{-0.5ex}
\lvoc{I.3}, \lamc{3.1}\\ {\slshape 
also try \verb|\(x=y\)| and \verb|\[x=y\]| and environment \verb|equation*|}\\
\textt \verb|\|lstinputlisting[firstline=7, lastline=14,linerange=\{8-10\}]\{file.tex\}
}

\cprotect\twocolImg{
Math environments

\lstinputlisting[linerange={15-16,19-19,22-25}]{mdollar.tex}
}{mdollar}
\end{frame}


\begin{frame}[fragile]{Indexes}\relax

\newcommand{\appendTline}[3]{
\makebox[10em]{#1\hfill} \hfill\hfill \makebox[5em]{\hfill #2\hfill} \hfill \makebox[5em]{\hfill #3\hfill} \hfill\hfill \strut
    \hrule
    \vspace{1ex}
}

\cprotect[mm]\appendTline{upper ind}{\lstinline[basicstyle=\tt\normalsize,]|$x^2$|}{ $x^2$}

\cprotect[mm]\appendTline{lower ind}{\lstinline[basicstyle=\tt\normalsize,]|$x_2$|}{ $x_2$}
\cprotect[mm]\appendTline{lower and upper ind}{\lstinline[basicstyle=\tt\normalsize,]|$x^4_2$|}{ $x^4_2$}
\cprotect\inclassFrag{
Try to put more then one letter to the index}
\cprotect[mm]\appendTline{more letters in ind}{\lstinline[basicstyle=\tt\normalsize,]|$x_{ij}$|}{ $x_{ij}$}
\cprotect\inclassFrag{

How can you create ${}^3_2 He$?}[4]
\cprotect[mm]\appendTline{empty block}{\lstinline[basicstyle=\tt\normalsize,]|${}^3_2He$|}{ ${}^3_2He$}
\cprotect[mm]\appendTline{index in index}{\lstinline[basicstyle=\tt\normalsize,]|$x^{4^2}$|}{ $x^{4^2}$}

\skfootnote{You really don't want to know, what code generate these lines...}
\end{frame}

\begin{frame}[fragile]{Fractions and (square) root}\relax
    \begin{columns}
        \begin{column}{0.45\textwidth}
          \hfill \lstinline[basicstyle=\tt\normalsize]|$\frac{x+z^2}{y-1}$| 
        \end{column}
        \begin{column}{0.45\textwidth}
             $$\frac{x+z^2}{y-1}$$
        \end{column}
    \end{columns}
    \vphantom.
    \hrule
    \vphantom. 
    \begin{columns}
        \begin{column}{0.50\textwidth}
          \hfill \lstinline"$\left.\left(\frac{x}{y}\right)\right|^a_b$" 
        \end{column}
        \begin{column}{0.5\textwidth}
             $$\left.\left(\frac{x}{y}\right)\right|^a_b$$
        \end{column}
    \end{columns}
    \vphantom(
    \hrule
    \vphantom) 
    \begin{columns}
        \begin{column}{0.45\textwidth}
          \hfill \lstinline[basicstyle=\tt\normalsize]|$\sqrt{x}$| 
        \end{column}
        \begin{column}{0.45\textwidth}
             $$\sqrt{x}$$
        \end{column}
    \end{columns}
    \vphantom(
    \hrule
    \vphantom) 
    \begin{columns}
        \begin{column}{0.45\textwidth}
          \hfill \lstinline[basicstyle=\tt\normalsize]|$\sqrt[y]{x}$| 
        \end{column}
        \begin{column}{0.45\textwidth}
             $$\sqrt[y]{x}$$
        \end{column}
    \end{columns}
    
\cprotect\skfootnote{\slshape sometimes, if you want use fraction into fraction it is better to use \verb|\cfrac{x}{y}|.\\ See typeography section or \lvoc{II.4.1, II.5.1}
also $\sphericalangle$ \verb|\genfrac|
}
\end{frame}

\inclassFrame{
\begin{frame}{Type these equations}{simple}\relax
\begin{equation}
     x^2-2y = h_{l_0^k}\tag{1}
\end{equation}
\begin{equation}
     \sqrt{h/(2a)}=t\tag{2}
\end{equation}

\begin{equation}
     \frac{x+y}{z_2 - \frac{1}{2}}=m\tag{3}
\end{equation}

\begin{equation}
     \left(\frac{x+y}{z}\right)=q\tag{4}
\end{equation}

\end{frame}
}

\inclassFrame{
\begin{frame}{Type these equations}{harder}\relax
\begin{equation}
    T_{N_1}(t) = \left.\frac{d}{dx_i} f^{\sqrt{n}}(z)\right/\left(\sqrt{\frac{m+e}{a}}-N\right) \tag{1}
\end{equation}

\end{frame}
}

\begin{frame}[fragile]{More symbols}\relax
    \line{\hss\Large \url{http://detexify.kirelabs.org/classify.html}\hss}
    
    \inclassFrag{\centering
    Try to find the following symbols: \\ $\exists$,  $\to$, $\notin$, $\sim$, $\equiv$
    }[-1]
    
    \skfootnote{\vspace{-3ex}
    \url{https://en.wikibooks.org/wiki/LaTeX/Special_Characters} \\
    \url{ftp://sunsite.icm.edu.pl/pub/CTAN/info/symbols/comprehensive/symbols-a4.pdf} (338 pages) \\ \url{https://www.rpi.edu/dept/arc/training/latex/LaTeX_symbols.pdf} (22 pages)}
\end{frame}

\begin{frame}[fragile]{More symbols}{Greek letters}\relax
\footnotesize

\newlength{\myboxlen}%
\setlength{\myboxlen}{9em}

\newcommand{\showmsym}[1]{%
\makebox[\myboxlen]{\hfill\makebox[0.45\myboxlen]{\csk\hfill\string#1}\makebox[0.25\myboxlen]{$\to$}\makebox[0.3\myboxlen]{$#1$\hfill}\hfill}%
}
\showmsym{\alpha}\hfill
\showmsym{\beta}\hfill
\showmsym{\gamma}\hfill
\showmsym{\delta}\hfill
\showmsym{\epsilon}\hfill
\showmsym{\zeta}\hfill
\showmsym{\eta}\hfill
\showmsym{\iota}\hfill
\showmsym{\kappa}\hfill
\showmsym{\lambda}\hfill
\showmsym{\mu}\hfill
\showmsym{\nu}\hfill
\showmsym{\xi}\hfill
\showmsym{\tau}\hfill
\showmsym{\upsilon}\hfill
\showmsym{\rho}\hfill
\showmsym{\sigma}\hfill
\showmsym{\pi}\hfill
\showmsym{\phi}\hfill
\showmsym{\chi}\hfill
\showmsym{\psi}\hfill
\showmsym{\omega}\hfill
\makebox[\myboxlen]{}\hfill
\makebox[\myboxlen]{}\hfill
\makebox[\myboxlen]{}\hfill

\showmsym{\varepsilon}\hfill
\showmsym{\vartheta}\hfill
\showmsym{\varkappa}\hfill
\showmsym{\varrho}\hfill
\showmsym{\varphi}\hfill

~\\
\showmsym{\Gamma}\hfill
\showmsym{\Delta}\hfill
\showmsym{\Lambda}\hfill
\showmsym{\Xi}\hfill
\showmsym{\Sigma}\hfill
\showmsym{\Upsilon}\hfill
\showmsym{\Pi}\hfill
\showmsym{\Phi}\hfill
\showmsym{\Psi}\hfill
\showmsym{\Omega}\hfill
\makebox[\myboxlen]{}\hfill
\makebox[\myboxlen]{}\hfill
\makebox[\myboxlen]{}\hfill

\end{frame}

\begin{frame}[fragile]{More symbols}{Other common used symbols}\relax
\footnotesize

\newlength{\myboxlen}%
\setlength{\myboxlen}{9em}

\newcommand{\showmsym}[1]{%
\makebox[\myboxlen]{\hfill\makebox[0.45\myboxlen]{\csk\hfill\string#1}\makebox[0.25\myboxlen]{$\to$}\makebox[0.3\myboxlen]{$#1$\hfill}\hfill}%
}
\showmsym{\infty}\hfill
\showmsym{\nabla}\hfill
\showmsym{\forall}\hfill
\showmsym{\partial}\hfill
\showmsym{\backslash}\hfill
\showmsym{\aleph}\hfill
\showmsym{\hbar}\hfill
\showmsym{\Re}\hfill
\showmsym{\Im}\hfill
\showmsym{\cdot}\hfill
\showmsym{\cdots}\hfill
\showmsym{\ldots}\hfill
\showmsym{\to}\hfill
\showmsym{\times}\hfill
\makebox[\myboxlen]{}\hfill
\makebox[\myboxlen]{}\hfill
\makebox[\myboxlen]{}\hfill

\normalsize
\pause \showmsym{\imath}\hfill \showmsym{\jmath}\hfill are useful for accents.

\end{frame}

\begin{frame}[fragile]{More symbols}{Accents}\relax
% \footnotesize

\skfootnote{\knuthc{17}}
\newlength{\myboxlen}%
\setlength{\myboxlen}{9em}

\newcommand{\showmacc}[2]{%
\makebox[\myboxlen]{\hfill\makebox[0.45\myboxlen]{\hfill{\csk\string#1}\{#2\}}\makebox[0.25\myboxlen]{$\to$}\makebox[0.3\myboxlen]{$#1{#2}$\hfill}\hfill}%
}
\showmacc{\hat}{a}\hfill
\showmacc{\check}{a}\hfill
\showmacc{\tilde}{a}\hfill
\showmacc{\acute}{a}\hfill
\showmacc{\grave}{a}\hfill
\showmacc{\dot}{a}\hfill
\showmacc{\ddot}{a}\hfill
\showmacc{\breve}{a}\hfill
\showmacc{\vec}{a}\hfill
~\\[2ex]
\pause

\setlength{\myboxlen}{12em}
\Large \showmacc{\check}{a}\hfill \to \hfill
\setlength{\myboxlen}{13em}
{\csk \textbackslash{}skew5\textbackslash{}check}\{a\}{$\to$}$\skew5\check{a}$%

\end{frame}


\begin{frame}[fragile]{More symbols}{Math fonts}\relax
    \begin{tabbing}
    ааaaaaaaaaaaaaaaaaaaaaaaaaaaaa \= aaaaaaaaaaaaaaaaaaaaaaaaaaaaaa \= \kill
    \texttt{\textbackslash mathrm\{\textit{letters,etc}\}} \> $\mathrm{ABCabc,123,\hat{a},\tilde{b},\tilde{c},\Psi\Omega}$ \\
    \texttt{\textbackslash mathbf\{\textit{letters,etc}\}} \> $\mathbf{ABCabc,123,\hat{a},\tilde{b},\tilde{c},\Psi\Omega}$ \\
    \texttt{\textbackslash mathsf\{\textit{letters,etc}\}} \> $\mathsf{ABCabc,123,\hat{a},\tilde{b},\tilde{c},\Psi\Omega}$ \\
    \texttt{\textbackslash mathtt\{\textit{letters,etc}\}} \> $\mathtt{ABCabc,123,\hat{a},\tilde{b},\tilde{c},\Psi\Omega}$ \\
    \texttt{\textbackslash mathit\{\textit{letters,etc}\}} \> $\mathit{ABCabc,123,\hat{a},\tilde{b},\tilde{c},\Psi\Omega}$ \\
    \texttt{\textbackslash mathnormal\{\textit{letters,etc}\}} \> $\mathnormal{ABCabc,123,\hat{a},\tilde{b},\tilde{c},\Psi\Omega}$ \\
    \texttt{\textbackslash mathcal\{\textit{capital letters}\}} \> $\mathcal{ABC}$ \\ 
    \texttt{\textbackslash mathds\{\textit{capital letters}\}} \> $\mathds{ABCRN}$ 
    \end{tabbing}
    \cprotect\skfootnote{\lstinline|\usepackage{dsfont}| for the last}
\end{frame}

\begin{frame}[fragile]{More symbols}{Limiters}\relax
\skfootnote{\knuthc{17}}

\newlength{\myboxlen}%
\setlength{\myboxlen}{6em}
% \small
\newcommand{\showmsym}[1]{%
\makebox[\myboxlen]{\hfill\makebox[0.45\myboxlen]{\csk\hfill\string#1}\makebox[0.25\myboxlen]{$\to$}\makebox[0.3\myboxlen]{$$#1$$\hfill}\hfill}%
}
\newcommand{\showidenti}[1]{%
\makebox[\myboxlen]{\hfill\makebox[0.45\myboxlen]{\csk\hfill$#1$}\makebox[0.25\myboxlen]{$\to$}\makebox[0.3\myboxlen]{$$#1$$\hfill}\hfill}%
}

\showidenti{(}\hfill
\showidenti{)}\hfill
\makebox[\myboxlen]{}\hfill
\makebox[\myboxlen]{}\hfill
\showidenti{[}\hfill
\showidenti{]}\hfill
\showmsym{\lbrack}\hfill
\showmsym{\rbrack}\hfill
\showmsym{\{}\hfill
\showmsym{\}}\hfill
\showmsym{\lbrace}\hfill
\showmsym{\rbrace}\hfill
\showmsym{\lfloor}\hfill
\showmsym{\rfloor}\hfill
\showmsym{\lceil}\hfill
\showmsym{\rceil}\hfill
\makebox[\myboxlen]{}\hfill
\makebox[\myboxlen]{}\hfill
\showmsym{\langle}\hfill
\showmsym{\rangle}\hfill

\end{frame}

\begin{frame}[fragile]{More symbols}{Operators}\relax
\skfootnote{\url{https://www.overleaf.com/learn/latex/Integrals,_sums_and_limits}}

\newlength{\myboxlen}%
\setlength{\myboxlen}{9em}

\newcommand{\showmsym}[1]{%
\makebox[\myboxlen]{\hfill\makebox[0.45\myboxlen]{\csk\hfill\string#1}\makebox[0.25\myboxlen]{$\to$}\makebox[0.3\myboxlen]{$$#1$$\hfill}\hfill}%
}
\showmsym{\prod}\hfill
\showmsym{\sum}\hfill
\showmsym{\int}\hfill
\showmsym{\iint}\hfill
\showmsym{\oint}\hfill
\showmsym{\idotsint}\hfill

\end{frame}


\begin{frame}[fragile]{More symbols}{Operators}\relax
\vspace{-5ex}
    \newcommand{\appendTline}[3]{ \vphantom{()}
    \makebox[10em]{#1\hfill} \hfill\hfill \makebox[5em]{\hfill #2\hfill} \hfill \makebox[5em]{\hfill #3\hfill} \hfill\hfill \strut
        \hrule
        \vspace{1ex}
    }
    
    \cprotect[mm]\appendTline{\csk source}{\csk operator}{\csk compare with}
    \hrule height 0.05pt
    \cprotect[mm]\appendTline{\lstinline[basicstyle=\tt\normalsize,]|$\sin({x})$|}{$\sin({x})$}{$sin({x})$}
    \cprotect[mm]\appendTline{\lstinline[basicstyle=\tt\normalsize,]|$\log({x})$|}{$$\log({x})$$}{$$ log({x}) $$}
    \cprotect[mm]\appendTline{\lstinline|$\lim({x})\limits_a^b$|}{$$\lim\limits_a^b({x})$$}{$$lim\limits_a^b({x})$$}
    
    \skfootnote{\url{https://oeis.org/wiki/List_of_LaTeX_mathematical_symbols}}
\end{frame}


\begin{frame}[fragile]{Formula in multiple line}\relax
\skfootnote{{ }\lvoc{II.4.2}\\ You can modify the line gaps, push string to left or right,..}

    \cprotect\twocolImg{
\lstinputlisting[linerange={13-14, 17-20}]{multilines.tex}
}{multilines}

\verb|AMS| --- American Math Society
\end{frame}


\begin{frame}[fragile]{Multiple formulas}\relax

% \vspace{-10ex}
    \cprotect\twocolImg{
\lstinputlisting[linerange={13-14, 17-21}]{gathermy.tex}
}{gathermy}

\pause
% \vspace{-10ex}
notice \lstinline[basicstyle=\tt\normalsize]|\notag| !

\cprotect\skfootnote{{} \lvoc{II.4.2}\\ \slshape  also see \verb'gather*' to ban numeration}

\end{frame}

\begin{frame}[fragile]{Multiple formulas and lines: alignment}\relax
\cprotect\twocolImg{
    \lstinputlisting[linerange={13-14, 17-20}]{alignmy.tex}
}{alignmy}
\cprotect\twocolImg{
    \lstinputlisting[linerange={13-14, 17-22}]{splitmy}
}{splitmy}

\inclassFrag{
What do you think {\csk \&} stands for?}
ampersand {\csk \&} is stands for indent (as in tables)
\cprotect\skfootnote{{ }\lvoc{II.4.2}\\ \slshape also see \verb'align*' to ban numeration}
\end{frame}

\begin{frame}[fragile, t]{Text inside equations}\relax

\cprotect\twocolImg{
    \lstinputlisting[linerange={13-14, 17-23}]{alignеtextwrong1.tex}
}{alignеtextwrong1}
\inclassFrag{What a problem can be there?}
\pause
Problem:
\cprotect\twocolImg{
    \lstinputlisting[linerange={13-14, 17-23}]{alignеtextwrong.tex}
}{alignеtextwrong}

\end{frame}
\begin{frame}[fragile]{Text inside equations}{solution}\relax
% \vspace{-1.3cm}
\cprotect\twocolImg{
    \lstinputlisting[linerange={13-14, 17-23}]{alignеtext.tex}
}{alignеtext}
\verb|\text| and \verb|\intertext|
\skfootnote{{} \lvoc{II.4.2}}
\end{frame}

\begin{frame}[fragile]{System of equations}\relax
% \vspace{-1.3cm}
\cprotect\twocolImg{
    \lstinputlisting[linerange={13-14, 17-24}]{equsis.tex}
}{equsis}
% \vspace{-2cm}
\cprotect\twocolImg{
    \lstinputlisting[linerange={13-14, 17-23}]{equcase}
}{equcase}
\skfootnote{{} \lvoc{II.4.2}}
\end{frame}


\begin{frame}[fragile]{Matrix}\relax
    \cprotect\twocolImg{
    \lstinputlisting[linerange={7-7, 11-18}]{matrixes}
}{matrixes}
\cprotect\skfootnote{{ } \lvoc{II.3}\\ \slshape Also see \lstinline!matrix! without borders and \lstinline!vmpatrix! for |border|}
     
\end{frame}

\begin{frame}[fragile]{Without AMS}\relax
     \cprotect\twocolImg{
        \lstinputlisting[linerange={11-17}]{matrixesno}
    }{matrixesno}
    \cprotect\twocolImg{
    \lstinputlisting[linerange={9-16}]{equcaseno}
    }{equcaseno}
\pause
Please, notice this {\csk \{ccc\}, \{ccl\}}.  

\cprotect\skfootnote{{ } \lvoc{II.3, II.4.2}}
\end{frame}

\cprotect\inclassFrame{
\begin{frame}[fragile]{Try to type the following}\relax
\small
\verb!$$\sum_a^b x$$!, \hfill\verb!$$\sum\nolimits_a^b x$$!, \hfill\verb!$$\sum\limits_a^b x$$!

\verb!$\sum_a^b x$!, \hfill\verb!$\sum\nolimits_a^b x$!, \hfill\verb!$\sum\limits_a^b x$!

\verb!$$\int_a^b x$$!, \hfill\verb!$$\int\nolimits_a^b x$$!, \hfill\verb!$$\int\limits_a^b x$$!

\verb!$\int_a^b x$!, \hfill\verb!$\int\nolimits_a^b x$!, \hfill\verb!$\int\limits_a^b x$!
~\\[2ex]
compare the index position and the operator size
     
\end{frame}
}

\begin{frame}[fragile]{One over another}{operators}\relax

\newcommand{\appendTline}[2]{\vspace*{10pt}\begin{columns}
        \begin{column}{0.45\textwidth}
          \hfill #1 
        \end{column}
        \begin{column}{0.45\textwidth}
             \hfill #2\hfill \hfill
        \end{column}
    \end{columns}
    \vphantom.
    \hrule
    }

    \cprotect[mm]\appendTline{\csk source}{\csk operator}
    \hrule height 0.05pt
    \cprotect[mm]\appendTline{\lstinline[basicstyle=\tt\normalsize,]|$$\int\limits_0^\pi$$|}{$$\int\limits_0^\pi$$}
    \cprotect[mm]\appendTline{\lstinline[basicstyle=\tt\normalsize,]|$$\int\nolimits_0^\pi$$|}{$$\int\nolimits_0^\pi$$}

\end{frame}

\begin{frame}[fragile]{One over another}\relax

\newcommand{\appendTline}[2]{\vspace*{10pt}\begin{columns}
        \begin{column}{0.45\textwidth}
          \hfill #1 
        \end{column}
        \begin{column}{0.45\textwidth}
             \hfill #2\hfill \hfill
        \end{column}
    \end{columns}
    \vphantom.
    \hrule
    }

    \cprotect[mm]\appendTline{\csk source}{\csk operator}
    \hrule height 0.05pt
    \cprotect[mm]\appendTline{\lstinline|$\stackrel {\Leftrightarrow}{A}$|}{$\stackrel {\Leftrightarrow}{A}$}
    \cprotect[mm]\appendTline{\lstinline|$A \stackrel{a'}{\rightarrow} D$|}{$A \stackrel{a'}{\rightarrow} D$}
    \cprotect[mm]\appendTline{\lstinline|$\underbrace{a+\overbrace{b+c}+d}_{m}$|}{$\underbrace{a+\overbrace{b+c}+d}_{m}$}
    
\end{frame}
