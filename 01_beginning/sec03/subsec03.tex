% \knuthc  knuth the TeXBook
% \lvoc   Lvovsky
% \lamc  lamport latex 
% \slshape different font for footnote
\graphicspath{{sec03/images/s3/}{sec03/code/s3/}}
\lstset{inputpath=sec03/code/s3/}

\begin{frame}[fragile]{Input some elements}\relax
\begin{itemize}
     \item Code 
     \item Figures (pictures)
     \item Tabels
\end{itemize}
\end{frame}

\begin{frame}[fragile]{Code}\relax

\verb|\usepackage|

\hspace{\fill} \verb|{verbatim}| \hfill \verb|{listings}| \hfill \verb|{minted}|

inline

\hspace{\fill} \verb|\verb!code!| \hfill \verb!\lstinline|code|! \hfill \verb|\mintinline{LaTeX}{Code}|

{\csk \small note: \verb"||, !!" are any characters not found in \verb|code|}

environment (\verb|\begin{env} ... \end{env}|)

\hspace{\fill} \verb|{verbatim}| \hfill \verb|{lstlisting}| \hfill \verb|{minted}|

file

\hspace{\fill} \verb|{verbatiminput}| \hfill \verb|{lstinputlisting}| \hfill \verb|{inputminted}|

\skfootnote{\url{https://en.wikibooks.org/wiki/LaTeX/Source_Code_Listings}}
\end{frame}

\begin{frame}[fragile]{Verbatim}\relax

\cprotect\twocolImg{
    \lstinputlisting[linerange={6-6, 11-11},basicstyle=\tt\small]{vemy}
}{vemy}
\end{frame}

\begin{frame}[fragile]{Listings}\relax

\cprotect\twocolImg{
    \lstinputlisting[linerange={6-6, 11-11},basicstyle=\tt\small]{lstmyf}
}{lstmyf}
\end{frame}

\begin{frame}[fragile]{Listings (also)}\relax

\cprotect\twocolImg{
    \lstinputlisting[linerange={6-18, 22-22}]{lstmys}
}{lstmys}
\end{frame}

\begin{frame}[fragile]{Minted}\relax

\cprotect\twocolImg{
    \lstinputlisting[linerange={6-6, 11-11}]{mintmy}
}{mintmy}
\end{frame}

\begin{frame}[fragile]{Comparison}\relax
    \begin{description}
         \item[verbatim] is the default package, when you need just to add code 
         \item[minted] is the verbatim on steroids --- it will color your code in proper style, but in works through a python lybrary
         \item[listings] is a package, that you must tune by yourself, but it is the most ``tunnable'' package
    \end{description}
    
    \skfootnote{\url{https://en.wikibooks.org/wiki/LaTeX/Source_Code_Listings#Settings} setting for lstlisting}
\end{frame}

\begin{frame}[fragile]{Tips about the code packages}\relax
    \begin{itemize}
         \item you must use {\csk \verb|[fragile]|} option in presentation slides (beamer) in the slides with code 
         \item use {\csk \verb|\cprotect|} package and command if you want to bring code to the command
         \item you can include only part of the code and provide a path to your src folder
         \item you can can find lots about the code usage in the source of this presentation  
    \end{itemize}
\end{frame}

\begin{frame}[fragile]{Include graphics}\relax
     \cprotect\twocolImg{
    \lstinputlisting[linerange={6-7, 11-11}]{incgraph}
    }{incgraph}

\end{frame}

\begin{frame}[fragile]{Include graphics}{params}\relax

     {
\scriptsize

\begin{tabbing}
\lstinline|\includegraphics[|\=\lstinline|width=|\=\lstinline|\textwidth,|\=\lstinline|height=|\=\lstinline|0.5\textheight,|\=\lstinline|keepaspectratio]{|\=\lstinline|papeeria}|\+\\
the width of the picture\+\\ 
means ``for whole width of the text''\+\\ 
the height of the picture\+\\ 
half of the whole page height\+\\ 
the ratio will remain the same\+\\ 
file name. You can ommit \\ ~~~the extension
\end{tabbing}

     \cprotect\skfootnote{\url{https://en.wikibooks.org/wiki/LaTeX/Importing_Graphics} \url{https://www.tug.org/TUGboat/tb17-1/tb50reck.pdf}\\ 
     you also can rotate, take only part of pic,.. \verb|\graphicspath{{path1/}{path2/}}| --- how I provide path to images}
\end{frame}

\begin{frame}[fragile]{Floating}\relax

\cprotect\twocolImg{
    \lstinputlisting[linerange={6-8, 11-18}]{floatgraph}
}{floatgraph}
\pause
~\\~\\
Figure appeared not where it was declaired!
\end{frame}

\begin{frame}[fragile]{Tips}\relax
\begin{itemize}
    \item {\csk \texttt \textbackslash{}caption} generate caption to the figure 
    \item \LaTeX{} doesn't care of what is inside the {\csk \texttt figure}. You are responsable of the content.
    \item You can knidly ask \LaTeX\ to put the illustration where you want: 
    \begin{itemize}
        \item[t] top
        \item[b] bottom
        \item[p] separate page
        \item[h] in place
    \end{itemize}
\end{itemize}

    \skfootnote{\url{https://en.wikibooks.org/wiki/LaTeX/Floats,_Figures_and_Captions}\\ 
    \lvoc{IV.8} for basis and \lvoc{IX.7.2} for advanced --- max fraction, gaps between text and illustration,..}
\end{frame}


\begin{frame}[fragile]{Wrapping graphics}\relax
    \cprotect\twocolImg{
    \lstinputlisting[linerange={6-7, 12-16}]{wrapgraph}
}{wrapgraph}

\skfootnote{\url{https://www.overleaf.com/learn/latex/Wrapping_text_around_figures}}
\end{frame}

\begin{frame}[fragile]{Tables: Floating and wrapping}\relax

\verb|\begin{table}table\end{table}|

\verb|\begin{wraptable}table\end{wraptable}|

\skfootnote{\url{https://www.overleaf.com/learn/latex/Tables} \\ \url{http://texdoc.net/texmf-dist/doc/latex/wrapfig/wrapfig-doc.pdf}}     
\end{frame}

\begin{frame}[fragile]{Tables}{tabbing}\relax
    \cprotect\twocolImg{
    \lstinputlisting[linerange={10-14}]{tabbingmy}
}{tabbingmy}
     \pause
     \begin{itemize}
         \item ommit \verb|\kill| to show the first line
         \item You can also reinstall tabular position inside the tabbing.
     \end{itemize}
     \skfootnote{\lvoc{VI.1}}
\end{frame}

\begin{frame}[fragile]{Tables}{tabbing}\relax
    \cprotect\twocolImg{
    \lstinputlisting[linerange={10-15}]{tabularmy}
}{tabularmy}
     \skfootnote{\lvoc{VI.2}}
\end{frame}

\begin{frame}[fragile]{Tables}{tabbing}\relax
     \begin{itemize}
         \item Line: \verb!o & x & o\\\hline!
         \begin{itemize}
             \item {\csk \&} --- moves to the next cell 
             \item {\csk \verb!\\!} --- movest to the next line 
             \item {\csk \verb!\hline!} --- provide a horizontal line between cells. You can ommit it.
         \end{itemize}
         \item preambula {\csk \verb!{||c|c|c|}!} 
         \begin{itemize}
             \item number of letters --- number of columns
             \item {\csk \verb!|!} stand for vertical line 
             \item available letters:
             \begin{itemize}
                 \item[l] pressed to the left
                 \item[r] pressed to the right
                 \item[c] centered
                 \item[p\{<size>\}] place for a paragraph with some <size> width 
             \end{itemize}
         \end{itemize}
     \end{itemize}
     \skfootnote{\lvoc{VI.2}}
\end{frame}

\begin{frame}[fragile]{Tabular: what else?}
     \begin{itemize}
         \item It is hard to make a table without WYSIWYG. Use this \url{https://www.tablesgenerator.com/}
         \item To color every second row: \url{https://tex.stackexchange.com/questions/5363/how-to-create-alternating-rows-in-a-table}
         \item You can use multicells, change the size between cells, make horizontal line for not all cells,.. See \lvoc{VI.2}
          
     \end{itemize}
\end{frame}