% \knuthc  knuth the TeXBook
% \lvoc   Lvovsky
% \lamc  lamport latex 
% \slshape different font for footnote
\graphicspath{{sec03/images/s3/}{sec03/code/s3/}}
\lstset{inputpath=sec03/code/s3/}

\begin{frame}[fragile]{Input some elements\preMagicPage}\relax
\begin{itemize}
     \item Code 
     \item Figures (pictures)
     \item Tabels
     \item \TeX\ files
\end{itemize}
\end{frame}

\begin{frame}[fragile]{Code\magicPage}\relax
\begin{center}
\begin{tabular}{c|c|c}
\hline
\multicolumn{3}{c}{{\ccol{\usepackage}} }\\\hline
 verbatim  &  listings  &  minted  \\\hline
\multicolumn{3}{c}{{\csk inline} }\\\hline
\textbackslash{}verb!code! & \textbackslash{}lstinline|code| & \textbackslash{}mintinline\{LaTeX\}\{Code\}\\\hline

\multicolumn{3}{c}{{\csk environment (\textbackslash{}begin\{env\} code \textbackslash{}end\{env\})} }\\\hline
\{verbatim\} & \{lstlisting\} &\{minted\}\\\hline
\multicolumn{3}{c}{{\csk file} }\\\hline
\textbackslash{}verbatiminput & \textbackslash{}lstinputlisting& \textbackslash{}inputminted\\\hline
\end{tabular}
\end{center}

\cprotect\skfootnote{\wikiC{https://en.wikibooks.org/wiki/LaTeX/Source_Code_Listings}\\ note: \verb"||, !!" are any characters that cannot be found in \verb|code|}
\end{frame}

\begin{frame}[fragile]{Verbatim\magicPage}\relax

\cprotect\twocolImg{
    \lstinputlisting[linerange={6-6, 11-11},basicstyle=\tt\small]{vemy}
}{vemy}

\skfootnote{\lmanc{8.27}[87]}
\end{frame}

\begin{frame}[fragile]{Listings\magicPage}\relax

\cprotect\twocolImg{
    \lstinputlisting[linerange={6-6, 11-11},basicstyle=\tt\small]{lstmyf}
}{lstmyf}
\end{frame}

\begin{frame}[fragile]{Listings (also)\magicPage}\relax

\cprotect\twocolImg{
    \lstinputlisting[linerange={6-18, 22-22}]{lstmys}
}{lstmys}

 \skfootnote{\wikiC{https://en.wikibooks.org/wiki/LaTeX/Source_Code_Listings\#Settings} setting for lstlisting}
 
\end{frame}

\begin{frame}[fragile]{Minted\magicPage}\relax

\cprotect\twocolImg{
    \lstinputlisting[linerange={6-6, 11-11}]{mintmy}
}{mintmy}
\skfootnote{\overC{https://www.overleaf.com/learn/latex/Code_Highlighting_with_minted} \stExC{https://tex.stackexchange.com/questions/389191/minted-vs-listings-pros-and-cons}}
\end{frame}

\begin{frame}[fragile]{Comparison\magicPage}\relax
    \begin{description}
         \item[verbatim] is the default package, when you need just to add code 
         \item[minted] is the verbatim on steroids --- it will color your code in proper style, but in works through a python lybrary
         \item[listings] is a package, that you must tune by yourself, but it is the most ``tunnable'' package
    \end{description}
    
\end{frame}

\begin{frame}[fragile]{Tips about the code packages\magicPage}\relax
    \begin{itemize}
         \item you must use {\csk \verb|[fragile]|} option in presentation slides (beamer) in the slides with code 
         \item use {\csk \verb|\cprotect|} package and command if you want to bring code to the command
         \item you can include only part of the code and provide a path to your src folder
         \item you can can find lots about the code usage in the source of this presentation  
    \end{itemize}
    \skfootnote{\normalfont \url{http://mirror.macomnet.net/pub/CTAN/macros/latex/contrib/cprotect/cprotect.pdf}}
\end{frame}

\begin{frame}[fragile]{Include graphics}\relax
     \cprotect\twocolImg{
    \lstinputlisting[linerange={6-7, 11-11}]{incgraph}
    }{incgraph}
    
\inclassFrag{
Write a bunch of paragraphs (don't forget about \url{https://en.lipsum.com/feed/html})

Upload a picture into your papeeria project and try to include it into the pdf}[-1]

\end{frame}

\begin{frame}[fragile]{Include graphics}{params}\relax

     {
\scriptsize

\begin{tabbing}
\lstinline|\includegraphics[|\=\lstinline|width=|\=\lstinline|\textwidth,|\=\lstinline|height=|\=\lstinline|0.5\textheight,|\=\lstinline|keepaspectratio]{|\=\lstinline|papeeria}|\+\\
the width of the picture\+\\ 
means ``for whole width of the text''\+\\ 
the height of the picture\+\\ 
half of the whole page height\+\\ 
the ratio will remain the same\+\\ 
file name. You can ommit \\ ~~~the extension
\end{tabbing}
     \cprotect\skfootnote{\wikiC{https://en.wikibooks.org/wiki/LaTeX/Importing_Graphics} \lmanc{22.3.1}[193] \url{https://www.tug.org/TUGboat/tb17-1/tb50reck.pdf}\\ 
     you also can rotate, take only part of pic,.. \verb|\graphicspath{{path1/}{path2/}}| --- how I provide path to images}
\end{frame}

\begin{frame}[fragile]{Floating}\relax

\cprotect\twocolImg{
    \lstinputlisting[linerange={6-8, 11-18}]{floatgraph}
}{floatgraph}
~\\~\\
\inclassFrag{Try to wrap your figure with this environment}
\pause
Figure appeared not where it was declaired! 

\end{frame}

\begin{frame}[fragile]{Tips\preMagicPage}\relax
\begin{itemize}
    \item {\csk \texttt \textbackslash{}caption} generate caption to the figure 
    \item \LaTeX{} doesn't care of what is inside the {\csk \texttt figure}. You are responsable of the content.
    \item You can knidly ask \LaTeX\ to put the illustration where you want: 
    \begin{itemize}
        \item[t] top
        \item[b] bottom
        \item[p] separate page
        \item[h] in place
    \end{itemize}
\end{itemize}

    \skfootnote{\lmanc{8.10.62}[62] \lvoc{IV.8}[185] \wikiC{https://en.wikibooks.org/wiki/LaTeX/Floats,_Figures_and_Captions} \overC{https://www.overleaf.com/learn/latex/Positioning_images_and_tables} (advanced: \lvoc{IX.7.2}[329] \lmanc{5}[38] -- max fraction, gaps between text and illustration,.., \knuthc{15}[115])}
\end{frame}


\begin{frame}[fragile]{Wrapping graphics\magicPage}\relax
    \cprotect\twocolImg{
    \lstinputlisting[linerange={6-7, 12-16}]{wrapgraph}
}{wrapgraph}

\skfootnote{\lvoc{IV.8.2}[188] \overC{https://www.overleaf.com/learn/latex/Wrapping_text_around_figures} \wikiC{https://en.wikibooks.org/wiki/LaTeX/Floats,_Figures_and_Captions\#Wrapping_text_around_figures}}
\end{frame}

\begin{frame}[fragile]{Tables: Floating and wrapping\preMagicPage}\relax

\verb|\begin{table}table\end{table}|

\verb|\begin{wraptable}table\end{wraptable}|

\skfootnote{\lvoc{IV.8.1}[187] \lmanc{8.22}[86] \wikiC{https://en.wikibooks.org/wiki/LaTeX/Floats,_Figures_and_Captions\#Tables} \overC{https://www.overleaf.com/learn/latex/Tables} \normalfont \url{http://texdoc.net/texmf-dist/doc/latex/wrapfig/wrapfig-doc.pdf}}     
\end{frame}

\begin{frame}[fragile]{Tables\magicPage}{tabbing}\relax
    \cprotect\twocolImg{
    \lstinputlisting[linerange={10-14}]{tabbingmy}
}{tabbingmy}
     \pause
     \begin{itemize}
         \item ommit \verb|\kill| to show the first line
         \item You can also reinstall tabular position inside the tabbing.
     \end{itemize}
     \skfootnote{\lvoc{VI.1}[206] \lmanc{8.21}[83] (advanced: \knuthc{22}[241])}
\end{frame}

\begin{frame}[fragile]{Tables}{tabular}\relax
    \cprotect\twocolImg{
    \lstinputlisting[linerange={10-15}]{tabularmy}
}{tabularmy}

\inclassFrag{Guess, what {\csk \&} and \ccol{\\} hear for?}[-1]
\end{frame}

\begin{frame}[fragile]{Tables\preMagicPage}{tabular}\relax
     \begin{itemize}
         \item Line: \verb!o & x & o\\\hline!
         \begin{itemize}
             $\left.
             \begin{tabular}{p{.4\textwidth}} \vspace{-2ex}
             \item {\csk \&} --- moves to the next cell
             \item {\csk \verb!\\!} --- moves to the next line
             \end{tabular}
             \right\}\text{{\csk it is common, remember!}}$
             \item {\csk \verb!\hline!} --- provide a horizontal line between cells. You can ommit it.
         \end{itemize}
         \item preambula {\csk \verb!{||c|c|c|}!} 
         \begin{itemize}
             \item number of letters --- number of columns
             \item {\csk \verb!|!} stand for vertical line 
             \item available letters:
             \begin{itemize}
                 \item[l] pressed to the left
                 \item[r] pressed to the right
                 \item[c] centered
                 \item[p\{<size>\}] place for a paragraph with some <size> width 
             \end{itemize}
         \end{itemize}
     \end{itemize}
     \skfootnote{\lvoc{VI.2}[212] \lmanc{8.23}[87] \wikiC{https://en.wikibooks.org/wiki/LaTeX/Tables} \overC{https://www.overleaf.com/learn/latex/Tables} (advanced: \knuthc{22}[232])\\ 
     for this bracket (\}) I also used tabular!}
\end{frame}

\begin{frame}[fragile]{Tabular: what else?\magicPage}{WYSiWYG}\relax
It is hard to make a table without WYSIWYG. Use this {\csk\url{https://www.tablesgenerator.com/}}
\end{frame}

\begin{frame}[fragile]{Tabular: what else?\magicPage}{color}\relax

\cprotect\twocolImg{
    \lstinputlisting[linerange={6-6, 10-18}]{tabularcolored}
}{tabularcolored}
    \skfootnote{\stExC{https://tex.stackexchange.com/questions/5363/how-to-create-alternating-rows-in-a-table}. because of \normalfont \url{https://imgur.com/gallery/ZY8dKpA}}
\end{frame}

\begin{frame}[fragile]{Tabular: what else?\magicPage}{more complex cells}\relax

\cprotect\twocolImg{
    \lstinputlisting[linerange={10-14}]{tabularcomplex}
}{tabularcomplex}

\begin{itemize}
     \item \ccol{\cline} is as \ccol{\hline} for several columns
     \item \ccol{\multicolumn} is a multi column
\end{itemize}


\end{frame}

\begin{frame}[fragile]{Tabular: what else?\magicPage}{Preambula}\relax

\cprotect\twocolImg{
    \lstinputlisting[linerange={10-16}]{tabularcopy}
}{tabularcopy}

The word ``lunch'' isn't inside cells!
\end{frame}


\begin{frame}{Including a \LaTeX\ file}

\centering\Huge\ccol{\input}\{filename\}
     
     \skfootnote{\lmanc{24}[208] \overC{https://www.overleaf.com/learn/latex/Multi-file_LaTeX_projects} \stExC{https://tex.stackexchange.com/questions/123226/how-to-include-tex-files} \stExC{https://tex.stackexchange.com/questions/246/when-should-i-use-input-vs-include}}
\end{frame}