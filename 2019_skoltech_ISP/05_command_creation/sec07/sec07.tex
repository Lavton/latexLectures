%% It is just an empty TeX file.
%% Write your code here.
\graphicspath{{sec07/images/}{sec07/code/}}
\lstset{inputpath=sec07/code/}

\begin{frame}[fragile]{New command with optional arguments\magicPage}\relax
\twocolImg{
\lstinputlisting[linerange={10-16}]{commandoptdef.tex}
}{commandoptdef}

Use syntax \verb|\newcommand{\<cmdname>}[total_arg_num][defaults]{<code>}|
\end{frame}

\begin{frame}[fragile]{New command with optional arguments\magicPage}{using package}\relax
\twocolImg{
\lstinputlisting[linerange={7-7, 11-18}]{commandoptpack.tex}
}{commandoptpack}

Use \ncol\usepackage{xargs}\ and \ccol\newcommandx\ (notice \textbf{x} at the end)

\end{frame}

\begin{frame}[fragile]{New command with key=value syntax\magicPage}{keyval package}\relax
\twocolImg{
\lstinputlisting[linerange={9-16, 22-22}]{commandkeyval.tex}
}{commandkeyval}

Use \ncol\usepackage{keyval}\ and \ccol{\define@key}, \ccol{\setkeys}

\skfootnote{\stExC{https://tex.stackexchange.com/questions/58069/newcommand-key-value}}
\end{frame}


\begin{frame}[fragile]{New command with key=value syntax\magicPage}{keyval package}\relax
\twocolImg{
\lstinputlisting[linerange={9-9, 15-16}]{commandkeyvalpgf.tex}
}{commandkeyvalpgf}

Use \ncol\usepackage{pgfkeys}

\skfootnote{\stExC{https://tex.stackexchange.com/questions/34312/how-to-create-a-command-with-key-values}}
\end{frame}


\begin{frame}[fragile]{New package with key=value syntax\magicPage}\relax

\lstinputlisting[basicstyle=\tt\tiny]{packexample.sty}

\twocolImg{
\lstinputlisting[linerange={8-8, 12-12}]{packagekeyval.tex}
}{packagekeyval}
\twocolImg{
\lstinputlisting[linerange={8-8, 12-12}]{packagekeyval2.tex}
}{packagekeyval2}
\twocolImg{
\lstinputlisting[linerange={8-8, 12-12}]{packagekeyval3.tex}
}{packagekeyval3}


\end{frame}