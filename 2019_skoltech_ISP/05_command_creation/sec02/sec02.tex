\graphicspath{{sec02/images/}{sec02/code/}}
\lstset{inputpath=sec02/code/}

\begin{frame}{Create new command\lW}{without arguments}\relax
\twocolImg{
\lstinputlisting[linerange={8-12, 14-17, 19-19}]{commandwithout.tex}
}{commandwithout}
\ccol\newcommand\{<\ccol\commandname>\}\{<code>\} to create new macros

\inclassFrag{Try it out!}[-1]

\skfootnote{\lmanc{12.1}[113] \lvoc{VII.1.1}[234] \wikiC{https://en.wikibooks.org/wiki/LaTeX/Macros} \overC{https://www.overleaf.com/learn/latex/Commands\#Defining_a_new_command}}
\end{frame}

\begin{frame}{Recreate new command\lW}\relax
\inclassFrag{Try to add new command to previous example:

\lstinputlisting[linerange={8-11}]{recommandwithoutTASK.tex}
}[1]
\twocolImg{
\lstinputlisting[linerange={8-12, 15-17, 20-20}]{recommandwithout.tex}
}{recommandwithout}

\ccol\renewcommand\ to recreate already created command

\end{frame}

\begin{frame}{Create new command\lW}{with arguments}\relax
     \twocolImg{
\lstinputlisting[linerange={8-12, 14-17, 19-19}]{commandwith.tex}
}{commandwith}

\ocol\newcommand\{<commandname>\}[<number of args]\{<code>\}. Refer to arg as \#1, \#2, ...

\inclassFrag{Try it!}[-1]
\skfootnote{to define command with star use \ccol{\@ifstar} \lmanc{12.4}[116]}
\end{frame}

\inclassFrame{
\begin{frame}{Try it out!}

Try to type the following, using the macros creation:
\newcommand{\mypart}[2]{\frac{\partial^2 #1}{\partial #2^2}}
\newcommand{\lName}{Mantius de Virr, the Future Lord of the Universe and conqueror of the Galaxy}
    $$\Delta f = \mypart{f}{r} = \mypart{f}{x}+\mypart{f}{y}+\mypart{f}{z}$$
    {
    -- Hello. I'm \lName.\\
    -- What do you wish, \lName?\\
    -- \lName\ wishes a big Cola and French fries\\
    -- Here are they, \lName! \\
}     

\end{frame}
}

\inclassFrame{
\begin{frame}{Try it out(2)!}

Try to type the following, using the macros creation:
\newcommand{\mypart}[2]{\frac{\partial^2 #1}{\partial #2^2}}
\newcommand{\lName}{Lusy}
    $$\Delta g = \mypart{g}{r} = \mypart{g}{x}+\mypart{g}{y}+\mypart{g}{z}$$
    {
    -- Hello. I'm \lName.\\
    -- What do you wish, \lName?\\
    -- \lName\ wishes a big Cola and French fries\\
    -- Here are they, \lName! \\
}     

\end{frame}
}


\begin{frame}[fragile]{Command creation inside command creation}\relax
As simple as \verb|\newcommand{\name}{\newcommand{\othername}{smth}}|

\begin{enumerate}
    \item In the inner command, you can refer to the argument of outer command as \#1
    \item In the inner command, you can refer to the argument of inner command as \#\#1
\end{enumerate}
\incPause
Sometimes you can see something like
\lstinline[basicstyle=\tt\small]|\newcommand{\photo}[1]{\renewcommand{\photo}[#1]}|

\inclassFrag{What it is for?}[3] It provides the following usage: You can store something at first usage as \verb|\photo{myface.png}| and then refer to it as just \verb|\photo|
\end{frame}

\begin{frame}[fragile]{The scope}\relax
     \cprotect\inclassFrag{
     Try the following:
     
     \lstinline[basicstyle=\tt]|\newcommand{\htext}[1]{\Huge text}|
     
     \lstinline[basicstyle=\tt]|text \htext{text} text|
     }[1]
     The braces at command definition and at command usage ommited. If you want your code to have local effect -- provide an extra braces:
     
     not \\  \lstinline[basicstyle=\tt]|\newcommand{\htext}[1]{\Huge text}|
     
     but \\  \lstinline[basicstyle=\tt]|\newcommand{\htext}[1]{{\Huge text}}|
\end{frame}

\cprotect\inclassFrame{
\begin{frame}[fragile]{Modify the previous code:}

use \verb|\usepackage{color}| and \verb|\color{green}|

Try to type the following, using the macros creation:
\newcommand{\mypart}[2]{\frac{\partial^2 #1}{\partial #2^2}}
\newcommand{\lName}{{\color{green}Lusy}}
    $$\Delta g = \mypart{g}{r} = \mypart{g}{x}+\mypart{g}{y}+\mypart{g}{z}$$
    {
    -- Hello. I'm \lName.\\
    -- What do you wish, \lName?\\
    -- \lName\ wishes a big Cola and French fries\\
    -- Here are they, \lName! \\
}     

\end{frame}
}

\begin{frame}{New environment\magicPage}\relax

use \ccol\newenvironment\{<name>\}\{<code at begin>\}\{<code at end>\}

or \ccol\renewenvironment     

\skfootnote{\lmanc{12.8}[118]}
     
\end{frame}

\begin{frame}{Where to put command creation}\relax

\begin{enumerate}
    \item You can put it into document preamble
    \item You can put it inside document whenever you want. Then:
    \begin{itemize}
        \item The command can be used only after it's definition
        \item The command definition is LOCAL: the scope of the visibility is the GROUP
    \end{itemize}
    \item You can put it into style or class files
     
\end{enumerate}
     
\end{frame}