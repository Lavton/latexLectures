\graphicspath{{sec03/images/}{sec03/code/s3/}}
\lstset{inputpath=sec03/code/s3/}

\begin{frame}{Spaces}\relax
{\LARGE \ccol{glue} and \ccol{kern} provides spaces between boxes.}
     \twocolImg{
\lstinputlisting[linerange={11-11}]{intro.tex}
}{intro}

\skfootnote{\tugC{https://www.tug.org/utilities/plain/cseq.html\#kern-rp} \knuthc{12}[79]}
\end{frame}

\cprotect\inclassFrame{
\begin{frame}[fragile]{write the following}\relax

\verb|I want page top and bottom layouts to be| \verb|\hskip0.5em plus 0.4mm minus 3mm respectively|
     
\end{frame}
}

\begin{frame}{What is glue}\relax
Glue is more than just ``spaces'' between the boxes.

Glue is a {\bfseries \csk tensile} spaces between boxes.

\incPause
Glue syntax is: {\csk <normal-length>} [{\csk plus <how-can-it-stretch>}] [{\csk minus <how-can-it-shrink>}].
     
\end{frame}

\begin{frame}{\ccol\relax}\relax
     Don't want unexpected adding to you glue? (Or two you commands?) Use \ccol\relax! \ccol\relax\ does nothing by itself but says \TeX\ ``This is the end of what you've been doing''
     
     \skfootnote{\tugC{https://www.tug.org/utilities/plain/cseq.html\#relax-rp} }
\end{frame}


\begin{frame}{Where glue adds implicitly?}\relax
\obeylines
\hbox spread -40pt {Between words and sentences. Here are lots' of glue.}
\hbox spread -20pt {Between words and sentences. Here are lots' of glue.}
\hbox spread -10pt {Between words and sentences. Here are lots' of glue.}
\hbox spread -5pt {Between words and sentences. Here are lots' of glue.}
\hbox{Between words and sentences. Here are lots' of glue.}
\hbox spread 5pt {Between words and sentences. Here are lots' of glue.}
\hbox spread 10pt {Between words and sentences. Here are lots' of glue.}
\hbox spread 20pt {Between words and sentences. Here are lots' of glue.}
\hbox spread 40pt {Between words and sentences. Here are lots' of glue.}
\hrule
\small P.S. here is \string\hbox\ spread in range -40pt---40pt
\small Between paragraphs, there is also a glue. Notice: between sentences the glue is bigger, than between words.
\end{frame}

\begin{frame}{How to use glue in your own work}\relax
\twocolImg{
\lstinputlisting[linerange={11-22}]{skipsmy.tex}
}{skipsmy}
For horizontal space use \ccol\hskip\ or \ccol\hspace\\
For vertical space use \ccol\vskip\ or \ccol\vspace.
\skfootnote{\tugC{https://www.tug.org/utilities/plain/cseq.html\#hskip-rp} \tugC{https://www.tug.org/utilities/plain/cseq.html\#vskip-rp} \lmanc{19.2}[167] \lmanc{19.14}[176]\\ 
use \ccol{\vspace*} if you want space at begin of the page!
}
\end{frame}

\inclassFrame{
\begin{frame}{Try known and some special glue!}
     \lstinputlisting[linerange={1-11}]{fillTASK.tex}
     
     use in the place {\csk <glue1> and <glue2>} in different combinations the following: 10em, 1fil, 2fil, 1fill, 1filll
     
\end{frame}
}

\begin{frame}[fragile]{Infinite glue}\relax
      
\twocolImg{
\lstinputlisting[linerange={11-16}]{fillmy.tex}
}{fillmy}

\vspace{-6ex}\hangindent=0.5\textwidth
\hangafter=-2\ccol{fil}, \ccol{fill}, \ccol{filll} are infinity with different ``power''. Both ``plus'' and ``minus'' are allowed. Notice: alignment without tabbular!

\skfootnote{\knuthc{12}[83] }
\end{frame}

\begin{frame}{Abbreviations}\relax
     You can use:
     
     \ccol\hfil \hfill \ccol\hfill \hfill \ccol\hspace\{\ccol\fil\} \hfill  \ccol\hspace\{\ccol\fill\}
     
     \ccol\vfil  \hfill \ccol\vfill  \hfill \ccol\vspace\{\ccol\fil\} \hfill  \ccol\vspace\{\ccol\fill\}
\end{frame}



