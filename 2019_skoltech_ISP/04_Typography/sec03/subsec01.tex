\begin{frame}{Length}{absolute values}\relax
\def\showLength#1{\raise4pt\hbox{\vrule height 6pt depth2pt{\csk\rule{#1}{4pt}}\vrule height 6pt depth2pt} #1}

\centering
most common used:

\begin{tabular}{r|cc|l}
     pt& points & $\simeq$0.35mm & \showLength{12pt} \\
     mm& millimeters & $\simeq$2.84pt & \showLength{10mm} \\
     cm& centimeter & $\simeq$28.4pt, 10mm & \showLength{1cm} \\
     in& inch & $\simeq$72.27pt, $\simeq$, 25.4mm  & \showLength{1in} \\
\end{tabular}


\skfootnote{\wikiC{https://en.wikibooks.org/wiki/LaTeX/Lengths} \knuthc{10}[68] \lvoc{I.2.10}[26]}
\end{frame}


\begin{frame}{Length}{absolute values\magicPage}\relax
\def\showLength#1{\raise4pt\hbox{\vrule height 6pt depth2pt{\csk\rule{#1}{4pt}}\vrule height 6pt depth2pt} #1}

\centering
not so common used:

\begin{tabular}{r|cc|l}
     pt& points & $\simeq$0.35mm & \showLength{12pt} \\
     mm& millimeters & $\simeq$2.84pt & \showLength{10mm} \\\hline
     bp& big point & 1/72in, $\simeq$1.003pt & \showLength{12bp} \\
     pc& pica & 12pt, 4.2mm  & \showLength{1pc} \\
     dd& didot & $\simeq$1.07pt, $\simeq$0376mm  & \showLength{12dd} \\
     cc & cicero & 12dd & \showLength{1cc}\\ \hline
     sp & scaled point    & 1/$2^{16}$pt = 1/65536pt & \showLength{2097152sp} \\
\end{tabular}

(pt and mm here for comparison)

\inclassFrag{The \textbf{sp} seems to be more ``programming'', not ``typography'' unit, isn't it?} Every \TeX's length is a integer number of {\csk sp}

\skfootnote{\wikiC{https://en.wikibooks.org/wiki/LaTeX/Lengths} \knuthc{10}[68] \lvoc{I.2.10}[26]}
\end{frame}

\begin{frame}[fragile]{Length}{Relative values}\relax
\def\showLength#1{\raise4pt\hbox{\vrule height 6pt depth2pt{\csk\rule{#1}{4pt}}\vrule height 6pt depth2pt} #1}

\centering


\begin{tabular}{r|c|l}
     pt& points  & \showLength{12pt} \\
     mm& millimeters & \showLength{10mm} \\\hline
     em& roughly the {\csk width} of an {\csk 'M'} (uppercase) & \showLength{1em} \\
     ex& roughly the {\csk height} of an {\csk 'x'} & \showLength{1ex} \\\hline
\end{tabular}

\incPause example: \verb|\Huge|

\begin{tabular}{r|l}
     mm & \showLength{10mm} \\
     em & \showLength{1em} \\\hline
     \Huge mm & \Huge \showLength{10mm} \\
     \Huge em & \Huge \showLength{1em} \\
\end{tabular}

use {\csk em} for horizontal and {\csk ex} for vertical cases

\skfootnote{\wikiC{https://en.wikibooks.org/wiki/LaTeX/Lengths} \knuthc{10}[68] \lvoc{I.2.10}[26]}
\end{frame}

\begin{frame}[fragile]{Prebuild lengths}{Most used}\relax

\centering
\begin{tabular}{r|l}
    \multicolumn{2}{c}{\tiny\TeX's}\\\hline
     \ccol\parindent & The normal paragraph indentation\\
     \ccol\parskip & The extra vertical space between paragraphs \\
     \hline\multicolumn{2}{c}{\tiny\LaTeX's}\\\hline
     \ccol\textwidth & The width of the text on the page\\
     \ccol\textheight & The height of the text on the page\\
     \ccol\linewidth & The width of the text in this ``box''\\
     \ccol\lineheight & The height of the text in this ``box''\\
\end{tabular}

 By typography rules, don't put both parskip and parindent as paragraph separation.

     \skfootnote{\wikiC{https://en.wikibooks.org/wiki/LaTeX/Lengths\#LaTeX_default_lengths} \lmanc{5.5}[34] \stExC{https://tex.stackexchange.com/questions/16942/difference-between-textwidth-linewidth-and-hsize}\\ 
     \ccol\parskip\ is actually a glue.  You can define your own lengths (See next lecture.)}
\end{frame}

\begin{frame}[fragile]{Prebuild lengths\magicPage}{not so common used}
     \centering {\LaTeX}\par
     practicully every length --- the margins; footnote, footer/header place; distance between columns,..\\[2ex]
     \TeX\par
     \begin{tabular}{rl}
    \ccol\hsize, \ccol\vsize & the normal size of text in page \\
    \ccol\hoffset, \ccol\voffset & the offset of on page 
     \end{tabular}
     
     
     \skfootnote{\vspace{-3ex}\lvoc{IV.4}[166] \lmanc{5.5}[34]\\ 
     by the way, you can magnitude whole text by \ccol\mag=... parameter.
     Also there are \ccol\hangindent\ and \ccol\hangafter\ \TeX\ params, but we'll speak in paragraph section. 
     }
\end{frame}

\begin{frame}\relax\magicPage
\centering
\includegraphics[height=0.9\textheight]{Layout-dimensions}
     
\skfootnote{\overC{https://www.overleaf.com/learn/latex/Page_size_and_margins}}
\end{frame}

\begin{frame}[fragile,t]{How to use lengths}\relax

\inclassFrag{
compile a document with different \ccol\parindent\ lengths
\lstinputlisting{lengthsTASK.tex}
}[1]

\samePosPicture{lengthsAr1}{11-11,18-20}

    Just \ccol{\<length-command>=<length>}

\end{frame}


\begin{frame}[fragile,t]{How to use lengths}\relax

\samePosPicture{lengthsAr2}{11-11,18-21}

Arifmetics: {\csk <multiply-factor>\ccol\<length-command>}

\end{frame}

\begin{frame}[fragile,t]{How to use lengths}\relax

\samePosPicture{lengthsAr3}{11-11,18-22}

Arifmetics: BUT You can't use simple notation {\csk +, -, /, *,...}
     % lengthsAr 18-20, 21-21, 22-22, 23-24
\end{frame}

\begin{frame}[fragile,t]{How to use lengths}\relax

\samePosPicture{lengthsAr4}{11-11,18-24}

Arifmetics: \ccol{\dimexpr} allow to use ``normal'' notation. 

\skfootnote{\stExC{https://tex.stackexchange.com/questions/245635/formal-syntax-rules-of-dimexpr-numexpr-glueexpr}}
\end{frame}


