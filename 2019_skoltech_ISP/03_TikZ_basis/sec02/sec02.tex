
\graphicspath{{sec02/images/}{sec02/code/}}
\lstset{inputpath=sec02/code/}

\begin{frame}{Path}
     
     Path is the main TikZ essence.
     
    %  \twocolImg{
\lstinputlisting[linerange={8-8, 11-13}]{pathmy.tex}
% }{skipsmy}

\inclassFrag{what do you think ``(0,0)'', ``(0,1)'' for?}

``(0,0)'', ``(0,1)'' is the simplest coordinate assignment. The (x, y) coordinate in units (typically 1cm)

\inclassFrag{Try to run the code}[4]
The \ccol\path\ is not draw anything by itself!
\end{frame}

\begin{frame}{Draw}\relax
\twocolImg{
\lstinputlisting[linerange={8-8, 14-16}]{drawmy1.tex}
}{drawmy1}

\begin{columns}
\begin{column}{0.45\textwidth}
     \lstinputlisting[linerange={8-8, 14-16}]{drawmy2.tex}
\end{column}
 \begin{column}{0.45\textwidth}
 %nothing    
\end{column}
\end{columns}

Use \ccol{\draw} or \ocol\path[draw] to draw
\end{frame}


\begin{frame}{Fill}\relax
\twocolImg{
\lstinputlisting[linerange={8-8, 14-16}]{fillmy.tex}
}{fillmy}

Use \ccol{\fill} or \ocol\path[fill] to fill. 

And {\csk \verb|--cycle| to make the path close}
\end{frame}

\begin{frame}{Options}

Let us add some color!

\twocolImg{
\lstinputlisting[linerange={8-8, 14-17}]{fillcolormy.tex}
}{fillcolormy}

new command \ccol\filldraw.

Notice the \textit{key=value} syntax!
     
\end{frame}

\begin{frame}{Options}

Draw another line

\twocolImg{
\lstinputlisting[linerange={8-8, 14-16}]{drawoptmy.tex}
}{drawoptmy}

\end{frame}

\begin{frame}{Other figures}
     
\twocolImg{
\lstinputlisting[linerange={9-9, 15-18}]{circlerectmy.tex}
}{circlerectmy}
      
\end{frame}

\begin{frame}{Nodes}\relax

\twocolImg{
\lstinputlisting[linerange={9-9, 15-17}]{nodemy.tex}
}{nodemy}

\ccol\node\ or \ocol\path[node]. Without optional arguments a node has no border
     
\end{frame}

\inclassFrame{
\begin{frame}{Draw a snowman!}
\centering
\begin{tikzpicture}

\draw (0, 0) circle[radius=1.5cm];
\draw (0, 2.4) circle[radius=1cm];
\draw (0, 4) circle[radius=0.7cm];
\filldraw[fill=gray] (-0.5, 4.5) rectangle (0.5, 5.5);
\filldraw[fill=orange, draw=orange!50!red] (0.3, 4.1) -- (1.5, 4) -- (0.3, 3.9) -- cycle;
\draw[very thick, ->] (-4, 4.2) -- (-1, 3); 
\node at (-4, 4.5) {snowman!};
\path (1, 3) -- (4, 5);
\end{tikzpicture}
     
\end{frame}
}


\begin{frame}{Vertical and horizontal}\relax
\twocolImg{
\lstinputlisting[linerange={9-9, 15-17}]{curveverthor.tex}
}{curveverthor}

Use {\csk -|} for ``first horizontal, than vertical''. Use {\csk |-} for ``first vertical, than horizontal''


\skfootnote{\tikzc{14.2.2}[152]}
     
\end{frame}


\begin{frame}{Curves}\relax
\vspace{-0.5cm}
\twocolImg{
\lstinputlisting[linerange={9-9, 15-20}]{curveinout.tex}
}{curveinout}
\footnotesize
\ccol{to [out=.., in=..]} the angle on which curve flows out and the angle on which curve flows in.

\twocolImg{
\lstinputlisting[linerange={9-9, 15-19}]{curveibez.tex}
}{curveibez}

\ccol{ .. controls <coord> and <coord> .. }


\skfootnote{\url{https://en.wikipedia.org/wiki/Bezier_curve}}
     
\end{frame}


\subsection{Coordinates}

\begin{frame}{``standart''}{x,y}\relax
\twocolImg{
\lstinputlisting[linerange={15-21}]{coordinatesSimple.tex}
}{coordinatesSimple}
\ccol{(<x-coord>, <y-coord>)}
\skfootnote{\tikzc{13.2}[133]}
\end{frame}
\begin{frame}{``standart''}{x,y,z}\relax
\twocolImg{
\lstinputlisting[linerange={15-21}]{coordinatesSimple2.tex}
}{coordinatesSimple2}
\ccol{(<x-coord>, <y-coord>, <z-coord>)}
\skfootnote{\tikzc{13.2}[133]}
\end{frame}
\begin{frame}{``standart''}{$\theta$,r}\relax
\twocolImg{
\lstinputlisting[linerange={15-21}]{coordinatesSimple3.tex}
}{coordinatesSimple3}
\ccol{(<$\theta$-coord>:<r-coord>)}
\skfootnote{\tikzc{13.2}[133]}
\end{frame}

\begin{frame}[fragile]{``++'' and ``+=+'' cordinates}\relax
\cprotect\inclassFrag{Write

\begin{lstlisting}
\begin{tikzpicture}
\draw (-3,-3) rectangle (3,3);
\draw (2,0) node{1} -- (1,0) node{2}  -- (0,1)  node{3} -- (-1,0)  node{4} -- cycle;
\end{tikzpicture}
\end{lstlisting}

\begin{lstlisting}
\begin{tikzpicture}
\draw (-3,-3) rectangle (3,3);
\draw (2,0) node{1} -- ++(1,0) node{2}  -- ++(0,1)  node{3} -- ++(-1,0)  node{4} -- cycle;
\end{tikzpicture}
\end{lstlisting}

\begin{lstlisting}
\begin{tikzpicture}
\draw (-3,-3) rectangle (3,3);
\draw (2,0) node{1} -- +(1,0) node{2}  -- +(0,1)  node{3} -- +(-1,0)  node{4} -- cycle;
\end{tikzpicture}
\end{lstlisting}
}[1]
\twocolImg{
\lstinputlisting[linerange={14-21}]{coordinatesPlus.tex}
}{coordinatesPlus}
 \begin{itemize}
     \item ``++'' use relative coordinate and set this new coordinate as ``current''
     \item ``+'' use relative coordinate and DOESN't set this new coordinate as ``current''
      
 \end{itemize}
\end{frame}

\begin{frame}{``Node''-based}\relax
\twocolImg{
\lstinputlisting[linerange={16-22}]{coordinatesNode.tex}
}{coordinatesNode}
 
\begin{enumerate}
    \item label node \verb|\node| \ccol{(<label>)}
    \item refer to the node as \ccol{(node cs:name=<label>)} 
    \item you can also use things like <label>.west or <label>.right
     
\end{enumerate}
     
\end{frame}

\begin{frame}{Coordinate calculation\magicPage}\relax
\twocolImg{
\lstinputlisting[linerange={9-10, 15-20}]{coordinatesCalc.tex}
}{coordinatesCalc}
 
\begin{enumerate}
    \item \ccol\usetikzlibrary\{calc\} 
    \item syntax: {\csk \$<coord1>!fraction!<coord2>\$}
    \item in this slide you can also see \ccol\foreach!
     
\end{enumerate}
     
\end{frame}

\begin{frame}{Coordinate intersection\magicPage}\relax
\twocolImg{
\lstinputlisting[linerange={9-10, 15-22}]{coordinatesInter.tex}
}{coordinatesInter}
 
\end{frame}
