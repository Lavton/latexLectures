%https://tex.stackexchange.com/questions/114219/add-notes-to-latex-beamer
\documentclass[14pt, aspectratio=169]{beamer}
\usepackage{fontspec}
\usepackage{xunicode}
\usepackage{xltxtra}
\usepackage{dsfont}
\usepackage{multicol}
\usepackage{soul}
% \usepackage{pgfpages}
% \setbeameroption{show notes}
%\setbeameroption{hide notes} % Only slides
%\setbeameroption{show only notes} % Only notes
% \setbeameroption{show notes on second screen=right} % Both
%% USAGE: 
% \usepackage[logo="sklogo.png"]{beamerskoltech} 
%   if you have a stand-alone image file for Sk logo 
% or
% \usepackage[logo]{beamerskoltech} 
%   if you has no logo-file, but want LaTeX to generate it. 
%   In this case you probably will need to use `--enable-write18 -interaction=nonstopmode` arguments running the latex command.
%   in papeeria and overleaf all works fine
% or 
% \usepackage{beamerskoltech}
%   In case you don't want logo at all 
%
% provided commands:
% color `skoltechgreen` -- the dark-green color for structure elements 
% command `\logoname` -- the name of logo file if exist 
% command `{\csk <text>}` -- the shortcut from `\color{skoltechgreen}`
% command `skfootnote{text}` -- put some text for current slide
% `\renewcommand{\skbeforetitle}{\vspace{-3ex}}` is useful in case you use `aspectratio=169`
%%%%%%%%%%%%%%%%
\usepackage[logo]{beamerskoltech} 
\usepackage{lecturessk}
\renewcommand{\skbeforetitle}{\vspace{-3ex}}

\usepackage{xecyr}
\usepackage{hyperref}

\usepackage{polyglossia}
\usepackage{graphicx}
\usepackage{listings}



\begin{document}


\title{\LaTeX:\\ \Large from dummy to \TeX nician}
\subtitle{Conclusion words}
\author{Anton Lioznov and David Saykin}
\institute{Skoltech}
\date{ISP 2019}
\frame{\titlepage}
% \begin{frame}\frametitle{What we will know?}
% \tableofcontents[hideallsubsections]
% \end{frame}

\begin{frame}\relax
\Huge \centering What the lectures were about?
\end{frame}

\begin{frame}{Lecture \#1: Overview and basis}\relax
\begin{enumerate}
    \item What is \LaTeX?
    \item In what situations you need to use it?
    \item In what situations it is useless?
    \item The idea of not-WYSiWYG approaches
    \item How to write a simple document 
    \item How to include external files 
    \item How to write math
\end{enumerate}
\end{frame}

\begin{frame}{Lecture \#2: Create articles, presentations, posters}\relax

\begin{enumerate}
    \item How to create the structure of the document 
    \item How to use reference and citations
    \item How to create a presentation: themes, overlays, animation
    \item How to create a poster
\end{enumerate}
     
\end{frame}

\begin{frame}{Lecture \#3: The bases of TikZ}\relax

\begin{enumerate}
    \item How to create a picture in \LaTeX 
    \item What options TikZ has 
    \item What figures and paths TikZ can create 
    \item What coordinate system TikZ has 
    \item How to create beautiful graphs with TikZ
\end{enumerate}
     
\end{frame}

\begin{frame}{Lecture \#4: Typography}\relax

\begin{enumerate}
    \item Changing fonts in \LaTeX
    \item Position blocks
    \item Useful typography package 
    \item Core \TeX\ concepts: Boxes and Glue
    \item How and when to use boxes and glue 
    \item ``Modes'' in \TeX
    \item How paragraphs created
\end{enumerate}
     
\end{frame}

\begin{frame}{Lecture \#5: Command creation}\relax

\begin{enumerate}
    \item Creating macros
    \item Creating style and class files
    \item Creating counters, lights, boxes,...
    \item Using \TeX\ for programming
    \item Providing optional arguments
    \item Working with filesystem
\end{enumerate}
     
\end{frame}


\begin{frame}

\Huge \centering Feedback review
\end{frame}

\begin{frame}{What will be changed at the next run?}{The major parts}\relax

\begin{enumerate}
    \item The hours will be expanded from 2 per lecture to 3
    \item We will re-design the practical parts
    \item We will focus on the main useful command 
    \item The tasks will be split in such a way so all students will able to make them 
    \item There will be more use cases for complicated stuff
    \item There will be lots of smaller changes: they are all written and fixed
     
\end{enumerate}
\end{frame}

\inclassFrame{
\begin{frame}{Feedback: ones more}\relax
\Huge\centering Please, provide a feedback ones more
     
\end{frame}}

\begin{frame}{Distribution}\relax
\begin{itemize}
     \item the pdf-version of the presentation and all printed materials can be distributed under license Creative Commons Attribution-ShareAlike 4.0 \url{https://creativecommons.org/licenses/by-sa/4.0/}
     \item The source code of the presentation is available on {\csk\url{https://github.com/Lavton/latexLectures}} and can be distributed under the MIT license \url{https://en.wikipedia.org/wiki/MIT_License\#License_terms}
\end{itemize}
     
\end{frame}
\end{document} 